\chapter{Lagrangian Dynamics}\label{s10}
\section{Introduction}
This chapter describes an elegant reformulation of the laws of
Newtonian dynamics which is due to  the French/Italian mathematician Joseph Louis
Lagrange. This reformulation is particularly useful for finding the
equations of motion of complicated dynamical systems.

\section{Generalized Coordinates}
Let the $q_i$, for $i=1,{\cal F}$, be a set of coordinates which uniquely
specifies the instantaneous configuration of some dynamical system.
Here, it is assumed that each of the $q_i$ can vary {\em independently}. 
The  $q_i$ might  be Cartesian coordinates, or polar 
coordinates, or angles, or some mixture of all three types of coordinate, and are, therefore,
termed {\em generalized coordinates}. A dynamical system whose
instantaneous configuration is fully specified by ${\cal F}$ independent
generalized coordinates is said to have ${\cal F}$ {\em degrees of freedom}. 
For instance, the instantaneous position of a particle moving freely
in three dimensions is completely specified by its three Cartesian
coordinates, $x$, $y$, and $z$. Moreover, these coordinates are clearly
independent of one another. 
Hence, a dynamical
system consisting of  a single particle moving
freely in three dimensions has three degrees of freedom. If there are two freely moving
particles then the system has six degrees of freedom, and so on.

Suppose that we have a dynamical system consisting of $N$ particles
moving freely in three dimensions. This is an ${\cal F}= 3\,N$ degree of
freedom system whose instantaneous configuration can be specified
by ${\cal F}$ Cartesian coordinates. Let us denote these coordinates the
$x_j$, for $j=1,{\cal F}$. Thus, $x_1, x_2, x_3$ are the Cartesian coordinates
of the first particle, $x_4, x_5, x_6$  the Cartesian coordinates
of the second particle, {\em etc}. Suppose that the instantaneous configuration of the system can also be specified by ${\cal F}$ generalized
coordinates, which we shall denote the $q_i$, for $i=1,{\cal F}$. Thus, the
$q_i$ might be the spherical coordinates of the particles.
 In general,
we expect the $x_j$ to be functions of the $q_i$. In other words,
\begin{equation}
x_j = x_j(q_1,q_2,\cdots,q_{\cal F}, t)
\end{equation}
for $j=1,{\cal F}$. Here, for the sake of generality, we have included the
possibility that the functional relationship between the $x_j$ and the
$q_i$ might depend on the time, $t$, explicitly. This would be the
case if the dynamical system were subject to time varying constraints. For
instance, a system consisting of a particle constrained to move on a surface which is itself
moving. Finally, by the chain rule, the variation of the $x_j$ due to a variation of the $q_i$
(at constant $t$) is given by
\begin{equation}\label{e10.2}
\delta x_j = \sum_{i=1,{\cal F}} \frac{\partial x_j}{\partial q_i}\,\delta q_i,
\end{equation}
for $j=1,{\cal F}$. 

\section{Generalized Forces}
The work done on the dynamical system when its Cartesian coordinates
change by $\delta x_j$ is simply
\begin{equation}
\delta W = \sum_{j=1,{\cal F}} f_j\,\delta x_j
\end{equation}
Here, the $f_j$ are the Cartesian components of the forces acting on the
various particles making up the system. Thus, $f_1, f_2, f_3$ are the
components of the force acting on the first particle, $f_4, f_5, f_6$ 
the components of the force acting on the second particle, {\em etc}. 
Using Equation~(\ref{e10.2}), we can also write
\begin{equation}
\delta W = \sum_{j=1,{\cal F}} f_j\sum_{i=1,{\cal F}}\frac{\partial x_j}{\partial q_i}\,\delta q_i.
\end{equation}
The above expression can be rearranged to give
\begin{equation}
\delta W = \sum_{i=1,{\cal F}} Q_i\,\delta q_i,
\end{equation}
where 
\begin{equation}\label{e10.6}
Q_i = \sum_{j=1,{\cal F}} f_j\,\frac{\partial x_j}{\partial q_i}.
\end{equation}
Here, the $Q_i$ are termed {\em generalized forces}. Note that a generalized
force does not necessarily have the dimensions of force. However, the
product $Q_i\,q_i$ must have the dimensions of work. Thus, if 
a particular $q_i$ is a Cartesian coordinate then the associated $Q_i$ is a force.
Conversely, if a particular $q_i$ is an angle then the associated $Q_i$ is a torque. 

Suppose that the dynamical system in question is {\em conservative}. It follows that
\begin{equation}
f_j = -\frac{\partial U}{\partial x_j},
\end{equation}
for $j=1,{\cal F}$, where $U(x_1,x_2,\cdots,x_{\cal F},t)$ is the system's potential energy. Hence, according to Equation~(\ref{e10.6}), 
\begin{equation}\label{e10.8}
Q_i = - \sum_{j=1,{\cal F}} \frac{\partial U}{\partial x_j}\,\frac{\partial x_j}{\partial q_i} = - \frac{\partial U}{\partial q_i},
\end{equation}
for $i=1,{\cal F}$. 

\section{Lagrange's Equation}
The Cartesian equations of motion of our system take
the form
\begin{equation}\label{e10.10}
m_j\,\ddot{x}_j = f_j,
\end{equation}
for $j=1,{\cal F}$, where $m_1, m_2, m_3$ are each equal to the mass of the
first particle, $m_4, m_5, m_6$ are each equal to the mass of the
second particle, {\em etc.} Furthermore, the kinetic energy of the
system can be written
\begin{equation}\label{e10.11}
K = \frac{1}{2}\sum_{j=1,{\cal F}} m_j\,\dot{x}_j^{\,2}.
\end{equation}

Now, since $x_j=x_j(q_1,q_2,\cdots, q_{\cal F},t)$, we can write
\begin{equation}
\dot{x}_j= \sum_{i=1,{\cal F}} \frac{\partial x_j}{\partial q_i}\,\dot{q}_i
+ \frac{\partial x_j}{\partial t},
\end{equation}
for $j=1,{\cal F}$.
Hence, it follows that $\dot{x}_j = \dot{x}_j(\dot{q}_1,\dot{q}_2,\cdots,
\dot{q}_{\cal F},q_1,q_2,\cdots,q_{\cal F},t)$. According to the
above equation,
\begin{equation}\label{e10.13}
\frac{\partial \dot{x}_j}{\partial\dot{q}_i} = \frac{\partial x_j}{\partial q_i},
\end{equation}
where we are treating the $\dot{q}_i$ and the $q_i$ as {\em independent}\/
variables.

Multiplying Equation~(\ref{e10.13}) by $\dot{x}_j$, and then differentiating
with respect to time, we obtain
\begin{equation}\label{e10.14}
\frac{d}{dt}\!\left(\dot{x}_j\,\frac{\partial \dot{x}_j}{\partial \dot{q}_i}\right)
= \frac{d}{dt}\!\left(\dot{x}_j\,\frac{\partial x_j}{\partial q_i}\right)
=\ddot{x}_j\, \frac{\partial x_j}{\partial q_i} + \dot{x}_j\,\frac{d}{dt}\!\left(
\frac{\partial x_j}{\partial q_i}\right).
\end{equation}
Now,
\begin{equation}\label{e10.15}
\frac{d}{dt}\!\left(\frac{\partial x_j}{\partial q_i}\right) = \sum_{k=1,{\cal F}}
\frac{\partial^2 x_j}{\partial q_i\,\partial q_k}\,\dot{q}_k + 
\frac{\partial^2 x_j}{\partial q_i\,\partial t}.
\end{equation}
Furthermore,
\begin{equation}\label{e10.16}
\frac{1}{2} \,\frac{\partial\dot{x}_j^{\,2}}{\partial \dot{q}_i}
= \dot{x}_j\,\frac{\partial \dot{x}_j}{\partial \dot{q}_i},
\end{equation}
and
\begin{eqnarray}\label{e10.17}
\frac{1}{2}\,\frac{\partial \dot{x}_j^{\,2}}{\partial q_i}
= \dot{x}_j\,\frac{\partial \dot{x}_j}{\partial q_i} &=&
\dot{x}_j\,\frac{\partial}{\partial q_i}\!\left(\sum_{k=1,{\cal F}}\frac{\partial x_j}{\partial q_k}\,\dot{q}_k + 
\frac{\partial x_j}{\partial t}\right)\nonumber\\[0.5ex]& =& \dot{x}_j\left(\sum_{k=1,{\cal F}}\frac{\partial^2 x_j}{\partial q_i\,\partial q_k}\,\dot{q}_k + 
\frac{\partial^2 x_j}{\partial q_i\,\partial t}\right)\nonumber\\[0.5ex]
&=& \dot{x}_j\,\frac{d}{dt}\!\left(\frac{\partial x_j}{\partial q_i}\right),
\end{eqnarray}
where use has been made of Equation~(\ref{e10.15}). Thus, it follows
from Equations~(\ref{e10.14}), (\ref{e10.16}), and (\ref{e10.17}) that
\begin{equation}
\frac{d}{dt}\!\left(\frac{1}{2}\,\frac{\partial \dot{x}_j^{\,2}}{\partial \dot{q}_i}\right) = \ddot{x}_j\,\frac{\partial x_j}{\partial q_i} + \frac{1}{2}\,\frac{\partial \dot{x}_j^{\,2}}{\partial q_i}.
\end{equation}

Let us take the above equation, multiply by $m_j$, and then sum over all $j$. 
We obtain
\begin{equation}
\frac{d}{dt}\!\left(\frac{\partial K}{\partial \dot{q}_i}\right) = \sum_{j=1,{\cal F}}
f_j\,\frac{\partial x_j}{\partial q_i} + \frac{\partial K}{\partial q_i},
\end{equation}
where use has been made of Equations~(\ref{e10.10}) and (\ref{e10.11}). Thus, it follows from Equation~(\ref{e10.6}) that
\begin{equation}
\frac{d}{dt}\!\left(\frac{\partial K}{\partial \dot{q}_i}\right) = Q_i + \frac{\partial K}{\partial q_i}.
\end{equation}
Finally, making use of Equation~(\ref{e10.8}), we get
\begin{equation}\label{e10.21}
\frac{d}{dt}\!\left(\frac{\partial K}{\partial \dot{q}_i}\right) = -\frac{\partial U}{\partial q_i}+\frac{\partial K}{\partial q_i}.
\end{equation}

It is helpful to introduce a function $L$, called the {\em Lagrangian}, which
is defined as the difference between the kinetic and potential energies of the dynamical system under investigation:
\begin{equation}\label{e10.21a}
L = K - U.
\end{equation}
Since the potential energy $U$ is clearly  independent of the
$\dot{q}_i$, it follows from Equation~(\ref{e10.21}) that
\begin{equation}\label{e10.23}
\frac{d}{dt}\!\left(\frac{\partial L}{\partial \dot{q}_i}\right)  -\frac{\partial L}{\partial q_i} =0,
\end{equation}
for $i=1,{\cal F}$. This equation is known as {\em Lagrange's equation}.

According to the above analysis, if we can express the kinetic and
potential energies of our dynamical system solely in terms of our generalized
coordinates and their time derivatives then we can immediately write
down the equations of motion of the system, expressed in terms
of the generalized coordinates, using Lagrange's equation, (\ref{e10.23}).
Unfortunately, this scheme only works for conservative systems.
Let us now consider some examples. 

\section{Motion in a Central Potential}\label{s10.5}
Consider a particle of mass $m$ moving in  two dimensions in the central potential $U(r)$. This is clearly a two degree of freedom dynamical system.
As described in Section~\ref{s6.5}, the particle's instantaneous position
is most conveniently specified in terms of the plane polar
coordinates $r$ and $\theta$. These are our two generalized coordinates.
According to Equation~(\ref{e6.14}), the square of the particle's velocity
can be written
\begin{equation}
v^2 = \dot{r}^{\,2} + (r\,\dot{\theta})^2.
\end{equation}
Hence, the Lagrangian of the system takes the form
\begin{equation}\label{e10.24}
L = \frac{1}{2}\,m\,(\dot{r}^{\,2} + r^2\,\dot{\theta}^{\,2}) - U(r).
\end{equation}
Note that
\begin{eqnarray}
\frac{\partial L}{\partial\dot{r}} = m\,\dot{r},&~~~~~~&\frac{\partial L}{\partial r} = m\,r\,\dot{\theta}^{\,2} - \frac{dU}{dr},\\[0.5ex]
\frac{\partial L}{\partial \dot{\theta}} = m\,r^2\,\dot{\theta},&&\frac{\partial L}{\partial\theta} = 0.
\end{eqnarray}
Now, Lagrange's equation (\ref{e10.23}) yields the equations of motion,
\begin{eqnarray}
\frac{d}{dt}\!\left(\frac{\partial L}{\partial\dot{r}}\right)- \frac{\partial L}{\partial r} &=& 0,\\[0.5ex]
\frac{d}{dt}\!\left(\frac{\partial L}{\partial\dot{\theta}}\right)- \frac{\partial L}{\partial \theta} &=& 0.
\end{eqnarray}
Hence, we obtain
\begin{eqnarray}
\frac{d}{dt}\!\left(m\,\dot{r}\right) - m\,r\,\dot{\theta}^{\,2} + \frac{dU}{dr} &=&0,\\[0.5ex]
\frac{d}{dt}\!\left(m\,r^2\,\dot{\theta}\right) = 0,
\end{eqnarray}
or
\begin{eqnarray}\label{e10.31}
\ddot{r} - r\,\dot{\theta}^{\,2}&=& - \frac{d V}{dr},\\[0.5ex]
r^2\,\dot{\theta}&=& h,\label{e10.32}
\end{eqnarray}
where $V = U/m$, and $h$ is a constant. We  recognize Equations~(\ref{e10.31}) and (\ref{e10.32}) as  the equations
we derived in Chapter~\ref{skepler} for motion in a central potential.
The advantage of the Lagrangian method of deriving these equations is
that we avoid having to express the acceleration in terms of the generalized
coordinates $r$ and $\theta$. 

\section{Atwood Machines}
An Atwood machine consists of two weights, of mass $m_1$ and $m_2$, 
connected by a light inextensible cord of length $l$, which passes over
a pulley of radius $a\ll l$, and moment of inertia $I$. See Figure~\ref{att}.

\begin{figure}
\epsfysize=3in
\centerline{\epsffile{Chapter09/fig9.01.eps}}
\caption{\em An Atwood machine.}\label{att}
\end{figure}

Referring to the diagram, we can see that this is a one degree of freedom
system whose instantaneous configuration is specified by the coordinate $x$.
Assuming that the cord does not slip with respect to the pulley, the
angular velocity of pulley is $\dot{x}/a$. Hence, the kinetic
energy of the system is given by
\begin{equation}
K = \frac{1}{2}\,m_1\,\dot{x}^{\,2} + \frac{1}{2}\,m_2\,\dot{x}^{\,2}
+ \frac{1}{2}\,I\, \frac {\dot{x}^{\,2}}{a^2}.
\end{equation}
The potential energy of the system takes the form
\begin{equation}
U = -m_1\,g\,x - m_2\,g\,(l-x).
\end{equation}
It follows that the Lagrangian is written
\begin{equation}
L=\frac{1}{2}\left(m_1+ m_2+\frac{I}{a^2}\right)\dot{x}^{\,2} + g\,(m_1-m_2)\,x + {\rm const}.
\end{equation}
The equation of motion,
\begin{equation}
\frac{d}{dt}\!\left(\frac{\partial L}{\partial\dot{x}}\right) - \frac{\partial L}{\partial x} = 0,
\end{equation}
thus yields
\begin{equation}
\left(m_1+ m_2+\frac{I}{a^2}\right)\ddot{x} - g\,(m_1-m_2) = 0,
\end{equation}
or
\begin{equation}
\ddot{x} = \frac{g\,(m_1-m_2)}{m_1+m_2 + I/a^2},
\end{equation}
which is the correct answer.

\begin{figure}
\epsfysize=3.25in
\centerline{\epsffile{Chapter09/fig9.02.eps}}
\caption{\em A double Atwood machine.}\label{att1}
\end{figure}

Consider the dynamical system drawn in Figure~\ref{att1}. This is an Atwood
machine in which one of the weights has been replaced by a second
Atwood machine with a cord of length $l'$. The system now has two degrees of freedom, and
its instantaneous position is specified by the two coordinates $x$ and $x'$,
as shown. 

For the sake of simplicity, let us neglect the masses of the two pulleys.
Thus, the kinetic energy of the system is written
\begin{equation}
K = \frac{1}{2}\,m_1\,\dot{x}^{\,2} + \frac{1}{2}\,m_2\,(-\dot{x}+ \dot{x}')^2 + \frac{1}{2}\,m_3\,(-\dot{x} - \dot{x}')^2,
\end{equation}
whereas the potential energy takes the form
\begin{equation}
U = - m_1\,g\,x - m_2\,g\,(l-x+x')- m_3\,g\,(l-x+l'-x').
\end{equation}
It follows that the Lagrangian of the system is
\begin{eqnarray}
L &=& \frac{1}{2}\,m_1\,\dot{x}^{\,2} + \frac{1}{2}\,m_2\,(-\dot{x}+ \dot{x}')^2 + \frac{1}{2}\,m_3\,(-\dot{x} - \dot{x}')^2\nonumber\\[0.5ex]
&&+ g\,(m_1-m_2-m_3)\,x+g\,(m_2-m_3)\,x' + {\rm const}.
\end{eqnarray}
Hence, the equations of motion,
\begin{eqnarray}
\frac{d}{dt}\!\left(\frac{\partial L}{\partial\dot{x}}\right) - \frac{\partial L}{\partial x} &=& 0,\\[0.5ex]
\frac{d}{dt}\!\left(\frac{\partial L}{\partial\dot{x}'}\right) - \frac{\partial L}{\partial x'} &=& 0,
\end{eqnarray}
yield
\begin{eqnarray}
m_1\,\ddot{x} + m_2\,(\ddot{x}- \ddot{x}') + m_3\,(\ddot{x}+\ddot{x}')
- g\,(m_1-m_2-m_3) &=& 0,\\[0.5ex]
m_2\,(-\ddot{x} + \ddot{x}')+m_3\,(\ddot{x}+\ddot{x}') - g\,(m_2-m_3) &=& 0.
\end{eqnarray}
The accelerations $\ddot{x}$ and $\ddot{x}'$ can be obtained from
the above two equations via simple algebra.

\section{Sliding down a Sliding Plane}\label{s10.7}
Consider the case of a particle of mass $m$ sliding down a smooth
inclined plane of mass $M$ which is, itself, free to slide on
a smooth horizontal surface, as shown in Figure~\ref{plane}.
This is a two degree of freedom system, so we need two
coordinates to specify the configuration. Let us choose $x$,
the horizontal distance of the plane from some reference point, and
$x'$, the parallel displacement of the particle from some reference point
on the plane.

\begin{figure}
\epsfysize=1.75in
\centerline{\epsffile{Chapter09/fig9.03.eps}}
\caption{\em A sliding plane.}\label{plane}
\end{figure}

Defining $x$- and $y$-axes, as shown in the diagram, the $x$- and 
$y$-components of the particle's velocity are clearly given by
\begin{eqnarray}
v_x &=& \dot{x} + \dot{x}'\,\cos\theta,\\[0.5ex]
v_y&=& -\dot{x}'\,\sin\theta,
\end{eqnarray}
respectively, where $\theta$ is the angle of inclination of the plane with
respect to the horizontal.
Thus,
\begin{equation}
v^2 = v_x^{\,2} + v_y^{\,2} = \dot{x}^{\,2} + 2\,\dot{x}\,\dot{x}'\,\cos\theta+ \dot{x}'^{\,2}.
\end{equation}
Hence, the kinetic energy of the system takes the form
\begin{equation}
K = \frac{1}{2}\,M\,\dot{x}^{\,2} + \frac{1}{2}\,m\,(\dot{x}^{\,2} + 2\,\dot{x}\,\dot{x}'\,\cos\theta+ \dot{x}'^{\,2}),
\end{equation}
whereas the potential energy is given by
\begin{equation}
U = - m\,g\,x'\,\sin\theta + {\rm const}.
\end{equation}
It follows that the Lagrangian is written
\begin{equation}\label{e10.51}
L = \frac{1}{2}\,M\,\dot{x}^{\,2} + \frac{1}{2}\,m\,(\dot{x}^{\,2} + 2\,\dot{x}\,\dot{x}'\,\cos\theta+ \dot{x}'^{\,2})+  m\,g\,x'\,\sin\theta + {\rm const}.
\end{equation}
The equations of motion,
\begin{eqnarray}
\frac{d}{dt}\!\left(\frac{\partial L}{\partial\dot{x}}\right) - \frac{\partial L}{\partial x} &=& 0,\\[0.5ex]
\frac{d}{dt}\!\left(\frac{\partial L}{\partial\dot{x}'}\right) - \frac{\partial L}{\partial x'} &=& 0,
\end{eqnarray}
thus yield
\begin{eqnarray}
M\,\ddot{x}+ m\,(\ddot{x} + \ddot{x}'\,\cos\theta) &=& 0,\\[0.5ex]
m\,(\ddot{x}'+ \ddot{x}\,\cos\theta) - m\,g\,\sin\theta &=& 0.
\end{eqnarray}
Finally, solving for $\ddot{x}$ and $\ddot{x}'$, we obtain
\begin{eqnarray}
\ddot{x} &=&  - \frac{g\,\sin\theta\,\cos\theta}{(m+M)/m-\cos^2\theta},\\[0.5ex]
\ddot{x}' &=& \frac{g\,\sin\theta}{1 - m\,\cos^2\theta/(m+M)}.
\end{eqnarray}

\section{Generalized Momenta}\label{s10.8}
Consider the motion of a single particle moving in one dimension. The
kinetic energy is
\begin{equation}
K = \frac{1}{2}\,m\,\dot{x}^{\,2},
\end{equation}
where $m$ is the mass of the particle, and $x$ its displacement.
Now, the particle's linear momentum is $p=m\,\dot{x}$. However,
this can also be written 
\begin{equation}
p = \frac{\partial K}{\partial \dot{x}}= \frac{\partial L}{\partial\dot{x}},
\end{equation}
since $L=K-U$, and the potential energy $U$ is independent of $\dot{x}$. 

Consider a dynamical system described by ${\cal F}$ generalized coordinates
$q_i$, for $i=1,{\cal F}$. By analogy with the above expression,  we can
define {\em generalized momenta} of the form
\begin{equation}
p_i = \frac{\partial L}{\partial\dot{q}_i},
\end{equation}
for $i=1,{\cal F}$. Here, $p_i$ is sometimes called the momentum {\em conjugate}\/ to the coordinate $q_i$. Hence, Lagrange's equation (\ref{e10.23}) can be written
\begin{equation}\label{e10.61}
\frac{d p_i}{dt} = \frac{\partial L}{\partial q_i},
\end{equation}
for $i=1,{\cal F}$. Note that a generalized momentum does not necessarily have
the dimensions of linear momentum.

Suppose that the Lagrangian $L$ does not depend explicitly on some coordinate
$q_k$. It follows from Equation~(\ref{e10.61}) that
\begin{equation}
\frac{d p_k}{dt} = \frac{\partial L}{\partial q_k}=0.
\end{equation}
Hence,
\begin{equation}
p_k = {\rm const.}
\end{equation}
The coordinate $q_k$ is said to be {\em ignorable}\/ in this case.
Thus, we conclude that the generalized momentum associated with
an ignorable coordinate is a constant of the motion.

For example, in Section~\ref{s10.5}, the Lagrangian (\ref{e10.24}) for a
particle moving in a central potential is independent of the angular
coordinate $\theta$. Thus, $\theta$ is an ignorable coordinate, 
and
\begin{equation}
p_\theta = \frac{\partial L}{\partial\dot{\theta}} = m\,r^2\,\dot{\theta}
\end{equation}
is a constant of the motion. Of course, $p_\theta$ is the angular momentum
about the origin. This is conserved because a central force exerts no torque
about the origin.

Again, in Section~\ref{s10.7}, the Lagrangian (\ref{e10.51}) for a mass
sliding down a sliding slope is independent
of the Cartesian coordinate $x$. It follows that $x$ is an ignorable coordinate,
and
\begin{equation}
p_x = \frac{\partial L}{\partial \dot{x}} = M\,\dot{x} + m\,(\dot{x}+\dot{x}'\,\cos\theta)
\end{equation}
is a constant of the motion. Of course, $p_x$ is the total linear momentum in the $x$-direction. This is conserved because there is no external force acting on
the system in the $x$-direction.

\section{Spherical Pendulum}
Consider a pendulum consisting of a  compact mass $m$ on the end of light
inextensible string of length $l$. Suppose that the mass is free to move
in any direction (as long as the string remains taut). Let the
fixed end of the string be located at the origin of our coordinate system. 
We can define Cartesian coordinates, ($x$, $y$, $z$), such that
the $z$-axis points vertically upward. We can also define spherical
 coordinates, ($r$, $\theta$, $\phi$), whose axis points along the $-z$-axis. The latter coordinates are the most convenient, since $r$ is constrained to always take the value $l$. However, the two angular coordinates,
$\theta$ and $\phi$, are free to vary independently. Hence, this is
 a two degree of freedom system. 

The Cartesian coordinates can be written in terms of the angular coordinates
$\theta$ and $\phi$. In fact, 
\begin{eqnarray}
x &=& l\,\sin\theta\,\cos\phi,\\[0.5ex]
y&=& l\,\sin\theta\,\sin\phi.\\[0.5ex]
z &=& -l\,\cos\theta.
\end{eqnarray}
Hence, the potential energy of the system is
\begin{equation}
U = m\,g\,z= - m\,g\,l\,\cos\theta.
\end{equation}
Also,
\begin{equation}
v^2 = \dot{x}^{\,2} + \dot{y}^{\,2} + \dot{z}^{\,2} = l^2\,(\dot{\theta}^{\,2}+ \sin^2\theta\,\dot{\phi}^{\,2}).
\end{equation}
Thus, the Lagrangian of the system is written
\begin{equation}
L = \frac{1}{2}\,m\,l^2\,(\dot{\theta}^{\,2}+ \sin^2\theta\,\dot{\phi}^{\,2})
+ m\,g\,l\,\cos\theta.
\end{equation}

Note that the Lagrangian is independent of the angular coordinate $\phi$. It
follows that
\begin{equation}
p_\phi = \frac{\partial L}{\partial \dot{\phi}} = m\,l^2\,\sin^2\theta\,\dot{\phi}
\end{equation}
is a constant of the motion. Of course, $p_\phi$ is the angular momentum of
the system about the $z$-axis. This is conserved because neither the tension
in the string nor the force of gravity exert a torque about the $z$-axis. 
Conservation of angular momentum about the $z$-axis implies that
\begin{equation}\label{e10.73}
\sin^2\theta\,\dot{\phi} = h,
\end{equation}
where $h$ is a constant.

The equation of motion of the system,
\begin{equation}
\frac{d}{dt}\!\left(\frac{\partial L}{\partial \dot{\theta}}\right) - \frac{\partial L}{\partial\theta} = 0,
\end{equation}
yields
\begin{equation}
\ddot{\theta} + \frac{g}{l}\,\sin\theta - \sin\theta\,\cos\theta\,\dot{\phi}^{\,2} = 0,
\end{equation}
or
\begin{equation}\label{e10.76}
\ddot{\theta} + \frac{g}{l}\,\sin\theta - h^2\,\frac{\cos\theta}{\sin^3\theta} = 0,
\end{equation}
where use has been made of Equation~(\ref{e10.73}).

Suppose that $\phi=\phi_0 = {\rm const}$. It follows that $\dot{\phi}=h=0$. Hence, Equation~(\ref{e10.76}) yields
\begin{equation}
\ddot{\theta} + \frac{g}{l}\,\sin\theta= 0.
\end{equation}
This, of course, is the equation of a simple pendulum whose motion is
restricted to the vertical plane $\phi = \phi_0$---see Section~\ref{s4.8}.

Suppose that $\theta=\theta_0 = {\rm const}$.  It follows from Equation~(\ref{e10.73}) that $\dot{\phi} = \dot{\phi}_0 = {\rm const}.$: {\em i.e.}, the pendulum bob rotates uniformly in a horizontal plane. According
to Equations~(\ref{e10.73}) and (\ref{e10.76}),
\begin{equation}
\dot{\phi}_0= \sqrt{\frac{g}{d}},
\end{equation}
where $d=l\,\cos\theta_0$ is the vertical distance of the plane of rotation below the pivot point. This type of pendulum is usually called a {\em conical
pendulum}, since the string attached to the pendulum bob sweeps out a
cone as the bob rotates. 

Suppose, finally, that the motion is almost conical: {\em i.e.}, the value
of $\theta$ remains close to the value $\theta_0$. Let
\begin{equation}
\theta = \theta_0 + \delta\theta.
\end{equation}
Taylor expanding Equation~(\ref{e10.76}) to first order in $\delta\theta$, the zeroth
order terms cancel out, and we are left with
\begin{equation}
\delta\ddot{\theta} + \dot{\phi}_0^{\,2}\,(1+ 3\,\cos^2\theta_0)\,\delta\theta \simeq 0.
\end{equation}
Hence, solving the above equation, we obtain
\begin{equation}
\theta \simeq \theta_0 + \delta\theta_0\,\cos({\mit\Omega}\,t),
\end{equation}
where
\begin{equation}
{\mit\Omega} = \dot{\phi}_0 \,\sqrt{1+ 3\,\cos^2\theta_0}.
\end{equation}
Thus, the angle $\theta$ executes simple harmonic motion about its
mean value $\theta_0$ at the angular frequency ${\mit\Omega}$. 

Now the azimuthal angle, $\phi$, increases by
\begin{equation}
{\mit\Delta}\phi \simeq \dot{\phi}_0\,\frac{\pi}{{\mit\Omega}}= \frac{\pi}{\sqrt{1+3\,\cos^2\theta_0}}
\end{equation}
as the angle of inclination to the vertical, $\theta$, goes between
successive maxima and minima. Suppose that $\theta_0$ is small.
In this case, ${\mit\Delta}\phi$ is slightly greater than $\pi/2$. Now
if ${\mit\Delta}\phi$ were exactly $\pi/2$ then the pendulum bob would
trace out the outline of a slightly {\em warped circle}: {\em i.e.}, something like the
outline of a potato chip or a saddle. The fact that ${\mit\Delta}\phi$
is slightly greater than $\pi/2$ means that this shape {\em precesses}\/ about
the $z$-axis in the {\em same}\/ direction as the direction rotation of the bob. The precession rate increases as 
the angle of inclination $\theta_0$ increases. Suppose, now, that $\theta_0$
is slightly less than $\pi/2$. (Of course, $\theta_0$ can never exceed $\pi/2$).
In this case, ${\mit\Delta}\phi$ is slightly less than $\pi$. Now
if ${\mit\Delta}\phi$ were exactly $\pi$ then the pendulum bob would
trace out the outline of a slightly {\em tilted circle}. The fact that
${\mit\Delta}\phi$ is slightly less than $\pi$ means that this shape
{\em precesses}\/ about the $z$-axis in the {\em opposite}\/ direction to the
direction of rotation of the bob. The precession rate increases as the
angle of inclination $\theta_0$ decreases below $\pi/2$. 

\section{Exercises}
{\small
\renewcommand{\theenumi}{9.\arabic{enumi}}
\begin{enumerate}
\item A horizontal rod $AB$ rotates with constant angular velocity $\omega$ about
its midpoint $O$. A particle $P$ is attached to it by equal strings $AP$, $BP$.
If $\theta$ is the inclination of the plane $APB$ to the vertical, prove that
$$
\frac{d^2\theta}{dt^2} -\omega^{\,2}\,\sin\theta\,\cos\theta = -\frac{g}{l}\,\sin\theta,
$$
where $l=OP$. Deduce the condition that the vertical position of $OP$ should be stable.

\item A double pendulum consists of two simple pendula, with one pendulum
suspended from the bob of the other. If the two pendula have equal lengths, $l$,
and have bobs of equal mass, $m$, and if both pendula are confined to move in the
same vertical plane, find Lagrange's equations of motion for the system. 
Use $\theta$ and $\phi$---the angles the upper and
lower pendulums  make with the downward vertical (respectively)---as the
generalized coordinates. Do
not assume small angles.

\item The surface of the Diskworld is a disk of radius $R$ which
rotates uniformly about a perpendicular axis passing through
its center with angular velocity ${\mit\Omega}$. Diskworld gravitational acceleration is of magnitude
$g$, and is everywhere directed normal to the disk.
Find the Lagrangian
of a projectile of mass $m$ using co-rotating cylindrical polar coordinates as the generalized
coordinates. What are the momenta conjugate to each coordinate? Are
any of these momenta conserved? Find Lagrange's equations of motion for
the projectile.

\item Find  Lagrange's equations of motion for  an elastic pendulum consisting of a particle
of mass $m$ attached to an elastic string of stiffness $k$ and unstretched
length $l_0$. Assume that the motion takes place in a vertical plane.

\item A disk of mass $M$ and radius $R$ rolls without slipping down a plane inclined at an angle $\alpha$ to the horizontal.
The disk has a short weightless axle of negligible radius. From this axle is suspended a simple pendulum
of length $l< R$ whose bob is of mass $m$. Assume that the motion of the pendulum takes place in the
plane of the disk. Find Lagrange's equations of motion of the system.

\epsfysize=1.75in
\centerline{\epsffile{Chapter09/fig9.04.eps}}
\item A vertical circular hoop of radius $a$ is rotated in a vertical plane about a point $P$ on its
circumference at the constant angular velocity $\omega$. A bead of
mass $m$ slides without friction on the hoop. Find the kinetic energy, the potential energy, the Lagrangian, and Largrange's equation
of motion of the bead, respectively,  in terms of the angular coordinate $\theta$
shown in the above diagram. Here, $x$ is a horizontal Cartesian coordinate,
$z$ a vertical Cartesian coordinate, and $C$  the center of the hoop.
Show that the beam oscillates like a pendulum about the point on the
rim diagrammatically opposite the point about which the hoop
rotates. What is the effective length of the pendulum?

\item Consider a spherical pendulum of length $l$. Suppose that the string is initially horizontal, and the bob is rotating horizontally with tangental velocity $v$. Demonstrate that, at its lowest subsequent point, the bob will have fallen a vertical height $l\,{\rm e}^{-u}$,
where
$$
\sinh u = \frac{v^2}{4\,g\,l}.
$$
Show that if $v^2$ is large compared to $4\,g\,l$ then this result becomes approximately
$2\,g\,l^2/v^2$. 

\item The kinetic energy of a rotating rigid object with an axis of symmetry
can be written
$$
K = \frac{1}{2}\left[I_\perp\,\dot{\theta}^{\,2} + (I_\perp\,\sin^2\theta + I_\parallel\,\cos^2\theta)\,\dot{\phi}^{\,2} + 2\,I_\parallel\,\cos\theta\,\dot{\phi}\,\dot{\psi} + I_\parallel\,\dot{\psi}^{\,2}\right],
$$
where $I_\parallel$ is the  moment of inertia about the symmetry axis,
$I_\perp$ is the moment of inertia about an axis perpendicular to the symmetry axis, and $\theta$, $\phi$, $\psi$ are the three Euler angles.
Suppose that the object is rotating freely.
Find the momenta conjugate to the Euler angles. Which of these
momenta are conserved? Find Lagrange's equations of motion for the
system. Demonstrate that if the system is precessing steadily (which
implies that $\theta$, $\dot{\phi}$, and $\dot{\psi}$ are constants) then
$$
\dot{\psi} = \left(\frac{I_\perp-I_\parallel}{I_\parallel}\right)\cos\theta\,\dot{\phi}.
$$

\item Consider a nonconservative system in which the 
dissipative forces take the form $f_i = -k_i\,\dot{x}_i$, where the $x_i$
are Cartesian coordinates, and the $k_i$ are all positive. Demonstrate that
the dissipative forces can be incorporated into the Lagrangian formalism 
provided that Lagrange's equations of motion are modified to read
$$
\frac{d}{dt}\left(\frac{\partial L}{\partial \dot{q}_i}\right) - \frac{\partial L}{\partial q_i} + \frac{\partial R}{\partial \dot{q}_i}=0,
$$
where
$$
R = \frac{1}{2} \sum_i k_i\,\dot{x}_i^{\,2}
$$
is termed the Rayleigh Dissipation Function.
\end{enumerate}
}