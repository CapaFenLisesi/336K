\chapter{The Three-Body Problem}\label{threeb}

\section{Introduction}
We saw earlier, in Section~\ref{sredm}, that  an isolated dynamical system consisting of two
freely moving point masses exerting forces on one another---which is usually referred to
as a {\em two-body problem}---can always be converted into an equivalent
one-body problem. In particular, this implies that we can {\em exactly solve}\/ a dynamical system containing {\em two}\/ gravitationally interacting
 point masses, since the equivalent one-body problem
is exactly soluble---see Chapter~\ref{skepler} and Section~\ref{sbin}. What about a system containing
{\em three}\/ gravitationally interacting point masses? Despite hundreds of years of research, no exact solution
of this famous problem---which is generally
known as the {\em three-body problem}---has ever been found. It is, however, possible
to make some progress by severely restricting the problem's scope.

\section{Circular Restricted Three-Body Problem}
Consider an isolated dynamical system consisting of three gravitationally  interacting point masses, $m_1$, $m_2$, and $m_3$.
Suppose, however, that the third mass, $m_3$, is so much smaller than the other two that it has a negligible
effect on their motion. Suppose, further, that the first two masses, $m_1$ and $m_2$, execute circular
orbits about their common center of mass. In the following, we shall investigate this simplified problem, which  is generally known as the {\em circular restricted
three-body problem}.  

\begin{figure}
\epsfysize=2.25in
\centerline{\epsffile{Chapter13/fig13.01.eps}}
\caption{\em The circular restricted three-body problem.}\label{f3b}
\end{figure}

Let us define a Cartesian coordinate system $(\xi,\,\eta,\,\zeta)$ in an inertial reference frame whose
origin coincides with the center of mass, $C$, of the two orbiting masses. Furthermore, let the orbital plane of these masses
coincide with the $\xi$-$\eta$ plane, and let them both lie on the $\xi$-axis at time $t=0$---see Figure~\ref{f3b}.
Suppose that $R$ is the constant distance between the two orbiting masses, $r_1$ the constant distance
between mass $m_1$ and the origin, and $r_2$ the constant distance between mass $m_2$ and the origin. Moreover,
let $\omega$ be the constant orbital angular velocity. It follows, from Section~\ref{sbin},
that
\begin{eqnarray}\label{e14.1}
\omega^2 &=& \frac{G\,M}{R^3},\\[0.5ex]
\frac{r_1}{r_2} &=& \frac{m_2}{m_1},
\end{eqnarray}
where $M=m_1+m_2$.

It is convenient to choose our unit of length such that $R=1$, and our unit of
mass such that $G\,M=1$. It follows, from Equation~(\ref{e14.1}), that $\omega =1$. However, we shall continue to
retain $\omega$ in our equations, for the sake of clarity. Let $\mu_1=G\,m_1$, and $\mu_2=G\,m_2=1-\mu_1$. It is easily demonstrated 
that $r_1 = \mu_2$, and $r_2=1-r_1=\mu_1$. Hence, the two orbiting masses, $m_1$ and $m_2$, have position
vectors ${\bf r}_1=(\xi_1,\,\eta_1,\,0)$ and ${\bf r}_2=(\xi_2,\,\eta_2,\,0)$, respectively, where (see Figure~\ref{f3b})
\begin{eqnarray}\label{e14.3}
{\bf r}_1 &=& \mu_2\,(-\cos \omega t,\,-\sin\omega t,\,0),\\[0.5ex]
{\bf r}_2 &=&\mu_1\,(\cos \omega t,\,\sin\omega t,\,0).\label{e14.4}
\end{eqnarray}
Let the third mass have position vector ${\bf r} = (\xi,\,\eta,\,\zeta)$. The
Cartesian components of the equation of motion of this mass are thus
\begin{eqnarray}\label{e14.5a}
\ddot{\xi}&=& -\mu_1\,\frac{(\xi-\xi_1)}{\rho_1^{\,3}} - \mu_2\,\frac{(\xi-\xi_2)}{\rho_2^{\,3}},\\[0.5ex]
\ddot{\eta}&=& -\mu_1\,\frac{(\eta-\eta_1)}{\rho_1^{\,3}} - \mu_2\,\frac{(\eta-\eta_2)}{\rho_2^{\,3}},\\[0.5ex]
\ddot{\zeta}&=& -\mu_1\,\frac{\zeta}{\rho_1^{\,3}} - \mu_2\,\frac{\zeta}{\rho_2^{\,3}},\label{e14.5c}
\end{eqnarray}
where
\begin{eqnarray}\label{e14.6}
\rho_1^{\,2} &=& (\xi-\xi_1)^2+(\eta-\eta_1)^2 + \zeta^2,\\[0.5ex]
\rho_2^{\,2} &=& (\xi-\xi_2)^2+(\eta-\eta_2)^2 + \zeta^2.\label{e14.7}
\end{eqnarray}

\section{Jacobi Integral}\label{sjac}
Consider the function
\begin{equation}\label{e14.10}
C = 2\left(\frac{\mu_1}{\rho_1}+\frac{\mu_2}{\rho_2}\right)  + 2\,\omega\,(\xi\,\dot{\eta}-\eta\,\dot{\xi})
-\dot{\xi}^2-\dot{\eta}^2-\dot{\zeta}^2.
\end{equation}
The time derivative of this function is
written
\begin{equation}
\dot{C} = - \frac{2\,\mu_1\,\dot{\rho}_1}{\rho_1^{\,2}}  - \frac{2\,\mu_2\,\dot{\rho}_2}{\rho_2^{\,2}} 
+ 2\,\omega\,(\xi\,\ddot{\eta} - \eta\,\ddot{\xi}) - 2\,\dot{\xi}\,\ddot{\xi} - 2\,\dot{\eta}\,\ddot{\eta}
- 2\,\dot{\zeta}\,\ddot{\zeta}.
\end{equation}
Moreover, it follows, from Equations~(\ref{e14.3})--(\ref{e14.4}) and  (\ref{e14.6})--(\ref{e14.7}),
that
\begin{eqnarray}
\rho_1\,\dot{\rho}_1 &=& -(\xi_1\,\dot{\xi}+\eta_1\,\dot{\eta}) + \omega\,(\xi\,\eta_1-\eta\,\xi_1) + \xi\,\dot{\xi}
 + \eta\,\dot{\eta} + \zeta\,\dot{\zeta},\\[0.5ex]
\rho_2\,\dot{\rho}_2 &=& -(\xi_2\,\dot{\xi}+\eta_2\,\dot{\eta}) + \omega\,(\xi\,\eta_2-\eta\,\xi_2) + \xi\,\dot{\xi}
 + \eta\,\dot{\eta} + \zeta\,\dot{\zeta}.
\end{eqnarray}
Combining Equations~(\ref{e14.5a})--(\ref{e14.5c}) with the above three expressions, we obtain (after considerable
algebra)
\begin{equation}
\dot{C} = 0.
\end{equation}
In other words, the function $C$---which is usually referred to as the {\em Jacobi integral}---is a {\em
constant of the motion}.

Now, we can rearrange Equation~(\ref{e14.10}) to give
\begin{equation}\label{e14.15}
{\cal E} \equiv \frac{1}{2}\,(\dot{\xi}^2+\dot{\eta}^2+\dot{\zeta}^2) -\left(\frac{\mu_1}{\rho_1}+\frac{\mu_2}{\rho_2}\right)
= \bomega\cdot{\bf h} - \frac{C}{2},
\end{equation}
where ${\cal E}$ is the energy (per unit mass) of mass $m_3$, ${\bf h} = {\bf r}\times \dot{\bf r}$ its angular momentum
(per unit mass), and $\bomega=(0,\,0,\,\omega)$  the orbital angular velocity of the other two masses.
Note, however, that ${\bf h}$ is {\em not}\/ a constant of the motion. Hence, ${\cal E}$ is not 
a constant of the motion either. In fact, the Jacobi integral is the {\em only}\/ constant of the
motion in the circular restricted three-body problem. Incidentally, the energy 
of mass $m_3$ is not a conserved quantity because the other two masses in the system
are {\em moving}. 

\section{Tisserand Criterion}
Consider a dynamical system consisting of three gravitationally
interacting point masses, $m_1$, $m_2$, and $m_3$. Let mass $m_1$ represent the Sun, mass $m_2$ the planet Jupiter, and mass $m_3$ a comet. Since the
mass of a comet is very much less than that of the Sun or Jupiter, and  the Sun and Jupiter
are in (almost) circular orbits about their common center of mass, the dynamical system in question satisfies
all of the necessary criteria to be considered an example of a restricted three-body problem. 

Now, the mass of the Sun is much greater than that of Jupiter. It follows that the gravitational
effect of Jupiter on the cometary orbit is {\em negligible}\/ unless the comet makes a very {\em close
approach}\/ to Jupiter. Hence, as described in Chapter~\ref{skepler}, before and after such an approach,  the comet executes a
standard elliptical orbit about the Sun with fixed orbital parameters: {\em i.e.}, fixed major radius, eccentricity, and
inclination to the ecliptic plane. However, in general, the orbital parameters before  and after the close approach will {\em not}\/ be the same
as one another. Let us investigate further.

Now, since $m_1\gg m_2$, we have $\mu_1=G\,m_1\simeq G\,(m_1+m_2)=1$, and $\rho_1\simeq r$.
Hence, according to Equations~(\ref{e6.51}) and (\ref{e6.57}), the (approximately) conserved
energy (per unit mass) of the comet before and after its close approach to Jupiter is
\begin{equation}\label{e14.16}
{\cal E} \equiv \frac{1}{2}\left(\dot{\xi}^{\,2}+\dot{\eta}^{\,2} + \dot{\zeta}^{\,2}\right) - \frac{1}{r} = - \frac{1}{2\,a}.
\end{equation}
Note that the comet's orbital energy is entirely determined by its major radius, $a$. (Incidentally, we are working
in units such that the major radius of Jupiter's orbit is unity.) Furthermore, the (approximately) conserved
angular momentum (per unit mass) of the comet before and after its approach to Jupiter
is written ${\bf h}$, where ${\bf h}$ is directed {\em normal}\/ to the comet's orbital plane,
and, from Equations~(\ref{e6.27}) and (\ref{e6.47}),
\begin{equation}
h^2 = a\,(1-e^2).
\end{equation}
Here, $e$ is the comet's orbital eccentricity. It follows that
\begin{equation}\label{e14.17}
\bomega\cdot {\bf h} = \omega\,h\,\cos I = \sqrt{a\,(1-e^2)}\,\cos I,
\end{equation}
since $\omega=1$ in our adopted system of units. Here, $I$ is the angle of
inclination of the normal to the comet's orbital plane to that of Jupiter's orbital
plane.

Let $a$, $e$, and $I$ be the major radius, eccentricity, and inclination angle of the cometary
orbit before the close encounter with Jupiter, and let $a'$, $e'$, and $I'$ be the corresponding
parameters after the encounter. It follows from Equations~(\ref{e14.15}), (\ref{e14.16}), and
(\ref{e14.17}), and the fact that $C$ is conserved during the encounter, whereas
${\cal E}$ and $h$ are not, that
\begin{equation}\label{e14.19}
\frac{1}{2\,a} + \sqrt{a\,(1-e^2)}\,\cos I = \frac{1}{2\,a'} + \sqrt{a'\,(1-e'^{\,2})}\,\cos I'.
\end{equation}
This result is known as the {\em Tisserand criterion}, and restricts the possible changes in the
orbital parameters of a comet due to a close encounter with Jupiter (or any other
massive planet).

The Tisserand criterion is very useful. For instance, whenever a new comet is discovered, astronomers
immediately calculate its {\em Tisserand parameter},
\begin{equation}
T_J = \frac{1}{a} + 2\,\sqrt{a\,(1-e^2)}\,\cos I.
\end{equation}
If this parameter has the same value as that of a previously observed comet then it
is quite likely that the new comet is, in fact, the same comet, but that
its orbital parameters have changed since it was last observed, due to a close encounter with Jupiter. Incidentally,
the subscript $J$ in the above formula is to remind us that we are dealing with the
Tisserand parameter for close encounters with {\em Jupiter}. (The parameter is, thus, evaluated
in a system of units in which the major radius of Jupiter's orbit is unity). Obviously,
it is also possible to calculate Tisserand parameters for close encounters with other planets.

The Tisserand criterion is also applicable to so-called {\em gravity assists}, in which a
space-craft gains energy due to a close encounter with a moving planet. Such assists
are often employed in missions to the outer planets  to reduce the amount of fuel
which the space-craft must carry in order to reach its destination. In fact, it is clear,
from Equations~(\ref{e14.16}) and (\ref{e14.19}), that a space-craft can make use of  a close encounter
with a moving planet to increase (or decrease) its orbital major radius $a$, and, hence, to increase
(or decrease)
its total orbital energy. 

\section{Co-Rotating Frame}
Let us transform to a non-inertial frame of reference rotating with angular
velocity $\omega$ about an axis normal
to the orbital plane of masses $m_1$ and $m_2$, and passing through their center of mass. 
It
follows that masses $m_1$ and $m_2$ appear {\em stationary}\/ in this new reference frame.
Let us define a Cartesian coordinate system $(x,\,y,\,z)$ in the rotating frame of reference which is
such that masses $m_1$ and $m_2$ always lie on the $x$-axis, and the $z$-axis
is parallel to the previously defined $\zeta$-axis. It follows that masses
$m_1$ and $m_2$ have the fixed position vectors ${\bf r}_1=\mu_2\,(-1,\,0,\,0)$ and
${\bf r}_2=\mu_1\,(1,\,0,\,0)$ in our new coordinate system. Finally, let the position vector of
mass $m_3$ be ${\bf r} = (x,\,y,\,z)$---see Figure~\ref{f3b1}.

\begin{figure}
\epsfysize=2.25in
\centerline{\epsffile{Chapter13/fig13.02.eps}}
\caption{\em The co-rotating frame.}\label{f3b1}
\end{figure}

According to Chapter~\ref{snoni}, the equation of motion of mass $m_3$ in the rotating
reference frame takes the form
\begin{equation}\label{e14.21}
\ddot{\bf r} + 2\,\bomega\times \dot{\bf r}= - \mu_1\,\frac{({\bf r}-{\bf r}_1)}{\rho_1^{\,3}} - 
\mu_2\,\frac{({\bf r}-{\bf r}_2)}{\rho_2^{\,3}} - \bomega\times(\bomega\times {\bf r}),
\end{equation}
where $\bomega = (0,\,0,\,\omega)$, and
\begin{eqnarray}
\rho_1^{\,2} &=& (x+\mu_2)^2+y^2 + z^2,\\[0.5ex]
\rho_2^{\,2} &=& (x-\mu_1)^2+y^2 + z^2.
\end{eqnarray}
Here, the second term on the left-hand side of Equation~(\ref{e14.21}) is the {\em Coriolis}\/ acceleration,
whereas the final term on the right-hand side is the {\em centrifugal}\/ acceleration. The components of Equation~(\ref{e14.21})
reduce to
\begin{eqnarray}
\ddot{x} - 2\,\omega\,\dot{y} &=& - \frac{\mu_1\,(x+\mu_2)}{\rho_1^{\,3}}- \frac{\mu_2\,(x-\mu_1)}{\rho_2^{\,3}}
+ \omega^2\,x,\\[0.5ex]
\ddot{y} + 2\,\omega\,\dot{x} &=& - \frac{\mu_1\,y}{\rho_1^{\,3}}- \frac{\mu_2\,y}{\rho_2^{\,3}}
+ \omega^2\,y,\\[0.5ex]
\ddot{z}  &=& - \frac{\mu_1\,z}{\rho_1^{\,3}}- \frac{\mu_2\,z}{\rho_2^{\,3}},
\end{eqnarray}
which yield
\begin{eqnarray}\label{e14.27}
\ddot{x} - 2\,\omega\,\dot{y} &=& -\frac{\partial U}{\partial x},\\[0.5ex]
\ddot{y} + 2\,\omega\,\dot{x} &=& -\frac{\partial U}{\partial y},\label{e14.28}\\[0.5ex]
\ddot{z}  &=& -\frac{\partial U}{\partial z},\label{e14.29}
\end{eqnarray}
where
\begin{equation}\label{e14.30}
U = - \frac{\mu_1}{\rho_1} - \frac{\mu_2}{\rho_2} - \frac{\omega^2}{2}\,(x^2+y^2)
\end{equation}
is the sum of the gravitational and centrifugal potentials.

Now, it follows from Equations~(\ref{e14.27})--(\ref{e14.29}) that
\begin{eqnarray}
\ddot{x}\,\dot{x} - 2\,\omega\,\dot{x}\,\dot{y} &=& -\dot{x}\,\frac{\partial U}{\partial x},\\[0.5ex]
\ddot{y}\,\dot{y} + 2\,\omega\,\dot{x}\,\dot{y} &=& -\dot{y}\,\frac{\partial U}{\partial y},\\[0.5ex]
\ddot{z}\,\dot{z}  &=& -\dot{z}\,\frac{\partial U}{\partial z}.
\end{eqnarray}
Summing the above three equations, we obtain
\begin{equation}
\frac{d}{dt}\left[\frac{1}{2}\left(\dot{x}^2+\dot{y}^2+\dot{z}^2\right) + U\right] = 0.
\end{equation}
In other words,
\begin{equation}\label{e14.35}
C = - 2\,U - v^2
\end{equation}
is a {\em constant of the motion}, where $v^2=\dot{x}^2+\dot{y}^2+\dot{z}^2$. In fact, $C$ is the
{\em Jacobi integral}\/ introduced in Section~\ref{sjac} [it is easily demonstrated that Equations~(\ref{e14.10}) and
(\ref{e14.35}) are identical].
 Note, finally, that
the mass $m_3$ is restricted to regions in which
\begin{equation}
-2\,U \geq C,
\end{equation} 
since $v^2$ is a positive definite quantity.

\section{Lagrange Points}
Let us search for possible {\em equilibrium  points}\/ of the mass $m_3$ in the rotating reference frame. Such points
are termed  {\em Lagrange points}. Thus, in the rotating  frame, the mass $m_3$   would remain at rest if placed at one of the Lagrange points. It is, thus, clear that these points are
{\em fixed}\/ in the rotating frame.
Conversely, in the inertial reference frame, the Lagrange points {\em rotate}\/ about the center of mass with
angular velocity $\omega$, and the mass $m_3$  would consequently also rotate about the center
of mass with angular velocity $\omega$ if placed at one of these points (with the appropriate
velocity). In the following,
we shall assumed, without loss of generality, that $m_1\geq m_2$.

The Lagrange points satisfy $\dot{\bf r} =\ddot{\bf r} = {\bf 0}$ in the rotating frame.
It thus follows, from Equations~(\ref{e14.27})--(\ref{e14.29}), that the Lagrange
points are the solutions of
\begin{equation}
\frac{\partial U}{\partial x} = \frac{\partial U}{\partial y}= \frac{\partial U}{\partial z}=0.
\end{equation}
Now, it is easily seen that
\begin{equation}
\frac{\partial U}{\partial z} = \left(\frac{\mu_1}{\rho_1^{\,3}}+ \frac{\mu_2}{\rho_2^{\,3}}\right) z.
\end{equation}
Since the term in brackets is positive definite, we conclude that the only solution to the
above equation is $z=0$. Hence, all of the Lagrange points lie in the $x$-$y$ plane.

If $z=0$ then it is readily demonstrated that
\begin{equation}
\mu_1\,\rho_1^{\,2} + \mu_2\,\rho_2^{\,2} = x^2+y^2+\mu_1\,\mu_2,
\end{equation}
where use has been made of the fact that $\mu_1+\mu_2=1$. 
Hence, Equation~(\ref{e14.30}) can also be written
\begin{equation}\label{e14.40}
U = - \mu_1\left(\frac{1}{\rho_1}+\frac{\rho_1^{\,2}}{2}\right)
- \mu_2\left(\frac{1}{\rho_2}+\frac{\rho_2^{\,2}}{2}\right) + \frac{\mu_1\,\mu_2}{2}.
\end{equation}
The Lagrange points thus satisfy
\begin{eqnarray}\label{e14.41}
\frac{\partial U}{\partial x} &=& \frac{\partial U}{\partial \rho_1}\,\frac{\partial \rho_1}{\partial x}+\frac{\partial U}{\partial \rho_2}\,\frac{\partial \rho_2}{\partial x}=0,\\[0.5ex]
\frac{\partial U}{\partial y} &=& \frac{\partial U}{\partial \rho_1}\,\frac{\partial \rho_1}{\partial y}+\frac{\partial U}{\partial \rho_2}\,\frac{\partial \rho_2}{\partial y}=0,\label{e14.42}
\end{eqnarray}
which reduce to
\begin{eqnarray}\label{e14.43}
\mu_1\left(\frac{1-\rho_1^{\,3}}{\rho_1^{\,2}}\right)\left(\frac{x+\mu_2}{\rho_1}\right)
+\mu_2\left(\frac{1-\rho_2^{\,3}}{\rho_2^{\,2}}\right)\left(\frac{x-\mu_1}{\rho_2}\right)&=&0, \\[0.5ex]
\mu_1\left(\frac{1-\rho_1^{\,3}}{\rho_1^{\,2}}\right)\left(\frac{y}{\rho_1}\right)
+\mu_2\left(\frac{1-\rho_2^{\,3}}{\rho_2^{\,2}}\right)\left(\frac{y}{\rho_2}\right)&=&0.\label{e14.44}
\end{eqnarray}

Now, one obvious solution of Equation~(\ref{e14.44}) is $y=0$, corresponding to a Lagrange
point which lies on the $x$-axis. It turns out that there are {\em three}\/ such points. $L_1$
lies between masses $m_1$ and $m_2$, $L_2$ lies to the right of mass $m_2$, and
$L_3$ lies to the left of mass $m_1$---see Figure~\ref{f3b1}. At the $L_1$ point,
we have $x = -\mu_2+\rho_1=\mu_1-\rho_2$ and $\rho_1=1-\rho_2$. Hence, from Equation~(\ref{e14.43}),
\begin{equation}\label{e14.45}
\frac{\mu_2}{3\,\mu_1} = \frac{\rho_2^{\,3}\,(1-\rho_2+\rho_2^{\,2}/3)}{(1+\rho_2+\rho_2^{\,2})\,(1-\rho_2)^3}.
\end{equation}
Assuming that $\rho_2\ll 1$, we can find an approximate solution of Equation (\ref{e14.45})
by expanding in powers of $\rho_2$:
\begin{equation}
\alpha = \rho_2 + \frac{\rho_2^{\,2}}{3} + \frac{\rho_2^{\,3}}{3} + \frac{51\,\rho_2^{\,4}}{81}+ {\cal O}(\rho_2^{\,5}).
\end{equation}
This equation can be inverted to give
\begin{equation}
\rho_2 = \alpha - \frac{\alpha^2}{3} - \frac{\alpha^3}{9} -\frac{23\,\alpha^4}{81}+{\cal O}(\alpha^5),
\end{equation}
where
\begin{equation}
\alpha = \left(\frac{\mu_2}{3\,\mu_1}\right)^{1/3}.
\end{equation}
is assumed to be a small parameter.

At the $L_2$ point,
we have $x = -\mu_2+\rho_1=\mu_1+\rho_2$ and $\rho_1=1+\rho_2$.
Hence, from Equation~(\ref{e14.43}),
\begin{equation}
\frac{\mu_2}{3\,\mu_1} = \frac{\rho_2^{\,3}\,(1+\rho_2+\rho_2^{\,2}/3)}{(1+\rho_2)^2\,(1-\rho_2^{\,3})}.
\end{equation}
Again, expanding in powers of $\rho_2$, we obtain
\begin{eqnarray}
\alpha &=& \rho_2 -\frac{\rho_2^{\,2}}{3} + \frac{\rho_2^{\,3}}{3} + \frac{\rho_2^{\,4}}{81}+ {\cal O}(\rho_2^{\,5}),\\[0.5ex]
\rho_2 &=& \alpha + \frac{\alpha^2}{3} - \frac{\alpha^3}{9} -\frac{31\,\alpha^4}{81}+{\cal O}(\alpha^5).
\end{eqnarray}

Finally, at the $L_3$ point,
we have $x = -\mu_2-\rho_1=\mu_1-\rho_2$ and $\rho_2=1+\rho_1$. Hence, from Equation~(\ref{e14.43}),
\begin{equation}
\frac{\mu_2}{\mu_1} = \frac{(1-\rho_1^{\,3})\,(1+\rho_1)^2}{\rho_1^{\,3}\,(\rho_1^{\,2}+3\,\rho_1+3)}.
\end{equation}
Let $\rho_1=1-\beta$. Expanding in powers of $\beta$, we obtain
\begin{eqnarray}
\frac{\mu_2}{\mu_1} &=&  \frac{12\,\beta}{7} + \frac{144\,\beta^2}{49} + \frac{1567\,\beta^3}{343} + {\cal O}(\beta^4),\\[0.5ex]
\beta &=&  \frac{7}{12}\left(\frac{\mu_2}{\mu_1}\right) - \frac{7}{12}\left(\frac{\mu_2}{\mu_1}\right)^2
+\frac{13223}{20736}\left(\frac{\mu_2}{\mu_1}\right)^3 + {\cal O}\left(\frac{\mu_2}{\mu_1}\right)^4,
\end{eqnarray}
where $\mu_2/\mu_1$ is assumed to be a small parameter.

Let us now search for Lagrange points which {\em do not lie}\/ on the $x$-axis. One obvious solution of Equations~(\ref{e14.41})
and (\ref{e14.42}) is 
\begin{equation}
\frac{\partial U}{\partial \rho_1} = \frac{\partial U}{\partial \rho_2} = 0,
\end{equation}
giving, from Equation~(\ref{e14.40}),
\begin{equation}
\rho_1 = \rho_2 =1,
\end{equation}
or
\begin{equation}
(x+\mu_2)^2 + y^2 = (x-1+\mu_2)^2 + y^2 = 1,
\end{equation}
since $\mu_1=1-\mu_2$.
The two solutions of the above equation are
\begin{eqnarray}
x &=& \frac{1}{2}-\mu_2,\\[0.5ex]
y &=& \pm \frac{\sqrt{3}}{2},
\end{eqnarray}
and specify the positions of the Lagrange points designated $L_4$ and $L_5$. Note that  point $L_4$ and the masses
$m_1$ and $m_2$ lie at the apexes of an {\em equilateral triangle}. The same is
true for   point $L_5$. We have now found all of the possible Lagrange points.

\begin{figure}
\epsfysize=2.75in
\centerline{\epsffile{Chapter13/fig13.03.eps}}
\caption{\em The masses $m_1$ and $m_2$, and the five Lagrange points, $L_1$ to $L_5$, calculated
for $\mu_2=0.1$.}\label{flag}
\end{figure}

Figure~\ref{flag} shows the positions of the two masses, $m_1$ and $m_2$, and
the five Lagrange points, $L_1$ to $L_5$, calculated for the case where $\mu_2=0.1$.

\section{Zero-Velocity Surfaces}
Consider the surface
\begin{equation}\label{e14.60}
V(x,y,z) = C,
\end{equation}
where
\begin{equation}
V = - 2\,U = \frac{2\,\mu_1}{\rho_1} 
+ \frac{2\,\mu_2}{\rho_2} +x^2+y^2.
\end{equation}
Note that $V\geq 0$.
It follows, from Equation~(\ref{e14.35}), that if the mass $m_3$ has the Jacobi
integral $C$, and lies on the surface specified in Equation~(\ref{e14.60}),
then it must have {\em zero}\/ velocity. Hence, such a surface is termed a
{\em zero-velocity surface}. The zero-velocity surfaces are important
because they form the boundary of regions from which
the mass $m_3$ is dynamically {\em excluded}: {\em i.e.}, regions in which $V< C$. 
Generally speaking, the regions from which $m_3$ is excluded grow in area as
$C$ increases, and {\em vice versa}.

Let  $C_i$ be the value of $V$ at the $L_i$ Lagrange point, for $i=1,5$. When $\mu_2\ll 1$,
it is easily demonstrated that
\begin{eqnarray}
C_1&\simeq & 3 + 3^{4/3}\,\mu_2^{\,2/3}-10\,\mu_2/3,\\[0.5ex]
C_2&\simeq & 3 + 3^{4/3}\,\mu_2^{\,2/3}-14\,\mu_2/3,\\[0.5ex]
C_3&\simeq & 3 + \mu_2,\\[0.5ex]
C_4&\simeq & 3 - \mu_2,\\[0.5ex]
C_4&\simeq & 3 - \mu_2.
\end{eqnarray}
Note that $C_1>C_2>C_3>C_4=C_5$.

\begin{figure}
\epsfysize=3.in
\centerline{\epsffile{Chapter13/fig13.04.eps}}
\caption{\em The zero-velocity surface $V=C$, where $C> C_1$, calculated for $\mu_2=0.1$. The mass $m_3$ is excluded
from the region lying between the two inner curves and the outer curve.}\label{fz1}
\end{figure}

\begin{figure}
\epsfysize=3.in
\centerline{\epsffile{Chapter13/fig13.05.eps}}
\caption{\em The zero-velocity surface $V=C$, where $C= C_1$, calculated for $\mu_2=0.1$. The mass $m_3$ is excluded
from the region lying between the  inner and  outer curves.}\label{fz2}
\end{figure}

\begin{figure}
\epsfysize=3.in
\centerline{\epsffile{Chapter13/fig13.06.eps}}
\caption{\em The zero-velocity surface $V=C$, where $C= C_2$, calculated for $\mu_2=0.1$.  The mass $m_3$ is excluded
from the region lying between the inner and  outer curve.}\label{fz3}
\end{figure}

\begin{figure}
\epsfysize=3.in
\centerline{\epsffile{Chapter13/fig13.07.eps}}
\caption{\em The zero-velocity surface $V=C$, where $C =C_3$, calculated for $\mu_2=0.1$.  The mass $m_3$ is excluded
from the regions lying inside the  curve.}\label{fz4}
\end{figure}

\begin{figure}
\epsfysize=3.in
\centerline{\epsffile{Chapter13/fig13.08.eps}}
\caption{\em The zero-velocity surface $V=C$, where $C_4< C < C_3$, calculated for $\mu_2=0.1$.  The mass $m_3$ is excluded
from the regions lying inside the two curves.}\label{fz5}
\end{figure}

Figures \ref{fz1}--\ref{fz5} show the intersection of the zero-velocity
surface $V=C$ with the $x$-$y$ plane for various different values of $C$, and
illustrate how the region from which $m_3$ is dynamically excluded---which we shall term the {\em excluded region}---evolves as the value of
$C$ is varied. Of course, any point not in the excluded region is in the so-called {\em allowed region}.
For $C> C_1$, the allowed region consists of two
separate oval regions centered on $m_1$ and $m_2$, respectively,  plus an
outer region which lies beyond a
large circle centered on the origin. All three allowed regions are separated
from one another by an excluded region---see Figure~\ref{fz1}. When $C=C_1$,
the two inner allowed regions merge at the $L_1$ point---see Figure~\ref{fz2}.
When $C=C_2$, the inner and outer allowed regions merge at the $L_2$ point, forming a  horseshoe-like excluded region---see Figure~\ref{fz3}.
When $C=C_3$, the excluded region splits in two at the $L_3$ point---see Figure~\ref{fz4}.
For $C_4 < C < C_3$, the two excluded  regions are localized about the
$L_4$ and 
$L_5$ points---see Figure~\ref{fz5}. Finally, for $C < C_4$, there is no excluded
region.

\begin{figure}
\epsfysize=5in
\centerline{\epsffile{Chapter13/fig13.09.eps}}
\caption{\em The zero-velocity surfaces and Lagrange points calculated for $\mu_2=0.01$.}\label{fzer}
\end{figure}

Figure~\ref{fzer} shows the zero-velocity surfaces and Lagrange points
calculated for the case $\mu_2=0.01$. It can be seen that, at very small values of
$\mu_2$, the $L_1$ and $L_2$ Lagrange points are almost {\em equidistant}\/ from mass $m_2$.
Furthermore, mass $m_2$, and the  $L_3$, $L_4$, and $L_5$ Lagrange points all lie  approximately  
on a {\em unit circle}, 
centered on mass $m_1$. It follows that, when $\mu_2$ is small, the Lagrange points $L_3$, $L_4$ and $L_5$ all
share the orbit of mass $m_2$ about $m_1$ (in the inertial frame) with $C_3$ being directly opposite $m_2$,
$L_4$ (by convention) $60^\circ$ ahead of $m_2$, and $L_5$ $60^\circ$ behind.

\section{Stability of Lagrange Points}
We have seen that the five Lagrange points, $L_1$ to $L_5$, are the equilibrium points
of mass $m_3$ in the co-rotating frame. Let us now determine whether or not
these equilibrium points are {\em stable}\/ to small displacements. 

Now, the equations of motion of mass $m_3$ in the co-rotating frame are
specified in Equations~(\ref{e14.27})--(\ref{e14.29}). Note that the motion
in the $x$-$y$ plane is complicated by presence of  the Coriolis acceleration. However, the motion parallel to the $z$-axis simply
corresponds to motion in the potential $U$. Hence, the condition for 
the stability of the Lagrange points (which all lie at $z=0$) to small displacements parallel
to the $z$-axis is simply (see Section~\ref{gpotn})
\begin{equation}
\left(\frac{\partial^2 U}{\partial z^2}\right)_{z=0} = \frac{\mu_1}{\rho_1^{\,3}} + \frac{\mu_2}{\rho_2^{\,3}}>0.
\end{equation}
This condition is satisfied everywhere in the $x$-$y$ plane. Hence, the Lagrange points are all
{\em stable}\/ to small displacements parallel to the $z$-axis. It, thus, remains to investigate
their stability to small displacements lying within the $x$-$y$ plane.

Suppose that a Lagrange point is situated in the $x$-$y$ plane at coordinates $(x_0,\,y_0,\,0)$.
Let us consider small amplitude $x$-$y$ motion in the vicinity of this point by writing
\begin{eqnarray}\label{e14.68}
x&=& x_0 + \delta x,\\[0.5ex]
y&=&y_0 + \delta y,\label{e14.69}\\[0.5ex]
z &=&0,\label{e14.69a}
\end{eqnarray}
where $\delta x$ and $\delta y$ are infinitesimal. Expanding $U(x,y,0)$ about the Lagrange point as a Taylor series, and retaining terms up to second-order in small
quantities,  we obtain
\begin{equation}
U = U_0 + U_x\,\delta x+ U_y\,\delta y + \frac{1}{2}\,U_{xx}\,(\delta x)^2+ U_{xy}\,\delta x\,\delta y
+ \frac{1}{2}\,U_{yy}\,(\delta y)^2,
\end{equation}
where $U_0=U(x_0,y_0,0)$, $U_x=\partial U(x_0,y_0,0)/\partial x$, $U_{xx}=\partial^2 U(x_0,y_0,0)/\partial x^2$, 
{\em etc.} However, by definition, $U_x=U_y=0$ at a Lagrange point, so the expansion simplifies to
\begin{equation}\label{e14.71}
U = U_0 + \frac{1}{2}\,U_{xx}\,(\delta x)^2+ U_{xy}\,\delta x\,\delta y
+ \frac{1}{2}\,U_{yy}\,(\delta y)^2.
\end{equation}
Finally, substitution of Equations~(\ref{e14.68})--(\ref{e14.69a}), and (\ref{e14.71})
into the equations of $x$-$y$ motion, (\ref{e14.27}) and (\ref{e14.28}), yields
\begin{eqnarray}
\delta\ddot{x} - 2\,\delta\dot{y} &=& - U_{xx}\,\delta x -U_{xy}\,\delta y,\\[0.5ex]
\delta\ddot{y} + 2\,\delta\dot{x} &=& - U_{xy}\,\delta x -U_{yy}\,\delta y,
\end{eqnarray}
since $\omega = 1$. 

Let us search for a solution of the above pair of equations
of the form $\delta x(t) = \delta x_0\,\exp(\gamma\,t)$ and
$\delta y(t) = \delta y_0\,\exp(\gamma\,t)$. We obtain
\begin{equation}
\left(
\begin{array}{cc}
\gamma^2 + U_{xx}& -2\,\gamma+U_{xy}\\[0.5ex]
2\,\gamma + U_{xy}& \gamma^2+ U_{yy}
\end{array}
\right)
\left(
\begin{array}{c}
\delta x_0\\[0.5ex]
\delta y_0
\end{array}
\right) = \left(
\begin{array}{c}
 0\\[0.5ex]
 0
\end{array}
\right).
\end{equation}
This equation only has a non-trivial solution if the determinant of the
matrix is zero. Hence, we get
\begin{equation}\label{e14.75}
\gamma^4 + (4+U_{xx}+U_{yy})\,\gamma^2 + (U_{xx}\,U_{yy}-U_{xy}^{\,2}) = 0.
\end{equation}
Now, it is convenient to define
\begin{eqnarray}
A &=& \frac{\mu_1}{\rho_1^{\,3}} + \frac{\mu_2}{\rho_2^{\,3}},\\[0.5ex]
B &=& 3\left[ \frac{\mu_1}{\rho_1^{\,5}} + \frac{\mu_2}{\rho_2^{\,5}}\right]y^2,\\[0.5ex]
C &=& 3\left[\frac{\mu_1\,(x+\mu_2)}{\rho_1^{\,5}}+\frac{\mu_2\,(x-\mu_1)}{\rho_2^{\,5}}\right]y,\\[0.5ex]
D &=& 3\left[\frac{\mu_1\,(x+\mu_2)^2}{\rho_1^{\,3}}+\frac{\mu_2\,(x-\mu_1)^2}{\rho_2^{\,3}}\right],
\end{eqnarray}
where all terms are evaluated at the point $(x_0,\,y_0,\,0)$. It thus follows that
\begin{eqnarray}
U_{xx} &=& A - D - 1,\\[0.5ex]
U_{yy} &=& A - B - 1,\\[0.5ex]
U_{xy} &=& - C.
\end{eqnarray}

Consider the co-linear Lagrange points, $L_1$, $L_2$, and $L_3$. These all
lie on the $x$-axis, and are thus characterized by
$y=0$, $\rho_1^{\,2} = (x+\mu_2)^2$, and $\rho_2^{\,2} = (x-\mu_1)^2$. It follows,
from the above equations, that $B=C=0$ and $D=3\,A$. Hence, $U_{xx}=-1-2\,A$,
$U_{yy} = A-1$, and $U_{xy}=0$. Equation (\ref{e14.75}) thus yields
\begin{equation}
\Gamma^2 + (2-A)\,\Gamma + (1-A)\,(1+2\,A) = 0,
\end{equation}
where $\Gamma=\gamma^2$. Now, in order for a Lagrange point to be stable
to small displacements, all four of the roots, $\gamma$,  of Equation~(\ref{e14.75}) must
be {\em purely imaginary}. This, in turn, implies that the two roots of
the above equation,
\begin{equation}
\Gamma = \frac{A-2\pm\sqrt{A\,(9\,A-8)}}{2},
\end{equation}
 must both be {\em real and negative}.
Thus, the stability criterion is
\begin{equation}
\frac{8}{9}\leq A \leq 1.
\end{equation}
Figure~\ref{fa} shows $A$  calculated at the three co-linear Lagrange points  as a function of $\mu_2$, for all
allowed values of this parameter ({\em i.e.}, $0<\mu_2\leq 0.5$). It can be seen that $A$
is always greater than unity for all three points. Hence, we conclude that the co-linear
Lagrange points, $L_1$, $L_2$, and $L_3$, are intrinsically {\em unstable}\/ equilibrium points in the co-rotating
frame.

\begin{figure}
\epsfysize=2.5in
\centerline{\epsffile{Chapter13/fig13.10.eps}}
\caption{\em The solid, short-dashed, and long-dashed curves show $A$ as a function of $\mu_2$ at the
$L_1$, $L_2$, and $L_3$ Lagrange points.}\label{fa}
\end{figure}

Let us now consider the triangular Lagrange points, $L_4$ and $L_5$. These points
are characterized by $\rho_1=\rho_2=1$. It follows that $A=1$, $B=9/4$, $C=\pm\sqrt{27/16}\,(1-2\,\mu_2)$,
and $D=3/4$. Hence, $U_{xx} = -3/4$, $U_{yy}=-9/4$, and $U_{xy} = \mp\sqrt{27/16}\,(1-2\,\mu_2)$,
where the upper/lower signs corresponds to $L_4$ and $L_5$, respectively.
Equation~(\ref{e14.75}) thus yields
\begin{equation}
\Gamma^2 + \Gamma + \frac{27}{4}\,\mu_2\,(1-\mu_2) = 0
\end{equation}
for both points,
where $\Gamma=\gamma^2$. As before, the stability criterion is that the two roots of the
above  equation must both be
real and negative. This is the case provided that $1 > 27\,\mu_2\,(1-\mu_2)$, which
yields the stability criterion
\begin{equation}
\mu_2 < \frac{1}{2}\left(1-\sqrt{\frac{23}{27}}\right) = 0.0385.
\end{equation}
In unnormalized units, this criterion becomes
\begin{equation}
\frac{m_2}{m_1+ m_2} < 0.0385.
\end{equation}

We thus conclude that the $L_4$ and $L_5$ Lagrange points are {\em stable}\/ equilibrium
points, in the co-rotating frame, provided that mass $m_2$ is less than about $4\%$ of
mass $m_1$. If this is the case then mass $m_3$ can orbit around these points
indefinitely. In the inertial frame, the mass will  share
the orbit of mass $m_2$ about mass $m_1$, but will stay approximately $60^\circ$ {\em ahead of}\/ 
mass $m_2$, if it is orbiting the $L_4$ point, or $60^\circ$ {\em behind}, if it is orbiting the $L_5$ point---see Figure~\ref{fzer}. This type of behavior has been observed in the Solar System. For instance,
there is a sub-class of asteroids, known as the {\em Trojan  asteroids}, which are trapped
in the vicinity of the $L_4$ and $L_5$ points of the Sun-Jupiter system (which easily
satisfies the above stability criterion), and consequently share Jupiter's orbit around the Sun,
staying approximately $60^\circ$ {\em ahead of}, and  $60^\circ$ {\em behind}, Jupiter, respectively.
Furthermore, the $L_4$ and $L_5$ points of the Sun-Earth system are occupied by clouds of dust.
