\chapter{One-Dimensional Motion}
\section{Introduction}
This chapter employs Newton's laws  to 
investigate one-dimensional motion. Particular
attention is given to the various mathematical techniques commonly used
to analyze  oscillatory motion.

\section{Motion in a General One-Dimensional Potential}\label{gpotn}
Consider a point particle of mass $m$ moving in the $x$-direction, say,
under the action of some $x$-directed force $f(x)$. 
Suppose that $f(x)$ is a conservative force: {\em e.g.}, gravity. In this case,
according to Equation~(\ref{e3.16}), we can write
\begin{equation}\label{egrad}
f(x) = - \frac{dU(x)}{dx},
\end{equation}
where $U(x)$ is the potential energy of the particle at position $x$. 

Let the curve $U(x)$ take the form shown in Figure~\ref{f43}.
For instance, this curve might represent the gravitational potential energy 
of a cyclist freewheeling in a hilly region. Observe that we have set the
potential energy at infinity to zero (which we are generally
free to do, since
potential energy is undefined to an arbitrary additive constant). This is a  fairly common  convention.
What can we deduce about the motion of the particle in this potential?

Well, we know that the total energy, $E$---which is the sum of the kinetic
energy, $K$, and the potential energy, $U$---is a {\em constant} of the motion---see Equation~(\ref{e3.22}).
Hence, we can write
\begin{equation}\label{e555}
K(x) = E - U(x).
\end{equation}
However, we also know that a kinetic energy can never be negative [since $K=(1/2)\,m\,v^2$, and neither
$m$ nor $v^2$ can be negative]. Hence, the above
expression tells us that the particle's motion  is restricted to the
region (or regions) in which the potential energy curve $U(x)$ falls
below the value $E$. This idea is illustrated in Figure~\ref{f43}.
Suppose that the total energy of the system is $E_0$. It is clear, from
the figure, that the particle is trapped inside one or other of the two dips
in the potential---these dips are
generally referred to as {\em potential wells}. 
Suppose that we now raise the energy to $E_1$. In this
case, the particle is free to enter or leave each of the potential wells, but
its motion is still {\em bounded}\/ to some extent, since it clearly cannot move off to
infinity. Finally, let us raise the energy to $E_2$. Now the
particle is {\em unbounded}: {\em i.e.}, it can move off to infinity. In conservative systems
in which it makes sense to adopt the convention that the potential
energy at infinity is zero, bounded systems are characterized
by $E<0$, whereas unbounded systems are characterized by  $E>0$. 

\begin{figure}
\epsfysize=2.25in
\centerline{\epsffile{Chapter03/fig3.01.eps}}
\caption{\em A potential energy curve.}\label{f43}   
\end{figure}

The above discussion suggests that the motion of an particle moving in a potential
generally becomes less bounded as the total energy $E$ of the system increases. 
 Conversely, we would expect the motion to become more bounded as $E$ decreases.
In fact, if the energy becomes sufficiently small then it appears likely that the
system will settle down in some {\em equilibrium state}\/ in which the particle is stationary.
Let us try  to identify any  prospective equilibrium states in Figure~\ref{f43}.
If the particle remains stationary then it must be subject to zero force (otherwise
it would accelerate). Hence, according to Equation~(\ref{egrad}), an equilibrium
state is characterized by
\begin{equation}
\frac{dU}{dx} = 0.
\end{equation}
In other words, a equilibrium state corresponds to either a {\em maximum}\/
or a {\em minimum}\/ of the potential energy curve $U(x)$. It can
be seen that the $U(x)$ curve shown in Figure~\ref{f43} has
three associated equilibrium states located at
$x=x_0$, $x=x_1$, and $x=x_2$. 

Let us now make a distinction between {\em stable}\/ equilibrium points
and {\em unstable}\/ equilibrium points. When the particle is slightly
displaced from a stable equilibrium point then the resultant force $f$ acting
on it
must always be such as to  return it to this point.
In other words, if $x=x_0$ is an equilibrium point then we require
\begin{equation}
\left.\frac{df}{dx}\right|_{x_0} <0
\end{equation}
for stability: {\em i.e.}, if the particle is displaced to the right, so that $x-x_0>0$,
then the force must act to the left, so that $f <0$, and {\em vice versa}.
Likewise, if 
\begin{equation}
\left.\frac{df}{dx}\right|_{x_0} >0
\end{equation}
then the equilibrium point $x=x_0$ is unstable. It follows, from
 Equation~(\ref{egrad}), that stable equilibrium points are
characterized by
\begin{equation}
\frac{d^2 U}{dx^2}>0.
\end{equation}
In other words, a stable equilibrium point corresponds to a {\em minimum}\/
of the potential energy curve $U(x)$. Likewise, an unstable
equilibrium point corresponds to a {\em maximum}\/ of the $U(x)$ curve. Hence,
we conclude that,
in Figure~\ref{f43},  $x=x_0$ and $x=x_2$ are stable equilibrium points, whereas $x=x_1$ is an unstable equilibrium point. 
Of course, this makes perfect sense if we think of $U(x)$ as
a gravitational potential energy curve,  so that $U$ is
directly proportional to height. In this case, all we are saying is that it is
easy to confine a low energy mass at the bottom of a valley,
but very difficult to balance the same mass on the top of
a hill (since any slight displacement of the mass will cause it
to fall down the hill). Note, finally, that if
\begin{equation}
\frac{dU}{dx}=\frac{d^2 U}{dx^2}=0
\end{equation}
at any point (or in any region) then we have what is known as a {\em neutral equilibrium}\/
point. We can move the particle slightly away from such a point and it will still
remain in equilibrium ({\em i.e.}, it will neither attempt to return to
its initial state,  nor will it continue to move). A neutral equilibrium point
corresponds to a {\em flat spot}\/ in a $U(x)$ curve. See Figure~\ref{f44}.

\begin{figure}
\epsfysize=2.in
\centerline{\epsffile{Chapter03/fig3.02.eps}}
\caption{\em Different types of equilibrium point.}\label{f44}   
\end{figure}

The equation of motion of an particle moving in one dimension
under the action of a conservative force is, in principle, integrable. Since
$K=(1/2)\,m\,v^2$, the energy
conservation equation (\ref{e555}) can be rearranged to give
\begin{equation}
v = \pm\left(\frac{2\,[E-U(x)]}{m}\right)^{1/2},
\end{equation}
where the $\pm$ signs correspond to motion to the left and to the right, respectively. However, since
$v=dx/dt$, this expression can be integrated to give
\begin{equation}
t=\pm\left(\frac{m}{2\,E}\right)^{1/2} \int_{x_0}^x\frac{dx'}{\sqrt{1-U(x')/E}},
\end{equation}
where $x(t=0)=x_0$. For sufficiently simple potential functions, $U(x)$, the
above equation can be solved to give $x$ as a function of $t$.
For instance, if $U=(1/2)\,k\,x^2$,  $x_0=0$, and the plus sign is chosen, then
\begin{equation}
t = \left(\frac{m}{k}\right)^{1/2}\int_0^{(k/2\,E)^{1/2}\,x}\frac{dy}{\sqrt{1-y^2}}= \left(\frac{m}{k}\right)^{1/2} \sin^{-1}\left(\left[\frac{k}{2\,E}\right]^{1/2} x\right),
\end{equation}
which can be inverted to give
\begin{equation}
x = a\,\sin (\omega\,t),
\end{equation}
where $a = \sqrt{2\,E/k}$ and $\omega = \sqrt{k/m}$. 
Note that the particle reverses direction each time it reaches one of the so-called
{\em turning points}\/ ($x=\pm a$) at which $U=E$ (and, so $K=0$). 

\section{Velocity Dependent Forces}\label{svelyd}
Consider a particle of mass $m$ moving in one dimension under the
action of a force, $f$, which is a function of the particle's speed, $v$, but not
of its displacement, $x$. Note that such a force is intrinsically non-conservative [since it clearly cannot be expressed as minus the gradient of 
some potential function, $U(x)$]. 
Now, the particle's equation of motion is written
\begin{equation}
m\,\frac{dv}{dt} = f(v).
\end{equation}
Integrating this equation, we obtain
\begin{equation}
\int_{v_0}^v \frac{dv'}{f(v')} = \frac{t}{m},
\end{equation}
where $v(t=0)=v_0$. In principle, the above equation can be solved to
give $v(t)$. The equation of motion is also written
\begin{equation}
m\,v\,\frac{dv}{dx} = f(v),
\end{equation}
since $v=dx/dt$. Integrating this equation, we obtain
\begin{equation}
\int_{v_0}^v \frac{v'\,dv'}{f(v')} = \frac{x-x_0}{m},
\end{equation}
where $x(t=0) = x_0$. In principle, the above equation
can be solved to give $v(x)$. 

Let us now consider a specific example. Suppose that an object of mass $m$ falls
vertically under gravity. Let $x$ be the height through which the object
has fallen since $t=0$, at which time the object is assumed to be at rest. It follows that
$x_0=v_0=0$. Suppose that, in addition to the force of gravity, $m\,g$, where $g$ is the gravitational acceleration, our
object is subject to a retarding air resistance force which is proportional
to the square of its instantaneous velocity. The object's equation
of motion is thus
\begin{equation}
m\,\frac{dv}{dt} = m\, g - c\,v^2,
\end{equation}
where $c>0$. This equation can be integrated to give
\begin{equation}
\int_0^v \frac{dv'}{1-(v'/v_t)^2} = g\,t,
\end{equation}
where $v_t=(m\,g/c)^{1/2}$. Making a change of variable, we obtain
\begin{equation}
\int_0^{v/v_t}\frac{dy}{1-y^2} = \frac{g}{v_t}\,t.
\end{equation}
The left-hand side of the above equation is now a standard integral, which can be solved to give
\begin{equation}
\tanh^{-1}\left(\frac{v}{v_t}\right) = \frac{g\,t}{v_t},
\end{equation}
or
\begin{equation}
 v = v_t\,\tanh\left(\frac{g\,t}{v_t}\right).
 \end{equation}
 Thus, when $t\ll v_t/g$, we obtain the standard result $v\simeq g\,t$,
 since $\tanh x\simeq x$ for $x\ll 1$. However, when $t\gg v_t/g$, we get
 $v\simeq v_t$, since $\tanh x\simeq 1$ for $x\gg 1$. It follows that
 air resistance prevents the downward velocity of our object from
 increasing indefinitely as it falls. Instead, at large times, the velocity asymptotically approaches
 the so-called {\em terminal velocity}, $v_t$ (at which the gravitational
 and air resistance forces balance). 
 
 The equation of motion of our falling object is also  written
 \begin{equation}
 m\,v\,\frac{dv}{dx} = m\,g - c\,v^2.
 \end{equation}
 This equation can be integrated to give
 \begin{equation}
 \int_0^v\frac{v'\,dv'}{1-(v'/v_t)^2} = g\,x.
 \end{equation}
 Making a change of variable, we obtain
 \begin{equation}
 \int_0^{(v/v_t)^2} \frac{dy}{1-y} = \frac{x}{x_t},
 \end{equation}
 where $x_t = m/ (2\,c)$. The left-hand side of the above equation is now a standard integral, which can
 be solved to give
 \begin{equation}
 -\ln\!\left[1-\left(\frac{v}{v_t}\right)^2\right] = \frac{x}{x_t},
\end{equation}
or
\begin{equation}
v = v_t\left(1-{\rm e}^{-x/x_t}\right)^{1/2}.
\end{equation}
It follows  that our object needs to fall a distance
of order $x_t$ before it achieves its terminal velocity.

Incidentally, it is quite easy to account for an air resistance force
which scales as the square of projectile velocity. Let us imaging that
our projectile is moving sufficiently rapidly that air does not have enough
time to flow around it, and is instead simply knocked out of the way.
If our projectile has cross-sectional area $A$,  perpendicular to the direction of its motion,
and is moving with speed $v$, then the mass of air that it knocks out of
its way per second is $\rho_a\,A\,v$, where $\rho_a$ is the mass density
of air. Suppose that the air knocked out of the way is pushed in the direction
of the projectile's motion with a speed of order $v$. It follows that the
air gains momentum per unit time $\rho_a\,A\,v^2$ in the direction
of the projectile's motion. Hence, by Newton's third law, the projectile
loses the same momentum per unit time in the direction of its motion.
In other words, the projectile is subject to a drag force of magnitude
\begin{equation}
f_{drag} = C\,\rho_a\,A\,v^2
\end{equation}
acting in the opposite direction to its motion.
Here, $C$ is an ${\cal O}(1)$ dimensionless constant, known as the
{\em drag coefficient}, which depends on the exact shape of the
projectile. Obviously, streamlined  projectiles, such as arrows, have small drag coefficients,
whereas non-streamlined projectiles, such as bricks, have large drag
coefficients. From before, the terminal velocity of our projectile is $v_t=
(m\,g/c)^{1/2}$, where $m$ is its mass, and $c=C\,\rho_a\,A$.
Writing $m = A\,d\,\rho$, where $d$ is the typical linear dimension
of the projectile, and $\rho$ its mass density, we obtain
\begin{equation}
v_t = \left(\frac{\rho\,g\,d}{\rho_a\,C}\right)^{1/2}.
\end{equation}
The above expression tells us that large, dense, streamlined projectiles ({\em e.g.}, medicine balls)
tend to have large terminal velocities, and small, rarefied, non-streamlined
projectiles ({\em e.g.}, feathers)  tend to have small terminal velocities. Hence, the former
type of projectile is relatively less affected by air resistance than the
latter.

\section{Simple Harmonic Motion}
Consider the motion of a point particle of mass $m$ which is
slightly displaced from a stable equilibrium point at $x=0$. 
Suppose that the particle is moving in the conservative force-field $f(x)$. According to the analysis of Section~\ref{gpotn}, in order for $x=0$ to be a stable equilibrium point
we require both
\begin{equation}\label{e4.8}
f(0) = 0,
\end{equation}
and
\begin{equation}\label{e4.9}
\frac{d f(0)}{dx} < 0.
\end{equation}

Now, our particle obeys Newton's second law of motion,
\begin{equation}
m\,\frac{d^2 x}{d t^2} = f(x).
\end{equation}
Let us assume that it always stays fairly close to its equilibrium
point. In this case, to a good approximation, we can represent $f(x)$
via a truncated Taylor expansion about this point. In other words,
\begin{equation}
f(x) \simeq f(0) + \frac{df(0)}{dx}\,x + {\cal O}(x^2).
\end{equation}
However, according to (\ref{e4.8}) and (\ref{e4.9}), the above
expression can be written
\begin{equation}\label{e4.12a}
f(x) \simeq - m\,\omega_0^{\,2}\,x,
\end{equation}
where $df(0)/dx = -m\,\omega_0^{\,2}$. 
Hence, we conclude that our particle satisfies the following approximate
equation of motion,
\begin{equation}\label{e4.13}
\frac{d^2 x}{dt^2}+ \omega_0^{\,2}\,x\simeq 0,
\end{equation}
provided that it does not stray too far from its equilibrium point: {\em i.e.},
provided $|x|$ does not become too large.

Equation~(\ref{e4.13}) is called the {\em simple harmonic equation}, and
governs the motion of {\em all}\/ one-dimensional conservative systems which are slightly
perturbed from some stable equilibrium state. The solution of Equation~(\ref{e4.13})
is well-known:
\begin{equation}\label{e4.12}
x(t) = a\,\sin(\omega_0\,t -\phi_0).
\end{equation}
The pattern of motion described by above expression, 
which is called {\em simple harmonic motion},
is {\em periodic}\/ in time, with repetition period
$T_0 = 2\pi/\omega_0$, and {\em oscillates}\/ between $x=\pm a$. Here, $a$
is called the {\em amplitude}\/ of the motion. The parameter $\phi_0$,
known as the {\em phase angle}, 
simply shifts the pattern of motion backward and forward in time. 
Figure~\ref{fshm} shows some examples of simple harmonic motion.
Here, $\phi_0 = 0$, $+\pi/4$, and $-\pi/4$ correspond to the
solid, short-dashed, and long-dashed curves, respectively.

Note that the frequency, $\omega_0$---and, hence, the period, $T_0$---of
simple harmonic motion is determined by the parameters appearing in the simple harmonic equation,
(\ref{e4.13}). However, the amplitude, $a$, and the phase angle, $\phi_0$,
are the two integration constants  of this second-order ordinary differential
equation, and are, thus, determined by the initial conditions: {\em i.e.}, by the particle's initial displacement and velocity.

\begin{figure}
\epsfysize=2.5in
\centerline{\epsffile{Chapter03/fig3.03.eps}}
\caption{\em Simple harmonic motion.}\label{fshm}   
\end{figure}

Now, from Equations~(\ref{egrad}) and (\ref{e4.12a}), the potential
energy of our particle at position $x$ is approximately
\begin{equation}
U(x) \simeq \frac{1}{2}\,m\,\omega_0^{\,2}\,x^2.
\end{equation}
Hence, the total energy is written
\begin{equation}
E = K + U = \frac{1}{2}\,m\left(\frac{dx}{dt}\right)^2+  \frac{1}{2}\,m\,\omega_0^{\,2}\,x^2,
\end{equation}
giving
\begin{equation}\label{e4.37x}
E = \frac{1}{2}\,m\,\omega_0^{\,2}\,a^2\,\cos^2(\omega_0\,t-\phi_0)
+ \frac{1}{2}\,m\,\omega_0^{\,2}\,a^2\,\sin^2(\omega_0\,t-\phi_0)
= \frac{1}{2}\,m\,\omega_0^{\,2}\,a^2,
\end{equation}
where use has been made of Equation~(\ref{e4.12}), and the trigonometric
identity $\cos^2\theta+\sin^2\theta \equiv 1$. Note that the
total energy is {\em constant}\/ in time, as is to be expected for a
conservative system, and is proportional to the {\em amplitude squared}\/
of the motion.

\section{Damped Oscillatory Motion}\label{s4.4}
According to Equation~(\ref{e4.12}), a one-dimensional conservative system which is
slightly perturbed from a stable equilibrium point (and then left alone) oscillates about this
point with a fixed frequency and a constant amplitude. In other words,
the oscillations never die away. This is not very realistic, since we
know that, in practice, if we slightly perturb a  dynamical system (such as a
pendulum) from a stable equilibrium point then it will indeed oscillate about this point,
but  these oscillations will  eventually die away due to frictional effects,
which are present in virtually all real dynamical systems. In order to model
this process, we need to include some sort of frictional drag force in
our perturbed equation of motion, (\ref{e4.13}).

The most common model for a frictional drag force is  one which is
always directed in the {\em opposite direction}\/ to the instantaneous velocity
of the object upon which it acts, and is {\em directly proportional}\/ to the magnitude
of this velocity. Let us adopt this model.
So, our drag force can be written
\begin{equation}
f_{drag} = - 2\,m\,\nu\,\frac{dx}{dt},
\end{equation}
where $\nu$ is a positive constant with the dimensions of frequency. Including such a force in our
perturbed equation of motion, (\ref{e4.13}),  we obtain
\begin{equation}\label{e4.19}
\frac{d^2 x}{dt^2} + 2\,\nu\,\frac{dx}{dt} + \omega_0^{\,2}\,x
= 0.
\end{equation}
Thus, the positive constant $\nu$ parameterizes the strength of the frictional
damping  in our dynamical system.

Equation (\ref{e4.19}) is a linear second-order ordinary differential
equation, which we suspect possesses oscillatory solutions. There is
a standard trick for solving such an equation. We search for complex
 solutions of the form
\begin{equation}\label{e4.20}
x = a\,{\rm e}^{-{\rm i}\,\omega\,t},
\end{equation}
where the constants $\omega$ and $a$ are both, in general, complex. Of course,
the physical solution is the {\em real part}\/ of the above expression: {\em i.e.}, 
\begin{equation}
x = |a|\,\cos[{\rm arg}(a)-{\rm Re}(\omega)\,t]\,{\rm e}^{\,{\rm Im}(\omega)\,t}.
\end{equation}
Clearly, the modulus and argument of the complex amplitude, $a$, determine
the amplitude (at $t=0$) and phase of the oscillation, respectively, whereas the
real and imaginary parts of the complex frequency, $\omega$, determine its
 frequency and growth-rate, respectively.
Note that this method of solution is only appropriate for {\em linear}\/ differential
equations. Incidentally, the method works because
\begin{equation}
{\rm Re}[{\cal L}(x)]\equiv {\cal L}({\rm Re}[x]),
\end{equation}
where $x$ is a complex variable, and ${\cal L}$ some real linear differential
operator which acts on this variable. [A linear operator satisfies  ${\cal L}(a\,x)=a\,{\cal L}(x)$ for all $a$ and $x$, where $a$ is a constant. The differential operator appearing
in Equation~(\ref{e4.19}) is clearly of this type.]

Substituting Equation~(\ref{e4.20}) into Equation~(\ref{e4.19}), we obtain
\begin{equation}
a \left[-\omega^2-{\rm i}\,2\,\nu\,\omega + \omega_0^{\,2}\right]
{\rm e}^{-{\rm i}\,\omega\,t} = 0,
\end{equation}
which reduces to the following quadratic equation for $\omega$ [since $a\,\exp(-{\rm i}\,\omega\,t)\neq 0$]:
\begin{equation}
\omega^2 + {\rm i}\,2\,\nu\,\omega - \omega_0^{\,2} = 0.
\end{equation}
The solution to this equation is 
\begin{equation}
\omega_\pm = - {\rm i}\,\nu \pm \sqrt{\omega_0^{\,2}-\nu^2}.
\end{equation}
Thus, the most general physical solution to Equation~(\ref{e4.19}) is
\begin{equation}
x(t) = {\rm Re}\left[a_+\,{\rm e}^{-{\rm i}\,\omega_+\,t}
+ a_-\,{\rm e}^{-{\rm i}\,\omega_-\,t}\right],
\end{equation}
where $a_\pm$ are two arbitrary complex constants.

We can distinguish three different cases. In the first case, $\nu < \omega_0$, and the motion is said to be {\em underdamped}. The most general
solution is written
\begin{equation}\label{e4.47x}
x(t) = x_0\,{\rm e}^{-\nu\,t}\,\cos(\omega_r\,t) + \left(\frac{v_0+\nu\,x_0}{\omega_r}\right){\rm e}^{-\nu\,t}\,\sin(\omega_r\,t),
\end{equation}
where $\omega_r = \sqrt{\omega_0^{\,2}-\nu^2}$, $x_0 = x(0)$, and
$v_0 = dx(0)/dt$. It can be seen that the solution oscillates at some real
frequency, $\omega_r$, which is somewhat less than the natural frequency
of oscillation of the undamped system, $\omega_0$, but also {\em decays}
exponentially in time at a rate proportional to the damping coefficient, $\nu$. 

In the second case, $\nu=\omega_0$, and the motion is said to be {\em
critically damped}. The most general solution is written
\begin{equation}
x(t) = \left[x_0 \,(1+\omega_0\,t) + v_0\,t\right] {\rm e}^{-\omega_0\,t}.
\end{equation}
It can be seen that the solution now decays without oscillating. 

In the third case, $\nu>\omega_0$, and the motion is said to be {\em overdamped}. The most general solution is written
\begin{equation}
x(t) = -\left(\frac{v_0 + \nu_-\,x_0}{\nu_+-\nu_-}\right){\rm e}^{-\nu_+\,t}
+ \left(\frac{v_0+\nu_+\,x_0}{\nu_+-\nu_-}\right){\rm e}^{-\nu_-\,t},
\end{equation}
where $\nu_\pm = \nu\pm \sqrt{\nu^2-\omega_0^{\,2}}$. It can be seen
that the solution again decays without oscillating, except there are now
{\em two}\/ independent decay rates. The largest, $\nu_+$, is always greater than the
critically damped decay rate, $\omega_0$, whereas the smaller, $\nu_-$,
is always less than this decay rate. This means that, in general,
the critically damped solution is more rapidly damped than either 
the underdamped or  overdamped solutions.

\begin{figure}
\epsfysize=2.5in
\centerline{\epsffile{Chapter03/fig3.04.eps}}
\caption{\em Damped oscillatory motion.}\label{fosc}   
\end{figure}

Figure~\ref{fosc} shows typical examples of underdamped ({\em i.e.}, $\nu=\omega_0/4$), critically damped ({\em i.e.}, $\nu=\omega_0$), and
overdamped ({\em i.e.}, $\nu=4\,\omega_0$) solutions, calculated with the
initial conditions $x_0=1$ and $v_0=0$. Here, $T_0= 2\pi/\omega_0$. 
The three solutions correspond to the solid, short-dashed, and long-dashed curves,
respectively.

\section{Quality Factor}\label{squal}
The total energy of a damped oscillator is the sum of its kinetic and
potential energies: {\em i.e.}, 
\begin{equation}
E = \frac{1}{2}\,m\left(\frac{dx}{dt}\right)^2 + \frac{1}{2}\,m\,\omega_0^{\,2}\,x^2.
\end{equation}
Differentiating the above expression with
respect to time, we obtain
\begin{equation}
\frac{dE}{dt} = m\,\frac{dx}{dt}\,\frac{d^2 x}{dt^2} + m\,\omega_0^{\,2}\,x\,\frac{dx}{dt} = m\,\frac{dx}{dt}\left(\frac{d^2 x}{dt^2} + \omega_0^{\,2}\,x\right).
\end{equation}
It follows from Equation~(\ref{e4.19}) that
\begin{equation}\label{e4.52x}
\frac{dE}{dt} = -2\,m\,\nu\left(\frac{dx}{dt}\right)^2.
\end{equation}
We conclude that the presence of damping causes the oscillator energy to
decrease monotonically in time, and, hence, causes the   amplitude of the oscillation to eventually become negligibly small [see Equation~(\ref{e4.37x})].

The energy loss rate of a weakly damped ({\em i.e.}, $\nu\ll\omega_0$) oscillator is conveniently 
characterized in terms of a parameter, $Q$, which is known as the 
{\em quality factor}. This parameter is defined to be $2\pi$ times the
energy stored in the oscillator, divided by the energy lost in a single
 oscillation period. If the oscillator is weakly damped then
the energy lost per period is relatively small, and $Q$ is therefore
much larger than unity.  Roughly speaking, $Q$ is the number of oscillations
 that the oscillator typically completes, after being set in motion, before its
 amplitude decays to a negligible value.
 Let us find an expression for $Q$.

Now, the most general solution for a weakly damped 
oscillator can be written in the form
[{\em cf.}, Equation~(\ref{e4.47x})]
\begin{equation}
x = x_0\,{\rm e}^{-\nu\,t}\,\cos(\omega_r\,t-\phi_0),
\end{equation}
where $x_0$ and $\phi_0$ are constants, and $\omega_r=\sqrt{\omega_0^{\,2}-\nu^2}$. It follows that
\begin{equation}
\frac{dx}{dt} =- x_0\,\nu\,{\rm e}^{-\nu\,t}\,\cos(\omega_r\,t-\phi_0)-
x_0\,\omega_r\,{\rm e}^{-\nu\,t}\,\sin(\omega_r\,t-\phi_0).
\end{equation}
Thus, making use of Equation~(\ref{e4.52x}), the energy lost during a single oscillation period is
\begin{eqnarray}
\Delta E &=& -\int_0^{T_r} \frac{dE}{dt}\,dt\\[0.5ex]
&=& 2\,m\,\nu\,x_0^{\,2}\int_0^{T_r}{\rm e}^{-2\,\nu\,t}\left[\nu\,\cos(\omega_r\,t-\phi_0) + \omega_r\,\sin(\omega_r\,t-\phi_0)\right]^2 dt,\nonumber
\end{eqnarray}
where $T_r=2\pi/\omega_r$. 
In the weakly damped limit, $\nu\ll \omega_r$, the exponential factor is approximately
unity in the interval $t=0$ to $t=T_r$, so that
\begin{equation}
\Delta E \simeq \frac{2\,m\,\nu\,x_0^{\,2}}{\omega_r}\int_0^{2\pi} \left(
\nu^2\,\cos^2\theta + 2\,\nu\,\omega_r\,\cos\theta\,\sin\theta + \omega_r^{\,2}\,\sin^2\theta\right)d\theta.
\end{equation}
Thus,
\begin{equation}
\Delta E \simeq \frac{2\pi\,m\,\nu\,x_0^{\,2}}{\omega_r}\,(\nu^2+\omega_r^{\,2}) =2 \pi\,m\,\omega_0^{\,2}\,x_0^{\,2}\left(\frac{\nu}{\omega_r}\right),
\end{equation}
since  $\cos^2\theta$ and $\sin^2\theta$ both have the average values $1/2$ in the interval
$0$ to $2\pi$, whereas  $\cos\theta\,\sin\theta$ has the average value $0$. 
According to Equation~(\ref{e4.37x}), the energy stored in the oscillator
(at $t=0$) is
\begin{equation}
E = \frac{1}{2}\,m\,\omega_0^{\,2}\,x_0^{\,2}.
\end{equation}
It follows that
\begin{equation}
Q = 2\pi\,\frac{E}{\Delta E} = \frac{\omega_r}{2\,\nu}\simeq \frac{\omega_0}{2\,\nu}.
\end{equation}

\section{Resonance}\label{eres}
We have seen that when a one-dimensional dynamical system is slightly perturbed
from a stable equilibrium point (and then left alone), it eventually returns to this point at
a rate controlled by the amount of damping in the system. Let us now
suppose that the same system is subject to {\em continuous}\/ oscillatory constant amplitude
external forcing at some fixed frequency, $\omega$. In this
case, we would expect the system to eventually settle down to some steady oscillatory pattern of motion
with the same frequency as the external forcing. Let us investigate the properties of this type of driven oscillation.

Suppose that our system is subject to an external force of the form
\begin{equation}\label{e4.29}
f_{ext}(t) = m\,\omega_0^{\,2}\,X_1\,\cos(\omega\,t).
\end{equation}
Here, $X_1$ is the  amplitude of the oscillation at which the external force matches the restoring
force, (\ref{e4.12a}). Incorporating the above force into our perturbed
equation of motion, (\ref{e4.19}), we obtain
\begin{equation}
\frac{d^2 x}{dt^2} + 2\,\nu\,\frac{dx}{dt} + \omega_0^{\,2}\,x
= \omega_0^{\,2}\,X_1\,\cos(\omega\,t).
\end{equation}
Let us  search for a solution of the form (\ref{e4.20}), and
represent the right-hand side of the above equation as
$\omega_0^{\,2}\,X_1\,\exp(-{\rm i}\,\omega\,t)$. It is again understood
that the physical solutions are the {\em real parts}\/ of these expressions. 
Note that $\omega$ is now a real parameter. We obtain
\begin{equation}
a\left[-\omega^2-{\rm i}\,2\,\nu\,\omega + \omega_0^{\,2}\right] {\rm e}^{-{\rm i}\,\omega\,t}
= \omega_0^{\,2}\,X_1\,{\rm e}^{-{\rm i}\,\omega\,t}.
\end{equation}
Hence,
\begin{equation}\label{e4.32}
a = \frac{\omega_0^{\,2}\,X_1}{\omega_0^{\,2}-\omega^2 - {\rm i}\,2\,\nu\,\omega}.
\end{equation}
In general, $a$ is a complex quantity. Thus, we can write
\begin{equation}\label{e4.33}
a = x_1\,{\rm e}^{\,{\rm i}\,\phi_1},
\end{equation}
where $x_1$ and $\phi_1$ are both real. It follows from 
Equations~(\ref{e4.20}),  (\ref{e4.32}), and (\ref{e4.33}) that the physical solution takes the
form
\begin{equation}
x(t) = x_1\,\cos(\omega\,t-\phi_1),
\end{equation}
where 
\begin{equation}\label{e4.35}
x_1 = \frac{\omega_0^{\,2}\,X_1}{\left[(\omega_0^{\,2}-\omega^2)^2
+ 4\,\nu^2\,\omega^2\right]^{1/2}},
\end{equation}
and
\begin{equation}\label{e4.36}
\phi_1 = \tan^{-1}\left(\frac{2\,\nu\,\omega}{\omega_0^{\,2}-\omega^2}\right).
\end{equation}
We conclude that, in response to the applied sinusoidal force, (\ref{e4.29}), the system
executes a sinusoidal pattern of motion at the {\em same}\/ frequency, with fixed amplitude $x_1$, and phase-lag $\phi_1$ (with respect to
the external force).

\begin{figure}
\epsfysize=2.25in
\centerline{\epsffile{Chapter03/fig3.05.eps}}
\caption{\em Resonance.}\label{fres}   
\end{figure}
Let us investigate the variation of $x_1$ and $\phi_1$ with the forcing frequency,
$\omega$.  This is most easily done graphically. Figure~\ref{fres} shows $x_1$ and $\phi_1$ as functions of $\omega$ for
various values of $\nu/\omega_0$. Here, $\nu/\omega_0 = 1$, $1/2$, $1/4$, $1/8$, and $1/16$ correspond to the solid, dotted, short-dashed, long-dashed,
and dot-dashed curves, respectively. It can be seen that as the amount of
damping in the system is decreased, the amplitude of the response becomes
progressively more peaked at the natural frequency of oscillation of the system, $\omega_0$. This effect is known as {\em resonance}, and
$\omega_0$ is termed the {\em resonant frequency}. Thus,
a weakly damped system ({\em i.e.}, $\nu\ll \omega_0$) can be driven to large amplitude by the application of a relatively
small external force which oscillates at a frequency close to the resonant frequency. Note that the response of the system is in phase ({\em i.e.}, $\phi_1\simeq 0$)
with the external driving force for driving frequencies well below the resonant
frequency, is in phase quadrature
({\em i.e.}, $\phi_1=\pi/2$)
at the resonant frequency, and is in anti-phase ({\em i.e.}, $\phi_1\simeq \pi$)
for frequencies well above the resonant frequency.

According to Equation~(\ref{e4.35}),
\begin{equation}
\frac{x_1(\omega=\omega_0)}{x_1(\omega = 0)} = \frac{\omega_0}{2\,\nu}
=Q.
\end{equation}
In other words, the ratio of the driven amplitude at the resonant frequency
to that at a typical non-resonant frequency (for the same drive amplitude)
is of order the quality factor. Equation~(\ref{e4.35}) also implies that,
for a weakly damped oscillator ($\nu\ll \omega_0$),
\begin{equation}
\frac{x_1(\omega)}{x_1(\omega=\omega_0)}\simeq \frac{\nu}{[(\omega-\omega_0)^2 + \nu^2]^{1/2}},
\end{equation}
provided $|\omega-\omega_0|\ll \omega_0$. 
Hence, the width of the resonance
peak (in frequency) is $\Delta\omega = 2\,\nu$, where the edges of peak are defined as the points at which the driven amplitude
is reduced to $1/\sqrt{2}$ of its maximum value. It follows that the
fractional width is
\begin{equation}
\frac{\Delta \omega}{\omega_0} = \frac{2\,\nu}{\omega_0} = \frac{1}{Q}.
\end{equation}
We conclude that the
height and width of the resonance peak of a weakly damped ($Q\gg 1$) oscillator scale as $Q$ and
$Q^{-1}$, respectively. Thus, the area under the resonance curve stays
approximately constant as $Q$ varies. 
 
\section{Periodic Driving Forces}
In the last section, we investigated the response of a one-dimensional dynamical system, close to a stable equilibrium point, 
to an external force which varies as $\cos(\omega\,t)$.  Let us
now examine the response of the same system to a more complicated
external force.

Consider a general external force which is {\em periodic} in time, with
period $T$. By analogy with Equation~(\ref{e4.29}), we can write such a 
force as
\begin{equation}\label{e4.37}
f_{ext}(t) = m\,\omega_0^{\,2}\,X(t),
\end{equation}
where
\begin{equation}\label{eper}
X(t+T) = X(t)
\end{equation}
for all $t$. 

It is convenient to represent $X(t)$ as a {\em Fourier series}\/ in time, so that
\begin{equation}\label{e4.39}
X(t) = \sum_{n=0}^{\infty} X_{n}\,\cos(n\,\omega\,t),
\end{equation}
where $\omega = 2\pi/T$. By writing $X(t)$ in this form, we {\em automatically}
satisfy the periodicity constraint (\ref{eper}). [Note that by choosing a cosine
Fourier series we are limited to even functions in $t$: {\em i.e.}, $X(-t)=X(t)$.
Odd functions in $t$ can be represented by sine Fourier series, and
mixed functions require a combination of cosine and sine Fourier series.]
 The constant coefficients
$X_n$ are known as  {\em Fourier coefficients}. But,
how do we determine these coefficients for a given functional form, $X(t)$?

Well, it follows from the periodicity of the cosine function that
\begin{equation}\label{e4.40}
\frac{1}{T} \int_0^T \cos (n\,\omega\,t)\,dt = \delta_{n\,0},
\end{equation}
where $\delta_{n\,n'}$ is unity if $n=n'$, and zero otherwise, and
is known as the {\em Kronecker delta function}. 
Thus, integrating Equation~(\ref{e4.39}) 
in $t$ from $t=0$ to $t=T$, and making use of Equation~(\ref{e4.40}),
we obtain
\begin{equation}\label{e4.41}
X_0 = \frac{1}{T}\int_0^T X(t) \,dt.
\end{equation}

It is also easily demonstrated that
\begin{equation}\label{e4.42}
\frac{2}{T} \int_0^T \cos(n\,\omega\,t)\,\cos(n'\,\omega\,t) \,dt= \delta_{n\,n'},
\end{equation}
provided $n, n'>0$. Thus, multiplying Equation~(\ref{e4.39}) by
$\cos(n\,\omega\,t)$, integrating in $t$ from $t=0$ to $t=T$, and
making use of Equations~(\ref{e4.40}) and (\ref{e4.42}), we obtain
\begin{equation}\label{e4.43}
X_n = \frac{2}{T} \int_0^T X(t)\,\cos(n\,\omega\,t)\,dt
\end{equation}
for $n>0$.  Hence, we have now determined the Fourier coefficients of
the general periodic function $X(t)$. 

We can incorporate the periodic external force (\ref{e4.37}) into our
perturbed equation of motion by writing
\begin{equation}
\frac{d^2 x}{dt^2} + 2\,\nu\,\frac{dx}{dt} + \omega_0^{\,2}\,x
= \omega_0^{\,2}\,\sum_{n=0}^\infty X_n\,{\rm e}^{-{\rm i}\,n\,\omega\,t},
\end{equation}
where we are again using the convention that the physical solution
corresponds to the {\em real part}\/ of the complex solution. Note that
the above differential equation is {\em linear}. This means that if $x_a(t)$
and $x_b(t)$ represent two independent solutions to this equation then
any linear combination of $x_a(t)$ and $x_b(t)$ is also a solution. We can
exploit the linearity of the above equation to write the
solution in the form
\begin{equation}
x(t) = \sum_{n=0}^\infty X_n\,a_n\,{\rm e}^{-{\rm i}\,n\,\omega\,t},
\end{equation}
where the $a_n$ are the complex amplitudes of the solutions to
\begin{equation}
\frac{d^2 x}{dt^2} + 2\,\nu\,\frac{dx}{dt} + \omega_0^{\,2}\,x
= \omega_0^{\,2}\,{\rm e}^{-{\rm i}\,n\,\omega\,t}.
\end{equation}
In other words, $a_n$ is obtained by substituting $x = a_n\,\exp(-{\rm i}\,n\,\omega\,t)$ into the above equation.
Hence, it follows that
\begin{equation}\label{e4.47}
a_n = \frac{\omega_0^{\,2}}{\omega_0^{\,2} - n^2\,\omega^2 - 
{\rm i}\,2\,\nu\,n\,\omega}.
\end{equation}
Thus, the physical solution takes the form
\begin{equation}\label{e4.48}
x(t) = \sum_{n=0}^{\infty} X_n\,x_n\,\cos(n\,\omega\,t-\phi_n),
\end{equation}
where
\begin{equation}
a_n = x_n \,{\rm e}^{\,{\rm i}\,\phi_n},
\end{equation}
and $x_n$ and $\phi_n$ are real parameters. It follows from
Equation~(\ref{e4.47}) that
\begin{equation}
x_n = \frac{\omega_0^{\,2}}{\left[(\omega_0^{\,2}-n^2\,\omega^2)^2
+ 4\,\nu^2\,n^2\,\omega^2\right]^{1/2}},
\end{equation}
and
\begin{equation}
\phi_n = \tan^{-1}\left(\frac{2\,\nu\,n\,\omega}{\omega_0^{\,2}-n^2\,\omega^2}\right).
\end{equation}
We have now fully determined the response of our dynamical system to
a general periodic driving force.

As an example, suppose that the external force periodically delivers a brief kick to
the system. For instance, let
$X(t) = A$ for $0\leq t\leq T/10$ and $9\,T/10<t<T$, and  $X(t)=0$ otherwise (in the period $0\leq t\leq T$). 
It follows from Equation~(\ref{e4.41}) and (\ref{e4.43}) that, in this case,
\begin{equation}
X_0 = 0.2\,A,
\end{equation}
and
\begin{equation}
X_n = \frac{2\,\sin(n\,\pi/5)\,A}{n\,\pi},
\end{equation}
for $n>0$. Obviously, to obtain an exact solution, we would have to include
every Fourier harmonic in Equation~(\ref{e4.48}), which would be impractical. However, we can obtain
a fairly accurate approximate solution by {\em truncating}\/ the Fourier series ({\em i.e.},
by neglecting all the terms with $n>N$, where $N\gg 1$).

\begin{figure}
\epsfysize=2.25in
\centerline{\epsffile{Chapter03/fig3.06.eps}}
\caption{\em Periodic forcing.}\label{ffou}   
\end{figure}

Figure~\ref{ffou} shows an example calculation in which the Fourier
series is truncated after 100 terms. The parameters used
in this calculation are $\omega = 1.2\,\omega_0$ and $\nu= 0.8\,\omega_0$.
The left panel shows the Fourier reconstruction of the driving force, $X(t)$. The
glitches at the rising and falling edges of the pulses are called {\em Gibbs
phenomena}, and are an inevitable consequence of attempting to
represent a discontinuous periodic function as a Fourier series. The
right panel shows the Fourier reconstruction of the response, $x(t)$,  of the dynamical
system to the applied force.

\section{Transients}\label{strans}
We saw, in Section~\ref{eres}, that when a one-dimensional dynamical system,
close to a stable equilibrium point, is subject to a sinusoidal external
force of the form (\ref{e4.29}) then the equation of motion of the
system is written
\begin{equation}\label{e4.54}
\frac{d^2 x}{dt^2} + 2\,\nu\,\frac{dx}{dt} + \omega_0^{\,2}\,x
= \omega_0^{\,2}\,X_1\,\cos(\omega\,t).
\end{equation}
We also found that the solution to this equation which oscillates in sympathy
with the applied force takes the form
\begin{equation}\label{e4.55}
x(t) = x_1\,\cos(\omega\,t-\phi_1),
\end{equation}
where $x_1$ and $\phi_1$ are specified in Equations~(\ref{e4.35}) and (\ref{e4.36}), respectively.
However, (\ref{e4.55}) is not the most general solution to Equation~(\ref{e4.54}).
It should be clear that we can take the above solution and add to it
any solution of Equation~(\ref{e4.54}) calculated with the right-hand side set to zero,
and the result will also be a solution of Equation~(\ref{e4.54}). 
Now, we investigated the solutions to (\ref{e4.54}) with the right-hand
set to zero in Section~\ref{s4.4}. In the underdamped regime ($\nu<\omega_0$),
we found that the most general such solution takes the
form
\begin{equation}
x(t) = A\,{\rm e}^{-\nu\,t}\,\cos(\omega_r\,t) + B\,{\rm e}^{-\nu\,t}\,
\sin(\omega_r\,t),
\end{equation}
where $A$ and $B$ are two arbitrary constants [they are in fact the integration constants
 of the second-order ordinary differential equation (\ref{e4.54})],
and $\omega_r = \sqrt{\omega_0^{\,2}-\nu^2}$. Thus,
the most general solution to Equation~(\ref{e4.54}) is written
\begin{equation}\label{e4.56}
x(t) = A\,{\rm e}^{-\nu\,t}\,\cos(\omega_r\,t) + B\,{\rm e}^{-\nu\,t}\,
\sin(\omega_r\,t) + x_1\,\cos(\omega\,t-\phi_1).
\end{equation}
The first two terms on the right-hand side of the above equation
are called {\em transients}, since they decay in time. The transients
are determined by the {\em initial conditions}. However, if we wait
long enough after setting the system into motion then the transients will 
always decay away, leaving the time-asymptotic solution (\ref{e4.55}), which is independent of the
initial conditions.

\begin{figure}
\epsfysize=2.5in
\centerline{\epsffile{Chapter03/fig3.07.eps}}
\caption{\em Transients.}\label{ftrans}   
\end{figure}

As an example, suppose that we set the system into motion at time
$t=0$ with the initial conditions $x(0)=dx(0)/dt=0$. 
Setting $x(0)=0$ in Equation~(\ref{e4.56}), we obtain
\begin{equation}
A = - x_1\,\cos\phi_1.
\end{equation}
Moreover, setting $dx(0)/dt=0$ in Equation~(\ref{e4.56}), we get
\begin{equation}
B = -\frac{x_1\,(\nu\,\cos\phi_1+\omega\,\sin\phi_1)}{\omega_r}.
\end{equation}
Thus, we have now determined the constants $A$ and $B$, and, hence,
fully specified the solution for $t>0$. Figure~\ref{ftrans}
shows  this solution (solid curve) calculated for
$\omega = 2\,\omega_0$ and 
$\nu=0.2\,\omega_0$. Here,  $T_0=2\pi/\omega_0$. The associated  time-asymptotic solution (\ref{e4.55})
is also shown for the sake of comparison (dashed curve). It can
be seen that the full solution quickly converges to the time-asymptotic
solution.

\begin{figure}
\epsfysize=2.5in
\centerline{\epsffile{Chapter03/fig3.08.eps}}
\caption{\em A simple pendulum.}\label{f98}  
\end{figure}
\section{Simple Pendulum}\label{s4.8}
Consider a compact mass $m$ suspended from a light inextensible string of length $l$, such that the
mass is free to swing from side to side in a vertical plane, as shown in 
Figure~\ref{f98}.
This setup is known as a {\em simple pendulum}. 
 Let $\theta$ be the angle subtended between the string and
the downward vertical. Obviously, the stable equilibrium state of the simple pendulum corresponds to
the situation in which the mass is stationary, and hangs vertically down ({\em 
i.e.}, $\theta=0$).
From elementary mechanics, the angular equation of motion of the pendulum is 
\begin{equation}
I\,\frac{d^2{\theta}}{dt^2} = \tau,
\end{equation}
where $I$ is the moment of inertia of the mass (see Section~\ref{mom}), and $\tau$ is the torque acting 
about the pivot point (see Section~\ref{svecp}).
For the
case in hand, given that the mass is essentially a point particle, and is situated a distance $l$ from
the axis of rotation ({\em i.e.}, the pivot point), it is easily seen that 
$I=m\,l^2$. 

The two forces acting on the mass are the downward gravitational force, $m\,g$, where $g$ is the acceleration due to gravity, 
 and the tension, $T$, in the string.
Note, however, that the tension makes no contribution to the torque, 
since its line of action clearly passes
through the pivot point. From simple trigonometry, 
the line of action of the gravitational force passes a distance $l\,\sin\theta$ 
from the
pivot point. Hence, the magnitude of the gravitational torque is $m\,g\,l\,
\sin\theta$.
Moreover, the gravitational torque is  a {\em restoring torque}: {\em i.e.}, if 
the mass is
displaced slightly from its equilibrium state ({\em i.e.}, $\theta =0$) then the
 gravitational torque clearly acts
 to push the mass back toward that state. Thus, we can write
\begin{equation}
\tau = - m\,g\,l\,\sin\theta.
\end{equation}
Combining the previous two equations, we obtain the following  angular equation 
of motion of the pendulum:
\begin{equation}\label{epend}
l\,\frac{d^2{\theta}}{dt^2} +g\,\sin\theta=0.
\end{equation}
Note that, unlike all of the other equations of motion which we have
examined in this chapter, the above equation is {\em nonlinear} [since $\sin(a\,\theta)\neq a\,\sin\theta$, except when $\theta\ll 1$].

Let us assume, as usual, that the system does not stray very far from
its equilibrium state ($\theta=0$). If this is the case then we
can make the small angle approximation $\sin\theta\simeq \theta$, and
the above equation of motion simplifies to
\begin{equation}
\frac{d^2\theta}{dt^2} + \omega_0^{\,2}\,\theta\simeq 0,
\end{equation}
where $\omega_0 = \sqrt{g/l}$. Of course, this is just the
simple harmonic equation. Hence, we can immediately write the solution
as
\begin{equation}\label{e4.64}
\theta(t) = \theta_0\,\cos(\omega_0\,t).
\end{equation}
We conclude that the pendulum swings back and forth at a fixed frequency, $\omega_0$, which depends on $l$ and $g$, but is {\em independent}\/ of the amplitude,
$\theta_0$, of the motion.

Suppose, now, that we desire a more accurate solution of Equation~(\ref{epend}).
One way in which we could achieve this would be to include
more terms in the small angle expansion of $\sin\theta$, which is
\begin{equation}
\sin\theta = \theta - \frac{\theta^{\,3}}{3!} + \frac{\theta^{\,5}}{5!} +\cdots.
\end{equation}
For instance, keeping the first two terms in this expansion, Equation~(\ref{epend})
becomes
\begin{equation}\label{e4.66}
\frac{d^2\theta}{dt^2} + \omega_0^{\,2}\,(\theta-\theta^{\,3}/6)\simeq 0.
\end{equation}
By analogy with (\ref{e4.64}), let us try a trial solution of the
form
\begin{equation}
\theta(t) = \vartheta_0\,\cos(\omega\,t).
\end{equation}
Substituting this into Equation~(\ref{e4.66}), 
and making use of the trigonometric identity
\begin{equation}
\cos^3 u \equiv (3/4)\,\cos u+ (1/4)\,\cos(3\,u),
\end{equation}
we obtain
\begin{equation}
\vartheta_0\left[\omega_0^{\,2}-\omega^2 - (1/8)\,\omega_0^{\,2}\,\vartheta_0^{\,2} \right] \cos(\omega\,t)- (1/24)\,\omega_0^{\,2}\,\vartheta_0^{\,3} \,\cos(3\,\omega\,t)\simeq 0.
\end{equation}
It is evident that the above equation cannot be satisfied for all values of $t$,
except in the trivial case $\vartheta_0=0$. However, the
form of this expression does suggest a better trial solution, namely
\begin{equation}\label{e4.70}
\theta(t) = \vartheta_0\,\cos(\omega\,t) + \alpha\,\vartheta_0^{\,3}\,\cos(3\,\omega\,t),
\end{equation}
where $\alpha$ is ${\cal O}(1)$. Substitution of this expression into 
Equation~(\ref{e4.66}) yields
\begin{eqnarray}
\vartheta_0\left[\omega_0^{\,2}-\omega^2 - (1/8)\,\omega_0^{\,2}\,\vartheta_0^{\,2} \right] \cos(\omega\,t) + &&\nonumber\\[0.5ex]\vartheta_0^{\,3}\left[\alpha\,\omega_0^{\,2} - 9\,\alpha\,\omega^2-
(1/24)\,\omega_0^{\,2}\right] \cos(3\,\omega\,t) + {\cal O}(\vartheta_0^{\,5})&\simeq& 0.
\end{eqnarray}
We can only satisfy the above equation at all values of $t$, and for non-zero $\vartheta_0$, by setting the two expressions in square brackets to zero.
This yields
\begin{equation}\label{e4.72}
\omega \simeq \omega_0 \,\sqrt{1-(1/8)\,\vartheta_0^{\,2}},
\end{equation}
and
\begin{equation}
\alpha\simeq -\frac{\omega_0^{\,2}}{192}.
\end{equation}
Now, the amplitude of the motion is given by
\begin{equation}
\theta_0 = \vartheta_0 + \alpha\,\vartheta_0^{\,3} = \vartheta_0-\frac{\omega_0^{\,2}}{192}\,\vartheta_0^{\,3}.
\end{equation}
Hence, Equation~(\ref{e4.72}) simplifies to
\begin{equation}
\omega = \omega_0\left[1-\frac{\theta_0^{\,2}}{16} + {\cal O}(\theta_0^{\,4})\right].
\end{equation}
The above expression is only approximate, but it illustrates an important
point: {\em i.e.}, that the frequency of oscillation of a simple pendulum
is {\em not}, in fact, amplitude independent. Indeed, the frequency
goes down slightly as the amplitude increases.

The above example illustrates how we might go about solving a
nonlinear equation of motion by means of an expansion in a
small parameter (in this case, the amplitude of the motion).

\section{Exercises}
{\small
\renewcommand{\theenumi}{3.\arabic{enumi}}
\begin{enumerate}
\item If a train of mass $M$ is subject to a retarding force $M\,(a+b\,v^2)$, show that if
the engines are shut off when the speed is $v_0 $ then the train will
come to rest in a time
$$
\frac{1}{\sqrt{a\,b}}\,\tan^{-1}\left(\sqrt{\frac{b}{a}}\,v_0\right),
$$
after traveling a distance
$$
\frac{1}{2\,b}\,\ln\left(1+\frac{b\,v_0^{\,2}}{a}\right).
$$

\item A particle is projected vertically upward from the Earth's surface with a 
velocity which would, if gravity were uniform, carry it to a height $h$.
Show that if the variation of gravity with height is allowed for, but the
resistance of air is neglected, then the height reached will be greater by $h^2/(R-h)$, where
$R$ is the Earth's radius. 

\item A particle is projected vertically upward from the Earth's surface with a velocity
just sufficient for it to reach infinity (neglecting air resistance). Prove that the time needed to
reach a height $h$ is
$$
\frac{1}{3}\left(\frac{2\,R}{g}\right)^{1/2}\,\left[\left(1+\frac{h}{R}\right)^{3/2}-1\right].
$$
where $R$ is the Earth's radius, and $g$ its surface gravitational acceleration.

\item A particle of mass $m$ is constrained to move in one dimension such that its instantaneous displacement is $x$. The particle is 
 released at rest from $x=b$, and is
 subject to a force of the form  $f(x) = - k\,x^{-2}$. Show that the time required
for the particle to reach the origin is
$$
\pi\left(\frac{m\,b^3}{8\,k}\right)^{1/2}.
$$
\item A block of mass $m$ slides along a horizontal surface which is lubricated with
heavy oil such that the block suffers a viscous retarding force
of the form
$$
F = - c\,v^n,
$$
where $c>0$ is a constant, and $v$ is the block's instantaneous velocity.
If the initial speed is $v_0$ at time $t=0$, find $v$ and the displacement
$x$ as functions of time $t$. Also find $v$ as a function of $x$. Show
that for $n=1/2$ the block does not travel further than $2\,m\,v_0^{3/2}/(3\,c)$.

 \item A particle is projected vertically upward in a constant gravitational
 field with an initial speed $v_0$. Show that if there is a retarding force
 proportional to the square of the speed then the speed of the
 particle when it returns to the initial position is
 $$
 \frac{v_0\,v_t}{\sqrt{v_0^{\,2} + v_t^{\,2}}},
 $$
 where $v_t$ is the terminal speed.
 
 \item A particle of mass $m$ moves (in one dimension) in a medium under the influence of a
 retarding force of the form $m\,k\,(v^3+a^2\,v)$, where $v$ is the
 particle speed, and $k$ and $a$ are positive constants. Show that
 for any value of the initial speed the particle will
 never move a distance greater than $\pi/(2\,k\,a)$, and will only come to rest as $t\rightarrow \infty$.

\item Two light springs have spring constants $k_1$ and $k_2$, respectively, and are used in a vertical
orientation to support an object of mass $m$. Show that the angular frequency of oscillation
is $[(k_1+k_2)/m]^{1/2}$ if the springs are in parallel, and $[k_1\,k_2/(k_1+k_2)\,m]^{1/2}$
if the springs are in series.

\item A body of uniform cross-sectional area $A$ and mass density $\rho$ floats in a liquid
of density $\rho_0$ (where $\rho<\rho_0$), and at equilibrium displaces a volume $V$. Show
that the period of small oscillations about the equilibrium position is
$$
T = 2\pi\,\sqrt{\frac{V}{g\,A}}.
$$

\item Show that the ratio of two successive maxima in the displacement of a damped
harmonic oscillator is constant.

\item If the amplitude of a damped harmonic oscillator decreases to $1/e$ of its initial
value after $n$ periods show that the ratio of the period of oscillation to the period
of the same oscillator with no damping is
$$
\left(1+\frac{1}{4\pi^2\,n^2}\right)^{1/2}\simeq 1 + \frac{1}{8\pi^2\,n^2}.
$$

\item Consider a damped driven oscillator whose equation of motion is
$$
\frac{d^2 x}{dt^2} + 2\,\nu\,\frac{dx}{dt} + \omega_0^{\,2} \,x = F(t).
$$
Let $x=0$ and $dx/dt = v_0$ at $t=0$.
 \begin{enumerate}
\item Find the solution for $t>0$ when $F = \sin(\omega\,t)$.
\item Find the solution for $t>0$ when $F= \sin^2(\omega\,t)$.
\end{enumerate}

\item Obtain the time asymptotic response of a damped  linear oscillator of natural frequency $\omega_0$ and damping coefficient
$\nu$ to a
square-wave periodic forcing function of amplitude $F_0=m\,\omega_0^{\,2}\,X_0$  and frequency $\omega$. Thus, $F(t) = F_0$ for
$-\pi/2< \omega\,t< \pi/2$, $3\pi/2<\omega\,t<5\pi/2$, {\em etc.}, and
$F(t)=-F_0$ for $\pi/2<\omega\,t<3\pi/2$, $5\pi/2<\omega\,t<7\pi/2$, {\em etc.}
\end{enumerate}
}
