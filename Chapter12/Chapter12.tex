\chapter{Gravitational Potential Theory}\label{spotn}
\section{Introduction}
This chapter employs gravitational potential theory in combination
with Newtonian dynamics
to examine various interesting phenomena in the Solar System.

\section{Gravitational Potential}
Consider two point masses, $m$ and $m'$, located at position vectors
${\bf r}$ and ${\bf r}'$, respectively. According to Section~\ref{snewt},
the acceleration ${\bf g}$ of mass $m$ due to the gravitational force exerted on it by mass $m'$ takes the form
\begin{equation}
{\bf g} = G\,m'\,\frac{({\bf r}'-{\bf r})}{|{\bf r}'-{\bf r}|^{\,3}}.
\end{equation} 
Now, the $x$-component of this acceleration is written
\begin{equation}
g_x = G\,m'\,\frac{(x'-x)}{[(x'-x)^2+(y'-y)^2+(z'-z)^2]^{\,3/2}},
\end{equation}
where ${\bf r}=(x,\, y,\, z)$ and ${\bf r}' = (x',\,y',\,z')$.
However, as is
easily demonstrated,
\begin{eqnarray}
\frac{(x'-x)}{[(x'-x)^2+(y'-y)^2+(z'-z)^2]^{\,3/2}}\equiv
 \frac{\partial}{\partial x}\!\left(\frac{1}{[(x'-x)^2+(y'-y)^2+(z'-z)^2]^{\,1/2}}\right).\nonumber\\[0.5ex]&&
\end{eqnarray}
Hence,
\begin{equation}
g_x = G\,m'\,\frac{\partial}{\partial x}\!\left(\frac{1}{|{\bf r}'-{\bf r}|}\right),
\end{equation}
with analogous expressions for $g_y$ and $g_z$. It follows that \begin{equation}\label{e13.5}
{\bf g} = -\nabla\Phi,
\end{equation}
where
\begin{equation}\label{e13.6}
\Phi = - \frac{G\,m'}{|{\bf r}'-{\bf r}|}
\end{equation}
is termed the {\em gravitational potential}. Of course,
we can only write ${\bf g}$ in the form (\ref{e13.5})  because gravity
is a {\em conservative}\/ force---see Chapter~\ref{sfun}.
Note that gravitational potential, $\Phi$, is
directly related to  gravitational potential energy, $U$. In fact, the
potential energy of mass $m$ is
$U=m\,\Phi$. 

Now, it is well-known that gravity is a {\em superposable}\/ force. In other
words, the gravitational force exerted on some test mass by a collection
of point masses is simply the sum of the forces exerted on the test mass
by each point mass taken in isolation. It follows that
the gravitational potential generated by a collection of point masses
at a certain location in space is the sum of the potentials generated at that
location by each point mass taken in isolation. Hence, using Equation~(\ref{e13.6}), if there are $N$
point masses, $m_i$ (for $i=1, N$), located at position vectors ${\bf r}_i$,
then the gravitational potential generated at position vector ${\bf r}$ is simply
\begin{equation}\label{e13.7}
\Phi({\bf r}) = - G\sum_{i=1,N} \frac{m_i}{|{\bf r}_i-{\bf r}|}.
\end{equation}

Suppose, finally, that, instead of having a collection of point masses, we have
a {\em continuous}\/ mass distribution. In other words, let the mass at position
vector ${\bf r}'$ be $\rho({\bf r}')\,d^3{\bf r}'$, where $\rho({\bf r}')$
is the local mass density, and $d^3{\bf r}'$  a volume element. 
Summing over all space, and taking the limit $d^3{\bf r}'\rightarrow 0$,
Equation~(\ref{e13.7}) yields
\begin{equation}\label{e13.8}
\Phi({\bf r}) = - G\int\frac{\rho({\bf r}')}{|{\bf r}'-{\bf r}|}\,d^3{\bf r}',
\end{equation}
where the integral is taken over all space.
This is the general expression for the gravitational potential, $\Phi({\bf r})$, generated by
a continuous mass distribution, $\rho({\bf r})$.

\section{Axially Symmetric Mass Distributions}\label{saxial}
At this point, it is convenient to adopt standard spherical  coordinates, $(r,\, \theta,\, \phi)$, aligned along the $z$-axis. These coordinates are related to
regular Cartesian coordinates as follows (see Section~\ref{spolar}):
\begin{eqnarray}\label{e13.9}
x &=& r\,\sin\theta\,\cos\phi,\\[0.5ex]
y &=&r\,\sin\theta\,\sin\phi,\\[0.5ex]
z &=& r\,\cos\theta.\label{e13.11}
\end{eqnarray}

Consider an {\em axially symmetric}\/ mass distribution: {\em i.e.}, a $\rho({\bf r})$
which is {\em independent}\/ of the azimuthal angle, $\phi$. We would expect
such a mass distribution to generated an axially symmetric gravitational
potential, $\Phi(r,\theta)$. Hence, without loss of generality, we can
set $\phi=0$ when evaluating $\Phi$ from Equation~(\ref{e13.8}). In fact,
given that $d^3{\bf r}' = r'^{\,2}\,\sin\theta'\,dr'\,d\theta'\,d\phi'$
in spherical  coordinates, this equation yields
\begin{equation}
\Phi(r,\theta) = - G\int_0^\infty\int_0^\pi\int_0^{2\pi}
\frac{r'^{\,2}\,\rho(r',\theta')\,\sin\theta'}{|{\bf r}-{\bf r}'|}\,dr'\,d\theta'\,d\phi',
\end{equation}
with the right-hand side evaluated at $\phi=0$. However, since
$\rho(r',\theta')$ is independent of $\phi'$, the above equation
can also be written
\begin{equation}\label{e13.13}
\Phi(r,\theta) = - 2\pi\,G\int_0^\infty\int_0^\pi
r'^{\,2}\,\rho(r',\theta')\,\sin\theta'\,\langle|{\bf r}-{\bf r}'|^{-1}\rangle\,dr'\,d\theta',
\end{equation}
where $\langle\cdots\rangle$ denotes an average over the azimuthal angle,
$\phi'$. 

Now, 
\begin{equation}
|{\bf r}'-{\bf r}|^{-1} = (r^{2}-2\,{\bf r}\cdot{\bf r}' + r'^{\,2})^{-1/2},
\end{equation}
and 
\begin{equation}
{\bf r}\cdot{\bf r}' = r\,r'\,F,
\end{equation}
where (at $\phi=0$)
\begin{equation}
F = \sin\theta\,\sin\theta'\,\cos\phi' + \cos\theta\,\cos\theta'.
\end{equation}
Hence,
\begin{equation}
|{\bf r}'-{\bf r}|^{-1} = (r^{2}-2\,r\,r'\,F + r'^{\,2})^{-1/2}.
\end{equation}

Suppose that $r > r'$. In this case, we can expand $|{\bf r}'-{\bf r}|^{-1}$
as a convergent power series in $r'/r$, to give
\begin{equation}
|{\bf r}'-{\bf r}|^{-1}= \frac{1}{r}\left[
1 + \left(\frac{r'}{r}\right)F + \frac{1}{2}\left(\frac{r'}{r}\right)^2(3\,F^2-1)
+ {\cal O}\left(\frac{r'}{r}\right)^3\right].
\end{equation}
Let us now average this expression over the azimuthal angle, $\phi'$. Since
$\langle 1\rangle =1$, $\langle\cos\phi'\rangle = 0$, and $\langle \cos^2\phi'\rangle = 1/2$, it is easily seen that
\begin{eqnarray}
\langle F\rangle &=&\cos\theta\,\cos\theta',\\[0.5ex]
\langle F^2\rangle &=& \frac{1}{2}\,\sin^2\theta\,\sin^2\theta'
+ \cos^2\theta\,\cos^2\theta'\nonumber\\[0.5ex]
&=& \frac{1}{3}+ \frac{2}{3}\left(\frac{3}{2}\,\cos^2\theta-\frac{1}{2}\right)\left(\frac{3}{2}\,\cos^2\theta'-\frac{1}{2}\right).
\end{eqnarray}
Hence,
\begin{eqnarray}\label{e13.21}
\left\langle |{\bf r}'-{\bf r}|^{-1}\right\rangle&=& \frac{1}{r}\left[
1 + \left(\frac{r'}{r}\right)\cos\theta\,\cos\theta' \right.\\[0.5ex]
&&+\left.\left(\frac{r'}{r}\right)^2\left(\frac{3}{2}\cos^2\theta-\frac{1}{2}\right)\left(\frac{3}{2}\cos^2\theta'-\frac{1}{2}\right)
+ {\cal O}\left(\frac{r'}{r}\right)^3\right].\nonumber
\end{eqnarray}

Now, the well-known {\em Legendre polynomials}, $P_n(x)$, are defined
\begin{equation}
P_n(x) = \frac{1}{2^n\,n!}\,\frac{d^n}{dx^n}\!\left[(x^2-1)^n\right],
\end{equation}
for $n=0,\infty$.
It follows that
\begin{eqnarray}
P_0(x) &=& 1,\label{e13.23}\\[0.5ex]
P_1(x) &=& x,\\[0.5ex]
P_2(x) &=& \frac{1}{2}\,(3\,x^2-1),\label{e13.25}
\end{eqnarray}
{\em etc.}
The Legendre polynomials are {\em mutually
orthogonal}: {\em i.e.},
\begin{equation}\label{e13.26}
\int_{-1}^1 P_n(x)\,P_m(x)\,dx = \int_0^\pi P_n(\cos\theta)\,P_m(\cos\theta)\,\sin\theta\,d\theta = \frac{\delta_{nm}}{n+1/2}.
\end{equation}
Here, $\delta_{nm}$ is 1 if $n=m$, and 0 otherwise. The Legendre polynomials also form a {\em complete set}: {\em i.e.}, any well-behaved function
of $x$ can be represented as a weighted sum of the $P_n(x)$. Likewise,
any well-behaved (even) function of $\theta$ can be represented as a weighted
sum of the $P_n(\cos\theta)$.

A comparison of Equation~(\ref{e13.21}) and Equations~(\ref{e13.23})--(\ref{e13.25}) makes it reasonably clear  that, when $r>r'$, the complete  expansion
of $\langle|{\bf r}'-{\bf r}|^{-1}\rangle$  is
\begin{equation}\label{e13.27}
\left\langle|{\bf r}'-{\bf r}|^{-1}\right\rangle = \frac{1}{r}\sum_{n=0,\infty}
\left(\frac{r'}{r}\right)^n P_n(\cos\theta)\,P_n(\cos\theta').
\end{equation}
Similarly, when $r < r'$, we can expand in powers of $r/r'$ to obtain
\begin{equation}\label{e13.28}
\left\langle|{\bf r}'-{\bf r}|^{-1}\right\rangle = \frac{1}{r'}\sum_{n=0,\infty}
\left(\frac{r}{r'}\right)^n P_n(\cos\theta)\,P_n(\cos\theta').
\end{equation}
It follows from Equations~(\ref{e13.13}), (\ref{e13.27}), and (\ref{e13.28})
that
\begin{equation}\label{e13.29}
\Phi(r,\theta) = \sum_{n=0,\infty} \Phi_n(r)\,P_n(\cos\theta),
\end{equation}
where
\begin{eqnarray}\label{e13.30}
\Phi_n(r) &=& -\frac{2\pi\,G}{r^{n+1}}\int_0^r \int_0^\pi r'^{\,n+2}\,
\rho(r',\theta')\,P_n(\cos\theta')\,\sin\theta'\,dr'\,d\theta'\nonumber\\[0.5ex]
&&-2\,\pi\,G\,r^n\int_r^\infty \int_0^\pi r'^{\,1-n}\,
\rho(r',\theta')\,P_n(\cos\theta')\,\sin\theta'\,dr'\,d\theta'.
\end{eqnarray}

Now, given that the $P_n(\cos\theta)$ form a complete set, we can always
write
\begin{equation}\label{e13.31}
\rho(r,\theta) = \sum_{n=0,\infty} \rho_n(r)\,P_n(\cos\theta).
\end{equation}
This expression can be inverted, with the aid of Equation~(\ref{e13.26}), to
give
\begin{equation}
\rho_n(r) = (n+1/2)\int_0^\pi\rho(r,\theta)\,P_n(\cos\theta)\,\sin\theta\,d\theta.
\end{equation}
Hence, Equation~(\ref{e13.30}) reduces to
\begin{equation}\label{e13.33}
\Phi_n(r) = -\frac{2\pi\,G}{(n+1/2)\,r^{n+1}}\int_0^r r'^{\,n+2}\,\rho_n(r')\,dr'-\frac{2\,\pi\,G\,r^n}{n+1/2}\int_r^\infty r'^{\,1-n}\,\rho_n(r')\,dr'.
\end{equation}
Thus, we now have a general expression for the gravitational potential,
$\Phi(r,\theta)$,
generated by any axially symmetric mass distribution, $\rho(r,\theta)$.

\section{Potential Due to a Uniform Sphere}
Let us calculate the gravitational potential generated by a sphere of
uniform mass density $\gamma$ and radius $a$. Expressing $\rho(r,\theta)$
in the form (\ref{e13.31}), it is clear that
\begin{equation}
\rho_0(r) = \left\{\begin{array}{ccc}
\gamma&\mbox{\hspace{0.5cm}}&\mbox{for $r\leq a$}\\
0&&\mbox{for $r>a$}\end{array}\right.,
\end{equation}
with $\rho_n(r)=0$ for $n>0$. Thus, from (\ref{e13.33}),
\begin{equation}
\Phi_0(r) = -\frac{4\pi\,G\,\gamma}{r}\int_0^r r'^{\,2}\,dr' - 4\pi\,G\,\gamma
\int_r^a r'\,dr'
\end{equation}
for $r\leq a$, and
\begin{equation}
\Phi_0(r) = - \frac{4\pi\,G\,\gamma}{r}\int_0^a r'^{\,2}\,dr'
\end{equation}
for $r>a$, with $\Phi_n(r)=0$ for $n>0$. Hence,
\begin{equation}\label{e13.37}
\Phi(r) = - \frac{2\pi\,G\,\gamma}{3}\,(3\,a^2-r^2) = - G\,M\,\frac{(3\,a^2-r^2)}{2\,a^3}
\end{equation}
for $r\leq a$, and
\begin{equation}\label{e13.38}
\Phi(r) = -\frac{4\pi\,G\,\gamma}{3}\,\frac{a^3}{r} = - \frac{G\,M}{r}
\end{equation}
for $r>a$. Here, $M=(4\pi/3)\,a^3\,\gamma$ is the total mass of the sphere.

According to Equation~(\ref{e13.38}), the gravitational potential {\em outside}\/ a uniform sphere of mass $M$ is the same as that generated by a point mass $M$ located
at the sphere's center. It turns out that this is a general result for {\em any}\/
finite spherically symmetric mass distribution. 
Indeed,  from the
previous analysis, it is clear that 
$\rho(r)=\rho_0(r)$ and $\Phi(r) = \Phi_0(r)$ for a spherically symmetric mass distribution.  Suppose that the mass
distribution extends out to $r=a$. It immediately follows, from Equation~(\ref{e13.33}),
that
\begin{equation}
\Phi(r) = - \frac{G}{r}\int_0^a 4\pi\,r'^{\,2}\,\rho(r')\,dr' = -\frac{G\,M}{r}
\end{equation}
for $r>a$, 
where $M$ is the total mass of the distribution. We, thus, conclude that
Newton's laws of motion, in their primitive form, apply not only to
point masses, but also to extended spherically symmetric masses. In fact, this
is something which we have implicitly assumed all along in this book.

According to Equation~(\ref{e13.37}), the gravitational potential {\em inside}\/ a uniform
sphere is {\em quadratic}\/ in $r$. This implies that if a narrow shaft were
drilled though the center of the sphere then a test mass, $m$, moving in this
shaft would experience a gravitational force acting toward the
center which scales {\em linearly}\/ in $r$. In fact, the
force in question is given by $f_r = -m\,\partial\Phi/\partial r = -(G\,m\,M/a^3)\,r$. It follows that a test mass dropped into the shaft executes simple
harmonic motion about the center of the sphere with period
\begin{equation}
T = 2\pi\,\sqrt{\frac{a}{g}},
\end{equation}
where $g=G\,M/a^2$ is the gravitational acceleration at the sphere's surface.

\section{Potential Outside a Uniform Spheroid}
Let us now calculate the gravitational potential generated outside a {\em spheroid}\/
of uniform mass density $\gamma$ and mean radius $a$. A spheroid is
the solid body produced by rotating an ellipse about a major or
minor axis. Let the axis of rotation coincide with the $z$-axis,
and let the  outer boundary of the spheroid satisfy
\begin{equation}\label{e13.41}
r = a_\theta(\theta) = a\left[1-\frac{2}{3}\,\epsilon\,P_2(\cos\theta)\right],
\end{equation}
where  $\epsilon$ is the termed the {\em ellipticity}. Here, we
are assuming that $|\epsilon|\ll 1$, so that the spheroid is very close to being a
sphere. If $\epsilon>0$ then the spheroid is
slightly squashed along its symmetry axis, and is termed {\em oblate}. Likewise, if $\epsilon<0$ then the spheroid is slightly elongated along its axis, and is
termed {\em prolate}---see Figure~\ref{fsph}.
Of course, if $\epsilon=0$ then the spheroid reduces to a sphere.

\begin{figure}
\epsfysize=2.25in
\centerline{\epsffile{Chapter12/fig12.01.eps}}
\caption{\em Prolate and oblate spheroids.}\label{fsph}
\end{figure}

Now, according to Equation~(\ref{e13.29}) and (\ref{e13.30}), the gravitational
potential generated {\em outside}\/ an axially symmetric mass distribution
can be written
\begin{equation}
\Phi(r,\theta) = \sum_{n=0,\infty} J_n\,\frac{P_n(\cos\theta)}{r^{n+1}},
\end{equation}
where
\begin{equation}\label{e13.43}
J_n = - 2\pi\,G\,\int\int r^{\,2+n}\,\rho(r,\theta)\,P_n(\cos\theta)\,\sin\theta\,dr\,d\theta.
\end{equation}
Here, the integral is taken over the whole cross-section of the distribution
in $r$--$\theta$ space.

It follows that for a uniform spheroid
\begin{equation}
J_n = - 2\pi\,G\,\gamma\int_0^\pi P_n(\cos\theta)\,\int_0^{a_\theta(\theta)}r^{\,2+n}\,dr\,\sin\theta\,d\theta
\end{equation}
Hence,
\begin{equation}
J_n = -\frac{2\pi\,G\,\gamma}{(3+n)}\int_0^\pi P_n(\cos\theta)\,a_\theta^{3+n}(\theta)\,\sin\theta\,d\theta,
\end{equation}
giving
\begin{equation}
J_n \simeq -\frac{2\pi\,G\,\gamma\,a^{3+n}}{(3+n)}\int_0^\pi P_n(\cos\theta)\left[P_0(\cos\theta)-\frac{2}{3}\,(3+n)\,\epsilon\,P_2(\cos\theta)\right]\sin\theta\,d\theta,
\end{equation}
to first-order in $\epsilon$. It is thus clear, from Equation~(\ref{e13.26}),
that, to first-order in $\epsilon$, the only non-zero $J_n$ are
\begin{eqnarray}
J_0 &=& - \frac{4\pi\,G\,\gamma\,a^3}{3} = - G\,M,\\[0.5ex]
J_2 &=&  \frac{8\pi\,G\,\gamma\,a^5\,\epsilon}{15} = \frac{2}{5}\,G\,M\,a^2\,\epsilon,\label{e13.48}
\end{eqnarray}
since $M = (4\pi/3)\,a^3\,\gamma$. 

Thus, the gravitational potential outside a uniform spheroid of
total mass $M$, mean radius $a$, and ellipticity $\epsilon$, is
\begin{equation}\label{e13.49}
\Phi(r,\theta) = - \frac{G\,M}{r} +\frac{2}{5}\frac{G\,M\,a^2}{r^3}\,\epsilon\,P_2(\cos\theta) + {\cal O}(\epsilon^2).
\end{equation}
In particular,
the gravitational potential on the surface of the spheroid is
\begin{equation}
\Phi(a_\theta, \theta) = - \frac{G\,M}{a_\theta} +\frac{2}{5}\frac{G\,M\,a^2}{a_\theta^{\,3}}\,\epsilon\,P_2(\cos\theta) + {\cal O}(\epsilon^2),
\end{equation}
which yields
\begin{equation}\label{e13.51}
\Phi(a_\theta,\theta) \simeq - \frac{G\,M}{a} \left[1+\frac{4}{15}\,\epsilon\,P_2(\cos\theta) + {\cal O}(\epsilon^2)\right],
\end{equation}
where use has been made of Equation~(\ref{e13.41}).

Consider a self-gravitating spheroid of mass $M$, mean radius $a$, and ellipticity $\epsilon$: {\em e.g.}, a star, or a planet. Assuming, for the sake of simplicity, that the
spheroid is composed of uniform density incompressible fluid, the gravitational potential on its surface is
given by Equation~(\ref{e13.51}). However, the condition for an equilibrium
state is that the potential be {\em constant}\/ over the surface. If this is not
the case then there will be gravitational forces acting {\em tangential}\/ to the
surface. Such forces cannot be balanced by internal pressure, which only
acts {\em normal}\/ to the surface. Hence, from (\ref{e13.51}), it is clear that the
condition for equilibrium is $\epsilon=0$. In other words, the equilibrium
configuration  of a self-gravitating mass is a {\em sphere}. Deviations
from this configuration can only be caused by forces in addition to self-gravity
and internal pressure: {\em e.g.}, centrifugal forces due to rotation, or tidal
forces due to orbiting masses.

\section{Rotational Flattening}\label{srotf}
Let us consider the equilibrium configuration of a self-gravitating spheroid,
composed of uniform density incompressible fluid, which is {\em rotating}\/
steadily
about some fixed axis. Let $M$ be the total mass, $a$ the mean radius,
$\epsilon$ the ellipticity, and ${\mit\Omega}$ the angular velocity of rotation. Furthermore, let the axis of rotation coincide with the axis of symmetry, which is assumed to run along the $z$-axis.

Let us transform to a non-inertial frame of reference which co-rotates with the spheroid about the $z$-axis, and in which the spheroid consequently appears to be stationary. From Chapter~\ref{snoni},
the problem is now analogous to that of a non-rotating spheroid, except that
the surface acceleration  is written ${\bf g} = {\bf g}_g + {\bf g}_c$,
where ${\bf g}_g=-\nabla\Phi(r,\theta)$ is the gravitational acceleration, and
${\bf g}_c$ the {\em centrifugal}\/ acceleration. The latter acceleration
is of magnitude $r\,\sin\theta\,\Omega^2$, and is everywhere directed
away from the axis of rotation (see Figure~\ref{frotn} and Chapter~\ref{snoni}).
The acceleration thus has components 
\begin{eqnarray}
{\bf g}_c = r\,\Omega^2\,\sin\theta\,(\sin\theta,\,\cos\theta,\,0)
\end{eqnarray}
in spherical polar coordinates. It follows that ${\bf g}_c = - \nabla\chi$,
where
\begin{equation}\label{e13.53}
\chi(r,\theta) = - \frac{\Omega^2\,r^2}{2}\,\sin^2\theta= -\frac{\Omega^2\,r^2}{3}\,\left[1-P_2(\cos\theta)\right]
\end{equation}
can be thought of as a sort of centrifugal potential. Hence, the
total surface acceleration is
\begin{equation}
{\bf g} = - \nabla(\Phi+\chi).
\end{equation}

As before, the criterion for an equilibrium state is that the surface lie at
a constant total potential, so as to eliminate tangential surface forces which
cannot be balanced by internal pressure. Hence, assuming that the
surface satisfies Equation~(\ref{e13.41}), the equilibrium configuration is specified by
\begin{equation}
\Phi(a_\theta,\theta)+\chi(a_\theta,\theta) = c,
\end{equation}
where $c$ is a constant. It follows from Equations~(\ref{e13.51}) and (\ref{e13.53}) that, to first-order in $\epsilon$ and ${\mit\Omega}^2$,
\begin{equation}
- \frac{G\,M}{a} \left[1+ \frac{4}{15}\,\epsilon\,P_2(\cos\theta)\right]
-\frac{\mit\Omega^2\,a^2}{3}\left[1-P_2(\cos\theta)\right]\simeq c,
\end{equation}
which yields
\begin{equation}
\epsilon =  \frac{5}{4}\,\frac{{\mit\Omega}^2\,a^3}{G\,M}.
\end{equation}

We conclude, from the above expression, that the equilibrium configuration
of a (relatively slowly) rotating self-gravitating mass distribution is an {\em oblate spheroid}: {\em i.e.}, a sphere
which is slightly flattened along its axis of rotation. The degree of flattening is proportional
to the square of the rotation rate. Now, from (\ref{e13.41}), the mean radius 
of the spheroid is
$a$, the radius at the poles ({\em i.e.}, along the axis of rotation) is $a_p=a\,(1-2\,\epsilon/3)$, and the radius at the
equator ({\em i.e.}, perpendicular to the axis of rotation) is $a_e = a\,(1+\epsilon/3)$---see Figure~\ref{frotn}. Hence, the degree of rotational flattening
can be written
\begin{equation}
\frac{a_e-a_p}{a} = \epsilon = \frac{5}{4}\,\frac{{\mit\Omega}^2\,a^3}{G\,M}.
\end{equation}
\begin{figure}
\epsfysize=2.25in
\centerline{\epsffile{Chapter12/fig12.02.eps}}
\caption{\em Rotational flattening.}\label{frotn}
\end{figure}

Now, for the Earth, $a=6.37\times 10^6\,{\rm m}$, $\Omega = 7.27\times 10^{-5}\,{\rm rad./s}$, and $M = 5.97\times 10^{24}\,{\rm kg}$. 
Thus, we predict that
\begin{equation}
\epsilon = 0.00429,
\end{equation}
corresponding to a difference between equatorial and polar radii
of
\begin{equation}
\Delta a = a_e-a_p = \epsilon\,a = 27.3\,{\rm km}.
\end{equation}
In fact, the observed degree of flattening of the Earth is $\epsilon=0.00335$,
corresponding to a difference between equatorial and polar radii
of $21.4\,{\rm km}$. The main reason that our analysis has overestimated the
degree of rotational flattening of the Earth is that it models the terrestrial interior as a uniform density incompressible fluid. In reality,
the Earth's core is much denser than its crust (see Exercise 12.1). 

For Jupiter, $a=6.92\times 10^7\,{\rm m}$, $\Omega = 1.76\times 10^{-4}\,{\rm rad./s}$, and $M=1.90\times 10^{27}\,{\rm kg}$. Hence,
we predict that
\begin{equation}
\epsilon= 0.101.
\end{equation}
Note that this degree of flattening is much larger than that of the Earth, due to
Jupiter's relatively large radius (about 10 times that of Earth), combined with its relatively
short rotation period (about 0.4 days). In fact, the polar flattening of
Jupiter is clearly apparent from images of this planet. The observed degree of
polar flattening of Jupiter is actually $\epsilon=0.065$. Our estimate of $\epsilon$ is probably
slightly too large because Jupiter, which is mostly gaseous, has a mass distribution which is strongly concentrated
at its core (see Exercise 12.1).

\section{McCullough's Formula}\label{smcl}
According to Equations~(\ref{e13.43}) and (\ref{e13.48}), if the Earth
is modeled as spheroid of uniform density $\gamma$ then its ellipticity
is given by
\begin{eqnarray}
\epsilon& =& - \left.\int r^2\,\gamma\,P_2(\cos\theta)\,d^{3}{\bf r}\right/ I_0
= - \left.\frac{1}{2}\int r^2\,\gamma\,\left(3\,\cos^2\theta - 1\right)\,d^{3}{\bf r}\right/ I_0,
\end{eqnarray}
where the integral is over the whole volume of the Earth, and
$I_0 = (2/5)\,M\,a^2$ would be the Earth's moment of inertia  were it
exactly spherical. Now, the Earth's moment of inertia about its
axis of rotation is given by
\begin{equation}
I_\parallel = \int (x^2+y^2)\,\gamma\,d^3{\bf r} = \int r^2\,\gamma\,(1-\cos^2\theta)\,d^3{\bf r}.
\end{equation}
Here, use has been made of Equations~(\ref{e13.9})--(\ref{e13.11}). Likewise,
the Earth's moment of inertia about an axis perpendicular to its
axis of rotation (and passing through the Earth's center) is
\begin{eqnarray}
I_\perp&= &\int (y^2+z^2)\,\gamma\,d^3{\bf r} = \int r^2\,\gamma\,(\sin^2\theta\,\sin^2\,\phi + \cos^2\theta)\,d^3{\bf r}\nonumber\\[0.5ex]
&=& \int r^2\,\gamma\,\left(\frac{1}{2}\,\sin^2\theta+ \cos^2\theta\right)\,d^3{\bf r} = \frac{1}{2}\int r^2\,\gamma\,(1+ \cos^2\theta)\,d^3{\bf r},
\end{eqnarray}
since the average of $\sin^2\phi$ is $1/2$ for an axisymmetric mass distribution. It follows from the above three equations that
\begin{equation}\label{emcll}
\epsilon = \frac{I_\parallel-I_\perp}{I_0}\simeq \frac{I_\parallel-I_\perp
}{I_\parallel}.
\end{equation}
This result, which is known as {\em McCullough's formula}, demonstrates
that the Earth's ellipticity is directly related to the difference between
its principle moments of inertia. It turns out that McCullough's formula
holds for {\em any}\/ axially symmetric mass distribution, and not
just a spheroidal distribution with uniform density. Finally, McCullough's
formula can be combined with Equation~(\ref{e13.49}) to give
\begin{equation}\label{e13.66a}
\Phi(r,\theta) = - \frac{G\,M}{r} + \frac{G\,(I_\parallel - I_\perp)}{r^3}\,P_2(\cos\theta).
\end{equation}
This is the general expression for the gravitational potential generated outside
an axially symmetric mass distribution. The first term on the right-hand
side is the {\em monopole}\/ gravitational field which would be generated
if all  of the mass in the distribution were concentrated at its center of mass,
whereas the second term is the {\em quadrupole}\/ field generated by any deviation from spherical symmetry in the distribution.

\section{Tidal Elongation}
Consider two point masses, $m$ and $m'$, executing circular orbits
about their common center of mass, $C$, with angular
velocity $\omega$. Let $R$ be the distance between
the masses, and $\rho$ the distance between point $C$ and mass $m$---see Figure~\ref{ftide}.
We know, from Section~\ref{sbin}, that
\begin{equation}\label{e13.61}
\omega^2 = \frac{G\,M}{R^3},
\end{equation}
and
\begin{equation}\label{e13.62}
\rho = \frac{m'}{M}\,R,
\end{equation}
where $M=m+m'$.

\begin{figure}
\epsfysize=1.in
\centerline{\epsffile{Chapter12/fig12.03.eps}}
\caption{\em Two orbiting masses.}\label{ftide}
\end{figure}

Let us transform to a non-inertial frame of reference which {\em rotates}, about an axis perpendicular to the orbital plane and  passing through $C$, at
the angular velocity $\omega$. In this reference frame, both masses appear to be {\em stationary}. Consider mass $m$. In the rotating frame, this mass experiences
a gravitational acceleration
\begin{equation}
a_g = \frac{G\,m'}{R^2}
\end{equation}
directed {\em toward}\/ the center of mass, and a  centrifugal acceleration (see Chapter~\ref{snoni})
\begin{equation}
a_c = \omega^2\,\rho
\end{equation}
directed {\em away from}\/ the center of mass.
However, it is easily demonstrated, using Equations~(\ref{e13.61}) and (\ref{e13.62}), that
\begin{equation}
a_c=a_g. 
\end{equation}
In other words, the gravitational and centrifugal accelerations
{\em balance}, as must be the case if mass $m$ is to remain stationary in the
rotating frame. Let us investigate how this balance is affected if the masses $m$ and $m'$
have  {\em finite}\/ spatial extents.

Let the center of the mass distribution $m'$ lie at $A$, the center of the
mass distribution $m$ at B, and the center of mass at $C$---see Figure~\ref{ftide1}. We wish to calculate the centrifugal and gravitational
accelerations at some point $D$ in the vicinity of point $B$. It is
convenient to adopt spherical  coordinates, centered on point $B$,
and aligned such that the $z$-axis coincides with the line $BA$. 

\begin{figure}
\epsfysize=1.25in
\centerline{\epsffile{Chapter12/fig12.04.eps}}
\caption{\em Calculation of tidal forces.}\label{ftide1}
\end{figure}

Let us assume that the mass distribution $m$ 
is orbiting around $C$, but  is {\em not}\/ rotating about
an axis passing through its center, in order to
exclude rotational flattening from our analysis. If this is the
case then it is easily seen that each constituent point of $m$ executes
circular motion of angular velocity $\omega$ and radius $\rho$---see Figure~\ref{fcirc}. Hence, each  constituent point experiences the {\em same}\/
centrifugal acceleration: {\em i.e.}, 
\begin{equation}
{\bf g}_c = - \omega^2\,\rho\,{\bf e}_z.
\end{equation}
 It follows that 
\begin{equation}
{\bf g}_c = - \nabla\chi,
\end{equation}
where
\begin{equation}
\chi =  \omega^2\,\rho\,z 
\end{equation}
is the centrifugal potential, and $z=r\,\cos\theta$. The centrifugal potential
can also be written
\begin{equation}
\chi = \frac{G\,m'}{R}\frac{r}{R}\,P_1(\cos\theta).
\end{equation}

\begin{figure}
\epsfysize=2.75in
\centerline{\epsffile{Chapter12/fig12.05.eps}}
\caption{\em The center $B$ of the mass distribution $m$ orbits about the center of mass $C$ in a circle of radius $\rho$. If the mass distribution is non-rotating then a non-central point $D$ must maintain  a constant spatial relationship to $B$. It follows that point $D$ orbits some point $C'$, which has the same spatial relationship to $C$ that $D$ has to $B$, in a circle
of radius $\rho$.}\label{fcirc}
\end{figure}

The gravitational acceleration at point $D$ due to mass $m'$ is given by
\begin{equation}
{\bf g}_g = -\nabla\Phi',
\end{equation}
where the gravitational potential takes the form
\begin{equation}
\Phi' = -\frac{G\,m'}{R'}.
\end{equation}
Here, $R'$ is the distance between points $A$ and $D$. Note that the
gravitational potential generated by the mass distribution $m'$ is the
same as that generated by an equivalent point mass at $A$, as long
as the distribution is spherically symmetric, which we shall assume to
be the case.

Now, 
\begin{equation}
{\bf R}' = {\bf R} - {\bf r},
\end{equation}
where ${\bf R}'$ is the vector $\stackrel{\displaystyle \rightarrow}{DA}$,
and ${\bf R}$ the vector  $\stackrel{\displaystyle \rightarrow}{BA}$---see Figure~\ref{ftide1}.
It follows that
\begin{equation}
R'^{\,-1} = \left(R^2 - 2\,{\bf R}\cdot{\bf r}+ r^2\right)^{-1/2}
=  \left(R^2 - 2\,R\,r\,\cos\theta+ r^2\right)^{-1/2}.
\end{equation}
Expanding in powers of $r/R$, we obtain
\begin{equation}
R'^{\,-1} = \frac{1}{R}\sum_{n=0,\infty} \left(\frac{r}{R}\right)^n P_n(\cos\theta).
\end{equation}
Hence, 
\begin{equation}
\Phi' \simeq - \frac{G\,m'}{R}\left[1+ \frac{r}{R}\,P_1(\cos\theta) + \frac{r^2}{R^2}\,P_2(\cos\theta)\right]
\end{equation}
 to second-order in $r/R$. 

Adding $\chi$ and $\Phi'$, we obtain
\begin{equation}\label{e13.76}
\chi+\Phi' \simeq - \frac{G\,m'}{R}\left[1 + \frac{r^2}{R^2}\,P_2(\cos\theta)\right]
\end{equation}
 to second-order in $r/R$. Note that $\chi+\Phi'$ is the potential
due to the net {\em external}\/ force acting on the mass distribution $m$. This
potential is constant up to first-order in $r/R$, because the first-order
variations in $\chi$ and $\Phi'$ cancel one another. The
cancellation
is a manifestation of the balance between the centrifugal and gravitational
accelerations in the equivalent point mass problem discussed above. However, this balance
is only exact at the {\em center}\/ of the mass distribution $m$. Away from the
center, the centrifugal acceleration remains constant, whereas
the gravitational acceleration increases with increasing $z$. Hence, at positive $z$, the gravitational
acceleration is larger than the centrifugal, giving rise to a net acceleration
in the $+z$-direction. Likewise, at negative $z$, the centrifugal acceleration
is larger than the gravitational, giving rise to a net acceleration in the $-z$-direction.
It follows that the mass distribution $m$
is subject to a residual acceleration, represented by the second-order variation in Equation~(\ref{e13.76}), which acts to {\em elongate}\/ it along the $z$-axis. 
This effect is known as {\em tidal elongation}.

In order to calculate the tidal elongation of the mass distribution $m$ we
need to add the  potential, $\chi+\Phi'$, due to the external forces,  to
the gravitational potential, $\Phi$, generated by the distribution itself. Assuming that
the mass distribution is spheroidal with mass $m$, mean radius $a$,
and ellipticity $\epsilon$, it follows from Equations~(\ref{e13.41}), (\ref{e13.51}),
and (\ref{e13.76}) that the total surface potential
is given by
\begin{eqnarray}
\chi +\Phi'+\Phi &\simeq& - \frac{G\,m}{a} - \frac{G\,m'}{R}  \nonumber\\&&
-\frac{4}{15}\,\frac{G\,m}{a}\,\epsilon\,P_2(\cos\theta) - \frac{G\,m'\,a^2}{R^3}\,P_2(\cos\theta),
\end{eqnarray}
where we have treated $\epsilon$ and $a/R$ as small quantities. As before,
the condition for equilibrium is that the total potential be constant
over the surface of the spheroid. Hence, we obtain
\begin{equation}
\epsilon = -\frac{15}{4}\,\frac{m'}{m}\left(\frac{a}{R}\right)^3
\end{equation}
as our prediction for the ellipticity induced in a self-gravitating spherical
mass distribution of total mass $m$ and radius $a$ by a second mass, $m'$, 
which is in a circular orbit  of radius $R$ about the distribution. Thus, if $a_+$ is
the maximum radius of the distribution, and $a_-$ the minimum radius (see Figure~\ref{ftide2}), then
\begin{equation}
 \frac{a_+-a_-}{a} = -\epsilon =  \frac{15}{4}\,\frac{m'}{m}\left(\frac{a}{R}\right)^3.
\end{equation}
\begin{figure}
\epsfysize=1.2in
\centerline{\epsffile{Chapter12/fig12.06.eps}}
\caption{\em Tidal elongation.}\label{ftide2}
\end{figure}

Consider the tidal elongation of the Earth due to the Moon. In this
case, we have $a=6.37\times 10^6\,{\rm m}$, $R=3.84\times 10^8\,{\rm m}$, 
$m=5.97\times 10^{24}\,{\rm kg}$, and $m'=7.35\times 10^{22}\,{\rm kg}$. 
Hence, we calculate that $-\epsilon=2.1\times 10^{-7}$, or
\begin{equation}
\Delta a = a_+-a_- = -\epsilon\,a = 1.34\,{\rm m}.
\end{equation}
We, thus, predict that tidal forces due to the Moon cause the
Earth to elongate along the axis joining its center to the Moon by
about $1.3$ meters. Since water is obviously more fluid than rock
(especially on relatively short time-scales) most of this elongation
takes place in the oceans rather than in the underlying land. Hence,
the oceans rise, relative to the land, in the region of the Earth closest
to the Moon, and also in the region furthest away. Since the Earth
is rotating, whilst the tidal bulge of the oceans remains relatively
stationary, the Moon's tidal force causes the ocean at a given point
on the Earth's surface to rise and fall, by about a meter, {\em twice}\/ daily, giving rise to the
phenomenon known as the {\em tides}.

Consider the tidal elongation of the Earth due to the Sun. In this case, 
we have $a=6.37\times 10^6\,{\rm m}$, $R=1.50\times 10^{11}\,{\rm m}$, 
$m=5.97\times 10^{24}\,{\rm kg}$, and $m'=1.99\times 10^{30}\,{\rm kg}$. 
Hence, we calculate that $-\epsilon=9.6\times 10^{-8}$, or
\begin{equation}
\Delta a = a_+-a_- = -\epsilon\,a = 0.61\,{\rm m}.
\end{equation}
Thus, the tidal elongation due to the Sun is about half that due to the Moon.
It follows that the tides are particularly high when the Sun, the Earth, and
the Moon lie  approximately in a straight-line, so that the tidal effects of the Sun and the Moon reinforce one another. This occurs at a new moon,
or at a full moon. These type of tides are called {\em spring tides} (note that
the name has nothing to do with the season). Conversely, the
tides are particularly low when the Sun, the Earth, and the Moon
form a right-angle, so that the tidal effects of the
Sun and the Moon partially cancel one another. These type of tides are called {\em neap tides}. Generally
speaking, we would expect two spring tides and two neap tides per month.

In reality, the amplitude of the tides varies significantly from place to place on the Earth's surface, due to the presence of the continents, which impede the flow of the oceanic tidal bulge
around the Earth. Moreover, there is a time-lag of approximately 12 minutes
between the Moon being directly overhead (or directly below) and
high tide, because of the finite inertia of the oceans. Similarly, the time-lag
between a spring tide and a full moon, or a new moon, can be up to 2 days.

\section{Roche Radius}\label{sroche}
Consider a spherical moon of mass $m$ and radius $a$ which is in a circular
orbit of radius $R$ about a spherical planet of mass $m'$ and radius $a'$.
(Strictly speaking, the moon and the planet execute circular orbits about
their common center of mass. However, if the planet is much more massive
than the moon then the center of mass lies close to the
planet's center.) According to the analysis in the previous section, a constituent  element of
the moon experiences a force per unit mass, due to the gravitational
field of the planet, which takes the form
\begin{equation}
{\bf g}' = - \nabla(\chi+\Phi'),
\end{equation}
where
\begin{equation}
\chi+\Phi' = - \frac{G\,m'}{R^3}\,(z^2-x^2/2-y^2/2)  + {\rm const}.
\end{equation}
Here, ($x$, $y$, $z$) is a Cartesian coordinate system whose origin
is the center of the moon, and whose $z$-axis always points toward the
center of the planet. It follows that
\begin{equation}\label{e13.90x}
{\bf g}' = \frac{2\,G\,m'}{R^3}\left(-\frac{x}{2}\,{\bf e}_x- \frac{y}{2}\,{\bf e}_y+ z\,{\bf e}_z\right).
\end{equation}
This so-called {\em tidal force}\/ is generated by the spatial variation of the
planet's gravitational field  over the interior of the moon, and acts
 to elongate the moon along an axis joining its center to that of the planet, and to compress it in any direction perpendicular
to this axis. Note that the magnitude of the tidal force increases strongly
as the radius, $R$, of the moon's orbit decreases. Now, if the tidal
force becomes sufficiently strong then it can overcome the moon's self-gravity, and thereby  rip the moon
apart. It follows that there is a minimum radius, generally referred to
as the {\em Roche radius}, at which a moon can orbit
a planet without being destroyed by tidal forces.  

Let us derive an expression for the Roche radius. Consider a small mass
element at the point on the surface of the moon which lies closest to the planet, and at which the
tidal force is consequently largest ({\em i.e.}, $x=y=0$, $z=a$).  According to Equation~(\ref{e13.90x}), the
mass experiences an upward (from the moon's surface) tidal acceleration due to the gravitational attraction of the planet of the form
\begin{equation}
{\bf g}' = \frac{2\,G\,m'\,a}{R^3}\,{\bf e}_z.
\end{equation}
The mass also experiences a downward gravitational acceleration due to the gravitational influence of the moon
which is written
\begin{equation}
{\bf g} = - \frac{G\,m}{a^2}\,{\bf e}_z.
\end{equation}
Thus, the effective surface gravity at the point in question is
\begin{equation}
g_{\rm eff} = \frac{G\,m}{a^2}\left(1- 2\,\frac{m'}{m}\,\frac{a^3}{R^3}\right).
\end{equation}
Note that if $R< R_c$, where
\begin{equation}
R_c = \left(2\,\frac{m'}{m}\right)^{1/3} a,
\end{equation}
then the effective  gravity is {\em negative}. In other words, the
tidal force  due to the planet is strong enough to overcome surface gravity and  lift objects off the moon's surface. 
If this is the case, and the tensile strength of the moon is negligible,
then it is fairly clear that the tidal force will start to break the moon apart. Hence, $R_c$ is the Roche radius. Now,
$m'/m = (\rho'/\rho)\,(a'/a)^3$, where $\rho$ and $\rho'$ are the mean
mass densities of the moon and planet, respectively. Thus, the 
above expression for the Roche radius  can also be written
\begin{equation}
R_c = 1.41\left(\frac{\rho'}{\rho}\right)^{1/3} a'.
\end{equation}

The above calculation is somewhat inaccurate, since it fails to take into
account the  inevitable distortion of  the moon's shape in the presence of strong tidal
forces. (In fact, the calculation assumes that the moon
always remains spherical.) A more accurate calculation, which treats the moon
as a self-gravitating incompressible fluid, yields
\begin{equation}
R_c = 2.44\left(\frac{\rho'}{\rho}\right)^{1/3} a'.
\end{equation}
It follows that if the planet and the moon have the same mean
density then the Roche radius is 2.44 times the planet's radius. Note that small orbital bodies such as rocks, or even
very small moons,  can survive intact within the Roche radius because they
are held together by internal tensile forces rather than gravitational attraction.
However, this mechanism becomes progressively less effective as the size
of the body in question increases. Not surprisingly, virtually all large planetary moons
occurring  in  the Solar System have orbital radii which exceed the relevant Roche radius,
whereas virtually all planetary ring systems (which consist of myriads of small orbiting rocks)
have radii which 
lie inside the relevant Roche radius. 

\section{Precession and Forced Nutation of the Earth}\label{sprec}
Consider the Earth-Sun system---see Figure~\ref{fesun}. From a geocentric viewpoint, the Sun orbits the Earth  counter-clockwise  (looking from the north), once per year, in an approximately circular orbit
of radius $a_s= 1.50\times 10^{11}\,{\rm m}$. In astronomy, the plane of the Sun's
apparent orbit relative to the Earth is known as the {\em ecliptic plane}. 
Let us define {\em non-rotating}\/ Cartesian coordinates, centered on the Earth, which are such that the $x$- and $y$-axes lie in the
ecliptic plane, and the $z$-axis is normal to this plane (in the
sense that the Earth's north pole lies at positive $z$). It follows that
the $z$-axis is directed toward a  point in the sky (located in the constellation Draco) known as the {\em north
ecliptic pole}. In the following, we shall treat the $Oxyz$ system as inertial. This is a reasonable
approximation because the orbital acceleration of the Earth is much smaller than the acceleration due to its diurnal rotation.
It is convenient to parameterize the instantaneous position
of the Sun in terms of a counter-clockwise (looking from the north) azimuthal angle $\lambda_s$ that is zero on the positive $x$-axis---see Figure~\ref{fesun}.

\begin{figure}
\epsfysize=2.25in
\centerline{\epsffile{Chapter12/fig12.07.eps}}
\caption{\em The Earth-Sun system.}\label{fesun}
\end{figure}

Let $\bomega$ be the Earth's angular velocity vector due to its
daily rotation. This vector makes an angle $\theta$ with the $z$-axis,
where $\theta = 23.44^\circ$ is the mean inclination of the ecliptic to the
Earth's equatorial plane. Suppose that the projection of $\bomega$
onto the ecliptic plane subtends an angle $\phi$ with the $y$-axis,
where $\phi$ is measured in a counter-clockwise (looking from the north) sense---see Figure~\ref{fesun}.
The orientation of the Earth's axis of rotation (which is, of course, parallel
to $\bomega$) is thus determined by the two angles $\theta$ and $\phi$.
Note, however, that these two angles are also {\em Euler angles}, in
the sense given in Chapter~\ref{srigid}. Let us examine the Earth-Sun
system at an instant in time, $t=0$,  when $\phi=0$: {\em i.e.}, when
$\bomega$ lies in the $y$-$z$ plane. At this particular instant, the $x$-axis points towards the so-called {\em vernal equinox},
which is defined as the point in the sky where  the ecliptic plane crosses the projection of the Earth's
equator ({\em i.e.}, the plane normal to $\bomega$) from south to north. A counter-clockwise (looking from the north) angle in the
ecliptic plane that is zero at the vernal equinox is generally known as an {\em ecliptic longitude}. Thus, $\lambda_s$ is the
Sun's ecliptic longitude. 

According to Equation~(\ref{e13.66a}), the potential energy of
the Earth-Sun system is written
\begin{equation}
U = M_s\,\Phi = - \frac{G\,M_s\,M}{a_s} + \frac{G\,M_s\,(I_\parallel-I_\perp)}{a_s^{\,3}}\,P_2[\cos(\gamma_s)],
\end{equation}
where $M_s$ is the mass of the Sun, $M$ the mass of the Earth,
$I_\parallel$ the Earth's moment of inertia about its axis of rotation,
$I_\perp$ the Earth's moment of inertia about an axis lying in its
equatorial plane, and $P_2(x)=(1/2)\,(3\,x^2-1)$.  Furthermore, $\gamma_s$ is the angle subtended
between $\bomega$ and ${\bf r}_s$, where ${\bf r}_s$ is the 
position vector of the Sun relative to the Earth.

It is easily demonstrated that (with $\phi=0$)
\begin{equation}
\bomega  = \omega\,(0,\,\sin\theta,\,\cos\theta),
\end{equation}
and
\begin{equation}
{\bf r}_s = a_s\,(\cos\lambda_s,\,\sin\lambda_s,\,0).
\end{equation}
Hence,
\begin{equation}
\cos\gamma_s =\frac{\bomega\cdot{\bf r}_s}{|\bomega|\,|{\bf r}_s|}= \sin\theta\,\sin\lambda_s,
\end{equation}
giving
\begin{equation}
U = - \frac{G\,M_s\,M}{a_s} + \frac{G\,M_s\,(I_\parallel-I_\perp)}{2\,a_s^{\,3}}\,(3\,\sin^2\theta\,\sin^2\lambda_s -1).
\end{equation}
Now, we are primarily interested in the motion of the Earth's axis of rotation over time-scales that are {\em much
longer}\/ than a year, so we can average the above
expression over the Sun's orbit to give
\begin{equation}
U = - \frac{G\,M_s\,M}{a_s} + \frac{G\,M_s\,(I_\parallel-I_\perp)}{2\,a_s^{\,3}}\,\left(\frac{3}{2}\,\sin^2\theta -1\right)
\end{equation}
(since the average of $\sin^2\lambda_s$ over a year is $1/2$).
Thus, we obtain
\begin{equation}\label{e13.93}
U = U_0 - \epsilon\,\alpha_s\,\cos(2\,\theta),
\end{equation}
where $U_0$ is a constant, and
\begin{equation}\label{e13.95}
\alpha_s = \frac{3}{8}\,I_\parallel\,n_s^{\,2}.
\end{equation}
Here, 
\begin{equation}
\epsilon = \frac{I_{\parallel}-I_\perp}{I_\parallel}=0.00335
\end{equation}
is the Earth's ellipticity,
and
\begin{equation}
n_s =\frac{d\lambda_s}{dt}=  \left(\frac{G\,M_s}{a_s^{\,3}}\right)^{1/2}
\end{equation}
the Sun's apparent orbital angular velocity. 

According to Section~\ref{sgyro}, the rotational kinetic
energy of the Earth can be written
\begin{equation}\label{e13.96}
K = \frac{1}{2}\left(I_\perp\,\dot{\theta}^2 + I_\perp\,\sin^2\theta\,\dot{\phi}^2 + I_\parallel\,\omega^2\right),
\end{equation}
where the Earth's angular velocity
\begin{equation}\label{e13.97}
\omega = \cos\theta\,\dot{\phi} + \dot{\psi}
\end{equation}
is a constant of the motion.
Here, $\psi$ is the third Euler angle.
Hence, the Earth's Lagrangian takes the form
\begin{equation}\label{e13.98}
{\cal L} = K - U = \frac{1}{2}\left(I_\perp\,\dot{\theta}^2 + I_\perp\,\sin^2\theta\,\dot{\phi}^2 + I_\parallel\,\omega^2\right) +\epsilon\,\alpha_s\,\cos(2\,\theta)
\end{equation}
where any constant terms have been neglected. 
One equation of motion which can immediately be derived from this Lagrangian is
\begin{equation}
\frac{d}{dt}\!\left(\frac{\partial{\cal  L}}{\partial \dot{\theta}}\right)- \frac{\partial {\cal L}}{\partial\theta} = 0,
\end{equation}
which reduces to
\begin{equation}
I_\perp\,\ddot{\theta} - \frac{\partial {\cal  L}}{\partial \theta} = 0.
\end{equation}

Consider {\em steady precession}\/ of the Earth's rotational axis, which is characterized by $\dot{\theta}=0$, with  both $\dot{\phi}$  and $\dot{\psi}$ constant. It follows, from the above equation, that
such motion must satisfy the constraint
\begin{equation}
\frac{\partial {\cal L}}{\partial\theta} = 0.
\end{equation}
Thus, we obtain
\begin{equation}
\frac{1}{2}\,I_\perp\,\sin(2\,\theta)\,\dot{\phi}^{\,2} - I_\parallel\,\sin\theta\,\omega\,\dot{\phi} - 2\,\epsilon\,\alpha_s\,\sin(2\,\theta) = 0,
\end{equation}
where use has been made of Equations~(\ref{e13.97}) and (\ref{e13.98}).
Now, as can easily be verified after the fact, $|\dot{\phi}|\ll \omega$, so the above equation reduces to
\begin{equation}
\dot{\phi} \simeq
-\frac{4\,\epsilon\,\alpha_s\,\cos\theta}{I_{\parallel}\,\omega} = \Omega_\phi,
\end{equation}
which can be integrated to give
\begin{equation}
\phi \simeq- \Omega_\phi\,t,
\end{equation}
where
\begin{equation}
\Omega_\phi = \frac{3}{2}\,\frac{\epsilon\,n_s^{\,2}}{\omega}\,\cos\theta,
\end{equation}
and use has been made of Equation~(\ref{e13.95}).
According to the above expression, the mutual interaction between the Sun and
the quadrupole gravitational field generated by the Earth's slight oblateness causes
the Earth's axis of rotation to {\em precess}\/ steadily about the normal to the ecliptic
plane at the rate $-\Omega_\phi$. The fact that $-\Omega_\phi$ is
negative implies that the precession is in the {\em opposite}\/ direction
to the direction of  the Earth's rotation and the Sun's apparent orbit about the Earth. Incidentally,  the interaction causes a precession
of the Earth's rotational axis, rather than the plane of the Sun's orbit,
 because  the Earth's axial moment of inertia is much less than
 the Sun's orbital moment of inertia.
The precession period in years is given by
\begin{equation}
T_\phi({\rm yr}) = \frac{n_s}{\Omega_\phi} = \frac{2\,T_s({\rm day})}{3\,\epsilon\,\cos\theta},
\end{equation}
where $T_s({\rm day}) = \omega/n_s = 365.24$ is
the Sun's orbital period in days. Thus, given that $\epsilon=0.00335$ and
$\theta = 23.44^\circ$, we obtain
\begin{equation}
T_\phi\simeq 79,200\,\,{\rm years}.
\end{equation}
Unfortunately, the observed precession period of the Earth's axis of rotation about the normal to the ecliptic plane is   approximately 25,800 years, so something is clearly missing from
our model. It turns out that the missing factor is the influence of the {\em Moon}.

Using analogous arguments to those given above, the potential energy of the Earth-Moon system can be
written
\begin{equation}\label{e13xx1}
U = - \frac{G\,M_m\,M}{a_m} + \frac{G\,M_m\,(I_\parallel-I_\perp)}{a_m^{\,3}}\,P_2[\cos(\gamma_m)],
\end{equation}
where $M_m$ is the lunar mass, and $a_m$ the radius of the Moon's (approximately circular) orbit. Furthermore, $\gamma_m$ 
 is the angle subtended
between $\bomega$ and ${\bf r}_m$, where 
 \begin{equation}\label{e13xx2}
\bomega = \omega\,\left(-\sin\theta\,\sin\phi,\,\sin\theta\,\cos\phi,\,\cos\theta\right)
\end{equation}
is the Earth's angular velocity vector, and
 ${\bf r}_m$ is the 
position vector of the Moon relative to the Earth. Here, for the moment, we have retained the $\phi$
dependence in our expression for $\bomega$ (since we shall presently differentiate by $\phi$, before setting $\phi=0$). Now, the Moon's orbital plane
is actually slightly inclined to the ecliptic plane,  the angle of inclination being $\iota_m=5.16^\circ$. Hence, we can write
\begin{equation}\label{e13xx3}
{\bf r}_m \simeq a_m\,\left(\cos\lambda_m,\,\sin\lambda_m,\,\iota_m\,\sin(\lambda_m-\varpi_n)\right),
\end{equation}
to first order in $\iota_m$, where $\lambda_m$ is the Moon's ecliptic longitude, and $\varpi_n$ is the ecliptic longitude of the
lunar {\em ascending node}, which is defined as the point on the lunar orbit where  the
Moon crosses the ecliptic plane from south to north. Of course, $\lambda_m$ increases at the rate $n_m$,
where
\begin{equation}
n_m = \frac{d\lambda_m}{dt}\simeq \left(\frac{G\,M}{a_m^{\,3}}\right)^{1/2}
\end{equation}
is the Moon's orbital angular velocity. It turns out that the lunar ascending node precesses steadily,
in the opposite direction to the Moon's orbital rotation, in such a manner that it completes a
full circuit every $18.6$ years. This precession is caused by the perturbing influence of the
Sun---see Chapter~\ref{moon}. It follows that 
\begin{equation}\label{e13xx8}
\frac{d\varpi_n}{dt}= -\Omega_n, 
\end{equation}
where $2\pi/\Omega_n=18.6\,{\rm years}$. 
Now, from (\ref{e13xx2}) and (\ref{e13xx3}), 
\begin{equation}
\cos\gamma_m = \frac{\bomega\cdot{\bf r}_m}{|\bomega|\,|{\bf r}_m|}= \sin\theta\,\sin(\lambda_m-\phi)+\iota_m\,\cos\theta\,\sin(\lambda_m-\varpi_n),
\end{equation}
so (\ref{e13xx1}) yields
\begin{eqnarray}
U &\simeq& - \frac{G\,M_m\,M}{a_m} + \frac{G\,M_m\,(I_\parallel-I_\perp)}{2\,a_m^{\,3}}\left[3\,\sin^2\theta\,\sin^2(\lambda_m-\phi)\right.\nonumber\\[0.5ex]&&\left.
+ 3\,\iota_m\,\sin(2\,\theta)\,\sin(\lambda_m-\phi)\,\sin(\lambda_m-\varpi_n)-1\right]
\end{eqnarray}
to first order in $\iota_m$.
Given that we are interested in the motion of the Earth's axis of rotation on time-scales that are much longer than a month, we can average the above expression over the Moon's orbit to give
\begin{equation}\label{e13xx5}
U \simeq U_0' - \epsilon\,\alpha_m\,\cos(2\,\theta) + \epsilon\,\beta_m\,\sin(2\,\theta)\,\cos(\varpi_n-\phi),
\end{equation}
[since the average of $\sin^2(\lambda_m-\phi)$ over a month is $1/2$, whereas that of $\sin(\lambda_m-\phi)\,\sin(\lambda_m-\varpi_m)$
is $(1/2)\,\cos(\varpi_m-\phi)$]. Here, $U_0'$ is a constant,
\begin{eqnarray}
\alpha_m &=& \frac{3}{8}\,I_\parallel\,\mu_m\,n_m^{\,2},\\[0.5ex]
\beta_m &=& \frac{3}{4}\,I_\parallel\,\iota_m\,\mu_m\,n_m^{\,2},
\end{eqnarray}
 and 
  \begin{equation}
  \mu_m=\frac{M_m}{M} = 0.0123
  \end{equation}
  is the ratio of the lunar to the terrestrial mass. 
  Now, gravity is a superposable force, so the total potential energy of the Earth-Moon-Sun system is
 the sum of Equations~(\ref{e13.93}) and (\ref{e13xx5}). In other words,
\begin{equation}
U = U_0'' -\epsilon\,\alpha\,\cos(2\,\theta) +\epsilon\,\beta_m\,\sin(2\,\theta)\,\cos(\varpi_n-\phi),
\end{equation}
where $U_0''$ is a constant, and 
\begin{equation}
\alpha=\alpha_s+\alpha_m. 
\end{equation}
Finally, making use of (\ref{e13.96}), the Lagrangian of the Earth is written
\begin{equation}
{\cal L}= \frac{1}{2}\left(I_\perp\,\dot{\theta}^2 + I_\perp\,\sin^2\theta\,\dot{\phi}^2 + I_\parallel\,\omega^2\right) +\epsilon\,\alpha\,\cos(2\,\theta)- \epsilon\,\beta_m\,\sin(2\,\theta)\,\cos(\varpi_n-\phi),
\end{equation}
where any constant terms have been neglected. Recall that $\omega$  is  given by (\ref{e13.97}), and is a constant of the motion. 

Two equations of motion that can immediately be derived from the above Lagrangian are
\begin{eqnarray}
\frac{d}{dt}\!\left(\frac{\partial {\cal L}}{\partial \dot{\theta}}\right)-\frac{\partial{\cal L}}{\partial \theta} &=&0,\\[0.5ex]
\frac{d}{dt}\!\left(\frac{\partial {\cal L}}{\partial \dot{\phi}}\right)-\frac{\partial{\cal L}}{\partial \phi} &=&0.
\end{eqnarray}
(The third equation, involving $\psi$, merely confirms that $\omega$ is a constant of the motion.)
The above two equations yield
\begin{eqnarray}\label{e13xx6}
0&=&I_\perp\,\ddot{\theta} - \frac{1}{2}\,I_\perp\,\sin(2\,\theta)\,\dot{\phi}^{\,2} + I_{\parallel}\,\sin\theta\,\omega\,\dot{\phi}
+2\,\epsilon\,\alpha\,\sin(2\,\theta) \nonumber\\[0.5ex]&&+ 2\,\epsilon\,\beta_m\,\cos(2\,\theta)\,\cos(\varpi_n-\phi),\\[0.5ex]\label{e13xx7}
0&=&\frac{d}{dt}\!\left(I_\perp\,\sin^2\theta\,\dot{\phi} + I_{\parallel}\,\cos\theta\,\omega\right) + \epsilon\,\beta_m\,\sin(2\,\theta)\,\sin(\varpi_n-\phi),
\end{eqnarray}
respectively. 
Let
\begin{eqnarray}\label{e13xx9}
\theta(t)&=&\theta_0+ \epsilon\,\theta_1(t),\\[0.5ex]
\phi(t) &=& \epsilon\,\phi_1(t),\label{e13xx10}
\end{eqnarray}
where $\theta_0=23.44^\circ$ is the mean inclination of the ecliptic to the Earth's equatorial plane. To first order in $\epsilon$, 
Equations~(\ref{e13xx6}) and (\ref{e13xx7}) reduce to 
\begin{eqnarray}
0&\simeq&I_\perp\,\ddot{\theta}_1 + I_{\parallel}\,\sin\theta_0\,\omega\,\dot{\phi}_1
+2\,\alpha\,\sin(2\,\theta_0) + 2\,\beta_m\,\cos(2\,\theta_0)\,\cos(\Omega_n\,t),\\[0.5ex]
0&\simeq&I_\perp\,\sin^2\theta_0\,\ddot{\phi}_1 - I_{\parallel}\,\sin\theta_0\,\omega\,\dot{\theta}_1 - \beta_m\,\sin(2\,\theta_0)\,\sin(\Omega_n\,t),
\end{eqnarray}
respectively, where use has been made of Equation~(\ref{e13xx8}). 
However, as can easily be verified after the fact, $d/dt\ll \omega$,
so we obtain
\begin{eqnarray}
\dot{\phi}_1 &\simeq& -\frac{4\,\alpha\,\cos\theta_0}{I_\parallel\,\omega}-\frac{2\,\beta_m\,\cos(2\,\theta_0)}{I_\parallel\,\omega\,\sin\theta_0}\,\cos(\Omega_n\,t),\\[0.5ex]
\dot{\theta}_1&\simeq& -\frac{2\,\beta_m\,\cos\theta_0}{I_\parallel\,\omega}\,\sin(\Omega_n\,t).
\end{eqnarray}
The above equations can be integrated,  and then combined with
Equations~(\ref{e13xx9}) and (\ref{e13xx10}), to give
\begin{eqnarray}\label{e13xx11}
\phi(t) &=& - \Omega_\phi\,t - \delta\phi\,\sin(\Omega_n\,t),\\[0.5ex]
\theta(t) &=& \theta_0 + \delta\theta\,\cos(\Omega_n\,t),\label{e13xx12}
\end{eqnarray}
where
\begin{eqnarray}
\Omega_\phi &=& \frac{3}{2}\,\frac{\epsilon\,(n_s^{\,2}+\mu_m\,n_m^{\,2})}{\omega}\,\cos\theta_0,\\[0.5ex]
\delta\phi &=& \frac{3}{2}\,\frac{\epsilon\,\iota_m\,\mu_m\,n_m^{\,2}}{\omega\,\Omega_n}\,\frac{\cos(2\,\theta_0)}{\sin\theta_0},\\[0.5ex]
\delta\theta &=& \frac{3}{2}\,\frac{\epsilon\,\iota_m\,\mu_m\,n_m^{\,2}}{\omega\,\Omega_n}\,\cos\theta_0.
\end{eqnarray}
Incidentally, in the above,  we have assumed that the lunar ascending node coincides with the vernal
equinox at time $t=0$ ({\em i.e.}, $\varpi_m=0$ at $t=0$), in accordance with our previous assumption that $\phi=0$ at $t=0$. 

According to Equation~(\ref{e13xx11}), the combined gravitational interaction of the Sun and the Moon with the
quadrupole field generated by the Earth's slight oblateness causes the Earth's axis of rotation to
precess steadily about the normal to the ecliptic plane at the rate $-\Omega_\phi$. 
As before, the negative sign indicates that the precession is in the opposite direction to the (apparent) orbital
motion of the sun and moon. The period of the precession in years is given by
\begin{equation}
T_\phi({\rm yr}) = \frac{n_s}{\Omega_\phi} = \frac{2\,T_s({\rm day})}{\epsilon\,(1+\mu_m/[T_m({\rm yr})]^2)\,\cos\theta_0},
\end{equation}
where $T_m({\rm yr})= n_s/n_m=0.081$ is the Moon's (synodic) orbital period in years. Given
that $\epsilon=0.00335$, $\theta_0=23.44^\circ$, $T_s({\rm day})=365.24$, and $\mu_m=0.0123$, we
obtain
\begin{equation}
T_\phi\simeq 27,600\,{\rm years}.
\end{equation}
This prediction is fairly close to the observed precession period of $25,800\,{\rm years}$. The main reason that our estimate is slightly inaccurate is because we have  neglected to take into
account the small eccentricities of the Earth's orbit around the Sun, and the Moon's orbit around
the Earth. 

The point in the sky
toward which the Earth's axis of rotation points is known as the {\em north celestial pole}. Currently,
this point lies within about a degree of the fairly bright star {\em Polaris}, which is consequently sometimes known as the {\em north star}\/
or the {\em pole star}. It follows that Polaris appears to be almost stationary in the sky, always lying {\em due north}, and can thus
be used for navigational purposes. Indeed, mariners have relied on the north star for many hundreds
of years to determine direction at sea. Unfortunately, because of the precession of the
Earth's axis of rotation, the north celestial pole is not a fixed point in the sky, but instead traces out a circle,
of angular radius $23.44^\circ$, about the north ecliptic pole, with a period of 25,800 years.
Hence,  a few thousand years from now, the north celestial pole will no longer coincide with Polaris, and
there will be no convenient way of telling direction from the stars.

The projection of the ecliptic plane onto the sky is called the {\em ecliptic}, and coincides with the
apparent path of the Sun against the backdrop of the stars. Furthermore, the projection of the Earth's equator
onto the sky is known as the {\em celestial equator}. As has been previously mentioned, the ecliptic is inclined at $23.44^\circ$ to the
celestial equator. The two points in the sky at which the ecliptic crosses the celestial equator are
called the {\em equinoxes}, since  night and day are equally
long when the Sun lies at these  points. Thus, the Sun reaches the {\em vernal equinox}\/ on about 
March 21st, and this traditionally marks the beginning of spring. Likewise, the Sun reaches the
{\em autumn equinox}\/ on about September 22nd, and this traditionally marks the beginning of autumn.
However, the precession of the Earth's axis of rotation causes the
celestial equator (which is always normal to this axis) to precess in the sky, and thus also causes the equinoxes to precess along the ecliptic. This
effect is known as the {\em precession of the equinoxes}. The precession  is in the opposite direction to the Sun's apparent motion around the ecliptic,  and is of magnitude $1.4^\circ$ per century. Amazingly, this miniscule
effect was discovered by the Ancient Greeks (with the help of ancient Babylonian observations). In about 2000 BC, when the science of astronomy originated in ancient Egypt and Babylonia, the vernal equinox lay in the constellation Aries. Indeed, the
vernal equinox is still sometimes called the {\em first point of Aries}\/ in astronomical texts. About 90 BC,
the vernal equinox moved into the constellation Pisces, where it still remains. The equinox will move
into the constellation Aquarius (marking the beginning of the much heralded ``Age of Aquarius'') in about 2600 AD. Incidentally, the position of the vernal equinox in the
sky is of great significance in astronomy, since it is used as the zero of celestial longitude (much as
Greenwich is used as the zero of terrestrial longitude). 

Equations~(\ref{e13xx11}) and (\ref{e13xx12}) indicate that the small inclination of the lunar orbit to the ecliptic
plane, combined with the precession  of the lunar ascending node, causes the Earth's axis of rotation to wobble
sightly. This wobble is known as {\em nutation}, and is superimposed on the aforementioned precession.  In the absence of
precession, nutation would
cause the north celestial pole to periodically trace out a small ellipse on the sky, the sense of rotation being
{\em counter-clockwise}. The
nutation period  is 18.6 years: {\em i.e.}, the same as the precession period of the lunar ascending node. 
The nutation amplitudes in the polar and azimuthal angles $\theta$ and $\phi$ are
\begin{eqnarray}
\delta\theta &= &\frac{3}{2}\,\frac{\epsilon\,\iota_m\,\mu_m\,T_n({\rm yr})}{T_s({\rm day})\,[T_m({\rm yr})]^2}\,\cos\theta_0,\\[0.4ex]
\delta\phi &=&\frac{3}{2}\,\frac{\epsilon\,\iota_m\,\mu_m\,T_n({\rm yr})}{T_s({\rm day})\,[T_m({\rm yr})]^2}\,\frac{\cos(2\,\theta_0)}{\sin\theta_0},
\end{eqnarray}
respectively, where $T_n({\rm yr}) = n_s/\Omega_n = 18.6$. Given
that $\epsilon=0.00335$, $\theta_0=23.44^\circ$, $\iota_m=5.16^\circ$, $T_s({\rm day})=365.24$, $T_m({\rm yr})=0.081$, and $\mu_m=0.0123$, we
obtain
\begin{eqnarray}
\delta\theta &=& 8.2^{''},\\[0.5ex]
\delta\phi &=& 15.3^{''}.
\end{eqnarray}
The observed nutation amplitudes are $9.2^{''}$ and $17.2^{''}$, respectively. Hence, our estimates are quite close to the
mark.  Any inaccuracy is mainly due to the fact that we have neglected to take into
account the small eccentricities of the Earth's orbit around the Sun, and the Moon's orbit around
the Earth. The nutation of the Earth was discovered in 1728 by the English astronomer James Bradley, and
was explained theoretically about 20 years later by d'Alembert and L.~Euler. Nutation is important because the corresponding gyration
of the Earth's rotation axis appears to be transferred to celestial objects when they are viewed using terrestrial 
telescopes. This effect causes the celestial longitudes and latitudes of heavenly objects to oscillate sinusoidally by up to $20^{''}$ ({\em i.e.}, about the maximum angular size of Saturn) with a period
of 18.6\,{\rm years}. It is necessary to correct for this oscillation in order to accurately guide terrestrial telescopes
to particular objects. 

Note, finally, that the type of forced nutation discussed above, which is driven by an external torque, is
quite distinct from the free nutation described  in Section~\ref{sgyro}.


\section{Potential Due to a Uniform Ring}
Consider a uniform ring of mass $M$ and radius $r$, centered on the origin, and lying in the $x$-$y$ plane.
Let us consider the gravitational potential $\Phi(r)$ generated by such a ring in the $x$-$y$ plane
(which corresponds to $\theta = 90^\circ$). It follows, from Section~\ref{saxial}, that for $r>a$,
\begin{equation}
\Phi(r) = - \frac{G\,M}{a}\sum_{n=0,\infty} [P_n(0)]^2\left(\frac{a}{r}\right)^{n+1}.
\end{equation}
However, $P_0(0)=1$, $P_1(0) = 0$, $P_2(0)=-1/2$, $P_3(0)=0$, and $P_4(0)=3/8$. Hence,
\begin{equation}\label{e13.117}
\Phi(r) = - \frac{G\,M}{r}\left[1 + \frac{1}{4}\left(\frac{a}{r}\right)^2 + \frac{9}{64}\left(\frac{a}{r}\right)^4+\cdots\right].
\end{equation}
Likewise, for $r<a$, 
\begin{equation}
\Phi(r) = - \frac{G\,M}{a}\sum_{n=0,\infty} [P_n(0)]^2\left(\frac{r}{a}\right)^{n}, 
\end{equation}
giving
\begin{equation}\label{e13.119}
\Phi(r) = - \frac{G\,M}{a}\left[1 + \frac{1}{4}\left(\frac{r}{a}\right)^2 + \frac{9}{64}\left(\frac{r}{a}\right)^4+\cdots\right].
\end{equation}

\section{Perihelion Precession of the Planets}\label{splanp}
The Solar System consists of eight major planets (Mercury to Neptune) moving around the Sun in slightly elliptical orbits which are approximately
co-planar with one another. According to Chapter~\ref{skepler}, if we neglect the relatively weak interplanetary
gravitational interactions then the perihelia of the various planets ({\em i.e.}, the points on their orbits at which they  are closest to the Sun)  remain {\em fixed}\/ in space. However, once these
interactions are taken into account, it turns out that the planetary perihelia all
slowly {\em precess}\/ around the Sun. We can  calculate the approximate rate of perihelion precession
of a given planet by treating the other planets as {\em uniform concentric rings}, centered on the Sun, of mass equal to the planetary mass, and
radius equal to the mean orbital radius.\footnote{M.G.~Stewart, American Jou.\ Physics {\bf 73}, 730 (2005).} This is equivalent to averaging the interplanetary gravitational interactions
over the orbits of the other planets. It is reasonable to do this, since the precession period in question is
very much longer than the orbital period of any planet in the Solar System. Thus, by treating the other planets as
rings, we can calculate the mean gravitational perturbation due to these planets, and, thereby, determine the
desired precession rate.

We can conveniently index the planets in the Solar System such that Mercury is planet 1, and Neptune planet 8. Let the $M_i$ and the $R_i$, for $i=1,8$, be the planetary masses and orbital radii, respectively. Furthermore, let $M_0$ be the mass of the Sun.
 It follows, from the previous section, that the gravitational potential generated at the $i$th planet by the Sun and
the other planets is
\begin{eqnarray}
\Phi(R_i) &=& -\frac{G\,M_0}{R_i}-G \sum_{j< i}\frac{M_j}{R_i}\left[1+\frac{1}{4}\left(\frac{R_j}{R_i}\right)^2 + \frac{9}{64}\left(\frac{R_j}{R_i}\right)^4+\cdots\right]\nonumber\\[0.5ex]&&
 -G \sum_{j> i}\frac{M_j}{R_j}\left[1+\frac{1}{4}\left(\frac{R_i}{R_j}\right)^2 + \frac{9}{64}\left(\frac{R_i}{R_j}\right)^4+\cdots\right].
\end{eqnarray}
Now, the radial force per unit mass acting on the $i$th planet is written
$f(R_i) = - d\Phi(R_i)/dr$, giving
\begin{eqnarray}
f(R_i) &=& -\frac{G\,M_0}{R_i^{\,2}}-\frac{G}{R_i^{\,2}} \sum_{j< i}M_j\left[1+\frac{3}{4}\left(\frac{R_j}{R_i}\right)^2
+ \frac{45}{64}\left(\frac{R_j}{R_i}\right)^4 +\cdots\right]\nonumber\\[0.5ex]&&
 +\frac{G}{R_i^{\,2}} \sum_{j> i}M_j\,\left(\frac{R_i}{R_j}\right)\!\left[\frac{1}{2}\left(\frac{R_i}{R_j}\right)^2+ \frac{9}{16}\left(\frac{R_i}{R_j}\right)^4+\cdots\right].
\end{eqnarray}
Hence, we obtain
\begin{eqnarray}
R_i\,f'(R_i) &=& \frac{2\,G\,M_0}{R_i^{\,2}}+\frac{G}{R_i^{\,2}} \sum_{j< i}M_j\left[2+3\left(\frac{R_j}{R_i}\right)^2
+ \frac{135}{32}\left(\frac{R_j}{R_i}\right)^4+\cdots\right]\nonumber\\[0.5ex]&&
 + \frac{G}{R_i^{\,2}}\,\sum_{j> i}M_j\,\left(\frac{R_i}{R_j}\right)\!\left[\frac{1}{2}\left(\frac{R_i}{R_j}\right)^2+ \frac{27}{16}\left(\frac{R_i}{R_j}\right)^4+\cdots\right],
 \end{eqnarray}
where $'\equiv d/dr$. It follows that
\begin{eqnarray}
&&\left[3 + \frac{R_i\,f'(R_i)}{f(R_i)}\right]^{-1/2} =1 + \frac{3}{4}\sum_{j<i}\left(\frac{M_j}{M_0}\right)
\left(\frac{R_j}{R_i}\right)^2\left[1 + \frac{15}{8}\left(\frac{R_j}{R_i}\right)^2 + \frac{175}{64}\left(\frac{R_j}{R_i}
\right)^4+\cdots\right]\nonumber\\[0.5ex]&&+ \frac{3}{4}\sum_{j>i}\left(\frac{M_j}{M_0}\right)
\left(\frac{R_i}{R_j}\right)^3\left[1 + \frac{15}{8}\left(\frac{R_i}{R_j}\right)^2 + \frac{175}{64}\left(\frac{R_i}{R_j}
\right)^4+\cdots\right].
\end{eqnarray}
Thus, according to Equation~(\ref{e6.81}), the apsidal angle for the $i$th planet is 
\begin{eqnarray}
\psi_i &=& \pi\left\{1 + \frac{3}{4}\sum_{j<i}\left(\frac{M_j}{M_0}\right)
\left(\frac{R_j}{R_i}\right)^2\left[1 + \frac{15}{8}\left(\frac{R_j}{R_i}\right)^2 + \frac{175}{64}\left(\frac{R_j}{R_i}
\right)^4+\cdots\right]\right.\nonumber\\[0.5ex]
&&\left.+ \frac{3}{4}\sum_{j>i}\left(\frac{M_j}{M_0}\right)
\left(\frac{R_i}{R_j}\right)^3\left[1 + \frac{15}{8}\left(\frac{R_i}{R_j}\right)^2 + \frac{175}{64}\left(\frac{R_i}{R_j}
\right)^4+\cdots\right]\right\}.
\end{eqnarray}
Hence, the perihelion of the $i$th planet advances by 
\begin{eqnarray}
\delta\psi_i &=& \frac{3\pi}{2}\sum_{j<i}\left(\frac{M_j}{M_0}\right)
\left(\frac{R_j}{R_i}\right)^2\left[1 + \frac{15}{8}\left(\frac{R_j}{R_i}\right)^2 + \frac{175}{64}\left(\frac{R_j}{R_i}
\right)^4+\cdots\right]\nonumber\\[0.5ex]
&&+ \frac{3\pi}{2}\sum_{j>i}\left(\frac{M_j}{M_0}\right)
\left(\frac{R_i}{R_j}\right)^3\left[1 + \frac{15}{8}\left(\frac{R_i}{R_j}\right)^2 + \frac{175}{64}\left(\frac{R_i}{R_j}
\right)^4+\cdots\right]
\end{eqnarray}
radians per revolution around the Sun. Now, the time for one revolution is $T_i = 2\pi/\omega_i$, where $\omega_i^{\,2} = G\,M_0/R_i^{\,3}$. Thus, the rate of perihelion precession, in {\em arc seconds per year}, is given by
\begin{eqnarray}
\delta\dot{\psi}_i &=& \frac{75}{T_i({\rm yr})}\left\{\sum_{j<i}\left(\frac{M_j}{M_0}\right)
\left(\frac{R_j}{R_i}\right)^2\left[1 + \frac{15}{8}\left(\frac{R_j}{R_i}\right)^2 + \frac{175}{64}\left(\frac{R_j}{R_i}
\right)^4+\cdots\right]\right.\nonumber\\[0.5ex]
&&\left.+ \sum_{j>i}\left(\frac{M_j}{M_0}\right)
\left(\frac{R_i}{R_j}\right)^3\left[1 + \frac{15}{8}\left(\frac{R_i}{R_j}\right)^2 + \frac{175}{64}\left(\frac{R_i}{R_j}
\right)^4+\cdots\right]\right\}.\label{e13.126}
\end{eqnarray}

\begin{table}
\centering
\begin{tabular}{llll}\hline 
Planet  & $M/M_0$              & $T({\rm yr})$ & R({\rm au})\\[0.25ex]\hline
&&&\\[-2ex]
Mercury & $1.66\times 10^{-7}$ & $0.241$       & $0.387$\\[0.5ex]
Venus   & $2.45\times 10^{-6}$ & $0.615$       & $0.723$\\[0.5ex]
Earth   & $3.04\times 10^{-6}$ & $1.000$       & $1.00$\\[0.5ex]
Mars    & $3.23\times 10^{-7}$ & $1.881$       & $1.52$\\[0.5ex]
Jupiter & $9.55\times 10^{-4}$ & $11.86$       & $5.20$\\[0.5ex]
Saturn  & $2.86\times 10^{-4}$ & $29.46$       & $9.54$\\[0.5ex]
Uranus  & $4.36\times 10^{-5}$ & $84.01$       & $19.19$\\[0.5ex]
Neptune & $5.18\times 10^{-5}$ & $164.8$       & $30.07$\\[0.5ex]
\end{tabular}
\caption{\em Data for the major planets in the Solar System, giving the planetary mass relative to that of the Sun, the orbital period in years, and the mean orbital
radius relative to that of the Earth.}\label{t1a}
\end{table}

\begin{table}
\centering
\begin{tabular}{lll}\hline&&\\[-2ex]
Planet  & $(\delta\dot{\Psi})_{obs}$              & $(\delta\dot{\Psi})_{th}$ \\[0.5ex]\hline
Mercury & $5.75$    & $5.50$      \\[0.5ex]
Venus     & $2.04$    & $10.75$    \\[0.5ex]
Earth      & $11.45$   & $11.87$    \\[0.5ex]
Mars       & $16.28$  & $17.60$     \\[0.5ex]
Jupiter    & $6.55$    & $7.42$       \\[0.5ex]
Saturn     & $19.50$  & $18.36$     \\[0.5ex]
Uranus    & $3.34$    & $2.72$       \\[0.5ex]
Neptune  & $0.36$    & $0.65$       \\[0.5ex]
\end{tabular}
\caption{\em The observed perihelion precession rates of the planets compared with the theoretical precession rates calculated from Equation~(\ref{e13.126})
and Table~\ref{t1a}. The precession rates are  in arc seconds per year.}\label{t2a}
\end{table}

\begin{figure}
\epsfysize=3.in
\centerline{\epsffile{Chapter12/fig12.08.eps}}
\caption{\em The triangular points show the observed perihelion precession rates of the
major planets in the Solar System, whereas the square points show the theoretical
rates calculated from Equation~(\ref{e13.126})
and Table~\ref{t1a}. The precession rates are in arc seconds per
year.}\label{fprec}
\end{figure}

Table~\ref{t2a} and Figure~\ref{fprec} compare the observed perihelion precession rates 
with the theoretical rates calculated from Equation~(\ref{e13.126})
and the planetary data given in Table~\ref{t1a}. It can be seen that there is excellent agreement
between the two, except for the planet Venus. The main reason for this is that Venus
has an unusually low eccentricity ($e=0.0068$), which renders its perihelion point extremely sensitive to small perturbations.

\section{Perihelion Precession of Mercury}\label{smerc}
If the calculation described in the previous section is carried out more accurately, taking into account the slight
eccentricities of the planetary orbits, as well as their small mutual inclinations, and retaining
many more terms in the expansions (\ref{e13.117}) and (\ref{e13.119}), then the perihelion
precession rate of the planet Mercury is found to be $5.32$ arc seconds per year. However, the
observed precession rate is $5.75$ arc seconds per year. It turns out that the
cause of this discrepancy is the general relativistic correction to Newtonian gravity.

General relativity gives rise to a small correction to the force per unit mass
exerted by the Sun (mass $M_0$) on a planet in a circular orbit of radius $r$, and angular momentum
per unit mass $h$. In fact, the modified formula for $f$ is
\begin{equation}
f \simeq -\frac{G\,M_0}{r^2}-\frac{3\,G\,M_0\,h^2}{c^2\,r^4},
\end{equation}
where $c$ is the velocity of light in vacuum. It follows that
\begin{equation}
\frac{r\,f'}{f} = - 2\left(1+ \frac{3\,h^2}{c^2\,r^2}+ \cdots\right).
\end{equation}
Hence, from Equation~(\ref{e6.81}), the apsidal angle is
\begin{equation}
\psi \simeq \pi\left(1+ \frac{3\,h^2}{c^2\,r^2}\right).
\end{equation}
Thus, the perihelion advances by
\begin{equation}
\delta\psi \simeq \frac{6\pi\,G\,M_0}{c^2 \,r}
\end{equation}
radians per revolution due to the general relativistic correction to Newtonian gravity. Here,
use has been made of $h^2=G\,M_0\,r$.  It follows that the
rate of perihelion precession due to the general relativistic correction is
\begin{equation}
\delta\dot{\psi} \simeq \frac{0.0383}{R\,T}
\end{equation}
arc seconds per year, where $R$ is the mean orbital radius in mean Earth orbital radii,
and $T$ is the orbital period in years. Hence, from Table~\ref{t1a}, the general relativistic contribution to
$\delta\dot{\psi}$ for Mercury is
$0.41$ arc seconds per year. It is easily demonstrated that the corresponding contribution is negligible for
the other planets in the Solar System. If the above calculation is carried out sightly more accurately,
taking the eccentricity of Mercury's orbit into account, then the general relativistic contribution
to $\delta\dot{\psi}$ becomes
$0.43$ arc seconds per year. It follows that the total perihelion precession rate for Mercury
is $5.32+0.43= 5.75$ arc seconds per year. This is in exact agreement with the observed precession
rate. Indeed, the ability of general relativity to explain  the discrepancy between the observed perihelion precession
rate of Mercury, and that calculated from Newtonian dynamics, was one of the first major successes of
this theory.

\section{Exercises}
{\small 
\renewcommand{\theenumi}{12.\arabic{enumi}}
\begin{enumerate}
\item Show that 
$$
\epsilon = \frac{5-\alpha}{4}\,\frac{\Omega^2\,a^3}{G\,M}.
$$
for a self-gravitating, rotating spheroid of ellipticity $\epsilon\ll 1$, mass
$M$, mean radius $a$, and angular velocity $\Omega$ whose mass
density varies as $r^{-\alpha}$ (where $\alpha<3$). Demonstrate that the above formula  matches
the observed rotational flattening of the Earth when $\alpha=1.09$ and
of Jupiter when $\alpha=1.79$. 

\item The Moon's orbital period about the Earth is approximately 27.3 days,
and is in the same direction as the Earth's axial rotation (whose period is
24 hours). Use this data to show that high tides at a given point on the Earth
occur every 12 hours and 26 minutes.

\item Estimate the tidal elongation of the Moon due to the Earth.

\item Consider an artificial satellite in a circular orbit of radius $L$
about the Earth. Suppose that the normal to the plane of the orbit subtends
an angle $\theta$ with the Earth's axis of rotation. By approximating the
orbiting satellite as a uniform ring, demonstrate that the Earth's oblateness causes
the plane of the satellite's orbit to precess about the Earth's rotational
axis  at the rate
$$
\frac{\dot\phi}{\omega} \simeq - \frac{1}{2}\,\epsilon\left(\frac{R}{L}\right)^2\,\cos\theta.
$$
Here, $\omega$ is the satellite's orbital angular velocity, $\epsilon = 0.00335$
 the Earth's ellipticity, and $R$  the Earth's radius. Note that the Earth's
axial moment of inertial is $I_\parallel \simeq (1/3)\,M\,R^2$, where
$M$ is the mass of the Earth.

\item A {\em sun-synchronous}\/ satellite is one which always passes
over a given point on the Earth at the same local solar time. This is
achieved by fixing the precession rate of the satellite's orbital 
plane such that it matches the rate at which the Sun appears to move
against the background of the stars. What orbital altitude above
the surface of the Earth would such a satellite need to have in order
to fly over all latitudes between $50^\circ$ N and $50^\circ$ S? Is the
direction of the satellite orbit in the same sense as the Earth's rotation (prograde),
or the opposite sense (retrograde)?
\end{enumerate}
}
