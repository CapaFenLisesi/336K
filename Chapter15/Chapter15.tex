\chapter{The Chaotic Pendulum}\label{schaos}
\section{Introduction}
Up until now, we have only dealt with dynamical problems which are capable of analytic solution. Let us now investigate a problem
which is quite intractable analytically, and  in which meaningful progress can only be
made via numerical methods.

\section{Basic Problem}
Consider a simple pendulum consisting of a point mass $m$, at the end of a light
rigid rod of length $l$, attached to a fixed frictionless pivot
which allows the rod (and the mass) to move freely under gravity in the vertical plane. Such a
pendulum is sketched in Figure~\ref{cf1}. Let us parameterize the instantaneous position
of the pendulum via the angle $\theta$ the rod makes with the downward vertical. It is
assumed that the pendulum is free to swing through a full circle. Hence, $\theta$ and $\theta +2\pi$
both correspond to the same pendulum position.

The angular equation of motion of the pendulum is simply
\begin{equation}
m\,l\,\frac{d^2\theta}{dt^2} + m\,g\,\sin\theta = 0,
\end{equation}
where $g$ is the downward acceleration due to gravity---see Section~\ref{s4.8}. 
Suppose that the pendulum is embedded in  a viscous medium ({\em e.g.}, air). 
Let us assume that the viscous drag torque acting on the pendulum is
directly proportional to the pendulum's instantaneous velocity.
 It follows that, in the presence of viscous drag, the above
equation generalizes to
\begin{equation}
m\,l\,\frac{d^2\theta}{dt^2}+ \nu\,\frac{d\theta}{dt} + m\,g\,\sin\theta = 0,
\end{equation}
where $\nu$ is a positive constant parameterizing the viscosity of the medium
in question---see Section~\ref{s4.4}. Of course, viscous
damping will eventually drain all energy from the pendulum, leaving it in a stationary state. 
In order to maintain the motion against viscosity, it is necessary to add some external driving.
For the sake of simplicity, we choose a fixed amplitude periodic drive (which could arise, for
instance, via periodic oscillations of the pendulum's pivot point). Thus, the final equation
of motion of the pendulum is written
\begin{equation}\label{e16.3}
m\,l\,\frac{d^2\theta}{dt^2}+ \nu\,\frac{d\theta}{dt} + m\,g\,\sin\theta =  A\,\cos(\omega\, t),
\end{equation}
where $A$ and $\omega$ are  constants parameterizing the amplitude
and  frequency  of the external driving torque, respectively.

\begin{figure}
\epsfysize=2.5in
\centerline{\epsffile{Chapter15/fig15.01.eps}}
\caption{\em A simple pendulum.}\label{cf1}
\end{figure}

Let
\begin{equation}
\omega_0 = \sqrt{\frac{g}{l}}.
\end{equation}
Of course, we recognize $\omega_0$ as the natural  frequency of small
amplitude oscillations of the pendulum. We can conveniently normalize the
pendulum's equation of motion by writing,
\begin{eqnarray}
\hat{t} &=& \omega_0\,t,\\[0.5ex]
\hat{\omega} &=& \frac{\omega}{\omega_0},\\[0.5ex]
Q &=& \frac{m\,g}{\omega_0\,\nu},\\[0.5ex]
\hat{A} &=& \frac{A}{m\,g},
\end{eqnarray}
in which case Equation~(\ref{e16.3}) becomes
\begin{equation}\label{e16.9}
\frac{d^2\theta}{d\hat{t}^2} + \frac{1}{Q}\,\frac{d\theta}{d\hat{t}}
+ \sin\theta = \hat{A}\,\cos(\hat{\omega}\,\hat{t}).
\end{equation}
From now on, the hats on normalized quantities will be omitted, for ease of
notation. Note that, in normalized units, the natural frequency of small amplitude
oscillations is {\em unity}. Moreover, $Q$ is the familiar {\em quality-factor}, which is, roughly speaking,
the number of oscillations of the undriven system which must
elapse before its energy is significantly
reduced via the action of viscosity---see Section~\ref{squal}. The quantity $A$ is the amplitude of the external
torque measured in units of the maximum possible gravitational torque. Finally,
$\omega$ is the oscillation frequency of the external torque measured in units of the
pendulum's natural frequency.

Equation~(\ref{e16.9}) is clearly a second-order ordinary differential equation. It can, therefore, also be written as
two coupled first-order ordinary differential equations:
\begin{eqnarray}
\frac{d\theta}{dt} &=& v,\label{e16.11}\\[0.5ex]
\frac{dv}{dt} &=& -\frac{v}{Q} -\sin\theta + A\,\cos(\omega\, t).\label{e16.12}
\end{eqnarray}

\section{Analytic Solution}\label{s5.2}
Before attempting a numerical solution of the equations of motion of any dynamical system, it is a good idea to, first, investigate the equations as thoroughly as possible via standard analytic techniques. 
Unfortunately, Equations~(\ref{e16.11}) and (\ref{e16.12}) constitute a {\em non-linear}\/ dynamical
system---because of the presence of the $\sin\theta$ term on the right-hand side of
Equation~(\ref{e16.12}). This system, like most non-linear systems, does not possess a simple
analytic solution. Fortunately, however, if we restrict our attention to {\em small amplitude}\/ oscillations,
such that the approximation
\begin{equation}
\sin\theta \simeq \theta
\end{equation}
is valid,
then the system becomes {\em linear}, and can easily be solved analytically.

The linearized equations of motion of the pendulum take the form:
\begin{eqnarray}
\frac{d\theta}{dt} &=& v,\label{e16.13}\\[0.5ex]
\frac{dv}{dt} &=& -\frac{v}{Q} -\theta + A\,\cos(\omega \,t).\label{e16.14}
\end{eqnarray}
 Suppose
that the pendulum's position, $\theta(0)$, and velocity, $v(0)$,  are specified at time $t=0$. 
As is well-known, in this case, the above equations of motion can be solved analytically to give:
\begin{eqnarray}
\theta(t) &=&\left\{\theta(0) - \frac{A\,(1-\omega^2)}{[(1-\omega^2)^2+\omega^2/Q^2]}\right\}
{\rm e}^{-t/2Q}\,\cos(\omega_\ast\, t)\nonumber\\[0.5ex]
&&+\frac{1}{\omega_\ast}\left\{ v(0) + \frac{\theta(0)}{2Q}
-\frac{A\,(1-3\,\omega^2)/2Q}
{[(1-\omega^2)^2+ \omega^2/Q^2]}\right\}{\rm e}^{-t/2Q}\,\sin(\omega_\ast\,t) \nonumber\\[0.5ex]
 &&+\frac{A\left[(1-\omega^2)\,\cos(\omega\, t)+ (\omega/Q)\,\sin(\omega\, t)\right]}
{\left[(1-\omega^2)^2+\omega^2/Q^2\right]},\label{e16.15}\\[0.5ex]
v(t) &=& \left\{v(0) - \frac{A\,\omega^2/Q}{[(1-\omega^2)^2+\omega^2/Q^2]}\right\}
{\rm e}^{-t/2Q}\,\cos\omega_\ast t\nonumber\\[0.5ex]
&&-\frac{1}{\omega_\ast}\left\{ \theta(0) + \frac{v(0)}{2Q}
-\frac{A\,[(1-\omega^2)-\omega^2/2Q^2]}
{[(1-\omega^2)^2+ \omega^2/Q^2]}\right\}{\rm e}^{-t/2Q}\,\sin(\omega_\ast\, t)\nonumber\\[0.5ex]
&&+\frac{\omega\,A\left[-(1-\omega^2)\,\sin(\omega\, t)+ (\omega/Q)\,\cos(\omega\, t)\right]}
{\left[(1-\omega^2)^2+\omega^2/Q^2\right]}.\label{e16.16}
\end{eqnarray}
Here, 
\begin{equation}
\omega_\ast = \sqrt{1-\frac{1}{4Q^2}},
\end{equation}
and it is assumed that $Q>1/2$. 
It can be seen that the above expressions for $\theta$ and $v$  both consist  of {\em three}\/
terms. The first two terms clearly represent {\em transients},  since they depend on the initial
conditions, and {\em decay}\/ exponentially in time---see Section~\ref{strans}. In fact, the e-folding time for the decay of these
terms is $2\,Q$ (in normalized time units). 
The final term represents the {\em time-asymptotic motion}\/ of the pendulum, and is manifestly {\em independent}\/ of the initial conditions---see Section~\ref{s4.4}.

\begin{figure}
\epsfysize=2.5in
\centerline{\epsffile{Chapter15/fig15.02.eps}}
\caption{\em A phase-space plot of the periodic attractor for a linear  damped periodically
driven  pendulum. Data calculated analytically for $Q=4$ and $\omega=2$. }\label{cf2}
\end{figure}

It is often convenient to {\em visualize}\/ the motion of a dynamical system as an orbit, or  trajectory, in
{\em phase-space}, which is defined as the space of all of the dynamical variables required to
specify the instantaneous state of the system. For the case in hand, there are two dynamical
variables, $v$ and $\theta$, and so phase-space corresponds to the $\theta$-$v$ plane. Note that
each different point in this plane corresponds to a unique instantaneous state of the pendulum.
[Strictly speaking, we should also consider $t$ to be a dynamical variable, since it 
appears explicitly on the right-hand side of Equation~(\ref{e16.12}).]

It is clear, from Equations~(\ref{e16.15}) and (\ref{e16.16}), that if we wait long enough for all
of the transients to decay away then the motion of the pendulum settles down to the
following simple orbit in phase-space:
\begin{eqnarray}
\theta(t) &=&
\frac{A\left[(1-\omega^2)\,\cos(\omega\, t)+ (\omega/Q)\,\sin(\omega\, t)\right]}
{\left[(1-\omega^2)^2+\omega^2/Q^2\right]},\\[0.5ex]
v(t) &=& \frac{\omega\, A\left[-(1-\omega^2)\,\sin(\omega\, t)+ (\omega/Q)\,\cos(\omega\, t)\right]}
{\left[(1-\omega^2)^2+\omega^2/Q^2\right]}.
\end{eqnarray}
This orbit traces out the closed curve
\begin{equation}
\left(\frac{\theta}{\tilde{A}}\right)^2 + \left(\frac{v}{\omega\,\tilde{A}}\right)^2 =1,
\end{equation}
in phase-space, where 
\begin{equation}\label{e16.18}
\tilde{A} = \frac{A}{\sqrt{(1-\omega^2)^2+\omega^2/Q^2}}.
\end{equation}
As illustrated in Figure~\ref{cf2}, this curve is an {\em ellipse}\/ whose principal axes are 
aligned with the $v$ and $\theta$ coordinate axes.
Observe that the curve is {\em closed}, which suggests that the associated motion is {\em periodic}\/ in time. In fact, the
motion repeats itself exactly every
\begin{equation}
\tau = \frac{2\pi}{\omega}
\end{equation}
normalized time units. The maximum angular displacement of the pendulum from its undriven rest position
($\theta=0$) is $\tilde{A}$. 
As illustrated in Figure~\ref{cf3}, the variation of $\tilde{A}$ with driving frequency $\omega$
[see Equation~(\ref{e16.18})] displays all of the features of a classic resonance curve. The maximum
amplitude of the driven oscillation is proportional to the quality-factor, $Q$, and
is achieved when the driving frequency matches the natural
frequency of the pendulum ({\em i.e.}, when $|\omega|=1$). Moreover, 
the width of the resonance in $\omega$-space
is proportional to $1/Q$---see Section~\ref{eres}.

\begin{figure}
\epsfysize=2.5in
\centerline{\epsffile{Chapter15/fig15.03.eps}}
\caption{\em The maximum angular displacement of a linear damped periodically driven 
pendulum as a function of driving frequency. The solid curve corresponds to $Q=1$. The
short-dashed curve corresponds to $Q=5$. The long-dashed curve corresponds to $Q=10$.
Analytic data.}\label{cf3}
\end{figure}

The phase-space curve shown in Figure~\ref{cf2} is called a {\em periodic attractor}. It is 
termed an ``attractor'' because,
irrespective of the initial conditions, the trajectory of the system in phase-space tends
asymptotically to---in other words, is attracted to---this curve as $t\rightarrow\infty$. This
gravitation of phase-space trajectories towards the attractor is illustrated in Figures~\ref{cf4} and
\ref{cf5}. Of course, the attractor is termed ``periodic'' because it corresponds to motion which is
periodic in time.

\begin{figure}
\epsfysize=2.5in
\centerline{\epsffile{Chapter15/fig15.04.eps}}
\caption{\em The phase-space trajectory of a linear damped  periodically driven
pendulum. Data calculated analytically for $Q=1$ and $\omega=2$. Here, $v(0)/A=0$ and $\theta(0)/A=0$.
}\label{cf4}
\end{figure}
\begin{figure}
\epsfysize=2.5in
\centerline{\epsffile{Chapter15/fig15.05.eps}}
\caption{\em The phase-space trajectory of a linear damped periodically driven
pendulum. Data calculated analytically for $Q=1$ and $\omega=2$.
Here, $v(0)/A=0.5$ and $\theta(0)/A=0.5$. }\label{cf5}
\end{figure}

Let us summarize our findings so far. We have discovered that if a damped pendulum is
subject to a low amplitude periodic drive then its {\em time-asymptotic}\/ response ({\em i.e.},
its response after any transients have died away) is {\em periodic}, with the same period
as the driving torque. Moreover, the response exhibits {\em resonant}\/ behaviour as the driving
frequency approaches the natural frequency of oscillation of the pendulum. The amplitude
of the resonant response, as well as the width of the resonant window, is governed by the
amount of damping in the system. After a little reflection, we can easily appreciate that
all of these results are a direct consequence of the {\em linearity}\/ of the pendulum's
equations of  motion in the low amplitude limit. In fact, 
it is easily demonstrated that the time-asymptotic response of {\em any} intrinsically
stable linear system (with a discrete spectrum of normal modes) to a periodic drive is periodic,
with the same period as the drive. Moreover, if the driving frequency approaches
one of the natural frequencies of oscillation of the system then the response exhibits
resonant behaviour. But, is this the only allowable time-asymptotic response of a dynamical system to a
periodic drive? It turns out that it is not.
Indeed, the response of
a {\em non-linear}\/ system to a periodic drive is generally much richer  and
far more diverse than simple periodic motion. Since the majority of naturally
occurring dynamical systems are non-linear, it is clearly important that we gain a
basic understanding of this phenomenon. Unfortunately, we cannot achieve this goal via
a standard analytic approach, since non-linear equations of motion generally do not possess
simple analytic solutions. Instead, we must employ {\em numerical methods}. As an example, let us
investigate the dynamics of a damped pendulum, subject to a periodic drive, with {\em no
restrictions}\/ on the amplitude of the pendulum's motion.

\section{Numerical Solution}
In the following, we present numerical solutions of Equations~(\ref{e16.11}) and (\ref{e16.12})
obtained using a fairly standard fixed step-length fourth-order
Runge-Kutta integration scheme.\footnote{W.H.~Press, S.A.~Teukolsky,
W.T.~Vetterling, and B.P.~Flannery, {\em Numerical recipes in C: The
art of scientific computing}, 2nd Edition (Cambridge University Press,
Cambridge UK, 1992), Section~16.1.}

\section{Poincar\'{e} Section}
For the sake of definiteness, let us fix the normalized amplitude and frequency of the external
drive to be $A=1.5$ and $\omega=2/3$, respectively.\footnote{G.L.~Baker, {\em Control of
the chaotic driven pendulum},  Am.\ J.\ Phys.\
{\bf 63}, 832 (1995).}
 Furthermore, let us investigate any changes which
may develop in the nature of the
pendulum's time-asymptotic motion
 as the quality-factor $Q$ is varied. Of course, if $Q$ is made sufficiently small 
({\em i.e.}, if the pendulum is embedded in a sufficiently viscous medium) then we expect the
amplitude of the pendulum's time-asymptotic motion to become low enough that the linear analysis
outlined in Section~\ref{s5.2} is valid.  Indeed, we expect non-linear effects to manifest themselves
as $Q$ is gradually made larger, and the amplitude of the pendulum's motion 
consequently increases to such an extent that the
small angle approximation breaks down.


\begin{figure}
\epsfysize=2.5in
\centerline{\epsffile{Chapter15/fig15.06.eps}}
\caption{\em Equally spaced (in time) points on a time-asymptotic orbit
in phase-space. Data calculated numerically
 for $Q=0.5$, $A=1.5$, $\omega=2/3$, $\theta(0)=0$, and
$v(0)=0$. }\label{f18}
\end{figure}

Figure~\ref{f18} shows a time-asymptotic orbit in phase-space 
calculated numerically for a case where $Q$ is sufficiently
small ({\em i.e.}, $Q=1/2$) that the small angle approximation holds reasonably well. Not surprisingly,
the orbit is very similar to the analytic orbits
 described in Section~\ref{s5.2}. The fact that the orbit consists
of a {\em single}\/ loop, and forms a {\em closed}\/ curve in phase-space,
 strongly suggests that the corresponding
motion is periodic with the same period as the external drive---we term this type of motion
{\em period-1}\/ motion. More generally, period-$n$ motion consists of motion which
repeats itself exactly every $n$ periods of the external drive (and, obviously,
does not repeat itself on any time-scale less than $n$ periods). Of course, period-1 motion
is the only allowed time-asymptotic motion in the small angle limit. 

It would certainly be helpful
to possess a graphical test for period-$n$ motion. In fact, such a test was developed more than a
 hundred years ago by the
French mathematician Henry Poincar\'{e}. Nowadays, it is called a
{\em Poincar\'{e} section}\/ in his honour. The idea of a  Poincar\'{e} section, as applied
to a periodically driven pendulum,  is very simple.
As before, we calculate the time-asymptotic motion of the pendulum, and visualize it as a
series of points in $\theta$-$v$
phase-space. However, we only plot {\em one point per period}\/ of the external drive. To be more
exact, we only plot a point when
\begin{equation}
\omega\,t = \phi + k\,2\pi
\end{equation}
where $k$ is any integer, and $\phi$ is referred to as the {\em Poincar\'{e} phase}. 
For period-1 motion, in which the motion repeats itself exactly every period of the
external drive, we expect the Poincar\'{e} section to consist of only {\em one} point
in phase-space ({\em i.e.},  we expect all of the points to plot on top of one another). 
Likewise, for period-2 motion, in which the motion repeats itself exactly every two periods of the
external drive, we expect the Poincar\'{e} section to consist of  {\em two}\/ points
in phase-space ({\em i.e.},  we expect alternating  points to plot on top of one another).
Finally, for period-$n$ motion we expect the Poincar\'{e} section to consist of  $n$ points
in phase-space.

\begin{figure}
\epsfysize=2.5in
\centerline{\epsffile{Chapter15/fig15.07.eps}}
\caption{\em The Poincar\'{e} section of a time-asymptotic
orbit. Data calculated numerically for $Q=0.5$, $A=1.5$, $\omega=2/3$, $\theta(0)=0$,
$v(0)=0$, and $\phi=0$. }\label{f19}
\end{figure}

Figure~\ref{f19} displays the Poincar\'{e} section of the orbit shown in Figure~\ref{f18}. 
The fact that the section consists of a single point confirms that the motion
displayed in Figure~\ref{f18} is indeed period-1 motion.


\section{Spatial Symmetry Breaking}
\begin{figure}
\epsfysize=2.5in
\centerline{\epsffile{Chapter15/fig15.08.eps}}
\caption{\em The $v$-coordinate of the Poincar\'{e} section of a time-asymptotic orbit
plotted against the quality-factor $Q$. Data
 calculated numerically for
$A=1.5$, $\omega=2/3$, $\theta(0)=0$, $v(0)=0$, and $\phi=0$. }\label{f20}
\end{figure}
Suppose that we now gradually increase the quality-factor $Q$. What happens to the
simple orbit shown in Figure~\ref{f18}? It turns out that, at first,  nothing particularly exciting
happens. The size of the orbit gradually increases, indicating a corresponding increase in
the amplitude of the pendulum's motion, but the general nature of the motion remains unchanged.
However, something interesting does occur when $Q$ is increased beyond about $1.2$. 
Figure~\ref{f20} shows the $v$-coordinate of the orbit's Poincar\'{e} section  plotted
against  $Q$ in the range $1.2$ and $1.3$. Note the sharp downturn
in the curve at $Q\simeq 1.245$. What does this signify? Well, Figure~\ref{f21} shows
 the time-asymptotic phase-space orbit just before the downturn ({\em i.e.}, 
at $Q=1.24$), and Figure~\ref{f22} shows the orbit somewhat after the downturn
({\em i.e.}, at $Q=1.30$). It is clear that the downturn is associated with a
sudden change in the nature of the pendulum's time-asymptotic phase-space orbit. Prior to the downturn, the orbit
spends as much time in the region $\theta<0$ as in the region $\theta>0$. However,
after the downturn the orbit spends the majority of its time in the region $\theta<0$. 
In other words, after the downturn, the pendulum bob favours the region to the {\em left}\/ of
the pendulum's vertical. This is somewhat surprising, since there is nothing in the
pendulum's equations of motion which differentiates between the regions to the left
and to the right of the vertical. We refer to a solution of this type---{\em i.e.}, one which fails
to realize the full symmetry of the dynamical system in question---as a {\em symmetry
breaking}\/ solution. In this case, because the particular symmetry which is broken is
a {\em spatial}\/ symmetry, we refer to the process by which the symmetry breaking solution
suddenly appears, as the control parameter $Q$ is adjusted, as {\em spatial symmetry breaking}.
Needless to say, spatial symmetry breaking is an intrinsically {\em non-linear}\/ process---it
cannot take place in dynamical systems possessing linear equations of motion.

\begin{figure}
\epsfysize=2.5in
\centerline{\epsffile{Chapter15/fig15.09.eps}}
\caption{\em Equally spaced (in time) points on a time-asymptotic orbit in phase-space.
Data calculated numerically for $Q=1.24$, $A=1.5$, $\omega=2/3$, $\theta(0)=0$, and
$v(0)=0$. }\label{f21}
\end{figure}

\begin{figure}
\epsfysize=2.5in
\centerline{\epsffile{Chapter15/fig15.10.eps}}
\caption{\em Equally spaced (in time) points on a time-asymptotic orbit in phase-space.
Data calculated  
numerically  for $Q=1.30$, $A=1.5$, $\omega=2/3$, $\theta(0)=0$, and
$v(0)=0$. }\label{f22}
\end{figure}

It stands to reason that since the pendulum's equations of motion favour neither the
left nor the right then the left-favouring orbit pictured in Figure~\ref{f22} must be accompanied
by a mirror image right-favouring orbit. How do we obtain this mirror image orbit?
It turns out that all we have to do is keep the physical  parameters $Q$, $A$, and $\omega$
fixed, but {\em change the initial conditions} $\theta(0)$ and $v(0)$. Figure~\ref{f23} shows
a time-asymptotic phase-space orbit calculated with the same physical parameters  used in Figure~\ref{f22}, but
with the initial conditions $\theta(0)=0$ and $v(0)=-3$, instead of $\theta(0)=0$ and
$v(0)=0$. It can be seen that the orbit is indeed the mirror image of that pictured in Figure~\ref{f22}.

\begin{figure}
\epsfysize=2.5in
\centerline{\epsffile{Chapter15/fig15.11.eps}}
\caption{\em Equally spaced (in time) points on a time-asymptotic orbit in phase-space. Data
 calculated numerically for $Q=1.30$, $A=1.5$, $\omega=2/3$, $\theta(0)=0$, and
$v(0)=-3$. }\label{f23}
\end{figure}

Figure~\ref{fx24} shows the $v$-coordinate of the Poincar\'{e} section of a time-asymptotic orbit,
calculated with the same physical parameters used in Figure~\ref{f20}, versus
  $Q$ in the range $1.2$ and $1.3$. Data is shown for the two sets of
initial conditions discussed above. The figure is interpreted as
follows. When $Q$ is less than a critical value,
which is about $1.245$, then the two sets of initial conditions lead to motions
which converge on the {\em same}\/ left-right symmetric 
period-1 attractor. However, once
$Q$ exceeds the critical value then the attractor {\em bifurcates}\/ into two 
asymmetric mirror image  period-1
attractors.  Obviously, the bifurcation is indicated by the forking of the
curve shown in Figure~\ref{fx24}. The lower and upper branches correspond to the left- and right-favouring
 attractors, respectively.

\begin{figure}
\epsfysize=2.5in
\centerline{\epsffile{Chapter15/fig15.12.eps}}
\caption{\em The $v$-coordinate of the Poincar\'{e} section of a time-asymptotic orbit
plotted against the quality-factor $Q$. Data
 calculated numerically for
$A=1.5$, and $\omega=2/3$. Data is shown for two sets of initial
conditions: $\theta(0)=0$ and $v(0)=0$ (lower branch); and $\theta(0)=0$ and $v(0)=-3$ (upper branch).}\label{fx24}
\end{figure}

Spontaneous symmetry breaking, which is the fundamental non-linear process illustrated in the
above discussion,  plays an
important role in many areas of physics. For instance, symmetry breaking gives mass
to elementary particles in the unified theory of electromagnetic and weak interactions.\footnote{E.S.~Albers
and B.W.~Lee, Phys.\ Rep.\ 9{\bf C}, 1 (1973).}
Symmetry breaking also plays a pivotal role in the so-called ``inflation'' theory of the expansion
of the early universe.\footnote{P.~Coles, and F.~Lucchin, {\em Cosmology: The origin and evolution
of cosmic structure}, (J.~Wiley \& Sons, Chichester UK, 1995).}

\section{Basins of Attraction}
We have seen that when $Q=1.3$, $A=1.5$, and $\omega=2/3$ there are two co-existing
period-1 attractors in $\theta$--$v$ phase-space. The time-asymptotic trajectory of the pendulum 
through phase-space converges on one or other of these attractors depending on the initial conditions:
{\em i.e.}, depending on the values of $\theta(0)$ and $v(0)$. 
Let us define the {\em basin of attraction}\/ of a given attractor as the locus of all
points in the $\theta(0)$--$v(0)$ plane which lead to motion which ultimately converges on that attractor.
We have seen that in the low-amplitude ({\em i.e.}, linear) limit (see Section~\ref{s5.2}) there is only a
single period-1 attractor  in phase-space, and  all possible initial conditions lead to motion
which converges on this attractor. In other words,  the basin of attraction for the
low-amplitude attractor constitutes the entire $\theta(0)$--$v(0)$ plane. The present
case, in which there are {\em two} co-existing attractors in phase-space, is somewhat more complicated.

\begin{figure}
\epsfysize=3.5in
\centerline{\epsffile{Chapter15/fig15.13.eps}}
\caption{\em The basins of attraction for the asymmetric, mirror image, attractors
pictured in Figures~\ref{f22} and \ref{f23}. Regions of $\theta(0)$--$v(0)$
space which lead to motion converging on the left-favouring attractor shown
in Figure~\ref{f22} are coloured white: regions of $\theta(0)$--$v(0)$
space which lead to motion converging on the right-favouring attractor shown
in Figure~\ref{f23} are coloured black. Data calculated numerically
for $Q=1.3$, $A=1.5$, $\omega=2/3$, and $\phi=0$.
}\label{f25}
\end{figure}

Figure~\ref{f25} shows the basins of attraction, in $\theta(0)$--$v(0)$ space, 
of the asymmetric mirror image attractors
pictured in Figures~\ref{f22} and \ref{f23}. The basin of attraction
of  the left-favoring attractor shown in Figure~\ref{f22} is coloured black, whereas
the basin of attraction
of  the right-favoring attractor shown in Figure~\ref{f23} is coloured  white. It can
be seen that the two basins form a complicated interlocking pattern. Since we can
identify the angles $\pi$ and $-\pi$, the right-hand edge of the pattern connects
smoothly with its left-hand edge. In fact, we can think of the pattern as existing
on the surface of a {\em cylinder}.

\begin{figure}
\epsfysize=3.5in
\centerline{\epsffile{Chapter15/fig15.14.eps}}
\caption{\em  Detail of the basins of attraction for the asymmetric, mirror image, attractors
pictured in Figures~\ref{f22} and \ref{f23}. Regions of $\theta(0)$--$v(0)$
space which lead to motion converging on the left-favouring attractor shown
in Figure~\ref{f22} are coloured white: regions of $\theta(0)$--$v(0)$
space which lead to motion converging on the right-favouring attractor shown
in Figure~\ref{f23} are coloured black. Data calculated numerically
for $Q=1.3$, $A=1.5$, $\omega=2/3$, and $\phi=0$.
}\label{f26}
\end{figure}

Suppose that we take a diagonal from the bottom left-hand corner of Figure~\ref{f25}
to its top right-hand corner. This diagonal is intersected by a number of black
bands of varying thickness. Observe that the two narrowest bands ({\em i.e.}, the
fourth band from the bottom left-hand corner and the second band from the
upper right-hand corner) both exhibit structure which is not very well resolved in the
present picture. 

\begin{figure}
\epsfysize=3.5in
\centerline{\epsffile{Chapter15/fig15.15.eps}}
\caption{\em  Detail of the basins of attraction for the asymmetric, mirror image, attractors
pictured in Figures~\ref{f22} and \ref{f23}. Regions of $\theta(0)$--$v(0)$
space which lead to motion converging on the left-favouring attractor shown
in Figure~\ref{f22} are coloured white: regions of $\theta(0)$--$v(0)$
space which lead to motion converging on the right-favouring attractor shown
in Figure~\ref{f23} are coloured black. Data calculated numerically
for $Q=1.3$, $A=1.5$, $\omega=2/3$, and $\phi=0$.
}\label{f27}
\end{figure}

Figure~\ref{f26} is a blow-up of a region close to the lower left-hand corner of Figure~\ref{f25}. 
It can be seen that the unresolved band in the latter figure ({\em i.e.}, the second and
third bands from the right-hand side in the former figure) actually consists of a closely spaced 
{\em pair}\/ of bands. 
Note, however, that the narrower of these two bands exhibits structure which is not very well resolved in the
present picture. 

\begin{figure}
\epsfysize=3.5in
\centerline{\epsffile{Chapter15/fig15.16.eps}}
\caption{\em  Detail of the basins of attraction for the asymmetric, mirror image, attractors
pictured in Figures~\ref{f22} and \ref{f23}. Regions of $\theta(0)$--$v(0)$
space which lead to motion converging on the left-favouring attractor shown
in Figure~\ref{f22} are coloured white: regions of $\theta(0)$--$v(0)$
space which lead to motion converging on the right-favouring attractor shown
in Figure~\ref{f23} are coloured black. Data calculated numerically
for $Q=1.3$, $A=1.5$, $\omega=2/3$, and $\phi=0$.
}\label{f28}
\end{figure}

Figure~\ref{f27} is a blow-up of a region of Figure~\ref{f26}. 
It can be seen that the unresolved band in the latter figure ({\em i.e.}, the first and
second bands from the left-hand side in the former figure) actually consists of a closely spaced 
{\em pair} of bands. 
Note, however, that the broader of these two bands exhibits structure which is not very well resolved in the
present picture. 

Figure~\ref{f28} is a blow-up of a region of Figure~\ref{f27}. 
It can be seen that the unresolved band in the latter figure ({\em i.e.}, the first, second, and third
bands from the right-hand side in the former figure) actually consists of a closely spaced 
{\em triplet}\/ of bands. 
Note, however, that the narrowest of these  bands exhibits structure which is not very well resolved in the
present picture. 

It should be clear, by this stage, that no matter how closely we look at Figure~\ref{f25} we are
going to find structure which  we cannot resolve. In other words, the separatrix between the two
basins of attraction shown in this figure is a curve which exhibits structure {\em at all
scales}. Mathematicians have a special term for such a curve---they call it a 
{\em fractal}.\footnote{B.B.~Mandelbrot, {\em The fractal geometry of nature}, (W.H.~Freeman, 
New York NY, 1982).}

Many people think of fractals as mathematical toys whose principal use is the generation
of pretty pictures. However, it turns out that there is a close connection between fractals
and the dynamics of non-linear systems---particularly systems which exhibit chaotic
dynamics. We have just seen an example in which the boundary between the basins of attraction of two
co-existing attractors in phase-space is a fractal curve. This turns out to be a fairly
general result: {\em i.e.}, when multiple attractors exist in phase-space the separatrix
between their various basins of attraction is invariably fractal. What is this telling us
about the nature of non-linear dynamics? Well, returning to Figure~\ref{f25}, we can see that in the
region of phase-space in which the fractal behaviour of the separatrix manifests itself most
strongly ({\em i.e.}, the region where the light and dark bands fragment) the system exhibits
abnormal sensitivity to  initial conditions. In other words, we only have to change the initial
conditions slightly ({\em i.e.}, so as to move from a dark to a light band, or {\em vice versa})
in order to significantly alter the time-asymptotic motion of the pendulum ({\em i.e.}, to cause the
system to converge to a left-favouring  instead of  a right-favouring attractor, 
or {\em vice versa}). Fractals and extreme sensitivity to initial conditions
are themes which will reoccur in our investigation of non-linear dynamics.

\section{Period-Doubling Bifurcations}
\begin{figure}
\epsfysize=2.5in
\centerline{\epsffile{Chapter15/fig15.17.eps}}
\caption{\em The $v$-coordinate of the Poincar\'{e} section of a time-asymptotic orbit
plotted against the quality-factor $Q$. Data
 calculated numerically for
$A=1.5$, $\omega=2/3$, $\theta(0)=0$, $v(0)=0$, and $\phi=0$. }\label{f29}
\end{figure}
Let us now return to Figure~\ref{f20}. Recall, that as the quality-factor $Q$
is gradually increased,  the time-asymptotic orbit of the
pendulum through phase-space undergoes a sudden transition, at $Q\simeq 1.245$, from a
left-right symmetric period-1  orbit to a left-favouring period-1 orbit. What happens if
we continue to increase $Q$? Figure~\ref{f29} is basically a continuation of Figure~\ref{f20}. 
It can be seen that as $Q$ is increased the left-favouring period-1 orbit gradually
evolves until a critical value of $Q$, which is about $1.348$, is reached.
When $Q$ exceeds this critical value the nature of the orbit undergoes another sudden change:
this time
from a left-favouring period-1 orbit to a left-favouring {\em period-2}\/ orbit. 
Obviously, the
change is
indicated by the forking of the curve in Figure~\ref{f29}. This type of transition
is termed a {\em period-doubling bifurcation}, since it involves a 
sudden doubling of the
repetition period of the pendulum's time-asymptotic motion. 

We can represent period-1 motion schematically as $AAAAAA\cdots$, where $A$ represents a pattern
of motion which is repeated every period of the external drive. Likewise, we can represent
period-2 motion as $ABABAB\cdots$, where $A$ and $B$ represent {\em distinguishable}\/  patterns of motion
which are repeated every alternate period of the external drive. A period-doubling
bifurcation is represented: $AAAAAA\cdots\rightarrow ABABAB\cdots$. Clearly, all that happens
in such a bifurcation is that the pendulum suddenly decides to do something slightly different in
alternate periods of the external drive.

\begin{figure}
\epsfysize=2.5in
\centerline{\epsffile{Chapter15/fig15.18.eps}}
\caption{\em Equally spaced (in time) points on a time-asymptotic orbit in phase-space. Data
calculated numerically for $Q=1.36$, $A=1.5$, $\omega=2/3$, $\theta(0)=0$,
and $v(0)=-3$. }\label{f30}
\end{figure}

Figure~\ref{f30} shows the time-asymptotic phase-space orbit of the pendulum calculated for a
value of $Q$ somewhat higher than that required to trigger the above mentioned period-doubling bifurcation. It can
be seen that the orbit is left-favouring ({\em i.e.}, it spends the majority of its time on
the left-hand side of the plot), and takes the form of a closed curve consisting of
{\em two}\/ interlocked loops in phase-space.
Recall that for period-1 orbits there was only a single closed  loop in phase-space.
Figure~\ref{f31} shows the Poincar\'{e} section of the orbit shown in Figure~\ref{f30}. 
The fact that the section consists
of {\em two}\/ points confirms that the orbit does indeed correspond to period-2 motion.

\begin{figure}
\epsfysize=2.5in
\centerline{\epsffile{Chapter15/fig15.19.eps}}
\caption{\em The Poincar\'{e} section of a time-asymptotic
orbit. Data calculated numerically for $Q=1.36$, $A=1.5$, $\omega=2/3$, $\theta(0)=0$,
$v(0)=0$, and $\phi=0$. }\label{f31}
\end{figure}

A period-doubling bifurcation is an example of {\em temporal symmetry breaking}. The equations of
motion of the pendulum are invariant under the transformation $t\rightarrow t+\tau$, where
$\tau$ is the period of the external drive. In the low amplitude ({\em i.e.}, linear) limit, 
the time-asymptotic motion of the pendulum always respects this symmetry. However, as we have just seen, in the
non-linear regime it is possible to obtain solutions which spontaneously break this  symmetry.
Obviously, motion which repeats itself every two periods of the external drive is not
as temporally symmetric as motion which repeats every period of the drive. 

\begin{figure}
\epsfysize=2.5in
\centerline{\epsffile{Chapter15/fig15.20.eps}}
\caption{\em The $v$-coordinate of the Poincar\'{e} section of a time-asymptotic orbit
plotted against the quality-factor $Q$. Data
 calculated numerically for
$A=1.5$, $\omega=2/3$, and $\phi=0$. Data is shown for two sets of initial
conditions: $\theta(0)=0$ and $v(0)=0$ (lower branch); and $\theta(0)=0$ and $v(0)=-2$ (upper branch).}\label{f32}
\end{figure}

Figure~\ref{f32} is essentially a continuation of Fig~\ref{fx24}. Data is shown for two
sets of initial conditions which lead to motions converging on left-favouring (lower branch) and
right-favouring (upper branch) periodic attractors. We have already seen that the left-favouring
periodic attractor undergoes a period-doubling bifurcation at $Q=1.348$. It is clear from Figure~\ref{f32}
that the right-favouring attractor undergoes a similar bifurcation at almost exactly the
same $Q$-value. This is hardly surprising since, as has already
been mentioned, for fixed physical parameters ({\em i.e.}, $Q$, $A$, $\omega$), the
left- and right-favouring attractors are mirror-images of one another.

\section{Route to Chaos}\label{spd}
Let us return to Figure~\ref{f29}, which tracks the evolution of a left-favouring
periodic attractor as the quality-factor $Q$ is gradually increased. Recall that when
$Q$ exceeds a critical value, which is about $1.348$, then the attractor
undergoes a period-doubling bifurcation which converts it from a period-1 to a period-2
attractor. This bifurcation is indicated by the forking of the curve in
Figure~\ref{f29}. Let us now investigate what happens as we continue to increase $Q$. Figure~\ref{f34} is basically
a continuation of Figure~\ref{f29}. It can be seen that, as $Q$ is gradually
increased, the attractor undergoes a period-doubling bifurcation at $Q=1.348$, as before,
but then undergoes a {\em second}\/ period-doubling bifurcation (indicated by the 
second forking of
the curves) at $Q\simeq 1.370$, and a {\em third}\/ bifurcation at $Q\simeq 1.375$. Obviously, the second bifurcation
converts a period-2 attractor into a period-4 attractor (hence, two curves split apart to
give four curves). Likewise, the third bifurcation converts a period-4 attractor
into a period-8 attractor (hence, four curves split into eight curves). 
Shortly after the third
bifurcation, the various curves in the figure seem to expand explosively and  merge together
to produce an area of almost solid black. As we shall see, this behaviour is indicative
of the onset of {\em chaos}. 
 
\begin{figure}
\epsfysize=3.in
\centerline{\epsffile{Chapter15/fig15.21.eps}}
\caption{\em The $v$-coordinate of the Poincar\'{e} section of a time-asymptotic orbit
plotted against the quality-factor $Q$. Data
 calculated numerically for
$A=1.5$, $\omega=2/3$, $\theta(0)=0$, $v(0)=0$, and $\phi=0$.}\label{f34}
\end{figure}

Figure~\ref{f35} is a blow-up of Figure~\ref{f34}, showing more details of the onset of
chaos. The period-4 to period-8 bifurcation can be seen quite clearly. However,
we can also see a period-8 to period-16 bifurcation, at $Q\simeq 1.3755$. Finally,
if we look carefully, we can see a hint of a period-16 to period-32 bifurcation, just
before the start of the solid black region.
 Figures~\ref{f34} and \ref{f35} seem to suggest that the
onset of chaos is triggered by an {\em infinite series}\/ of period-doubling bifurcations.

\begin{figure}
\epsfysize=3.in
\centerline{\epsffile{Chapter15/fig15.22.eps}}
\caption{\em The $v$-coordinate of the Poincar\'{e} section of a time-asymptotic orbit
plotted against the quality-factor $Q$. Data
calculated numerically for
$A=1.5$, $\omega=2/3$, $\theta(0)=0$, $v(0)=0$,
 and $\phi=0$.}\label{f35}
\end{figure}

Table~\ref{tpd} gives some details of the sequence of period-doubling bifurcations
shown in Figures~\ref{f34} and \ref{f35}. Let us introduce a bifurcation index $n$: the
period-1 to period-2 bifurcation corresponds to $n=1$;  the
period-2 to period-4 bifurcation corresponds to $n=2$; and so on. Let $Q_n$ be
the critical value of the quality-factor $Q$ at which the $n$th bifurcation is
triggered. Table~\ref{tpd} shows the $Q_n$, determined from  Figures~\ref{f34} and \ref{f35}, for
$n=1$ to 5. Also shown is the ratio
\begin{equation}
F_n = \frac{Q_{n-1}-Q_{n-2}}{Q_{n}-Q_{n-1}}
\end{equation}
for $n=3$ to 5. It can be seen that Table~\ref{tpd} offers reasonably
convincing  evidence that this ratio takes the {\em constant}\/ value $F=4.69$.
It follows that we can estimate the critical $Q$-value required to trigger the $n$th
 bifurcation via the following formula:
\begin{equation}
Q_n = Q_1 + (Q_2-Q_1)\sum_{j=0}^{n-2}\frac{1}{F^j},
\end{equation}
for $n>1$. Note that the distance (in $Q$) between bifurcations decreases rapidly as
$n$ increases. In fact, the above
formula predicts an {\em accumulation}\/ of period-doubling bifurcations at $Q=Q_\infty$, where
\begin{equation}
Q_\infty = Q_1 + (Q_2-Q_1)\sum_{j=0}^{\infty}\frac{1}{F^j} \equiv Q_1 + (Q_2-Q_1)\,\frac{F}{F-1}=
1.3758.
\end{equation}
Note that our calculated  accumulation point corresponds almost exactly to the onset of the 
solid black region in Figure~\ref{f35}.
By the time that $Q$ exceeds $Q_\infty$,  we expect the attractor to have been
converted into a {\em period-infinity}\/
attractor via an infinite series of period-doubling bifurcations. 
A period-infinity  attractor is one  whose corresponding motion {\em never}\/ repeats
itself, no matter
how long we wait. In dynamics, such bounded aperiodic motion is generally referred to as {\em chaos}. 
Hence, a period-infinity attractor is sometimes called a {\em chaotic attractor}. 
Now,  period-$n$ motion is represented by $n$ separate curves in Figure~\ref{f35}. 
It is, therefore, not surprising that chaos ({\em i.e.}, period-infinity motion) is
represented by an infinite number of curves which merge together to form a region of solid black.


\begin{table}
\centering
\begin{tabular}{ccccc}\hline
Bifurcation & $n$ & $Q_n$ & $Q_{n} - Q_{n-1}$ & $F_n$\\\hline
period-1$\rightarrow$period-2 & $1$ & $1.34870$ & - & - \\
period-2$\rightarrow$period-4 & $2$ & $1.37003$ & 0.02133 & - \\
period-4$\rightarrow$period-8 & $3$ & $1.37458$ & 0.00455 & $4.69\pm 0.01$ \\
period-8$\rightarrow$period-16 & $4$ & $1.37555$ & 0.00097 & $4.69\pm 0.04$ \\
period-16$\rightarrow$period-32 & $5$ & $1.37575$ & 0.00020 & $4.9\pm 0.20$ \\
\hline
\end{tabular}
\caption{\em The period-doubling cascade.}\label{tpd}
\end{table}

Let us examine the onset of chaos in a little more detail. Figures~\ref{f36}--\ref{f39}
show details of the pendulum's time-asymptotic motion at various stages on the
period-doubling cascade discussed above. Figure~\ref{f36} shows period-4 motion:
note that the  Poincar\'{e} section consists of four points, and the associated sequence of
net rotations per period of the pendulum repeats itself every four periods.
 Figure~\ref{f37} shows period-8 motion:
now the Poincar\'{e} section consists of eight points, and the rotation sequence
 repeats itself every eight periods. Figure~\ref{f38} shows period-16 motion:
as expected, the Poincar\'{e} section consists of sixteen points, and the rotation sequence
 repeats itself every sixteen periods. Finally, Figure~\ref{f39} shows chaotic motion. Note that
the Poincar\'{e} section now consists of a set of four {\em continuous}\/ line segments, which are, presumably, made
up of an infinite number of points (corresponding to the infinite period of chaotic motion). 
Note, also, that the associated sequence of net rotations per period shows no obvious sign of ever
repeating itself. In fact, this sequence looks rather like one of the previously shown periodic
sequences with the addition of a small random component. The generation of {\em apparently
random}\/ motion from   equations of motion, such as Equations~(\ref{e16.11}) and
(\ref{e16.12}), which contain no overtly random elements is one of the most surprising features of
non-linear dynamics.

\begin{figure}
\epsfysize=2.in
\centerline{\epsffile{Chapter15/fig15.23.eps}}
\caption{\em The Poincar\'{e} section of a time-asymptotic orbit. Data 
calculated numerically for $Q=1.372$, $A=1.5$, $\omega=2/3$,
$\theta(0)=0$, $v(0)=0$, and $\phi=0$. Also, shown is the
net rotation per period, ${\mit\Delta}\theta/2\pi$,  calculated at the Poincar\'{e} phase
$\phi=0$.}\label{f36}
\end{figure}

\begin{figure}
\epsfysize=2.in
\centerline{\epsffile{Chapter15/fig15.24.eps}}
\caption{\em The Poincar\'{e} section of a time-asymptotic orbit.
Data calculated numerically  for $Q=1.375$, $A=1.5$, $\omega=2/3$,
$\theta(0)=0$, $v(0)=0$, and $\phi=0$. Also, shown is the
net rotation per period, ${\mit\Delta}\theta/2\pi$, calculated at the Poincar\'{e} phase
$\phi=0$.}\label{f37}
\end{figure}

\begin{figure}
\epsfysize=2.in
\centerline{\epsffile{Chapter15/fig15.25.eps}}
\caption{\em The Poincar\'{e} section of a time-asymptotic orbit. Data
calculated numerically for $Q=1.3757$, $A=1.5$, $\omega=2/3$,
$\theta(0)=0$, $v(0)=0$, and $\phi=0$. Also, shown is the
net rotation per period, ${\mit\Delta}\theta/2\pi$, calculated at the Poincar\'{e} phase
$\phi=0$.}\label{f38}
\end{figure}

\begin{figure}
\epsfysize=2.in
\centerline{\epsffile{Chapter15/fig15.26.eps}}
\caption{\em The Poincar\'{e} section of a time-asymptotic orbit. Data
calculated numerically for $Q=1.376$, $A=1.5$, $\omega=2/3$,
$\theta(0)=0$, $v(0)=0$, and $\phi=0$. Also, shown is the
net rotation per period, ${\mit\Delta}\theta/2\pi$, calculated at the Poincar\'{e} phase
$\phi=0$.}\label{f39}
\end{figure}

Many non-linear dynamical systems found in nature exhibit a transition
from periodic to chaotic motion as some control parameter is varied.
Moreover, there are various known mechanisms by which chaotic motion can arise from periodic
motion. A transition to chaos via an infinite series of period-doubling bifurcations, as
illustrated
above, is one of the most commonly occurring mechanisms. 
Around 1975, the physicist Mitchell Feigenbaum was investigating a simple mathematical
model, known as the {\em logistic map}, which exhibits a transition to chaos,
via a sequence of period-doubling bifurcations, as a control parameter
$r$ is increased. Let $r_n$ be the value of $r$ at which the
first $2^n$-period cycle appears. Feigenbaum noticed that the ratio
\begin{equation}
F_n =\frac{r_{n-1}-r_{n-2}}{r_{n}-r_{n-1}}
\end{equation}
converges rapidly to a constant value, $F=4.669$, as $n$ increases. Feigenbaum was
able to demonstrate  that this value of $F$ is common to a wide range of different
mathematic models which exhibit transitions to chaos via period-doubling
bifurcations.\footnote{M.J.~Feigenbaum, {\em Quantitative universality for a
class of nonlinear transformations}, J.\ Stat.\ Phys.\ {\bf 19}, 25 (1978).}
Feigenbaum  went on to argue that the {\em Feigenbaum ratio}, $F_n$,
should converge to the value $4.669$ in {\em any} dynamical system exhibiting a
transition to chaos via period-doubling bifurcations.\footnote{M.J.~Feigenbaum, 
{\em The universal metric properties of nonlinear transformations}, 
J.\ Stat.\ Phys.\ {\bf 21}, 69 (1979).} This amazing prediction has
been verified experimentally in a number of quite different physical
systems.\footnote{P.~Citanovic, {\em Universality in chaos}, (Adam Hilger,
Bristol UK, 1989).} Note that our best estimate of the
Feigenbaum ratio (see Table~\ref{tpd}) is $4.69\pm 0.01$, in good agreement with Feigenbaum's
prediction.

The existence of a universal ratio characterizing the transition to chaos via
period-doubling bifurcations is one of many pieces of evidence indicating that
chaos is a {\em universal}\/ phenomenon ({\em i.e.}, the onset and nature
of chaotic motion in different dynamical systems has many common features). 
This observation encourages us to believe that in studying the chaotic motion of 
a damped periodically driven pendulum we are  learning lessons which can
be applied to a wide range of non-linear dynamical systems.

\section{Sensitivity to Initial Conditions}
Suppose that we launch  our pendulum  and then wait until its motion has converged onto
a particular attractor. The subsequent motion can be visualized as a
trajectory $\theta_0(t)$, $v_0(t)$ through phase-space. Suppose that we then somehow
perturb the pendulum, at time $t=t_0$, such that its position in phase-space
is instantaneously changed from  $\theta_0(t_0)$, $v_0(t_0)$ to $\theta_0(t_0)+\delta\theta_0$, $v_0(t_0)
+\delta v_0$. The subsequent motion can be visualized as a second trajectory
   $\theta_1(t)$, $v_1(t)$ through phase-space. What is the relationship
between the original trajectory $\theta_0(t)$, $v_0(t)$ and the perturbed trajectory 
$\theta_1(t)$, $v_1(t)$?
In particular, does the phase-space separation between the two trajectories,
whose components are
\begin{eqnarray}
\delta\theta({\mit\Delta} t) &=& \theta_1(t_0+{\mit\Delta} t)-\theta_0(t_0+{\mit\Delta} t),\\[0.5ex]
\delta v({\mit\Delta} t) &=& v_1(t_0+{\mit\Delta} t)-v_0(t_0+{\mit\Delta} t),
\end{eqnarray}
grow in time,  decay in time, or stay more or less the same? What
we are really investigating is how sensitive  the time-asymptotic
 motion of the pendulum  is to initial conditions. 

According to the linear analysis of Section~\ref{s5.2},
\begin{eqnarray}
\delta\theta({\mit\Delta} t) &=& \delta\theta_0\,\cos(\omega_\ast\, {\mit\Delta} t)\,\,{\rm e}^{-{\mit\Delta} t/2Q}\nonumber\\[0.5ex]&&
+ \frac{1}{\omega_\ast}\left\{\delta v_0 + \frac{\delta\theta_0}{2Q}\right\}
\sin(\omega_\ast \,{\mit\Delta} t)\,\,{\rm e}^{-{\mit\Delta} t/2Q},\\[0.5ex]
\delta v({\mit\Delta} t) &=& \delta v_0\,\cos(\omega_\ast \,{\mit\Delta} t)\,\,{\rm e}^{-{\mit\Delta} t/2Q}
\nonumber\\[0.5ex]&&
- \frac{1}{\omega_\ast}\left\{\delta \theta_0 + \frac{\delta v_0}{2Q}\right\}
\sin(\omega_\ast \,{\mit\Delta} t)\,\,{\rm e}^{- {\mit\Delta} t/2Q},
\end{eqnarray}
assuming $\sin(\omega_\ast\,t_0)=0$.
It is clear that, in the linear regime, at least, the pendulum's 
time-asymptotic motion  is not particularly sensitive
to initial conditions. In fact, if we move the pendulum's phase-space trajectory slightly off
the linear attractor, as described above, then
 the perturbed
trajectory   decays  back to the attractor {\em exponentially}\/ in time.
In other words, if we wait long enough then the perturbed and unperturbed motions of the pendulum
become effectively indistinguishable. Let us now investigate whether this insensitivity to
initial conditions carries over into the non-linear regime.

\begin{figure}
\epsfysize=2.5in
\centerline{\epsffile{Chapter15/fig15.27.eps}}
\caption{\em The $v$-component of the separation between two neighbouring
phase-space trajectories (one of which lies on an attractor) plotted against normalized time. Data
 calculated numerically for $Q=1.372$, 
$A=1.5$, $\omega=2/3$, $\theta(0)=0$, and $v(0)=0$. The separation between the
two trajectories is initialized to $\delta\theta_0=\delta v_0 = 10^{-6}$ at
${\mit\Delta} t = 0$.}\label{f40}
\end{figure}

\begin{figure}
\epsfysize=2.5in
\centerline{\epsffile{Chapter15/fig15.28.eps}}
\caption{\em The $v$-component of the separation between two neighbouring
phase-space trajectories  (one of which lies on an attractor) plotted against normalized time. Data
 calculated numerically for $Q=1.375$, 
$A=1.5$, $\omega=2/3$, $\theta(0)=0$, and $v(0)=0$. The separation between the
two trajectories is initialized to $\delta\theta_0=\delta v_0 = 10^{-6}$ at
${\mit\Delta} t = 0$.}\label{f41}
\end{figure}

\begin{figure}
\epsfysize=2.5in
\centerline{\epsffile{Chapter15/fig15.29.eps}}
\caption{\em The $v$-component of the separation between two neighbouring
phase-space trajectories (one of which lies on an attractor) plotted against normalized time. Data
calculated numerically for $Q=1.3757$, 
$A=1.5$, $\omega=2/3$, $\theta(0)=0$, and $v(0)=0$. The separation between the
two trajectories is initialized to $\delta\theta_0=\delta v_0 = 10^{-6}$ at
${\mit\Delta} t = 0$.}\label{f42}
\end{figure}

\begin{figure}
\epsfysize=2.5in
\centerline{\epsffile{Chapter15/fig15.30.eps}}
\caption{\em The $v$-component of the separation between two neighbouring
phase-space trajectories (one of which lies on an attractor) plotted against normalized time. Data
 calculated numerically for $Q=1.376$, 
$A=1.5$, $\omega=2/3$, $\theta(0)=0$, and $v(0)=0$. The separation between the
two trajectories is initialized to $\delta\theta_0=\delta v_0 = 10^{-6}$ at
${\mit\Delta} t = 0$.}\label{fx43}
\end{figure}

Figures~\ref{f40}--\ref{fx43} show the results of the experiment described above,
in which the pendulum's phase-space trajectory is moved slightly off an attractor
and the phase-space separation between the perturbed and unperturbed trajectories
is then monitored as a function of time,
at various stages on the period-doubling cascade discussed in the previous section. 
To be more exact, the figures show the logarithm of the absolute magnitude of the
$v$-component of the phase-space separation between the perturbed and unperturbed
trajectories as  a function of normalized time.

Figure~\ref{f40} shows the time evolution of the $v$-component of the phase-space
separation, $\delta v$, between two neighbouring trajectories, one of which is the period-4
attractor illustrated in Figure~\ref{f36}. It can be seen that $\delta v$ decays rapidly
in time. In fact, the graph of $\log(|\delta v|)$ versus ${\mit\Delta} t$ can be plausibly
represented as a {\em straight-line} of gradient $\lambda$. In other words,
\begin{equation}
|\delta v({\mit\Delta} t)| \simeq \delta v_0\,{\rm e}^{\,\lambda\,{\mit\Delta t}},\label{lia}
\end{equation}
where the quantity $\lambda$ is known as the {\em Liapunov exponent}, and is obviously negative. Clearly, in this case,
$\lambda$ measures the strength of the exponential convergence of the two
trajectories in phase-space. 
Of course,
the  graph of $\log(|\delta v|)$ versus ${\mit\Delta} t$ is not exactly a straight-line. There
are deviations due to the fact that $\delta v$ oscillates, as well as decays, in time.
There are  
also deviations because the strength
of the exponential convergence between the two trajectories varies along the attractor. 

The above definition of the Liapunov exponent is rather inexact, for
two main reasons. In the first place,
the strength
of the exponential convergence/divergence between  two neighbouring trajectories in phase-space,
one of which is an attractor, generally {\em varies}\/ along the attractor. Hence, we should really
take formula (\ref{lia}) and somehow {\em average}\/ it over the attractor,
in order to obtain a more unambiguous definition of $\lambda$. In the second place,
since the dynamical system under investigation  is a second-order system, it actually possesses {\em two}\/
different Liapunov exponents. Consider the evolution of an infinitesimal circle of perturbed
initial conditions, centred on a point in phase-space
lying on an attractor. During its evolution, the circle will become
distorted into an infinitesimal ellipse. Let $\delta_k$, where $k=1,2$, denote the phase-space
length of the $k$th principal axis of the ellipse. The two Liapunov exponents, $\lambda_1$ and $\lambda_2$,
are defined via $\delta_k({\mit\Delta}t)\simeq \delta_k(0)\,\exp(\lambda_k\,{\mit\Delta}t)$. 
However, for large ${\mit\Delta}t$, the diameter of the ellipse is effectively controlled by the 
Liapunov exponent with the most positive real part. Hence, when we refer to
{\em the}\/ Liapunov exponent, $\lambda$, what we generally mean is the Liapunov 
exponent with the most positive real part. 

Figure~\ref{f41} shows the time evolution of the $v$-component of the phase-space
separation, $\delta v$, between two neighbouring trajectories, one of which is the period-8
attractor illustrated in Figure~\ref{f37}. It can be seen that $\delta v$ decays in time,
though not as rapidly as in Figure~\ref{f40}. Another way of saying this is that the
Liapunov exponent of the periodic  attractor shown in Figure~\ref{f37} 
is negative ({\em i.e.}, it has a negative real part),
but not as negative as that of the periodic attractor shown in Figure~\ref{f36}.

Figure~\ref{f42} shows the time evolution of the $v$-component of the phase-space
separation, $\delta v$, between two neighbouring trajectories, one of which is the period-16
attractor illustrated in Figure~\ref{f38}. It can be seen that $\delta v$ decays weakly in
time. In other words, the Liapunov exponent of the periodic attractor shown in Figure~\ref{f38} is
small and negative. 

Finally, Figure~\ref{fx43} shows the time evolution of the $v$-component of the phase-space
separation, $\delta v$, between two neighbouring trajectories, one of which is the chaotic
attractor illustrated in Figure~\ref{f39}. It can be seen that $\delta v$ {\em increases}\/   in
time. In other words, the  Liapunov exponent of the chaotic attractor shown in Figure~\ref{f39} is
{\em positive}. Further investigation reveals that, as the control parameter $Q$
is gradually increased, the Liapunov exponent changes sign and becomes positive at exactly the same
point that chaos ensues in Figure~\ref{f35}. 

The above discussion strongly suggests that periodic attractors are characterized by negative
Liapunov exponents, whereas chaotic attractors are characterized by positive exponents.
But, how can an attractor have a positive Liapunov exponent? Surely, a positive
exponent necessarily implies that neighbouring phase-space trajectories {\em diverge}\/
from the attractor (and, hence, that the attractor is not a true attractor)? 
It turns out that this is not the case. The chaotic attractor
shown in Figure~\ref{f39} is a true attractor, in the sense that neighbouring trajectories
rapidly converge onto it---{\em i.e.}, after a few periods of the external drive
their Poincar\'{e} sections plot out the same four-line segment shown in Figure~\ref{f39}. 
Thus, the exponential divergence of neighbouring trajectories, characteristic of chaotic
attractors, takes place {\em within the attractor}\/ itself. Obviously, this exponential divergence
must come to an end when the phase-space separation of the trajectories becomes comparable to the 
extent of the attractor.

A dynamical system characterized by a positive Liapunov exponent, $\lambda$,  has a {\em time horizon} beyond
which regular deterministic prediction breaks down. Suppose that we measure the initial
conditions of an experimental system very accurately. Obviously, no measurement is perfect: there
is always some error $\delta_0$ between our estimate and the true initial state. After a
time $t$, the discrepancy grows to $\delta(t)\sim \delta_0\,\exp(\lambda\,t)$. Let $a$ be
a measure of our tolerance: {\em i.e.}, a prediction within $a$ of the true state
is
considered acceptable. It follows that our prediction becomes unacceptable when
$\delta \gg a$, which occurs when
\begin{equation}
t> t_h \sim \frac{1}{\lambda}\ln\!\left(\frac{a}{\delta_0}\right).
\end{equation}
Note the {\em logarithmic}\/ dependence on $\delta_0$. This ensures that, in practice, no matter how
hard we work to reduce our initial measurement error, we cannot predict the behaviour of
the system for longer than a few multiples of $1/\lambda$. 

It follows, from the above discussion, that chaotic attractors are associated with motion which is
essentially {\em unpredictable}. In other words, if we attempt to integrate the equations of
motion of a chaotic system then even the slightest error made in the initial conditions
will be amplified exponentially over time, and will rapidly destroy the accuracy of our prediction. 
Eventually, all that we will be able to say is that the motion lies somewhere on the chaotic attractor
in phase-space, but exactly where it lies on the attractor at any given time will be unknown to
us. 

The hyper-sensitivity of chaotic systems to initial conditions  is
sometimes called the {\em butterfly effect}. The idea is that a butterfly flapping its wings
in a South American rain-forest could, in principle,  affect the weather in Texas (since the atmosphere 
exhibits chaotic dynamics). This idea was first publicized by the meteorologist Edward Lorenz,
who constructed a very crude model of the convection of the atmosphere when it is heated from
below by the ground.\footnote{E.~Lorenz, {\em Deterministic nonperiodic flow}, J.\ Atmospheric
Science {\bf 20}, 130 (1963).}
 Lorenz discovered, much to his surprise, that his model atmosphere exhibited
chaotic motion---which, at that time, was virtually unknown.
 In fact, Lorenz was essentially the first
scientist to fully understand the nature and ramifications of chaotic motion in
physical systems. In particular, Lorenz realized that the chaotic dynamics of the
atmosphere spells the doom of long-term weather forecasting: the best one can hope 
to achieve is
to predict the weather a few days in advance ($1/\lambda$ for the atmosphere is of order a
few days). 

\section{Definition of Chaos}
There is no universally agreed definition of chaos. However, most people would accept the
following working definition:
\begin{quote}
Chaos is aperiodic time-asymptotic behaviour in a deterministic system which
exhibits sensitive dependence on initial conditions.
\end{quote}
This definition contains three main elements:
\begin{enumerate}
\item Aperiodic time-asymptotic behaviour---this implies the existence of
phase-space trajectories which do not settle down to fixed points or
periodic orbits. For practical reasons, we  insist that these trajectories
are not too rare. We also require the trajectories to be {\em bounded}:
{\em i.e.}, they should not go off to infinity. 
\item Deterministic---this implies that the equations of motion of the system  possess no
random inputs. In other words, the irregular behaviour of the system   arises
from non-linear dynamics, and not from noisy driving forces. 
\item Sensitive dependence on initial conditions---this implies that nearby
trajectories in phase-space separate exponentially fast in time: {\em i.e.}, that the
system has a positive Liapunov exponent.
\end{enumerate}

\section{Periodic Windows}
\begin{figure}
\epsfysize=3.in
\centerline{\epsffile{Chapter15/fig15.31.eps}}
\caption{\em The $v$-coordinate of the Poincar\'{e} section of a time-asymptotic orbit
plotted against the quality-factor $Q$. Data
 calculated numerically for
$A=1.5$, $\omega=2/3$, $\theta(0)=0$, $v(0)=-0.75$,
 and $\phi=0$.}\label{fx44}
\end{figure}

Let us return to Figure~\ref{f35}. Recall, that this figure shows the onset of
chaos, via a cascade of period-doubling bifurcations, as the quality-factor $Q$
is gradually increased. Figure~\ref{fx44} is essentially a continuation of Figure~\ref{f35}
which shows the full extent of the chaotic region (in $Q$-$v$ space). It can be
seen that the chaotic region ends abruptly when $Q$ exceeds a critical
value, which is about $1.4215$. Beyond this critical value, the time-asymptotic motion appears to
revert to period-1 motion ({\em i.e.}, the solid black region collapses to a single curve). 
It can also be seen that the chaotic region contains many narrow  windows in which
chaos reverts to periodic motion  ({\em i.e.}, the solid black region collapses
to $n$ curves, where $n$ is the period of the motion) for a short interval in $Q$. The four widest
windows are indicated in the figure.

\begin{figure}
\epsfysize=3.in
\centerline{\epsffile{Chapter15/fig15.32.eps}}
\caption{\em The $v$-coordinate of the Poincar\'{e} section of a time-asymptotic orbit
plotted against the quality-factor $Q$. Data
 calculated numerically for
$A=1.5$, $\omega=2/3$, $\theta(0)=0$, $v(0)=-0.75$, and $\phi=0$.}\label{f45}
\end{figure}

Figure~\ref{f45} is a blow-up of the period-3 window shown in Figure~\ref{fx44}. It can be
seen that the window appears ``out of the blue'' as $Q$ is gradually increased. However,
it can also be seen that, as $Q$ is further increased,  the window breaks down,
and eventually disappears, due to the action of a cascade of period-doubling
bifurcations. The same basic mechanism operates here as in the original period-doubling
cascade, discussed in Section~\ref{spd}, except that now the orbits are of period $3\cdot 2^n$, instead of $2\cdot 2^n$. 
Note that all of the other periodic windows seen in Figure~\ref{fx44} break down in an analogous manner,
as $Q$ is increased.


\begin{figure}
\epsfysize=2.in
\centerline{\epsffile{Chapter15/fig15.33.eps}}
\caption{\em The Poincar\'{e} section of a time-asymptotic orbit. Data
calculated numerically for $Q=1.387976$, $A=1.5$, $\omega=2/3$,
$\theta(0)=0$, $v(0)=-0.75$, and $\phi=0$. Also, shown is the
net rotation per period, ${\mit\Delta}\theta/2\pi$, calculated at the Poincar\'{e} phase
$\phi=0$.}\label{f46}
\end{figure}

\begin{figure}
\epsfysize=2.in
\centerline{\epsffile{Chapter15/fig15.34.eps}}
\caption{\em The Poincar\'{e} section of a time-asymptotic orbit. Data
calculated numerically for $Q=1.387977$, $A=1.5$, $\omega=2/3$,
$\theta(0)=0$, $v(0)=-0.75$, and $\phi=0$. Also, shown is the
net rotation per period, ${\mit\Delta}\theta/2\pi$, calculated at the Poincar\'{e} phase
$\phi=0$.}\label{f47}
\end{figure}

\begin{figure}
\epsfysize=2.in
\centerline{\epsffile{Chapter15/fig15.35.eps}}
\caption{\em The Poincar\'{e} section of a time-asymptotic orbit. Data
calculated numerically for $Q=1.387978$, $A=1.5$, $\omega=2/3$,
$\theta(0)=0$, $v(0)=-0.75$, and $\phi=0$. Also, shown is the
net rotation per period, ${\mit\Delta}\theta/2\pi$, calculated at the Poincar\'{e} phase
$\phi=0$.}\label{f48}
\end{figure}

We now understand how periodic windows break down. But, how do they appear in the
first place? Figures~\ref{f46}--\ref{f48} show details of the pendulum's time-asymptotic
motion calculated {\em just before}\/ the appearance of the period-3 window (shown in Figure~\ref{f45}), {\em just at}\/ the appearance of the window, and {\em just after}\/ the appearance of the window,
respectively. It can be seen, from Figure~\ref{f46}, that just before the appearance of the
window the attractor is chaotic ({\em i.e.}, its Poincar\'e section consists of a line,
rather than a discrete set of points), and the time-asymptotic motion of the pendulum
consists of intervals of period-3 motion interspersed with intervals of chaotic motion.
Figure~\ref{f47} shows that just at the appearance of the window the attractor
loses much of its chaotic nature ({\em i.e.}, its Poincar\'e section  breaks up into a series 
of points), and the chaotic intervals  become shorter and much less frequent. Finally,
Figure~\ref{f48} shows that just after the appearance of the window the attractor
collapses to a period-3 attractor, and the chaotic intervals cease altogether. 
All of the other periodic windows seen in Figure~\ref{fx44} appear in an analogous manner to that
just described.

According to the above discussion,
the typical time-asymptotic motion seen just prior to the appearance of a period-$n$ window
consists of intervals of period-$n$ motion interspersed with intervals of chaotic
motion. This type of behaviour is called {\em intermittency}, and is observed in a
wide variety of non-linear systems. As we move away from the window, in parameter
space, the intervals of periodic motion become gradually shorter and more infrequent. 
Eventually, they cease altogether. Likewise, as we move towards the window, the
intervals of periodic motion become gradually longer and more frequent. Eventually,
the whole motion becomes periodic. 

In 1973, Metropolis and co-workers investigated a class of simple mathematical models which all
exhibit a transition to chaos, via a cascade of period-doubling bifurcations, as
some control parameter $r$ is increased.\footnote{N.~Metropolis,
M.L.~Stein, and P.R.~Stein, {\em On finite limit sets for transformations
on the unit interval}, J.\ Combin.\ Theor.\ {\bf 15}, 25 (1973).} They were able to demonstrate
that, for these models,  the order in which stable periodic orbits occur as $r$ is
increased  is fixed. 
That is, {\em stable periodic attractors always occur in the same sequence}\/ as
$r$ is varied. This sequence is
called the universal or {\em U-sequence}. It is possible to make a fairly
convincing argument that any physical system which exhibits a transition to chaos
via a sequence of period-doubling bifurcations should also exhibit the U-sequence
of stable periodic attractors.
Up to period-6, the U-sequence is
$$
1, \,2,\, 2\times 2,\, 6,\, 5,\, 3,\, 2\times 3,\, 5, \,6,\, 4,\, 6,\, 5,\, 6.
$$
The beginning of this sequence is familiar: periods 1, 2, $2\times 2$ are the
first stages of the  period-doubling cascade. (The later period-doublings
give rise to periods greater than 6, and so are omitted here). The next periods,
$6,5,3$ correspond to the first three periodic windows shown in Figure~\ref{fx44}. 
Period $2\times 3$ is the first component of the period-doubling cascade which
breaks up the period-3 window. The next period, 5, corresponds to the last
periodic window shown in Figure~\ref{fx44}. The remaining periods,
6, 4, 6, 5, 6, correspond to tiny periodic windows, which, in practice,
are virtually impossible to  observe. It follows that our
driven pendulum system exhibits the U-sequence of stable periodic orbits fairly
convincingly. This sequence has also been observed experimentally in other,
quite different, dynamical systems.\footnote{R.H.~Simoyi,
A.~Wolf, and H.L.~Swinney, {\em One-dimensional dynamics in a multi-component
chemical reaction}, Phys.\ Rev.\ Lett.\ {\bf 49}, 245 (1982).} The existence of
a universal sequence of stable periodic orbits in dynamical systems which
exhibit a transition to chaos via a cascade of period-doubling bifurcations
 is another indication that
chaos is a universal phenomenon.

\section{Further Investigation}
\begin{figure}
\epsfysize=3.in
\centerline{\epsffile{Chapter15/fig15.36.eps}}
\caption{\em The $v$-coordinate of the Poincar\'{e} section of a time-asymptotic orbit
plotted against the quality-factor $Q$. Data
 calculated numerically for
$A=1.5$, $\omega=2/3$, $\theta(0)=0$, $v(0)=0$, and $\phi=0$.}\label{f49}
\end{figure}
Figure~\ref{f49} shows the complete {\em bifurcation diagram}\/ for the damped, periodically
driven, pendulum (with $A=1.5$ and $\omega=2/3$). 
 It can be seen that the chaotic region investigated in the previous section is, in fact,
the first, and least extensive, of {\em three}\/ different chaotic regions. 

\begin{figure}
\epsfysize=2.in
\centerline{\epsffile{Chapter15/fig15.37.eps}}
\caption{\em Equally spaced (in time) points on a time-asymptotic orbit in phase-space. Data 
calculated numerically for $Q=1.5$, $A=1.5$, $\omega=2/3$, $\theta(0)=0$,
and $v(0)=0$.
Also shown is the time-asymptotic orbit calculated for the modified initial
conditions $\theta(0)=0$, and
$v(0)=-1$.}\label{f50}
\end{figure}

The interval
between the first and second chaotic regions is occupied by the period-1 orbits shown in
Figure~\ref{f50}. Note that these orbits differ somewhat  from previously encountered period-1 orbits
 in that the pendulum executes a complete rotation (either to
the left or to the right) every period of the external drive. 
Now, an $n,l$ periodic orbit is defined such that
$$
\theta(t+n\,\tau) = \theta(t) + 2\pi\,l
$$
for all $t$ (after the transients have died away). It follows that all of the periodic orbits which we 
encountered in previous sections were $l=0$ orbits: {\em i.e.}, their  associated motions did
not involve a net rotation of the pendulum. The orbits show in Figure~\ref{f50}
are $n=1, l=-1$ and $n=1, l=+1$ orbits, respectively. The existence of periodic
orbits in which the pendulum undergoes a net rotation, either to the left or to the right,
is another example of spatial symmetry breaking---there is nothing in the pendulum's equations
of motion which distinguishes between the two possible directions of rotation.

\begin{figure}
\epsfysize=3in
\centerline{\epsffile{Chapter15/fig15.38.eps}}
\caption{\em The Poincar\'{e} section of a time-asymptotic orbit. Data
calculated numerically for $Q=2.13$, $A=1.5$, $\omega=2/3$, $\theta(0)=0$,
$v(0)=0$, and $\phi=0$. }\label{f51}
\end{figure}

Figure~\ref{f51} shows the Poincar\'{e} section of a typical  attractor in the
second chaotic region shown in Figure~\ref{f49}. It can be seen that this attractor is far more
convoluted and extensive than the simple 4-line chaotic attractor pictured in Figure~\ref{f39}. 
In fact, the attractor shown in Figure~\ref{f51} is clearly a {\em fractal}\/ curve. It turns
out that virtually all chaotic attractors exhibit fractal structure.

The interval between the second and third chaotic regions shown in Figure~\ref{f49} is
occupied by $n=3$, $l=0$ periodic orbits. Figure~\ref{f52} shows the Poincar\'{e} section 
of a typical attractor in the
third chaotic region. It can be seen that this attractor is even more overtly fractal
in nature than that pictured in the previous figure. Note that the fractal nature
of chaotic attractors is closely associated with some of their  unusual properties. 
Trajectories on a chaotic attractor remain confined to a bounded region of phase-space,
and yet they separate from their neighbours exponentially fast (at least, initially).
How can trajectories diverge endlessly and still stay bounded? The basic mechanism
is described below.
If we imagine
a blob of initial conditions in phase-space then these undergo a series of
repeated {\em stretching and folding}\/ episodes, as the chaotic motion unfolds. The stretching is
what gives rise to the divergence of neighbouring trajectories. The folding is what ensures that
the trajectories remain bounded. 
The net
result is a phase-space structure which looks  a bit like filo pastry---in other words, a fractal
structure. 

\begin{figure}
\epsfysize=3in
\centerline{\epsffile{Chapter15/fig15.39.eps}}
\caption{\em The Poincar\'{e} section of a time-asymptotic
orbit. Data  calculated numerically for $Q=3.9$, $A=1.5$, $\omega=2/3$, $\theta(0)=0$,
$v(0)=0$,  and $\phi=0$. }\label{f52}
\end{figure}
