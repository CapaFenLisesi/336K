\chapter{Lunar Motion}\label{moon}
\section{Historical Background}
The motion of the Planets around the Sun is   fairly accurately described by {\em Kepler's laws} (see Chapter~\ref{skepler}).
Likewise, to a first approximation, the motion of the Moon around the Earth can also be accounted for via these laws.
However, unlike the planetary orbits, the deviations of the lunar orbit from a Keplerian ellipse are sufficiently large that they
are easily apparent to the naked eye. Indeed, the largest of these deviations, which is generally known as
{\em evection}, was discovered in ancient times by the Alexandrian astronomer Claudius Ptolemy (90--168 AD). Moreover, the next largest deviation,
which is called {\em variation}, was first observed by Tycho Brahe (1546--1601) without the
aid of a telescope. Another non-Keplerian feature of the lunar orbit, which is sufficient obvious that it was known to the
ancient Greeks, is the fact that the lunar {\em perigee}\/ ({\em i.e.}, the point of closest approach to the Earth)
{\em precesses}\/  ({\em i.e.}, orbits about the Earth in the same direction as the Moon) at such a rate that it completes a full circuit
 every $8.85$ years.\footnote{Note that this precession rate is about $10^4$ times greater than any of the planetary perihelion precession
 rates discussed in Section~\ref{splanp}.} The ancient Greeks also noticed that the lunar {\em ascending node}\/ ({\em i.e.}, the point
at which the Moon passes through the fixed plane of the Earth's orbit around the Sun from south to north)
{\em regresses}\/  ({\em i.e.}, orbits about the Earth in the opposite direction to the Moon) at such a rate that it completes a full circuit  every 18.6 years. Of course, according to standard two-body orbit  theory, both the lunar perigee and  ascending node should be
{\em stationary} (see Chapter~\ref{skepler}). 

Newton demonstrated, in Book III of his Principia, that the deviations of the lunar orbit from a Keplerian ellipse are due to the
gravitational influence of the Sun, which is  sufficiently large that it is not completely negligible compared with the mutual
gravitational attraction of the Earth and the Moon.
However, Newton was not able to give a complete account of these
deviations, due to the complexity of the equations of motion which arise in a system of three mutually
gravitating bodies (see Chapter~\ref{threeb}).  In fact, Clairaut (1713--1765) is generally credited with the first reasonably accurate and complete theoretical explanation  of the Moon's orbit. 
His method of calculation makes use of an  expansion of the lunar equations of motion in terms of various small parameters.
Clairaut, however, initially experienced difficulty in accounting for the precession of the lunar perigee. Indeed, his first calculation
overestimated the period of this precession  by a factor of about two, leading him to question Newton's
inverse-square law of gravitation. Later on, he realized that he could indeed account for the precession, in terms of
standard Newtonian dynamics, by
continuing his expansion in small parameters to higher order. After Clairaut, the theory of lunar motion was further elaborated in major works by
D'Alembert (1717--1783), Euler (1707--1783), Laplace (1749--1827), Damoiseau (1768--1846), Plana (1781--1864), Poisson (1781--1840), Hansen (1795--1874),  De Pont\'{e}coulant (1795--1824), J.~Herschel (1792--1871), Airy (1801--1892), Delaunay (1816--1872), G.W.~Hill (1836--1914), and E.W.~Brown (1836--1938). The fact that so many celebrated mathematicians and astronomers devoted so much time and effort to lunar
theory is a tribute to its inherent difficulty, as well as its great theoretical and practical interest. Indeed, for a period
of about one hundred years (between 1767 and about 1850) the
so-called {\em method of lunar distance}\/ was the principal means used by mariners to determine
longitude at sea. This method depends crucially on a precise   knowledge of the position of the
Moon in the sky as a function of time. Consequently, astronomers and mathematicians during the
period in question  were spurred to make ever more accurate observations of the Moon's orbit, and to develop lunar theory
to greater and greater precision. An important outcome of these activities were various  tables of lunar motion ({\em e.g.}, those
of Mayer, Damoiseau, Plana, Hansen, and Brown), most of which were 
published at public expense.  Finally, it is worth noting that the development  of lunar theory gave rise to many
significant advances in mathematics and  dynamics, and was, in many ways, a precursor to the discovery of 
chaotic motion in the 20th century (see Chapter~\ref{schaos}). 

This chapter contains an introduction to lunar theory in which approximate expressions for evection, variation,
the precession of the perigee,  and the regression of the ascending node are derived from the  laws of Newtonian dynamics.\footnote{For more information on lunar theory, see  {\em An Elementary Treatise on the Lunar Theory}, H.~Godfray (Macmillan \& Co.,  1853);  {\em An Introductory Treatise on the Lunar Theory}, E.W.~Brown (Cambridge University Press, 1896); {\em Lectures on the Lunar Theory}, J.C.~Adams (Cambridge University Press, 1900).}

\section{Preliminary Analysis}
Let ${\bf r}_E$ and ${\bf r}_M$ be the position vectors of the Earth and Moon, respectively, in a  {\em non-rotating}\/
reference frame in which the Sun is at rest at the origin.  Treating this reference frame as {\em inertial}\/  (which  is an
excellent  approximation, given that the  mass of the Sun is very much greater than that of the Earth or the Moon---see Chapter~\ref{ctwo}),
the Earth's
equation of motion becomes (see Chapter~\ref{skepler})
\begin{equation}\label{ev1}
\ddot{\bf r}_E  = - n'^{\,2}\,a'^{\,3}\,\frac{{\bf r}_E}{|{\bf r}_E|^{\,3}},
\end{equation}
where $n'=0.98560025^\circ$ per day and $a'=149,598,261\,{\rm km}$ are the mean angular velocity and major radius, respectively, of the
terrestrial orbit about the Sun. Here, $\ddot{~}\equiv d^2/dt^2$. On the other hand, the Moon's equation of motion takes the form
\begin{equation}\label{ev2}
\ddot{\bf r}_M  = - n^{\,2}\,a^{\,3}\,\frac{({\bf r}_M-{\bf r}_E)}{|{\bf r}_M-{\bf r}_E|^{\,3}} - n'^{\,2}\,a'^{\,3}\,\frac{{\bf r}_M}{|{\bf r}_M|^{\,3}},
\end{equation}
where $n=13.17639646^\circ$ per day  and $a=384,399\,{\rm km}$ are the mean angular velocity and major radius, respectively, of the lunar 
orbit about the Earth. Note that we have retained the perturbing influence ({\em i.e.}, acceleration) of the Sun in the
lunar equation of motion, (\ref{ev2}), whilst  neglecting the perturbing influence of the Moon in the terrestrial
equation of motion, (\ref{ev1}), since the former influence is significantly greater (by a factor $M_E/M_M\simeq 81$, where $M_E$ is the mass of the
Earth, and $M_M$ the mass of the Moon) than the latter. 

Let 
\begin{eqnarray}
{\bf r}&=&{\bf r}_M-{\bf r}_E,\\[0.5ex]
{\bf r}' &=& -{\bf r}_E,\label{ev4}
\end{eqnarray}
be the position
vectors of the Moon and Sun, respectively, relative to the Earth. It follows, from Equations~(\ref{ev1})--(\ref{ev4}), that in a non-inertial
 reference frame,   $S$ (say), in which the Earth is at rest at the origin, but the coordinate axes point
 in {\em fixed}\/ directions, the
lunar and solar equations of motion take the form
 \begin{eqnarray}\label{eMoon}
 \ddot{\bf r} &=&
 - n^{\,2}\,a^{\,3}\,\frac{{\bf r}}{|{\bf r}|^{\,3}}
  + n'^{\,2}\,a'^{\,3}
 \left[\frac{({\bf r}'-{\bf r})}{|{\bf r}'-{\bf r}|^{\,3}}-\frac{{\bf r}'}{|{\bf r}'|^{\,3}}\right],
\\[0.5ex]
\ddot{\bf r}' &=& - n'^{\,2}\,a'^{\,3}\,\frac{{\bf r}'}{|{\bf r}'|^{\,3}},\label{eSun}
\end{eqnarray}
 respectively.  
 
 Let us set up a conventional Cartesian coordinate system in $S$ 
 which is  such that the (apparent)  orbit of the
 Sun about the Earth lies in the $x$-$y$ plane. This implies that the
 $x$-$y$ plane corresponds to the so-called {\em ecliptic plane}. Accordingly, in $S$, the Sun
 appears to orbit the Earth at the mean angular velocity $\bomega'=n'\,{\bf e}_z$ (assuming that the $z$-axis points
 toward the so-called north ecliptic pole), whereas the
  projection of the Moon onto the ecliptic plane  orbits the Earth at the mean angular velocity $\bomega = n\,{\bf e}_z$.

In the following, for the sake of simplicity, we shall neglect the small eccentricity, 
 $e'=0.016711$, of the Sun's apparent orbit about the
Earth (which is actually the eccentricity of the Earth's orbit about the Sun), and approximate the solar orbit as a {\em circle}, centered on the Earth. Thus, if $x'$, $y'$, $z'$ are the Cartesian coordinates of the Sun in $S$ then
 an appropriate solution of the solar equation of motion, (\ref{eSun}), is
 \begin{eqnarray}\label{evx}
 x'&=&a'\,\cos(n'\,t),\\[0.5ex]
 y'&=&a'\,\sin(n'\,t),\\[0.5ex]
 z' &=&0.\label{evz}
 \end{eqnarray}
 
 \section{Lunar Equations of Motion}
  It is convenient to solve the lunar equation of motion, (\ref{eMoon}), in a {\em geocentric}\/ frame of reference, $S_1$ (say), which {\em rotates}\/ with respect to
 $S$ at the fixed angular velocity $\bomega$. Thus, if the lunar orbit  were a circle, centered on the Earth, and lying in the ecliptic plane,
 then the Moon would appear {\em stationary}\/ in $S_1$. In fact, the small eccentricity of the lunar orbit, $e=0.05488$, combined
 with its slight inclination to the ecliptic plane, $\iota=5.161^\circ$, causes the Moon to execute a small periodic orbit about the stationary point. 
 
  Let $x$, $y$, $z$ and $x_1$, $y_1$, $z_1$  be the Cartesian coordinates
 of the Moon in $S$ and $S_1$, respectively.
 It is easily demonstrated that (see Section~\ref{sgradx})
 \begin{eqnarray}\label{ev10}
 x&=&x_1\,\cos(n\,t) -y_1\,\sin(n\,t),\\[0.5ex]
 y&=&x_1\,\sin(n\,t)+y_1\,\cos(n\,t),\\[0.5ex]
 z&=&z_1.\label{ev12}
 \end{eqnarray}
 Moreover, if $x_1'$, $y_1'$, $z_1'$  are the Cartesian components of the Sun in $S_1$ then (see Section~\ref{stransx})
 \begin{eqnarray}
 x_1' &=&x'\,\cos(n\,t)+y'\,\sin(n\,t),\\[0.5ex]
 y_1' &=& -x'\,\sin(n\,t) + y'\,\cos(n\,t),\\[0.5ex]
 z_1'&=&z',
 \end{eqnarray}
 giving
 \begin{eqnarray}\label{evxx}
 x_1'&=&a'\,\cos[(n-n')\,t],\\[0.5ex]
 y_1'&=&-a'\,\sin[(n-n')\,t],\\[0.5ex]
 z_1'&=&0,\label{evzz}
 \end{eqnarray}
where use has been made of Equations~(\ref{evx})--(\ref{evz}).

Now, in the rotating frame $S_1$, the lunar equation of motion (\ref{eMoon}) transforms to (see Chapter~\ref{snoni})
 \begin{equation}
 \ddot{\bf r}+ 2\,\bomega\times\dot{\bf r} + \bomega\times (\bomega\times {\bf r})
 = - n^{\,2}\,a^{\,3}\,\frac{{\bf r}}{|{\bf r}|^{\,3}} + n'^{\,2}\,a'^{\,3}\left[\frac{({\bf r}'-{\bf r})}{|{\bf r}'-{\bf r}|^{\,3}}
 -\frac{{\bf r}'}{|{\bf r}'|^{\,3}}\right],\label{ev19}
 \end{equation}
 where $\dot{~}\equiv d/dt$. 
 Furthermore, expanding the final term on the right-hand side of (\ref{ev19}) to lowest order in the small parameter $a/a'=0.00257$, we obtain
 \begin{equation}
 \ddot{\bf r} +2\,\bomega\times\dot{\bf r}+\bomega\times (\bomega\times {\bf r})
 \simeq - n^{\,2}\,a^{\,3}\,\frac{{\bf r}}{|{\bf r}|^{\,3}} + \frac{n'^{\,2}\,a'^{\,3}}{|{\bf r'}|^{\,3}}\left[\frac{(3\,{\bf r}\cdot{\bf r}')\,{\bf r}'}{|{\bf r}'|^{\,2}} - {\bf r}\right].
 \end{equation}
 When written in terms of Cartesian coordinates, the above equation yields
 \begin{eqnarray}
 \ddot{x}_1 -2\,n\,\dot{y}_1-\left(n^2+n'^{\,2}/2\right)x_1&\simeq&-n^2\,a^3\,\frac{x_1}{r^3}+\frac{3}{2}\,n'^{\,2}\,\cos[2\,(n-n')\,t]\,x_1\nonumber\\[0.5ex]
 &&-\frac{3}{2}\,n'^{\,2}\,
 \sin[2\,(n-n')\,t]\,y_1,\label{ev21}\\[0.5ex]
 \ddot{y}_1+2\,n\,\dot{x}_1-\left(n^2+n'^{\,2}/2\right)y_1&\simeq&-n^2\,a^3\,\frac{y_1}{r^3}-\frac{3}{2}\,n'^{\,2}\,\sin[2\,(n-n')\,t]\,x_1\nonumber\\[0.5ex]
 &&-\frac{3}{2}\,n'^{\,2}\,
 \cos[2\,(n-n')\,t]\,y_1,\\[0.5ex]
 \ddot{z}_1 + n'^{\,2}\,z_1 &\simeq& - n^2\,a^3\,\frac{z_1}{r^3},\label{ev23}
 \end{eqnarray}
 where $r=(x_1^{\,2}+y_1^{\,2}+z_1^{\,2})^{1/2}$, and use has been made of Equations~(\ref{evxx})--(\ref{evzz}).
 
 It is convenient, at this stage, to normalize all lengths to $a$, and all times to $n^{-1}$. Accordingly, let 
 \begin{eqnarray}
 X&=&x_1/a,\label{evxxx}\\[0.5ex] 
 Y&=&y_1/a,\\[0.5ex]
 Z&=&z_1/a, \label{evzzz}
 \end{eqnarray}
 and $r/a=R=(X^2+Y^2+Z^2)^{1/2}$, and $T = n\,t$.
 In  normalized form, Equations~(\ref{ev21})--(\ref{ev23}) become
 \begin{eqnarray}\label{ev24}
 \ddot{X} -2\,\dot{Y}-(1+m^2/2)\,X&\simeq&-\frac{X}{R^3}+\frac{3}{2}\,m^{2}\,\cos[2\,(1-m)\,T]\,X\nonumber\\[0.5ex]
 &&-\frac{3}{2}\,m^{2}\,
 \sin[2\,(1-m)\,T]\,Y,\\[0.5ex]
 \ddot{Y}+2\,\dot{X}-(1+m^2/2)\,Y&\simeq&-\frac{Y}{R^3}-\frac{3}{2}\,m^{2}\,\sin[2\,(1-m)\,T]\,X\nonumber\\[0.5ex]
 &&-\frac{3}{2}\,m^{2}\,
 \cos[2\,(1-m)\,T]\,Y,\\[0.5ex]
 \ddot{Z} + m^2\,Z &\simeq& -\frac{Z}{R^3},\label{ev26}
 \end{eqnarray}
 respectively, 
 where $m=n'/n=0.07480$ is a measure of the perturbing influence of the Sun on the lunar orbit. Here, $\ddot{~}\equiv
 d^2/dT^2$ and $\dot{~}\equiv d/dT$. 
 
 Finally, let us write
 \begin{eqnarray}\label{ev27}
 X &=& X_0 + \delta X,\\[0.5ex]
Y &=& \delta Y,\\[0.5ex]
Z &=& \delta Z,\label{ev29}
 \end{eqnarray}
 where $X_0=(1+m^2/2)^{-1/3}$, and $|\delta X|$, $|\delta Y|$, $|\delta Z|\ll X_0$. Thus, if the lunar orbit were a circle,
 centered on the Earth, and lying in the ecliptic plane,
 then, in the rotating frame $S_1$, the Moon would appear stationary at the point $(X_0$, $0,$ $0)$. 
 Expanding Equations~(\ref{ev24})--(\ref{ev26}) to {\em second-order}\/ in $\delta X$, $\delta Y$,
  $\delta Z$, and neglecting terms of order $m^4$ and $m^2\,\delta X^{\,2}$, {\em etc.}, we obtain
 \begin{eqnarray}\label{ev30}
 \delta \ddot{X}-2\,\delta \dot{Y} - 3\,(1+m^2/2)\,\delta X &\simeq& \frac{3}{2}\,m^2\,\cos[2\,(1-m)\,T]+\frac{3}{2}\,m^2\,\cos[2\,(1-m)\,T]\,\delta X\nonumber\\[0.5ex]
 &&-\frac{3}{2}\,m^2\,\sin[2\,(1-m)\,T]\,\delta Y-3\,\delta X^{\,2} + \frac{3}{2}\,(\delta Y^{\,2} + \delta Z^{\,2}),\nonumber\\[0.5ex]&&\\[0.5ex]
 \delta \ddot{Y}+2\,\delta \dot{X} &\simeq& -\frac{3}{2}\,m^2\,\sin[2\,(1-m)\,T]-\frac{3}{2}\,m^2\,\sin[2\,(1-m)\,T]\,\delta X
\nonumber\\[0.5ex]
 &&-\frac{3}{2}\,m^2\,\cos[2\,(1-m)\,T]\,\delta Y +3\,\delta X\,\delta Y,\\[0.5ex]
 \delta\ddot{Z} + (1+3\,m^2/2)\,\delta Z &\simeq & 3\,\delta X\,\delta Z.\label{ev32}
 \end{eqnarray}
 
Now,  once the above three equations have been solved for $\delta X$, $\delta Y$, and $\delta Z$, the Cartesian coordinates, $x$, $y$, $z$, of the Moon in the non-rotating geocentric
 frame $S$ are obtained from Equations~(\ref{ev10})--(\ref{ev12}), (\ref{evxxx})--(\ref{evzzz}),  and (\ref{ev27})--(\ref{ev29}). However, it is  more convenient to write
 $x=r\,\cos\theta$, $y=r\,\sin\theta$, and $z=r\,\sin\beta$, where $r$ is the {\em radial distance}\/ between the Earth and Moon, and $\theta$ and $\beta$ are termed the Moon's {\em ecliptic
 longitude}\/ and {\em ecliptic latitude}, respectively. Moreover, it is easily seen that, to second-order in $\delta X$, $\delta Y$,
  $\delta Z$, and neglecting terms of order $m^4$, 
 \begin{eqnarray}\label{ev33}
 \frac{r}{a}-1+\frac{m^2}{6}&\simeq&\delta X +\frac{1}{2}\,\delta Y^{\,2}+\frac{1}{2}\,\delta Z^{\,2},\\[0.5ex]
 \theta - n\,t&\simeq&  \delta Y - \delta X\,\delta Y,\\[0.5ex]
 \beta&\simeq& \delta Z-\delta X\,\delta Z.\label{ev35}
 \end{eqnarray}
 
 \section{Unperturbed Lunar Motion}
 Let us, first of all,  neglect the perturbing influence of the Sun on the Moon's orbit by setting $m =0$ in the lunar equations of motion (\ref{ev30})--(\ref{ev32}). For the sake of simplicity, let us also neglect nonlinear effects in these equations by setting $\delta X^2=\delta Y^2=\delta Z^2=\delta X\,\delta Y=\delta X\,\delta Z=0$. In this case, the equations reduce to
  \begin{eqnarray}\label{ev36}
 \delta \ddot{X}-2\,\delta \dot{Y} - 3\,\delta X &\simeq& 0,\\[0.5ex]
 \delta \ddot{Y}+2\,\delta \dot{X} &\simeq& 0,\\[0.5ex]
 \delta\ddot{Z} + \delta Z &\simeq &0.\label{ev38}
 \end{eqnarray}
 By inspection, appropriate solutions are
 \begin{eqnarray}
 \delta X &\simeq &-e\,\cos(T-\alpha_0),\label{ev42x}\\[0.5ex]
 \delta Y &\simeq &2\,e\,\sin(T-\alpha_0),\\[0.5ex]
 \delta Z &\simeq &i\,\sin(T-\gamma_0),\label{ev44x}
 \end{eqnarray}
 where $e$, $\alpha_0$, $i$, and $\gamma_0$ are arbitrary constants. Recalling that $T=n\,t$, it follows from (\ref{ev33})--(\ref{ev35})
  that
 \begin{eqnarray}\label{ev42}
 r &\simeq&a\left[1-e\,\cos(n\,t-\alpha_0)\right],\\[0.5ex]
 \theta &\simeq& n\,t+2\,e\,\sin(n\,t-\alpha_0),\label{ev43}\\[0.5ex]
 \beta &\simeq & \iota\,\sin(n\,t-\gamma_0).\label{ev44}
 \end{eqnarray}
 However,  Equations~(\ref{ev42}) and (\ref{ev43}) are simply first-order (in $e$) approximations to the
 familiar Keplerian laws (see Chapter~\ref{skepler})
 \begin{eqnarray}
 r &=&\frac{a\,(1-e^2)}{1+e\,\cos(\theta-\alpha_0)},\\[0.5ex]
 r^2\,\dot{\theta} &=& (1-e^2)^{1/2}\,n\,a^2,
 \end{eqnarray}
 where $\dot{~}\equiv d/dt$. 
 Of course,  the above two laws describe a  body
 which executes an {\em elliptical}\/ orbit, confocal with the Earth, of major radius $a$, mean angular velocity $n$, and eccentricity
 $e$, such that the radius vector connecting the body to the Earth sweeps out equal areas in equal time intervals.
 We conclude, unsurprisingly, that the unperturbed lunar orbit is a Keplerian ellipse.
Note that the lunar {\em perigee}\/ lies at the  fixed ecliptic longitude $\theta=\alpha_0$. Equation (\ref{ev44}) is the first-order approximation to
 \begin{equation}
 \beta= \iota\,\sin(\theta-\gamma_0).
 \end{equation}
 This expression implies that the unperturbed lunar orbit is {\em co-planar}, but is {\em inclined}\/ at an angle $\iota$ to the ecliptic plane.
Moreover, the {\em ascending node}\/  lies at the fixed ecliptic  longitude $\theta=\gamma_0$.  Incidentally, the neglect of nonlinear terms in Equations~(\ref{ev36})--(\ref{ev38}) is only valid as long as $e$, $\iota\ll 1$: {\em i.e.}, provided that the unperturbed lunar orbit is only
 slightly elliptical, and slightly inclined to the ecliptic plane. In fact, the observed values of $e$ and
 $\iota$ are $0.05488$ and $0.09008$ radians, respectively, so this is indeed the case.
 
 \section{Perturbed Lunar Motion}
 The perturbed nonlinear lunar equations of motion, (\ref{ev30})--(\ref{ev32}), take the general form
 \begin{eqnarray}\label{ev51}
 \delta \ddot{X}-2\,\delta \dot{Y} - 3\,(1+m^2/2)\,\delta X &\simeq& R_X,\\[0.5ex]
 \delta \ddot{Y}+2\,\delta \dot{X} &\simeq&R_Y,\\[0.5ex]
 \delta\ddot{Z} + (1+3\,m^2/2)\,\delta Z &\simeq &R_Z,\label{ev53}
 \end{eqnarray}
 where 
 \begin{eqnarray}
 R_X &=& a_0 + \sum_{j>0}a_j\,\cos(\omega_j\,T-\alpha_j),\label{ev54}\\[0.5ex]
 R_Y &=&\sum_{j>0} b_j\,\sin(\omega_j\,T-\alpha_j),\\[0.5ex]
 R_Z &=& \sum_{j>0} c_j\,\sin(\Omega_j\,T-\gamma_j).\label{ev56}
 \end{eqnarray}
 Let us search for solutions of the general form
 \begin{eqnarray}
 \delta X &=& x_0 + \sum_{j>0}x_j\,\cos(\omega_j\,T-\alpha_j),\label{ev57}\\[0.5ex]
 \delta Y &=&\sum_{j>0} y_j\,\sin(\omega_j\,T-\alpha_j),\\[0.5ex]
 \delta Z &=&\sum_{j>0}z_j\,\sin(\Omega_j\,T-\gamma_j).\label{ev59}
 \end{eqnarray}
 Substituting expressions (\ref{ev54})--(\ref{ev59}) into Equations~(\ref{ev51})--(\ref{ev53}), it is easily demonstrated that
 \begin{eqnarray}\label{ev60}
 x_0 &=& -\frac{a_0}{3\,(1+m^2/2)},\\[0.5ex]
 x_j &=& \frac{\omega_j\,a_j-2\,b_j}{\omega_j\,(1-3\,m^2/2-\omega_j^{\,2})},\label{ev61}\\[0.5ex]
 y_j &=&\frac{(\omega_j^{\,2}+ 3+3\,m^2/2)\,b_j - 2\,\omega_j\,a_j}{\omega_j^{\,2}\,(1-3\,m^2/2-\omega_j^{\,2})},\label{ev62}\\[0.5ex]
 z_j &=& \frac{c_j}{1+3\,m^2/2-\Omega_j^{\,2}},\label{ev63}
 \end{eqnarray}
 where $j>0$. 
 
 \begin{table}\centering
 \begin{tabular}{c|cccc}\hline
 $j$ & $\omega_j$  &$\alpha_j$ & $\Omega_j$ & $\gamma_j$\\[0.5ex]\hline
 &&&&\\[-2ex]
 $1$ & $1+c\,m^2$ & $\alpha_0$  & $1+g\,m^2$ & $\gamma_0$\\[0.5ex]
 $2$ & $2\,(1+c\,m^2)$ & $2\,\alpha_0$ & $(c-g)\,m^2$ & $\alpha_0-\gamma_0$\\[0.5ex]
 $3$ & $2\,(1+g\,m^2)$ & $2\,\gamma_0$ & $2+(c+g)\,m^2$ & $\alpha_0+\gamma_0$\\[0.5ex]
 $4$ & $2-2\,m$ & $0$  &&\\[0.5ex]
 $5$ & $1-2\,m-c\,m^2$ & $-\alpha_0$ & $1-2\,m-g\,m^2$ & $-\gamma_0$
 \end{tabular}
 \caption{\em Angular frequencies and phase-shifts associated with the principal periodic driving terms appearing in
 the perturbed nonlinear lunar equations of motion.}\label{tv1}
 \end{table}
 
 The angular frequencies, $\omega_j$, $\Omega_j$, and phase shifts, $\alpha_j$, $\gamma_j$, of the principal periodic driving terms 
 that appear  on the right-hand sides of the perturbed nonlinear lunar equations of motion, (\ref{ev51})--(\ref{ev53}), are
 specified in Table~\ref{tv1}. Here, $c$ and $g$ are, as yet, unspecified ${\cal O}(1)$ constants associated with
 the precession of the lunar perigee, and the regression of the ascending node, respectively. 
 Note that $\omega_1$ and $\Omega_1$ are the  frequencies of the Moon's unforced motion in
 ecliptic longitude and latitude, respectively. Moreover, $\omega_4$ is the forcing frequency associated with the
 perturbing influence of the Sun. All other frequencies appearing in Table~\ref{tv1} are combinations of these
 three fundamental frequencies. In fact, $\omega_2=2\,\omega_1$, $\omega_3=2\,\Omega_1$, 
 $\omega_5=\omega_4-\omega_1$, $\Omega_2=\omega_1-\Omega_1$, $\Omega_3=\omega_1+\Omega_1$,
 and $\Omega_5=\omega_4-\Omega_1$.  Note that there is no $\Omega_4$.
 
 Now, a comparison of Equations (\ref{ev30})--(\ref{ev32}), (\ref{ev51})--(\ref{ev53}), and Table~\ref{ev1} reveals that
 \begin{eqnarray}
 R_X &=& \frac{3}{2}\,m^2\,\cos(\omega_4\,T-\alpha_4)+\frac{3}{2}\,m^2\,\cos(\omega_4\,T-\alpha_4)\,\delta X\nonumber\\[0.5ex]
 &&-\frac{3}{2}\,m^2\,\sin(\omega_4\,T-\alpha_4)\,\delta Y-3\,\delta X^{\,2} + \frac{3}{2}\,(\delta Y^{\,2}+\delta Z^{\,2}),\\[0.5ex]
 R_Y &=&- \frac{3}{2}\,m^2\,\sin(\omega_4\,T-\alpha_4)-\frac{3}{2}\,m^2\,\sin(\omega_4\,T-\alpha_4)\,\delta X\nonumber\\[0.5ex]
 &&-\frac{3}{2}\,m^2\,\cos(\omega_4\,T-\alpha_4)\,\delta Y +3\,\delta X\,\delta Y,\\[0.5ex]
 R_Z &=& 3\,\delta X\,\delta Z.
 \end{eqnarray}
 Substitution of the solutions (\ref{ev57})--(\ref{ev59}) into the above equations, followed by a comparison with expressions (\ref{ev54})--(\ref{ev56}),
yields  the amplitudes $a_j$, $b_j$, and $c_j$  specified in Table~\ref{tv2}. Note that, in calculating these amplitudes,
  we have neglected all contributions  to the periodic driving terms, appearing in  Equations~(\ref{ev51})--(\ref{ev53}) which  involve cubic, or higher order, combinations of $e$, $\iota$,  $m^2$, $x_j$, $y_j$, and $z_j$, since we only expanded Equations~(\ref{ev30})--(\ref{ev32})  to {\em second-order}\/ in $\delta X$, $\delta Y$, and $\delta Z$. 
 
 \begin{table}
 \centering
 \begin{tabular}{c|ccc}\hline
 $j$ & $a_j$ & $b_j$ & $c_j$\\[0.5ex]\hline
 &&&\\[-2ex]
 $0$ & $\frac{3}{2}\,e^2+\frac{3}{4}\,\iota^2$ & &\\[0.5ex]
 $1$ & $\frac{3}{4}\,m^2\,x_5 - \frac{3}{4}\,m^2\,y_5-3\,x_4\,x_5+\frac{3}{2}\,y_4\,y_5$ & $-\frac{3}{4}\,m^2\,x_5
 +\frac{3}{4}\,m^2\,y_5+\frac{3}{2}\,y_4\,x_5-\frac{3}{2}\,y_5\,x_4$ & $-\frac{3}{2}\,x_4\,z_5$\\[0.5ex]
 $2$ & $-\frac{9}{2}\,e^2$ & $-3\,e^2$ & $\frac{3}{2}\,e\,\iota$\\[0.5ex]
 $3$ & $-\frac{3}{4}\,\iota^2$ & $0$ & $-\frac{3}{2}\,e\,\iota$\\[0.5ex]
 $4$ & $\frac{3}{2}\,m^2$ & $-\frac{3}{2}\,m^2$ & 0\\[0.5ex]
 $5$ & $-\frac{9}{4}\,m^2\,e + 3\,e\,x_4+3\,e\,y_4$ & $\frac{9}{4}\,m^2\,e-3\,e\,x_4 - \frac{3}{2}\,e\,y_4$ &$-\frac{3}{2}\,\iota\,x_4$
 \end{tabular}
 \caption{\em Amplitudes of the periodic driving terms appearing in the perturbed nonlinear lunar equations of motion.}\label{tv2}
 \end{table}
 
 For $j=0$, it follows from Equation~(\ref{ev60}) and Table~\ref{tv2} that
 \begin{equation}
 x_0 = -\frac{1}{2}\,e^2 -\frac{1}{4}\,\iota^{\,2}.
 \end{equation}
 
 For $j=2$, making the approximation $\omega_2\simeq 2$ (see Table~\ref{tv1}), it follows from Equations (\ref{ev61}), (\ref{ev62})
 and Table~\ref{tv2} that
 \begin{eqnarray}
 x_2&\simeq & \frac{1}{2}\,e^2,\\[0.5ex]
 y_2 &\simeq & \frac{1}{4}\,e^2.
 \end{eqnarray}
 Likewise, making the approximation $\Omega_2\simeq 0$  (see Table~\ref{tv1}), it follows from Equation (\ref{ev63})
  and Table~\ref{tv2} that
 \begin{equation}
 z_2\simeq  \frac{3}{2}\,e\,\iota.
 \end{equation}
 
 For $j=3$, making the approximation $\omega_3\simeq 2$ (see Table~\ref{tv1}), it follows from Equations  (\ref{ev61}), (\ref{ev62})
 and Table~\ref{tv2} that
 \begin{eqnarray}
 x_3&\simeq & \frac{1}{4}\,\iota^2,\\[0.5ex]
 y_3 &\simeq & -\frac{1}{4}\,\iota^2.
 \end{eqnarray}
 Likewise, making the approximation $\Omega_3\simeq 2$  (see Table~\ref{tv1}), it follows from Equation (\ref{ev63})
  and Table~\ref{tv2} that
 \begin{equation}
 z_3\simeq  \frac{1}{2}\,e\,\iota.
 \end{equation}
 
 For $j=4$, making the approximation $\omega_4\simeq 2$ (see Table~\ref{tv1}), it follows from Equations (\ref{ev61}), (\ref{ev62})
 and Table~\ref{tv2} that
 \begin{eqnarray}
 x_4&\simeq &-m^2,\\[0.5ex]
 y_4 &\simeq & \frac{11}{8}\,m^2.
 \end{eqnarray}
 Thus, according to Table~\ref{tv2}, 
 \begin{eqnarray}
 a_5 &\simeq&-\frac{9}{8}\,m^2\,e,\label{ev76}\\[0.5ex]
 b_5 &\simeq&\frac{51}{16}\,m^2\,e,\label{ev77}\\[0.5ex]
 c_5 &\simeq & \frac{3}{2}\,m^2\,\iota.\label{ev78}
 \end{eqnarray}
 
 For $j=5$, making the approximation $\omega_5\simeq 1-2\,m$ (see Table~\ref{tv1}), it follows from Equations (\ref{ev61}), (\ref{ev62}), (\ref{ev76}), and (\ref{ev77}) that
 \begin{eqnarray}
 x_5&\simeq & -\frac{15}{8}\,m\,e,\\[0.5ex]
 y_5 &\simeq & \frac{15}{4}\,m\,e.
 \end{eqnarray}
 Likewise, making the approximation $\Omega_5\simeq 1-2\,m$  (see Table~\ref{tv1}), it follows from Equations  (\ref{ev63})
  and (\ref{ev78}) that
 \begin{equation}
 z_5\simeq  \frac{3}{8}\ m\,\iota.
 \end{equation}
 Thus, according to Table~\ref{tv2}, 
 \begin{eqnarray}
 a_1 &=&-\frac{135}{64}\,m^3\,e,\label{ev82}\\[0.5ex]
 b_1 &=&\frac{765}{128}\,m^3\,e,\label{ev83}\\[0.5ex]
 c_1 &=& \frac{9}{16}\,m^3\,\iota.\label{ev84}
 \end{eqnarray}
 
 Finally, for $j=1$, by analogy with Equations~(\ref{ev42x})--(\ref{ev44x}), we expect
 \begin{eqnarray}
 a_1 &=& -e,\label{ev85}\\[0.5ex]
 b_1 &=& 2\,e,\label{ev86}\\[0.5ex]
 c_1 &=& \iota.\label{ev87}
 \end{eqnarray}
 Thus, since $\omega_1=1+c\,m^2$ (see Table~\ref{tv1}), it follows from Equations~(\ref{ev61}), (\ref{ev82}),  (\ref{ev83}),
 and (\ref{ev85}) 
 that
 \begin{equation}
 -e \simeq \frac{ - (225/16)\,m^3\,e}{-(3/2)\,m^2-2\,c\,m^2},
 \end{equation}
 which yields
 \begin{equation}\label{ev89}
 c \simeq- \frac{3}{4} - \frac{225}{32}\,m + {\cal O}(m^2).
 \end{equation}
 Likewise, since 
$\Omega_1=1+g\,m^2$ (see Table~\ref{tv1}), it follows from Equations~(\ref{ev63}), (\ref{ev84}),   and (\ref{ev87})
 that
 \begin{equation}
 \iota \simeq \frac{ (9/16)\,m^3\,\iota}{(3/2)\,m^2-2\,g\,m^2},
 \end{equation}
 which yields
 \begin{equation}\label{ev91}
 g \simeq \frac{3}{4} + \frac{9}{32}\,m + {\cal O}(m^2).
 \end{equation}
 
 According to the above analysis, our final expressions for $\delta X$, $\delta Y$, and $\delta Z$  are
 \begin{eqnarray}
 \delta X &=& -\frac{1}{2}\,e^2-\frac{1}{4}\,\iota^2 - e\,\cos[(1+c\,m^2)\,T-\alpha_0] + \frac{1}{2}\,e^2\,\cos[2\,(1+c\,m^2)\,T-2\,\alpha_0]\nonumber\\[0.5ex]
 &&+ \frac{1}{4}\,\iota^2\,\cos[2\,(1+g\,m^2)\,T-2\,\gamma_0] - m^2\,\cos[2\,(1-m)\,T]\nonumber\\[0.5ex]
 &&-\frac{15}{8}\,m\,e\,\cos[(1-2\,m-c\,m^2)\,T+\alpha_0],\\[0.5ex]
  \delta Y &=& 2\,e\,\sin[(1+c\,m^2)\,T-\alpha_0] + \frac{1}{4}\,e^2\,\sin[2\,(1+c\,m^2)\,T-2\,\alpha_0]\nonumber\\[0.5ex]
 &&- \frac{1}{4}\,\iota^2\,\sin[2\,(1+g\,m^2)\,T-2\,\gamma_0] + \frac{11}{8}\,m^2\,\cos[2\,(1-m)\,T]\nonumber\\[0.5ex]
&&+\frac{15}{4}\,m\,e\,\sin[(1-2\,m-c\,m^2)\,T+\alpha_0],\\[0.5ex]
\delta Z &=& \iota\,\sin[(1+g\,m^2)\,T-\gamma_0] + \frac{3}{2}\,e\,\iota\,\sin[(c-g)\,m^2\,T-\alpha_0+\gamma_0]\nonumber\\[0.5ex]
&&+\frac{1}{2}\,e\,\iota\,\sin [(2+c\,m^2+g\,m^2)\,T-\alpha_0-\gamma_0]\nonumber\\[0.5ex]
&&+\frac{3}{8}\,m\,\iota\,\sin[(1-2\,m-g\,m^2)\,T+\gamma_0].
 \end{eqnarray}
 Thus, making use of Equations~(\ref{ev33})--(\ref{ev35}), we find that
 \begin{eqnarray}\label{ev95}
 \frac{r}{a} &=&1  - e\,\cos[(1+c\,m^2)\,T-\alpha_0] +\frac{1}{2}\,e^2 - \frac{1}{6}\,m^2- \frac{1}{2}\,e^2\,\cos[2\,(1+c\,m^2)\,T-2\,\alpha_0]\nonumber\\[0.5ex]
 &&- m^2\,\cos[2\,(1-m)\,T]-\frac{15}{8}\,m\,e\,\cos[(1-2\,m-c\,m^2)\,T+\alpha_0],\\[0.5ex]
  \theta &=& T + 2\,e\,\sin[(1+c\,m^2)\,T-\alpha_0] + \frac{5}{4}\,e^2\,\sin[2\,(1+c\,m^2)\,T-2\,\alpha_0]\nonumber\\[0.5ex]
 &&- \frac{1}{4}\,\iota^2\,\sin[2\,(1+g\,m^2)\,T-2\,\gamma_0] + \frac{11}{8}\,m^2\,\sin[2\,(1-m)\,T]\nonumber\\[0.5ex]
&&+\frac{15}{4}\,m\,e\,\sin[(1-2\,m-c\,m^2)\,T+\alpha_0],\label{ev96}\\[0.5ex]
\beta &=& \iota\,\sin[(1+g\,m^2)\,T-\gamma_0] + e\,\iota\,\sin[(c-g)\,m^2\,T-\alpha_0+\gamma_0]\nonumber\\[0.5ex]
&&+e\,\iota\,\sin [(2+c\,m^2+g\,m^2)\,T-\alpha_0-\gamma_0]\nonumber\\[0.5ex]
&&+\frac{3}{8}\,m\,\iota\,\sin[(1-2\,m-g\,m^2)\,T+\gamma_0].\label{ev97}
 \end{eqnarray}
 The above expressions are accurate up to {\em second-order}\/ in the small parameters $e$, $\iota$, and $m$. 
 
 \section{Description of Lunar Motion}
 In order to better understand the perturbed lunar motion derived in the previous section, it is helpful to introduce the concept of 
 the {\em mean moon}. This is an imaginary body which orbits the Earth, in the ecliptic plane,  at a
 {\em steady}\/ angular velocity that is equal to the Moon's mean orbital angular velocity, $n$. Likewise, the
 {\em mean sun}\/ is a second imaginary body which orbits the Earth, in the ecliptic plane, at a
 {\em steady}\/ angular velocity that is equal to the Sun's mean orbital angular velocity, $n'$. Thus, the
 ecliptic longitudes of the mean moon and the mean sun are
 \begin{eqnarray}
 \bar{\theta} &=& n\,t,\\[0.5ex]
 \bar{\theta}' &=&n'\,t,
 \end{eqnarray}
 respectively. Here, for the sake of simplicity, and also for the sake of consistency with our
 previous analysis, we have assumed that both objects are located at ecliptic longitude $0^\circ$ at time $t=0$. 

 Now, from Equation~(\ref{ev95}), to first-order in small parameters, the lunar perigee corresponds  to $(1+c\,m^2)\,n\,t -\alpha_0=j\,2\pi$, where
 $j$ is an integer. 
 However, this condition can also be written $\bar{\theta}=\alpha$, where
 \begin{equation}\label{ev101}
 \alpha = \alpha_0 + \alpha_1\,n'\,t,
 \end{equation}
 and, making use of Equation~(\ref{ev89}),  together with the definition $m=n'/n$, 
 \begin{equation}
  \alpha_1 = \frac{3}{4}\,m+ \frac{255}{32}\,m^2 + {\cal O}(m^3).
  \end{equation}
  Thus, we can identify $\alpha$ as the mean ecliptic longitude of the perigee. Moreover, according to Equation~(\ref{ev101}),
   the  perigee {\em precesses}\/  ({\em i.e.},  its longitude increases  in time) at the mean  rate
  of  $360\,\alpha_1$ degrees per year. (Of course, a year corresponds to $\Delta t=2\pi/n'$.) Furthermore,  it is clear that this precession is entirely due to the perturbing influence of the Sun, since it only depends on the parameter $m$,
  which is a measure of this influence. Given that $m=0.07480$, we find that the perigee advances by
  $34.36^\circ$ degrees per year. Hence, we predict that the perigee completes a full circuit about the Earth every $1/\alpha_1 = 10.5$ years.
  In fact, the lunar perigee completes a full circuit every $8.85$ years. Our prediction is somewhat inaccurate because
  our previous analysis neglected ${\cal O}(m^2)$,  and smaller, contributions to the parameter $c$ [see Equation~(\ref{ev89})],
  and these turn out to be significant. 
  
 From Equation~(\ref{ev97}), to first-order in small parameters, the Moon passes through its ascending node  when $(1+g\,m^2)\,n\,t -\gamma_0=\gamma_0+j\,2\pi$, where
 $j$ is an integer. 
 However, this condition can also be written $\bar{\theta}=\gamma$, where
 \begin{equation}\label{ev103}
 \gamma = \gamma_0 - \gamma_1\,n'\,t,
 \end{equation}
 and, making use of Equation~(\ref{ev91}), 
 \begin{equation}
  \gamma_1 = \frac{3}{4}\,m - \frac{9}{32}\,m^2 + {\cal O}(m^3).
  \end{equation}
  Thus, we can identify $\gamma$ as the mean ecliptic longitude of the ascending node. Moreover, according to Equation~(\ref{ev103}),
   the  ascending node {\em regresses}\/  ({\em i.e.},  its longitude decreases  in time) at the mean rate
  of  $360\,\gamma_1$ degrees per year.  As before,  it is clear that this regression is entirely due to the perturbing influence of the Sun.  Moreover, we   find that the ascending node retreats by
  $19.63^\circ$ degrees per year. Hence, we predict that the ascending node completes a full circuit about the Earth every $1/\gamma_1=18.3$ years.
  In fact, the lunar ascending node completes a full circuit every $18.6$ years, so our prediction is fairly accurate. 
  
  
  It is  helpful to introduce the lunar {\em mean anomaly}, 
  \begin{equation}
  M = \bar{\theta} - \alpha,
  \end{equation}
  which is defined as the  angular distance (in longitude) between the mean Moon and the perigee.
  It is also helpful to introduce the lunar {\em mean argument of latitude}, 
  \begin{equation}
  F = \bar{\theta}-\gamma,
  \end{equation}
  which is
  defined as the angular distance (in longitude) between the mean Moon and the ascending node.
   Finally, it is helpful to introduce the {\em mean elongation}\/ of the Moon,
 \begin{equation}
 D = \bar{\theta}-\bar{\theta}',
 \end{equation}
 which is defined as the difference between the longitudes of the mean Moon and the mean Sun.
  
  When expressed in terms of $M$, $F$, and $D$, our previous expression (\ref{ev96}) for the true ecliptic
  longitude of the Moon becomes
  \begin{equation}
  \theta = \bar{\theta}+\lambda,
  \end{equation}
  where
  \begin{eqnarray}\label{ev108}
   \lambda &=&  2\,e\,\sin M+ \frac{5}{4}\,e^2\,\sin 2\,M
- \frac{1}{4}\,\iota^2\,\sin 2\,F+ \frac{11}{8}\,m^2\,\sin 2\,D+\frac{15}{4}\,m\,e\,\sin(2\,D-M)\nonumber\\[0.5ex]&&
\end{eqnarray}
is the angular distance (in longitude) between the Moon and the mean Moon. The first three terms on the
right-hand side of the above expression are Keplerian ({\em i.e.}, they are independent of the perturbing
action of the Sun). In fact, the first is due to the eccentricity of the lunar orbit ({\em i.e.}, the fact that the
geometric center of the orbit is slightly shifted from the Earth),  the second is due to
the ellipticity of the orbit ({\em i.e.}, the fact that the orbit is slightly non-circular), and the third
is due to the slight inclination of the  orbit to the ecliptic plane. However, the final two
terms are caused by the perturbing action of the Sun. In fact, the fourth term corresponds to
{\em variation}, whilst the fifth corresponds to {\em evection}. Note that variation attains its maximal
amplitude around the so-called {\em octant points}, at which the
Moon's disk  is either one-quarter or three-quarters illuminated ({\em i.e.}, when $D=45^\circ$, $135^\circ$, 
$225^\circ$, or $315^\circ$). Conversely, the amplitude of variation is zero around the so-called {\em quadrant points},
at which  the Moon's disk is either fully illuminated, half illuminated, or not illuminated at all ({\em i.e.}, when $D=0^\circ$, $90^\circ$, 
$180^\circ$, or $270^\circ$). Evection can be thought of as causing a slight reduction in the eccentricity of the lunar
orbit around the times of the new moon and the full moon ({\em i.e.}, $D=0^\circ$ and $D=180^\circ$), and
causing a corresponding slight increase in the eccentricity around the times of the first and last quarter moons
 ({\em i.e.}, $D=90^\circ$ and $D=270^\circ$). This follows because the evection term in Equation~(\ref{ev108}) augments the eccentricity
 term, $2\,e\,\sin M$, when $\cos 2D=-1$, and reduces the term when $\cos 2D = +1$.
 The variation and evection terms appearing in expression (\ref{ev108}) oscillate sinusoidally with periods of half a synodic month,\footnote{A
 synodic month, which is $29.53$ days, is the mean period between successive new moons.}
 or $14.8$ days, and $31.8$ days, respectively. These periods are in good agreement with observations.
 Finally, the amplitudes of the variation and evection terms  (calculated using $m=0.07480$ and $e=0.05488$)
 are $1630$ and $3218$ arc seconds, respectively. However, the observed amplitudes are 
 $2370$ and $4586$ arc seconds, respectively. It turns out that our expressions for these amplitudes  are somewhat inaccurate because, for the
 sake of simplicity, we have only calculated the lowest order (in $m$) contributions to them. Recall that we  also
 neglected the slight eccentricity, $e'=0.016711$, of the Sun's apparent orbit about the
Earth in our calculation. In fact, the eccentricity of the solar orbit gives rise to a small addition term $-3\,m\,e'\,\sin M'$ on the right-hand side of (\ref{ev108}),
where $M'$ is the Sun's mean anomaly. This term, which is known as the {\em annual equation}, oscillates with a period
of a solar year, and has an amplitude of $772$ arc seconds. 

When expressed in terms of $D$ and $F$ our previous expression (\ref{ev97}) for the ecliptic latitude of the Moon
becomes
\begin{equation}
\beta = \iota\,\sin(F+\lambda) + \frac{3}{8}\,m\,\iota\,\sin(2\,D-F).
\end{equation}
The first term on the right-hand side of this expression is Keplerian ({\em i.e.}, it is independent of the perturbing
influence of the Sun). However, the final term, which is known as {\em evection in latitude}, is due to the
Sun's action. Evection in latitude can be thought of as causing a slight increase in the inclination of the lunar orbit
to the ecliptic at the times of the first and last quarter moons, and a slight decrease at the times of the new moon and
the full moon. The evection in latitude term oscillates sinusoidally with a period of 32.3 days, and has
an amplitude of 521 arc seconds. This period is in good agreement with observations. However, the observed
amplitude of the evection in latitude term is 624 arc seconds.  As before, our expression for the amplitude  is somewhat inaccurate because we have only calculated the lowest order (in $m$) contribution.