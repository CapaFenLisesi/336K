\chapter{Introduction}
\section{Intended Audience}
This book presents a single semester course  on Newtonian dynamics that is intended primarily for upper-division ({\em i.e.}, junior and senior) undergraduate students majoring in physics. A thorough  understanding
of physics at the lower-division level, including a basic working
knowledge of the laws of mechanics, is assumed. It is
also taken for granted that students are
 familiar with the fundamentals of integral and differential
calculus,  complex analysis,  ordinary differential equations, and linear algebra. 
On the other hand, vector analysis plays such a central role in the study of Newtonian
dynamics that a brief, but fairly comprehensive, 
review of this subject area is provided in Appendix~\ref{vector}. 
Likewise, those results in matrix eigenvalue theory which are helpful in the analysis of  rigid body motion
and coupled oscillations are discussed in Sections~\ref{smatrix} and
\ref{smat2}, respectively. Finally, the calculus of variations, an area of mathematics that 
is central to Hamiltonian dynamics, is
outlined in Section~\ref{s11.2}.

\section{Scope of Book}
The scope of this book is indicated by its title, ``Newtonian Dynamics''.   ``Dynamics''
is the study of the {\em motions}\/ of the various objects in the
world around us. Furthermore, by ``Newtonian", we understand that the theory
 which we are actually going to employ in our investigation of dynamics is that which was first formulated by Sir  Isaac Newton
in 1687. 

For the sake of simplicity, and brevity, we shall restrict our investigations to the motions
of  idealized {\em point particles}\/ and idealized {\em rigid bodies}.  To be more exact, we shall exclude from consideration any discussion of statics, the strength of materials, and the non-rigid motions of continuous media. We shall also concentrate, for the most
part, on motions which take place  under the influence of {\em conservative forces}, such as gravity, which can be
accurately represented in terms of simple mathematical formulae. Finally, with one major exception, we shall only
consider that subset of dynamical problems that can be solved by means
of conventional mathematical analysis. 

Newtonian dynamics was originally developed in order to predict
the motions of the objects which make up the {\em Solar System}. It turns out that this is an
ideal application of the theory, since the objects in question can
be modeled as being rigid to a fair degree of accuracy, and the motions
take place under the action of a single conservative force---namely, gravity---that has a simple mathematical form. In particular, the frictional forces which greatly
complicate the application of Newtonian dynamics to the motions of
everyday objects close to the Earth's surface are completely absent.
Consequently, in this book  we shall make a particular effort to describe how  Newtonian dynamics
can successfully  account  for a wide variety of different solar system phenomena. For example, during the
course of this book, we shall explain the  origins of Kepler's laws
of planetary motion (see Chapter~\ref{skepler}),
the rotational flattening of the Earth, the tides,  the Roche radius ({\em i.e.}, the  minimum radius at which a moon can orbit a planet without being destroyed by tidal forces), the forced precession and nutation of the Earth's axis of rotation,
and the forced perihelion precession of the planets (see Chapter~\ref{spotn}). We
shall also derive the Tisserand criterion
used to re-identify comets whose orbits have been modified by close encounters with massive planets,  account for the existence of the so-called
Trojan asteroids which share the orbit of Jupiter (see Chapter~\ref{threeb}), and analyze the motion of the Moon (see Chapter~\ref{moon}).

Virtually all of the  results described in this book were first 
obtained---either by Newton himself, or by scientists living in the 150, or so,
years immediately following the initial publication of his theory---by means of
conventional mathematical analysis. Indeed,
scientists at the beginning of the 20th century generally assumed that
they knew everything that there was to known about
Newtonian dynamics. However, they were mistaken. The advent of
fast electronic computers, in the latter half of the 20th century, allowed
scientists to solve {\em nonlinear}\/ equations of motion, for the first time, via numerical techniques. In general, such equations are insoluble using
standard analytic methods. The numerical investigation  of dynamical systems with nonlinear equations of motion
revealed the existence of a previously unknown type of motion
known as {\em deterministic chaos}. Such motion is quasi-random (despite being derived from deterministic equations of motion),
aperiodic, and exhibits extreme sensitivity to initial conditions. The discovery of chaotic motion  lead to a  renaissance in the study of Newtonian dynamics 
which started in the late 20th century and  is still ongoing. It is therefore appropriate that the  last chapter in this book  is 
devoted to an in-depth numerical investigation of a particular dynamical
system that exhibits chaotic motion (see Chapter~\ref{schaos}).

\section{Major Sources}
The material appearing in Appendix~\ref{vector} is largely based on the author's recollections
of a vector analysis course given by Dr.~Stephen Gull at the University
of Cambridge. Major sources for the material appearing in Chapters~\ref{sfun}--\ref{threeb}
include  
{\em Dynamics}, H.~Lamb, 2nd Edition (Cambridge University Press, Cambridge UK, 1923);
{\em Analytical Mechanics}, G.R.~Fowles (Holt, Rinehart, and Winston,
New York NY, 1977); 
{\em Solar System Dynamics}, C.D.~Murray, and S.F.~Dermott (Cambridge University Press,
Cambridge UK, 1999);
{\em Classical Mechanics}, 3rd Edition, H.~Goldstein, C.~Poole, and J.~Safko (Addison-Wesley,
San Fransisco CA, 2002);
{\em Classical Dynamics of Particles and Systems}, 5th Edition, S.T.~Thornton, and
 J.B.~Marion (Brooks/Cole---Thomson Learning, Belmont CA, 2004); and
{\em Analytical Mechanics}, 7th Edition,  G.R.~Fowles, and G.L.~Cassiday
(Brooks/Cole---Thomson Learning, Belmont CA, 2005). The various sources for
the material appearing in Chapters \ref{moon} and \ref{schaos} are identified in footnotes.

