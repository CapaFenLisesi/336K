\chapter{Rotating Reference Frames}\label{snoni}
\section{Introduction}
As we saw in Chapter~\ref{sfun}, Newton's second law of motion is
only valid in {\em inertial}\/ frames of reference. Unfortunately, we are sometimes
forced to observe motion in {\em non-inertial}\/  reference frames. For instance,
it is most convenient for us to observe the motions of the objects in our
immediate vicinity
in a reference frame which is {\em fixed}\/ relative to the {\em surface of the Earth}.
Such a frame is {\em non-inertial}\/ in nature, since it
{\em accelerates}\/ with respect to a standard inertial frame due to the Earth's daily rotation  about
its axis. (Note that the accelerations of this frame due to the Earth's orbital motion about the Sun, or
the Sun's orbital motion about the galactic center, {\em etc}., are
negligible compared to the acceleration due to the Earth's axial rotation.) Let us now investigate
motion in a {\em rotating}\/ reference frame.

\section{Rotating Reference Frames}\label{srot}
Suppose that a given object has position vector ${\bf r}$ in some {\em non-rotating inertial}\/ reference frame. Let us observe the motion of
this object in a {\em non-inertial}\/ reference frame which {\em  rotates}\/ with {\em constant}\/ angular
velocity $\bOmega$ about
an axis passing through the origin of the inertial frame. Suppose, first of all, that our object appears {\em stationary}\/ in the rotating reference frame. Hence, in the non-rotating frame,
the object's  position vector ${\bf r}$ will appear to {\em precess}\/ about the origin with
angular velocity $\bOmega$. It follows, from Equation~(\ref{e239}),
that in the non-rotating reference frame
\begin{equation}
\frac{d{\bf r}}{dt} =  \bOmega\times{\bf r}.
\end{equation}
Suppose, now, that our object appears to move in the rotating reference frame
with instantaneous velocity ${\bf v}'$. It is fairly obvious that the appropriate generalization of the above equation is simply
\begin{equation}
\frac{d{\bf r}}{dt} = {\bf v}'+   \bOmega\times{\bf r}.
\end{equation}

Let and $d/dt$ and  $d/dt'$ denote apparent time derivatives in the non-rotating and rotating frames of reference, respectively. Since an object which is
stationary in the rotating reference frame appears to move in the non-rotating
frame, it is clear that $d/dt\neq d/dt'$. Writing the apparent velocity, ${\bf v}'$, 
of our object in the rotating reference frame as $d{\bf r}/dt'$, the above
equation takes the form
\begin{equation}\label{e8.3}
\frac{d{\bf r}}{dt} = \frac{d{\bf r}}{dt'}+\bOmega\times{\bf r},
\end{equation}
or
\begin{equation}\label{e8.4}
\frac{d}{dt} =\frac{d}{dt'}+ \bOmega\times,
\end{equation}
since ${\bf r}$ is a general position vector. Equation~(\ref{e8.4}) expresses the
relationship between  apparent time derivatives in the non-rotating and
rotating reference frames.

Operating on the general position vector ${\bf r}$ with the time derivative (\ref{e8.4}), we get
\begin{equation}
{\bf v} = {\bf v}' + \bOmega\times{\bf r}.
\end{equation}
This equation relates the apparent velocity, ${\bf v}=d{\bf r}/dt$, of an object with
position vector ${\bf r}$ in the non-rotating reference frame to its
apparent velocity, ${\bf v}' = d{\bf r}/dt'$, in the rotating reference frame.

Operating twice on the position vector ${\bf r}$ with the time
derivative (\ref{e8.4}), we obtain
\begin{equation}
{\bf a} = \left(\frac{d}{dt'}+ \bOmega\times\right)\left({\bf v}' + \bOmega\times{\bf r}\right),
\end{equation}
or
\begin{equation}\label{e8.6}
{\bf a} = {\bf a}' + \bOmega\times(\bOmega\times{\bf r}) + 2\,\bOmega\times{\bf v}'.
\end{equation}
This equation relates the apparent acceleration, ${\bf a}= d^2{\bf r}/dt^2$, of an object with
position vector ${\bf r}$ in the non-rotating reference frame to its
apparent acceleration, ${\bf a}'= d^2{\bf r}/dt'^2$, in the rotating reference frame.

Applying Newton's second law of motion in the inertial ({\em i.e.}, non-rotating) reference frame, we obtain
\begin{equation}
m\,{\bf a} = {\bf f}.
\end{equation}
Here, $m$ is the mass of our object, and ${\bf f}$ is the (non-fictitious) force acting on it. Note that these quantities are the {\em same}\/ in both reference
frames.
Making use of Equation~(\ref{e8.6}), the  apparent equation of motion of our object in the
rotating reference frame takes the form
\begin{equation}\label{e8.8}
m\,{\bf a}' = {\bf f} - m\,\bOmega\times(\bOmega\times{\bf r}) -2\,m\,\bOmega\times{\bf v}'.
\end{equation}
The last two terms in the above equation are so-called ``fictitious forces''. Such forces
are always needed to account for motion observed in non-inertial reference
frames. Note that fictitious forces can always be distinguished from
non-fictitious forces in Newtonian dynamics because the former
have no associated reactions. 
 Let us now investigate the two fictitious forces appearing in Equation~(\ref{e8.8}).

\section{Centrifugal Acceleration}
Let our non-rotating inertial frame be one whose origin lies at the center
of the Earth, and let our rotating frame be one whose origin is fixed with respect
to some point,  of latitude $\lambda$, on the Earth's surface---see Figure~\ref{rot}.
The latter reference frame thus rotates with respect to the former (about an
axis passing through the Earth's center)
with an angular velocity vector, $\bOmega$, which points from  the center of the Earth toward its north pole, and is of magnitude
\begin{equation}
{\mit\Omega} = \frac{2\pi}{24\,\,{\rm hrs}}=7.27\times 10^{-5}\,{\rm rad./s}.
\end{equation}

\begin{figure}
\epsfysize=2.5in
\centerline{\epsffile{Chapter07/fig7.01.eps}}
\caption{\em Inertial and non-inertial reference frames.}\label{rot}
\end{figure}

Consider an object which appears stationary in our rotating reference frame:
{\em i.e.}, an object which is stationary with respect to the Earth's surface. 
According to Equation~(\ref{e8.8}), the object's apparent equation of motion in the
rotating frame takes the form
\begin{equation}
m\,{\bf a}' = {\bf f} - m\,\bOmega\times(\bOmega\times{\bf r}).
\end{equation}
Let the non-fictitious force acting on our object be the force of
gravity, ${\bf f}= m\,{\bf g}$. Here, the local gravitational acceleration, ${\bf g}$, 
points directly toward the center of the Earth. It follows, from the above,
that the apparent gravitational acceleration in the rotating frame is written
\begin{equation}\label{e8.12}
{\bf g}' = {\bf g} - \bOmega\times(\bOmega\times{\bf R}),
\end{equation}
where ${\bf R}$ is the displacement vector of the origin of the rotating
frame (which  lies on the Earth's surface) with respect to
the center of the Earth. Here, we are assuming that our object is situated
relatively close to the Earth's surface ({\em i.e.}, ${\bf r}\simeq {\bf R}$).

It can be seen, from Equation~(\ref{e8.12}), that the apparent gravitational acceleration
of a stationary object close to the Earth's surface has two components. First,
the true gravitational acceleration, ${\bf g}$, of magnitude
$g\simeq 9.8\,{\rm m/s^2}$, which always points directly toward the
center of the Earth. Second, the so-called  {\em centrifugal acceleration}, $-\bOmega\times(\bOmega\times{\bf R})$. This acceleration is normal to the Earth's axis of
rotation, and always points directly away from this axis. The magnitude
of the centrifugal acceleration is ${\mit\Omega}^2\,\rho = {\mit\Omega}^2\,R\,\cos\lambda$, where
$\rho$ is the perpendicular distance to the Earth's rotation axis, and $R=6.37\times 10^6\,{\rm m}$
is the Earth's radius---see Figure~\ref{cent}.

\begin{figure}
\epsfysize=2.5in
\centerline{\epsffile{Chapter07/fig7.02.eps}}
\caption{\em Centrifugal acceleration.}\label{cent}
\end{figure}

It is convenient to define Cartesian axes in the rotating reference frame such that the $z'$-axis
points vertically upward, and $x'$- and $y'$-axes are horizontal, with
the $x'$-axis pointing directly northward, and the $y'$-axis pointing directly westward---see Figure~\ref{rot}.
The Cartesian components of the Earth's angular velocity are thus
\begin{equation}
\bOmega = {\mit\Omega}\,(\cos\lambda,\,0,\,\sin\lambda),
\end{equation}
whilst the vectors ${\bf R}$ and ${\bf g}$ are written
\begin{eqnarray}
{\bf R} &=& (0,\,0,\,R),\\[0.5ex]
{\bf g} &=& (0,\,0,\,-g),
\end{eqnarray}
respectively.
 It follows that the Cartesian coordinates
of the apparent gravitational acceleration, (\ref{e8.12}), are
\begin{equation}\label{e8.13}
{\bf g}' = \left(-{\mit\Omega}^2\,R\,\cos\lambda\,\sin\lambda,\,0,\,
-g + {\mit\Omega}^2\,R\,\cos^2\lambda\right).
\end{equation}
The magnitude of this acceleration is approximately
\begin{equation}
g' \simeq g - {\mit\Omega}^2\,R\,\cos^2\lambda \simeq 9.8 - 0.034\,\cos^2\lambda\,\,\, {\rm m/s^2}.
\end{equation}
According to the above equation, the centrifugal acceleration causes the magnitude of the apparent gravitational
acceleration on the Earth's surface to {\em vary}\/ by about $0.3\%$, being largest
at the poles, and smallest at the equator. This variation in apparent
gravitational acceleration, due (ultimately) to the Earth's rotation, causes the
Earth itself to {\em bulge}\/ slightly at the equator (see Section~\ref{srotf}), which has the effect of further intensifying the variation, since a point on the surface of the Earth at the
equator is slightly further away  from the Earth's center than a similar point at one of the
poles (and, hence, the true gravitational acceleration is slightly weaker in the
former case).

Another consequence of centrifugal acceleration is that the apparent
gravitational acceleration on the Earth's surface has a {\em horizontal}\/
component aligned in the north/south direction. This horizontal component
ensures that the apparent gravitational acceleration {\em does not}\/ point
directly toward the center of the Earth. In other words,
a plumb-line on the surface of the Earth does not point vertically
downward, but is deflected slightly away from a true vertical in the north/south
direction. The angular deviation from true vertical can easily be 
calculated from Equation~(\ref{e8.13}):
\begin{equation}
\theta_{dev} \simeq - \frac{{\mit\Omega}^2\,R}{2\,g}\,\sin(2\,\lambda)\simeq 
-0.1^\circ\,\sin(2\,\lambda).
\end{equation}
Here, a positive angle denotes a northward deflection, and {\em vice versa}.
Thus, the deflection is {\em southward}\/ in the {\em northern hemisphere}\/ ({\em i.e.},
$\lambda>0$) and {\em northward}\/ in the {\em  southern hemisphere}\/ ({\em i.e.}, 
$\lambda < 0$). The deflection is
zero at the poles and at the equator, and reaches its maximum magnitude
(which is very small) at middle latitudes.

\section{Coriolis Force}
We have now accounted for the first fictitious force,
$-m\,\bOmega\times(\bOmega\times{\bf r})$, in Equation~(\ref{e8.8}).
Let us now investigate the second, which takes the form $-2\,m\,\bOmega\times{\bf v}'$, and is called the {\em Coriolis force}.
Obviously, this force only affects objects which are {\em moving}\/ in the rotating
reference frame.

Consider a particle of mass $m$ free-falling under gravity in our rotating reference frame. As before, we define  Cartesian axes in the rotating  frame such that the $z'$-axis
points vertically upward, and the $x'$- and $y'$-axes are horizontal, with
the $x'$-axis pointing directly northward, and the $y'$-axis pointing directly
westward. It follows, from Equation~(\ref{e8.8}), that the Cartesian equations of motion of the particle
in the rotating reference frame take the form:
\begin{eqnarray}\label{e8.18}
\ddot{x}' &=&2\,{\mit\Omega}\,\sin\lambda\,\dot{y}',\\[0.5ex]
\ddot{y}'&=&-2\,{\mit\Omega}\,\sin\lambda\,\dot{x}'+2\,{\mit\Omega}\,\cos\lambda\,
\dot{z}',\label{e8.19}\\[0.5ex]
\ddot{z}'&=& -g - 2\,{\mit\Omega}\,\cos\lambda\,\dot{y}'.\label{e8.20}
\end{eqnarray}
Here, $\dot{}\equiv d/dt$, and $g$ is the local acceleration due to gravity. In the
above, we have neglected the centrifugal acceleration, for the sake
of simplicity. This is reasonable, since the only effect of the centrifugal
acceleration is to slightly modify the magnitude and direction of the
local  gravitational acceleration. We have also neglected air resistance,
which is less reasonable.

Consider a particle which is dropped (at $t=0$) from rest a height $h$ above the Earth's
surface. The following solution method exploits the fact that the
Coriolis force is much smaller in magnitude that the force of gravity: hence,
${\mit\Omega}$ can be treated as a {\em small}\/ parameter. 
To lowest order ({\em i.e.}, neglecting ${\mit\Omega}$), the
particle's vertical motion satisfies $\ddot{z}'=-g$, which can be solved,
subject to the initial conditions, to give
\begin{equation}
z' = h - \frac{g\,t^{\,2}}{2}.
\end{equation}
Substituting this expression into Equations~(\ref{e8.18}) and (\ref{e8.19}), 
neglecting terms involving ${\mit\Omega}^2$, and solving subject to the
initial conditions, we obtain
$x'\simeq 0$, and
\begin{equation}
y' \simeq - g\,{\mit\Omega}\,\cos\lambda\,\frac{t^{\,3}}{3}.
\end{equation}
In other words, the particle is deflected {\em eastward}\/ ({\em i.e.}, in the negative
$y'$-direction). Now, the particle hits the ground when
$t\simeq \sqrt{2\,h/g}$. Hence, the net eastward deflection of the particle as strikes the ground is
\begin{equation}
d_{east} = \frac{{\mit\Omega}}{3}\,\cos\lambda\,\left(\frac{8\,h^3}{g}\right)^{1/2}.
\end{equation}
Note that this deflection is in the {\em same}\/ direction as the Earth's rotation ({\em i.e.}, west to east),
and is greatest at the equator, and zero at the poles. 
A particle dropped from a height of 100\,m at the equator is deflected by about
$2.2\,{\rm cm}$. 

Consider a particle launched {\em horizontally}\/ with some fairly large velocity
\begin{equation}
{\bf V} = V_0\,(\cos\theta,-\sin\theta,\,0).
\end{equation}
 Here, $\theta$ is the {\em compass
bearing}\/ of the velocity vector (so north is $0^\circ$, east is $90^\circ$, {\em etc.}). Neglecting any vertical motion, Equations~(\ref{e8.18}) and (\ref{e8.19}) yield
\begin{eqnarray}
\dot{v}_{x'} &\simeq& -2\,{\mit\Omega}\,V_0\,\sin\lambda\,\sin\theta,\\[0.5ex]
\dot{v}_{y}' &\simeq& -2\,{\mit\Omega}\,V_0\,\sin\lambda\,\cos\theta,
\end{eqnarray}
which can be integrated to give
\begin{eqnarray}
v_{x'}&\simeq& V_0\,\cos\theta -2\,{\mit\Omega}\,V_0\,\sin\lambda\,\sin\theta\,t,\\[0.5ex]
v_{y'}&\simeq&- V_0\,\sin\theta - 2\,{\mit\Omega}\,V_0\,\sin\lambda\,\cos\theta\,t.
\end{eqnarray}
To lowest order in ${\mit\Omega}$, the above equations are equivalent to
\begin{eqnarray}
v_{x'} &\simeq& V_0\,\cos(\theta +2\,{\mit\Omega}\,\sin\lambda\,t),\\[0.5ex]
v_{y'} &\simeq& -V_0\,\sin(\theta+2\,{\mit\Omega}\,\sin\lambda\,t).
\end{eqnarray}
If follows that the Coriolis force causes the compass
bearing of the particle's velocity vector to {\em rotate}\/ steadily as time progresses. The rotation rate is
\begin{equation}
\frac{d\theta}{dt} \simeq 2\,{\mit\Omega}\,\sin\lambda.
\end{equation}
Hence, the rotation is {\em clockwise}\/ (looking from above) in the
{\em northern hemisphere}, and {\em counter-clockwise}\/ in the
{\em southern  hemisphere}. The rotation rate is zero at the equator, and
greatest at the poles. 

The Coriolis force has a significant effect on terrestrial weather patterns. 
Near equatorial regions, the intense heating  of the Earth's surface due to the Sun results in hot
air rising. In the northern hemisphere, this causes cooler air to
move in a southerly direction toward the equator. The Coriolis
force deflects this moving air in a clockwise sense (looking from above),
resulting in the {\em trade winds}, which blow toward the {\em southwest}.
In the southern hemisphere, the cooler air moves northward, and
is deflected by the Coriolis force in a counter-clockwise sense, resulting
in trade winds which blow toward the {\em northwest}. 

Furthermore, as air flows from high to low pressure regions,
the Coriolis force deflects the air in a clockwise/counter-clockwise manner in the northern/southern
hemisphere, producing {\em cyclonic}\/  rotation---see Figure~\ref{cyc}. 
It follows that cyclonic rotation is {\em counter-clockwise}\/ in the {\em northern hemisphere},
and {\em clockwise}\/ in the {\em southern hemisphere}.  Thus, this
is the direction of rotation of tropical storms ({\em e.g.}, hurricanes,
typhoons) in each hemisphere.

\begin{figure}
\epsfysize=3.in
\centerline{\epsffile{Chapter07/fig7.03.eps}}
\caption{\em A cyclone in the northern hemisphere.}\label{cyc}
\end{figure}

\section{Foucault Pendulum}
Consider a pendulum consisting of a compact mass $m$ suspended from a light cable of length $l$ in such
a manner that the pendulum is free to oscillate in any plane whose
normal is parallel to the Earth's surface. The mass is
subject to three forces: first, the force of gravity $m\,{\bf g}$, which
is directed vertically downward (we are again ignoring centrifugal
acceleration); second, the tension ${\bf T}$ in the cable, which is directed upward
along the cable; and, third,  the Coriolis force. It follows that the
apparent equation of motion of the mass, in a frame of
reference which co-rotates with the Earth, is [see Equation~(\ref{e8.8})]
\begin{equation}
m\,\ddot{\bf r}' = m\,{\bf g} + {\bf T} - 2\,m\,\bOmega\times \dot{\bf r}'.
\end{equation}

Let us define our usual Cartesian coordinates ($x'$,\,$y'$,\,$z'$), and
let the origin of our coordinate system correspond to the equilibrium position
of the mass. If the pendulum cable is deflected from the downward vertical by a {\em small}\/ angle
$\theta$ then it is easily seen that $x'\sim  l\,\theta$, $y'\sim l\,\theta$,
and $z'\sim l\,\theta^{\,2}$. In other words, the change in height
of the mass, $z'$, is negligible compared to its horizontal displacement.
Hence, we can write $z'\simeq 0$, provided that $\theta\ll 1$. The
tension ${\bf T}$ has the vertical component $T\,\cos\theta\simeq T$,
and the horizontal component ${\bf T}_{hz}= - T\,\sin\theta\,{\bf r}'/r'\simeq
-T\,{\bf r}'/l$, since $\sin\theta \simeq r'/l$---see Figure~\ref{pend}. Hence, the
Cartesian equations of motion of the mass are written [{\em cf.}, Equations~(\ref{e8.18})--(\ref{e8.20})]
\begin{eqnarray}\label{e8.34}
\ddot{x}' &=& - \frac{T}{l\,m}\,x' + 2\,{\mit\Omega}\,\sin\lambda\,\dot{y}',
\\[0.5ex]
\ddot{y}' &=& - \frac{T}{l\,m}\,y' - 2\,{\mit\Omega}\,\sin\lambda\,\dot{x}',
\label{e8.35}\\[0.5ex]
0 &=& \frac{T}{m} - g - 2\,{\mit\Omega}\,\cos\lambda\,\dot{y}'.
\end{eqnarray}
To lowest order in ${\mit\Omega}$ ({\em i.e.}, neglecting ${\mit\Omega}$), the final equation, which is just vertical force balance, yields $T\simeq m\,g$.
Hence, Equations~(\ref{e8.34}) and (\ref{e8.35}) reduce to
\begin{eqnarray}\label{e8.37}
\ddot{x}' &\simeq& - \frac{g}{l}\,x' + 2\,{\mit\Omega}\,\sin\lambda\,\dot{y}',
\\[0.5ex]
\ddot{y}' &\simeq& - \frac{g}{l}\,y' - 2\,{\mit\Omega}\,\sin\lambda\,\dot{x}'.\label{e8.38}
\end{eqnarray}

\begin{figure}
\epsfysize=3.in
\centerline{\epsffile{Chapter07/fig7.04.eps}}
\caption{\em The Foucault pendulum.}\label{pend}
\end{figure}

Let
\begin{equation}\label{e8.39}
s = x' + {\rm i}\,y'.
\end{equation}
Equations ~(\ref{e8.37})  and (\ref{e8.38}) can be combined to give a single complex
equation for $s$:
\begin{equation}\label{e8.40}
\ddot{s} = -\frac{g}{l}\,s - {\rm i}\,2\,{\mit\Omega}\,\sin\lambda\,\dot{s}.
\end{equation}
Let us look for a sinusoidally oscillating solution of the form
\begin{equation}\label{e8.41}
s = s_0\,{\rm e}^{-{\rm i}\,\omega\,t}.
\end{equation}
Here, $\omega$ is the (real) angular frequency of oscillation, and $s_0$
is an arbitrary complex constant. Equations~(\ref{e8.40}) and (\ref{e8.41})
yield the following quadratic equation for $\omega$:
\begin{equation}
\omega^2 - 2\,{\mit\Omega}\,\sin\lambda\,\omega - \frac{g}{l}=0.
\end{equation}
The solutions are approximately
\begin{equation}
\omega_{\pm} \simeq {\mit\Omega}\,\sin\lambda \pm \sqrt{\frac{g}{l}},
\end{equation}
where we have neglected terms involving ${\mit\Omega}^{\,2}$. 
Hence, the general solution of (\ref{e8.41}) takes the form
\begin{equation}
s = s_+\, {\rm e}^{-{\rm i}\,\omega_+\,t}+s_-\, {\rm e}^{-{\rm i}\,
\omega_-\,t},
\end{equation}
where $s_+$ and $s_-$ are two arbitrary complex constants.

Making the specific choice $s_+=s_-=a/2$, where $a$ is real, the
above solution reduces to
\begin{equation}
s = a\,{\rm e}^{-{\rm i}\,{\mit\Omega}\,\sin\lambda\,t} \,\cos\left(\sqrt{\frac{g}{l}}\,t\right).
\end{equation}
Now, it is clear from Equation~(\ref{e8.39}) that $x'$  and $y'$ are the real and imaginary
parts of $s$, respectively. Thus, it follows from the above that
\begin{eqnarray}
x'&=& a\,\cos({\mit\Omega}\,\sin\lambda\,t)\,\cos\left(\sqrt{\frac{g}{l}}\,t\right),\\[0.5ex]
y'&=& -a\,\sin({\mit\Omega}\,\sin\lambda\,t)\,\cos\left(\sqrt{\frac{g}{l}}\,t\right).
\end{eqnarray}
These  equations describe sinusoidal oscillations, in a plane whose normal
is parallel to the Earth's surface, at the standard pendulum frequency $\sqrt{g/l}$. 
The Coriolis force, however, causes the plane of oscillation to slowly {\em precess}\/ at the
angular frequency ${\mit\Omega}\,\sin\lambda$. The period of the
precession is 
\begin{equation}
T = \frac{2\pi}{{\mit\Omega}\,\sin\lambda} = \frac{24}{\sin\lambda}\,\,{\rm hrs}.
\end{equation}
For example, according to the above equations, the pendulum oscillates
in the $x'$-direction ({\em i.e.}, north/south) at $t\simeq 0$, in the $y'$-direction
({\em i.e.}, east/west) at $t\simeq T/4$, in the $x'$-direction again at $t\simeq T/2$, {\em etc.} The precession is {\em clockwise}\/ (looking from above)
in the {\em northern hemisphere}, and {\em counter-clockwise}\/
in the  {\em southern
hemisphere}. 

The precession of the plane of oscillation of a pendulum, due to
the Coriolis force, is used in many museums and observatories to
demonstrate that the Earth is rotating. This method of making the
Earth's rotation manifest was first devised by Foucault in 1851.

\section{Exercises}
{\small 
\renewcommand{\theenumi}{7.\arabic{enumi}}
\begin{enumerate}
\item A pebble is dropped down an elevator shaft in the Empire State
Building ($h=1250$ ft, latitude $=41^\circ$ N). Find the pebble's horizontal deflection (magnitude and direction) 
due to the Coriolis force at the bottom of the shaft. Neglect air resistance.
\item If a bullet is fired due east, at an elevation angle $\alpha$, from a point
on the Earth whose latitude is $+\lambda$ show that it will strike the
Earth with a lateral deflection $4\,{\mit\Omega}\,v_0^{\,3}\,\sin\lambda\,\sin^2\alpha\,\cos\alpha/g^2$.
Here, ${\mit\Omega}$ is the Earth's angular velocity, $v_0$ is the bullet's initial speed, and
$g$ is the acceleration due to gravity. Neglect air resistance.
\item A particle is thrown vertically with initial speed $v_0$, reaches
a maximum height, and falls back to the ground. Show that the horizontal Coriolis
deflection of the particle when it returns to the ground is opposite in direction,
and four times greater in magnitude, than the Coriolis deflection when it
is dropped at rest from the same maximum height. Neglect air resistance.
\item The surface of the Diskworld is a disk which
rotates (counter-clockwise looking down) with angular frequency $\Omega$ about a perpendicular axis passing through
its center. Diskworld gravitational acceleration is of magnitude
$g$, and is everywhere directed normal to the disk.
A projectile is launched from the surface of the disk at a point whose radial distance from the axis of rotation is $R$.
The initial velocity of the projectile (in a co-rotating frame) is of magnitude $v_0$,  is directly radially
outwards, and is inclined at an angle  $\alpha$ to the horizontal. 
What are the radial and tangential displacements of the impact point
from that calculated by neglecting the centrifugal and Coriolis forces?
Neglect air resistance. You may assume that the displacements are small
compared to both $R$ and the horizontal range of the projectile. 
\item Demonstrate that the Coriolis force causes conical pendulums to
rotate clockwise and counter-clockwise with slightly different angular frequencies. What
is the frequency difference as a function of terrestrial latitude?
\item A satellite is in a circular orbit of radius $a$ about the Earth.
Let us define a set of co-moving Cartesian coordinates, centered on the satellite, such that
the $x$-axis always points toward the center of the Earth, the $y$-axis in the
direction of the satellite's orbital motion, and the $z$-axis in the direction
of the satellite's orbital angular velocity, $\bomega$. Demonstrate that the
equation of motion of a small mass in orbit about the satellite are
\begin{eqnarray}
\ddot{x} &=& 3\,\omega^2\,x + 2\,\omega\,\dot{y},\nonumber\\[0.5ex]
\ddot{y} &=& -2\,\omega\,\dot{x},\nonumber
\end{eqnarray}
assuming that $|x|/a\ll 1$ and $|y|/a\ll 1$. You may neglect the gravitational
attraction between the satellite and the mass.
Show that the mass
executes a retrograde ({\em i.e.}, in the opposite sense to the
satellite's orbital rotation)  elliptical orbit about the satellite whose 
period matches that of the satellite's orbit, and whose major and minor axes
are in the ratio $2:1$, and are  aligned along the $y$- and $x$-axes,
respectively. 
\end{enumerate}

}