\chapter{Newton's Laws of Motion}\label{sfun}
\section{Introduction}
This chapter discusses Newton's laws of motion.

\section{Newtonian Dynamics}
Newtonian dynamics is a mathematical model whose purpose is to 
predict  the motions of the various objects that we encounter in the world
around us. The general principles of this model were first enunciated
by Sir Isaac Newton in a work entitled {\em Philosophiae Naturalis Principia
Mathematica}\/ (Mathematical Principles of Natural Philosophy).  This work,
which was first published in 1687,  is nowadays more commonly referred to as the {\em Principa}.\footnote{An excellent discussion of the historical
development of Newtonian dynamics, as well as the physical and
philosophical assumptions which underpin this theory, is
given in {\em The Discovery of Dynamics: A Study from a Machian Point of View of the Discovery and the Structure of Dynamical Theories}, J.B.~Barbour (Oxford University Press, Oxford UK, 2001).}

Up until the beginning of the 20th century, Newton's theory of motion 
was thought to constitute a {\em complete}\/ description of all types of
motion occurring in the Universe. We now know that this is not the
case. The modern view is that Newton's theory is only an {\em approximation}\/ which is valid under certain circumstances. 
The theory breaks down when the velocities of the objects under
investigation approach the speed of light in vacuum, and must be modified in accordance with  Einstein's
{\em special theory of  relativity}. The theory also fails in regions of space which are sufficiently curved that the propositions of Euclidean geometry
do not hold to a good approximation, and must be augmented by Einstein's {\em general theory
of relativity}.
Finally, the theory breaks down on  atomic and subatomic length-scales, and must be replaced by {\em quantum mechanics}. 
In this book, we shall  neglect  relativistic and quantum effects all together.
It follows that  we must restrict our investigations to the motions  of {\em large}\/ (compared to an atom) {\em slow}\/ (compared to the speed of light) 
objects moving in {\em Euclidean}\/ space. Fortunately, virtually all of the motions which we commonly observe in the world around us fall into
this category. 

 Newton very deliberately modeled his approach in the Principia on
that taken in {\em Euclid's Elements}.
Indeed, Newton's theory of motion has much in common with a conventional {\em axiomatic system}\/ such  as Euclidean geometry. Like all such  systems, Newtonian dynamics starts from a set of terms that are {\em undefined}\/ within the
system. In  this case, the fundamental terms are {\em mass}, {\em position}, 
{\em time}, and {\em force}. It is taken for granted that  we understand what these terms mean,
and, furthermore, that they correspond to {\em measurable}\/ quantities which
can be ascribed to, or associated with, objects in the world
around us. In particular, it is assumed that the ideas of position
in space, distance in space, and position as a function of time in space, are
correctly described by the Euclidean vector algebra and vector calculus discussed in the
previous chapter.
The next component
of an axiomatic system is a set of {\em axioms}. These are a set of
{\em unproven}\/ propositions, involving the undefined terms, from which
all other propositions in the system can be derived via logic and mathematical
analysis. In the present case, the axioms are called {\em Newton's laws of
motion}, and can only be justified via experimental observation.
Note, incidentally, that Newton's laws, in their primitive form, are only applicable
to {\em point objects}. However, these laws can
be applied to extended objects by treating them as collections of point
objects.

One difference between an axiomatic system and a physical theory is that,
in the latter case, even if a given prediction has been shown to follow
necessarily from the axioms of the theory, it is still incumbent upon us to test the prediction against
experimental observations. Lack of agreement might indicate
faulty experimental data, faulty application of the theory (for instance, in the case of Newtonian dynamics, there
might be forces at work which we have not identified), or, as a last resort, 
incorrectness of the theory. Fortunately, 
Newtonian dynamics has been found to give predictions which are in excellent agreement with experimental
observations in all
situations in which  it is expected to be valid.

In the following, it is assumed that we know how to set up a rigid {\em Cartesian
frame of reference}, and how to measure the positions of point objects as 
functions of time within that frame. It is also taken for granted that we  have some basic
familiarity with the laws of mechanics, and standard mathematics up to, and including, calculus, as well
as
the vector analysis outlined in Appendix~\ref{vector}.

\section{Newton's Laws of Motion}
Newton's laws of motion,  in the rather obscure language of the Principia,  take the following form:
\begin{enumerate}
\item Every body continues in its state of rest, or uniform motion in a 
straight-line, unless
compelled to change that state by forces impressed upon it.
\item The change of motion ({\em i.e.}, momentum) of an object is proportional to the force impressed upon
it, and is made in the direction of the straight-line in which the force is impressed.
\item To every action there is always opposed an equal reaction; or, the mutual actions
of two bodies upon each other are always equal and directed to contrary parts.  
\end{enumerate}
Let us now examine how these laws can be applied to dynamical systems. 

\section{Newton's First Law of Motion}
Newton's first law of motion essentially states that a  point object
subject to zero net external force moves
in a straight-line  with a constant speed ({\em i.e.}, it does not accelerate). However, this is only true in special frames of reference called {\em inertial frames}.  Indeed, we can think of Newton's
first law as the {\em definition}\/ of an inertial frame: {\em i.e.},
an inertial frame of reference is one in which a point object subject to zero
net external force moves  in a straight-line with constant speed.

Suppose that we have found an inertial frame of reference. Let us
set up a Cartesian coordinate system in this frame. The motion
of a point object can now be specified by giving its position vector, ${\bf r}\equiv (x,\,y,\,z)$,
with respect to the origin of the coordinate system, as a function of time, $t$. 
Consider a second frame of reference moving with some 
{\em constant}\/ velocity
${\bf u}$ with respect to the first frame. Without loss of generality,
we can suppose that the Cartesian axes in the second frame are parallel
to the corresponding axes in the first frame, that ${\bf u} \equiv (u,0,0)$,
and, finally, that the origins of the two frames instantaneously coincide at $t=0$---see Figure~\ref{fun1}. Suppose that the position vector
of our point object is ${\bf r}'\equiv (x',\,y',\,z')$ in the second frame of reference. 
It is evident, from Figure~\ref{fun1}, that at any given time, $t$, the coordinates of the
object in the two reference frames satisfy
\begin{eqnarray}\label{e3.1}
x' &=& x - u\,t,\\[0.5ex]
y' &=& y,\\[0.5ex]
z' &=& z.\label{e3.3}
\end{eqnarray}
This coordinate transformation was first discovered by Galileo Galilei, and is  known as a {\em Galilean transformation}\/
in his honor. 

\begin{figure}
\epsfysize=2.25in
\centerline{\epsffile{Chapter02/fig2.01.eps}}
\caption{\em A Galilean coordinate transformation.}\label{fun1}
\end{figure}

By definition, the instantaneous velocity of the object in our first reference frame is given by
${\bf v} = d{\bf r}/dt \equiv (dx/dt,\,dy/dt,\,dz/dt)$, with an analogous
expression for the velocity, ${\bf v}'$, in the second frame. 
It follows from differentiation of Equations~(\ref{e3.1})--(\ref{e3.3}) with respect to time that the velocity components in the two frames satisfy
\begin{eqnarray}\label{e3.4}
v_x'& = &v_x - u,\\[0.5ex]
v_y' &=& v_y ,\\[0.5ex]
v_z'&=&v_z.\label{e3.6}
\end{eqnarray}
These equations can be written more succinctly as
\begin{equation}\label{e3.7}
{\bf v}' = {\bf v} - {\bf u}.
\end{equation}

Finally, by definition, the instantaneous acceleration of the object in our first reference frame is given by
${\bf a} = d{\bf v}/dt \equiv (dv_x/dt,\,dv_y/dt,\,dv_z/dt)$, with an analogous
expression for the acceleration, ${\bf a}'$, in the second frame. 
It follows from differentiation of Equations~(\ref{e3.4})--(\ref{e3.6}) with respect to time that the acceleration
components in the two frames satisfy
\begin{eqnarray}
a_x'& = &a_x,\\[0.5ex]
a_y' &=& a_y ,\\[0.5ex]
a_z'&=&a_z.
\end{eqnarray}
These equations can be written more succinctly as
\begin{equation}\label{e3.11}
{\bf a}' = {\bf a}.
\end{equation}

According to Equations~(\ref{e3.7}) and (\ref{e3.11}), if an
object is moving  in a straight-line with constant speed in our original
inertial frame ({\em i.e.}, if  ${\bf a}={\bf 0}$) then it also
moves in a (different) straight-line  with (a different) constant speed 
in the second frame of reference ({\em i.e.}, ${\bf a}'= {\bf 0}$). Hence,
we conclude that the second frame of reference is {\em also} an inertial frame.

A simple extension of the above argument allows us to conclude that there
are an {\em infinite}\/ number of different inertial frames moving with {\em constant
velocities}\/ with respect to one another. Newton through that one of these inertial frames was special, and
defined an absolute standard of rest: {\em i.e.}, a static object in this frame was in a state of absolute rest.
However, Einstein showed that this is not the case. In fact, there is no absolute standard of rest: {\em i.e.}, all
motion is relative---hence, the name ``relativity'' for Einstein's theory. Consequently, one inertial frame is
just as good as another as far as Newtonian dynamics is concerned. 

But, what happens if the second frame of reference {\em accelerates}\/ with
respect to the first? In this case, it is not hard to see that Equation~(\ref{e3.11})
generalizes to
\begin{equation}
{\bf a}' = {\bf a} - \frac{d{\bf u}}{dt},
\end{equation}
where ${\bf u}(t)$ is the instantaneous velocity of the second frame
with respect to the first. According to the above formula, if an object is
moving   in a straight-line with constant speed in the first frame ({\em i.e.}, if  ${\bf a}={\bf 0}$) then it does not move  in a
straight-line with constant speed
in the second frame ({\em i.e.}, ${\bf a}'\neq{\bf 0}$). Hence,
if the first frame is an inertial frame then the second is {\em not}. 

A simple extension of the above argument allows us to conclude that any
frame of reference which accelerates with respect to a given inertial
frame is not itself an inertial frame.

For most practical purposes, when studying the motions of objects close to the
Earth's surface, a reference frame which is  fixed with
respect to this  surface is approximately inertial. However, 
if the trajectory of a projectile within such a frame is measured to high
precision then it will be found to deviate slightly from the predictions
of Newtonian dynamics---see Chapter~\ref{snoni}. This deviation
is due to the fact that the Earth is rotating, and its surface is therefore
accelerating towards its axis of rotation.  When studying the motions of
objects in orbit around the Earth, a reference frame whose origin
is the center of the Earth, and whose coordinate axes are fixed with respect
to distant stars, is approximately inertial. However, if such
orbits are measured to extremely high precision then they will
again be found to deviate very slightly from the predictions of Newtonian
dynamics. In this case, the deviation is due to the Earth's orbital
motion about the Sun. When studying the orbits of the planets
in the Solar System, a reference frame whose origin is the center of the Sun,  and whose coordinate axes are fixed with respect
to distant stars, is approximately inertial. In this case, any deviations
of the orbits from the predictions of Newtonian dynamics
due to the orbital motion of the Sun about the galactic center are
far too small to be measurable. It should be noted that it is impossible
to identify an {\em absolute}\/ inertial frame---the best approximation to such
a frame  would be one in which the cosmic microwave background appears
to be (approximately) isotropic. However, for a given dynamical problem, it is always
possible to identify an {\em approximate}\/ inertial frame. Furthermore, any
deviations of such a frame from a true inertial frame can be incorporated
into the framework of Newtonian dynamics via the introduction of so-called fictitious forces---see Chapter~\ref{snoni}.

\section{Newton's Second Law of Motion}\label{new2}
Newton's second law of motion essentially states that if a point object
is subject to an external force, ${\bf f}$, then its equation of motion
is given by
\begin{equation}\label{en2}
\frac{d{\bf p}}{dt} = {\bf f},
\end{equation}
where the momentum, ${\bf p}$, is the product of the  object's inertial
mass, $m$,   and its velocity, ${\bf v}$. If $m$
is not a function of time then the above expression reduces to the
familiar equation
\begin{equation}\label{e3.14}
m\,\frac{d{\bf v}}{dt} = {\bf f}.
\end{equation}
Note that this equation is only valid in a {\em inertial frame}. 
Clearly, the inertial mass of an object measures its reluctance to deviate
from its preferred state of uniform motion in a straight-line (in an
inertial frame). Of course, the above equation of motion can only be solved if we have an independent expression for the force, ${\bf f}$  ({\em i.e.}, a law of force).
Let us suppose that this is the case.

An important corollary of Newton's second law is that force is a {\em vector
quantity}. This must be the case, since the law  equates force to the
product of a scalar (mass) and a vector (acceleration). 
Note that acceleration is obviously a vector because it is directly related to displacement, which is the prototype of all vectors---see Appendix~\ref{vector}. One consequence of force being a vector is
that two forces, ${\bf f}_1$ and ${\bf f}_2$, both acting at a given
point, have the same effect as a single force, ${\bf f} = {\bf f}_1+{\bf f}_2$,
acting at the same point, where the summation is performed according to the
laws of vector addition---see Appendix~\ref{vector}. Likewise, a single force, ${\bf f}$, acting at
a given point, has the same effect as two forces, ${\bf f}_1$ and ${\bf f}_2$,
acting at the same point, provided that ${\bf f}_1+{\bf f}_2={\bf f}$. This
method of combining and splitting forces is known as the {\em resolution of
forces}, and lies at the heart of many calculations in Newtonian dynamics.

Taking the scalar product of Equation~(\ref{e3.14}) with the velocity, ${\bf v}$,
we obtain
\begin{equation}
m\,{\bf v}\!\cdot\!\frac{d {\bf v}}{dt} = 
\frac{m}{2}\frac{d({\bf v}\cdot{\bf v})}{dt}=\frac{m}{2}\frac{d v^2}{dt} = 
{\bf f}\cdot{\bf v}.
\end{equation}
This can be written
\begin{equation}\label{e3.15}
\frac{d K}{dt} = {\bf f}\cdot {\bf v}.
\end{equation}
where 
\begin{equation}
K = \frac{1}{2}\,m\,v^2.
\end{equation}
The right-hand side of Equation~(\ref{e3.15}) represents the rate at
which the force does work on the object: {\em i.e.},
the rate at which the force transfers energy to the object. The quantity
$K$ represents the energy the object possesses by virtue of its motion.
This type of energy is generally known as {\em kinetic energy}. Thus, Equation~(\ref{e3.15}) states that any work done on  point object by an external force
goes to increase the object's kinetic energy.

Suppose that, under the action of the force, ${\bf f}$, our object moves
from point $P$ at time $t_1$ to point $Q$ at time $t_2$. The
net change in the object's kinetic energy is obtained by integrating
Equation~(\ref{e3.15}):
\begin{equation}\label{e3.18}
{\mit\Delta}K = \int_{t_1}^{t_2} {\bf f}\cdot{\bf v}\,dt
=\int_P^Q {\bf f}\cdot d{\bf r},
\end{equation}
since ${\bf v} = d{\bf r}/dt$.
Here, $d{\bf r}$ is an element of the object's path between points $P$
and $Q$, and the integral in ${\bf r}$ represents the net work done by the
force as the objects moves along the path from $P$ to $Q$. 

As described in Section~\ref{sgrad}, there are basically two kinds
of forces in nature. Firstly, those for which line integrals of the type
$\int_P^Q {\bf f}\cdot d{\bf r}$ depend on the end points, but not
on the path taken between these points. Secondly, those for which
line integrals of the type $\int_P^Q {\bf f}\cdot d{\bf r}$ depend
both on the end points, and the path taken between these points. 
The first kind of force is termed {\em conservative}, whereas the
second kind is termed {\em non-conservative}. It was also
demonstrated in Section~\ref{sgrad} that if the line integral $\int_P^Q {\bf f}\cdot d{\bf r}$ is {\em path independent}\/ then the force ${\bf f}$ can always
be written as the gradient of a scalar field. In other words, all
conservative forces satisfy
\begin{equation}\label{e3.16}
{\bf f} = - \nabla U,
\end{equation}
for some scalar field $U({\bf r})$.  
Note that
\begin{equation}
\int_P^Q \nabla U\cdot d{\bf r} \equiv {\mit\Delta} U = U(Q) - U(P),
\end{equation}
irrespective of the path taken between $P$ and $Q$. 
Hence, it follows from Equation~(\ref{e3.18})
that 
\begin{equation}
{\mit\Delta} K = - {\mit\Delta} U
\end{equation}
for conservative forces. Another way of writing this is 
\begin{equation}\label{e3.22}
E = K + U = {\rm constant}.
\end{equation}
Of course, we recognize this as an {\em energy conservation equation}: $E$
is the object's total energy, which is conserved; $K$ is the energy the
object has by virtue of its motion, otherwise know as its
{\em kinetic energy}; and $U$ is the energy the object has by
virtue of its position, otherwise known as its {\em potential
energy}. Note, however, that we can only write such  energy conservation
equations for conservative forces. Gravity is a good example of a conservative force.
Non-conservative forces, on the other hand,  do not conserve energy. In general, this
is because of some sort of frictional energy loss which drains energy
from the dynamical system whilst it remains in motion.
Note that potential energy is undefined to an arbitrary additive constant.
In fact, it is only the {\em difference}\/ in potential energy between
different points in space that is well-defined.

\section{Newton's Third Law of Motion}\label{new3}
Consider a system of $N$ mutually interacting point objects.
Let the $i$th object, whose mass is $m_i$, be located at vector displacement ${\bf r}_i$.
Suppose that this object exerts a force ${\bf f}_{ji}$ on the $j$th object. 
Likewise, suppose that the $j$th object exerts a force ${\bf f}_{ij}$ on the
$i$th object.
Newton's third
law of motion essentially states that these two forces are equal and opposite, irrespective
of their nature.
In other words, 
\begin{equation}\label{e966}
{\bf f}_{ij} = - {\bf f}_{ji}.
\end{equation}
One corollary of Newton's third law is that an object cannot exert
a force on itself. Another corollary is that all forces in the Universe have corresponding reactions. The only exceptions to this
rule are the fictitious forces which arise in non-inertial reference frames ({\em e.g.}, the centrifugal and Coriolis
forces which appear in  rotating reference frames---see Chapter~\ref{snoni}). Fictitious forces do not possess  reactions.

It should be noted that Newton's third law implies {\em action at a
distance}. In other words, if the force  that object
$i$ exerts on object $j$ suddenly changes then Newton's third law
demands that there must be an {\em immediate}\/
change in the force that object
$j$ exerts on object $i$. Moreover, this must be true irrespective of the
distance between the two objects. However, we now know that
Einstein's theory of relativity forbids information  from traveling through the
Universe faster than the velocity of light in vacuum. Hence, action at a distance is also forbidden. In other words,  if the force  that object
$i$ exerts on object $j$ suddenly changes then there must be a
{\em time delay}, which is at least as long as it takes a light ray to propagate
between the two objects, before  the force that object
$j$ exerts on object $i$ can respond. Of course, this means that
Newton's third law is not, strictly speaking, correct. However, as
long as we restrict our investigations to the motions of dynamical
systems on time-scales that are long compared to the time
required for  light-rays to traverse these systems, Newton's third
law can be regarded as being approximately correct. 

In an inertial frame, Newton's second law of motion applied to the $i$th object yields
\begin{equation}\label{e3.24}
m_i\,\frac{d^2 {\bf r}_i}{dt^2} = \sum_{j=1,N}^{j\neq i}\! {\bf f}_{ij}.
\end{equation}
Note that the summation on the right-hand side of the above equation excludes the case
$j=i$, since the $i$th object cannot exert a force on itself. Let us now take the above
equation and sum it over all objects. We obtain
\begin{equation}
\sum_{i=1,N} \!m_i\,\frac{d^2 {\bf r}_i}{dt^2}=\sum_{i,j=1,N}^{j\neq i}\! {\bf f}_{ij}.
\end{equation}
Consider the sum  over forces on the right-hand side of the above equation.
Each element of this sum---${\bf f}_{ij}$, say---can be paired with another element---${\bf f}_{ji}$,
in this case---which is equal and opposite, according to Newton's third law. In other words,
 the elements of the sum all cancel out in pairs. Thus, the net value of the sum is {\em zero}.  
It follows that the above equation can be written
\begin{equation}\label{e6yy}
M\,\frac{d^2 {\bf r}_{cm}}{dt^2}= {\bf 0},
\end{equation}
where $M = \sum_{i=1}^N m_i$ is the total mass.  The quantity ${\bf r}_{cm}$ is the vector displacement of the {\em center of mass} of
the system, which is an imaginary point whose coordinates are the mass weighted
averages of the coordinates of the objects that constitute  the system: {\em i.e.}, 
\begin{equation}\label{e3.27}
{\bf r}_{cm} = \frac{\sum_{i=1}^N m_i\,{\bf r}_i}{\sum_{i=1}^N m_i}.
\end{equation}
According to Equation~(\ref{e6yy}), the center of mass of the
system moves in a uniform straight-line, in accordance with
Newton's first law of motion, irrespective of the nature of the
forces acting between the various components of the system.

Now, if the center of mass moves in a uniform straight-line  then
the  center of mass velocity,
\begin{equation}\label{e6xy}
\frac{d{\bf r}_{cm} }{dt}= \frac{\sum_{i=1}^N m_i\, d{\bf r}_i/dt}{\sum_{i=1}^N m_i},
\end{equation}
is a constant of the motion. However, the momentum of the $i$th object takes the
form ${\bf p}_i = m_i\,d{\bf r}_i/dt$. Hence, the total momentum of the
system is written
\begin{equation}\label{e6yx}
{\bf P} = \sum_{i=1}^N m_i\,\frac{d {\bf r}_i}{dt}.
\end{equation}
A comparison of Equations~(\ref{e6xy}) and (\ref{e6yx}) suggests that ${\bf P}$ is also
a constant of the motion. In other words, the total momentum
of the system is a {\em conserved}\/ quantity, irrespective of the nature of the
forces acting between the various components of the system. This result (which only holds if there is no net external force acting on the system) is
a direct consequence of Newton's third law of motion.

Taking the vector product of Equation~(\ref{e3.24}) with the position vector ${\bf r}_i$, we
obtain
\begin{equation}\label{e988}
m_i\,{\bf r}_i\times \frac{d^2{\bf r}_i }{dt^2}= \sum_{j=1,N}^{j\neq i} {\bf r}_i\times {\bf f}_{ij}.
\end{equation}
However, it is easily seen that
\begin{equation}
m_i\,{\bf r}_i\times \frac{d^2{\bf r}_i }{dt^2}= \frac{d}{dt}\!\left( m_i\,{\bf r}_i\times
\frac{d{\bf r}_i}{dt}\right) = \frac{d{\bf l}_i}{dt},
\end{equation}
where
\begin{equation}
{\bf l}_i = m_i\,{\bf r}_i\times \frac{d{\bf r}_i}{dt}
\end{equation}
is the {\em angular momentum}\/ of the $i$th particle about the origin
of our coordinate system.
The total angular momentum of the system (about the origin) takes the form
\begin{equation}
{\bf L} = \sum_{i=1,N} {\bf l}_i
\end{equation}
Hence, summing Equation~(\ref{e988}) over all particles, we obtain
\begin{equation}\label{e977}
\frac{d {\bf L}}{dt} = \sum_{i,j = 1,N}^{i\neq j} {\bf r}_i\times {\bf f}_{ij}.
\end{equation}

Consider  the sum on the right-hand side of the above equation. A general term,
${\bf r}_i\times {\bf f}_{ij}$, in this sum can always be paired with a
matching term, ${\bf r}_j\times {\bf f}_{ji}$, in which the indices have been swapped.
Making use of Equation~(\ref{e966}), the sum of a general matched pair  can be written
\begin{equation}
{\bf r}_i\times {\bf f}_{ij}+{\bf r}_j\times {\bf f}_{ji} = 
({\bf r}_i-{\bf r}_j)\times {\bf f}_{ij}.
\end{equation}
Let us assume that the  forces acting between the
various components of the system are {\em central}\/ in nature, so
that  ${\bf f}_{ij}$ is parallel
to ${\bf r}_i-{\bf r}_j$. In other words, the force exerted on object $j$
by object $i$ either points directly toward, or directly away from, object $i$,
and {\em vice versa}. 
 This is a reasonable assumption,
since most of the forces which we encounter in the world around us  are of this type ({\em e.g.}, gravity). 
It follows that if the forces are central in nature then the vector product
on the right-hand side of the above expression is zero.
We conclude that
\begin{equation}
{\bf r}_i\times {\bf f}_{ij}+{\bf r}_j\times {\bf f}_{ji} = {\bf 0},
\end{equation}
for all values of $i$ and $j$. Thus, the sum on the right-hand side of Equation~(\ref{e977}) is zero for any kind of central force. We are left with
\begin{equation}\label{e955}
\frac{d {\bf L}}{dt} = {\bf 0}.
\end{equation}
In other words, the total angular momentum of the system is
a {\em conserved}\/ quantity, provided that the different components of the
system interact via {\em central}\/ forces (and there is no net external torque
acting on the system).

\section{Non-Isolated Systems}\label{new4}
Up to now, we have only considered {\em isolated}\/ dynamical systems, in which all of the forces acting
on the system originate  within the system itself. Let us now generalize our approach to
deal with {\em non-isolated}\/ dynamical systems, in which some of the forces  originate outside the system. Consider a system of $N$ mutually interacting point objects. Let
$m_i$ and ${\bf r}_i$ be the mass and position vector of the $i$th object, respectively. Suppose
that the $i$th object is subject to two forces. First, an {\em internal force}\/ which originates
from the other objects in the system, and second an {\em external force}\/ which originates
outside the system. In other words, let the force acting on the $i$th object take the form
\begin{equation}
{\bf f}_i = \sum_{j=1,N}^{j\neq i} {\bf f}_{ij} + {\bf F}_i,
\end{equation}
where ${\bf f}_{ij}$ is the internal force exerted by object $j$ on object $i$, and ${\bf F}_i$  the
net external force acting on object $i$. 

The equation of motion of the $i$th object is
\begin{equation}\label{eemon}
m_i\,\frac{d^2{\bf r}_i}{dt^2} = {\bf f}_i =  \sum_{j=1,N}^{j\neq i} {\bf f}_{ij} + {\bf F}_i.
\end{equation}
Summing over all objects, we obtain
\begin{equation}
\sum_{i=1,N} m_i\,\frac{d^2{\bf r}_i}{dt^2} =  \sum_{i,j=1,N}^{j\neq i} {\bf f}_{ij} + \sum_{i=1,N}{\bf F}_i,
\end{equation}
which reduces to
\begin{equation}\label{etrans}
\frac{d{\bf P}}{dt} = {\bf F},
\end{equation}
where
\begin{equation}
{\bf F} = \sum_{i=1,N} {\bf F}_i
\end{equation}
is the net external force acting on the system. Here, the sum over the internal forces has cancelled out in pairs
due to Newton's third law of motion. We conclude that the total system momentum  evolves in time according to the
simple equation (\ref{etrans})
when there is a net external force acting on the system, but is completely unaffected by the internal forces.
The fact that Equation~(\ref{etrans}) is similar in form to Equation~(\ref{en2}) suggests that the center of
mass of a system of many point objects has analogous dynamics to a point object.

Taking ${\bf r}_i\times$ Equation~(\ref{eemon}), and summing over all objects, we obtain
\begin{equation}\label{erotn}
\frac{d{\bf L}}{dt} = {\bf T},
\end{equation}
where
\begin{equation}
{\bf T} = \sum_{i=1,N} {\bf r}_i\times {\bf F}_i
\end{equation}
is the net external torque acting on the system. Here, the sum over the internal
torques has cancelled out in pairs, assuming that the internal forces are central in nature.
We conclude that the total system angular momentum  evolves in time according to the simple equation (\ref{erotn})
when there is a net external torque acting on the system, but is completely unaffected by the internal torques. 

\section{Exercises}
{\small
\renewcommand{\theenumi}{2.\arabic{enumi}}
\begin{enumerate}
\item Consider an isolated system of $N$ point objects interacting via
gravity. Let the mass and position vector of the $i$th object be
 $m_i$ and  ${\bf r}_i$, respectively. What is  the vector equation
 of motion of the $i$th object? Write expressions for the total
 kinetic energy, $K$, and potential energy, $U$, of the system.
 Demonstrate from the equations of motion  that $K+U$ is a conserved quantity.
 
 \item Consider a function of many variables $f(x_1,x_2,\cdots,x_n)$. 
 Such a function which satisfies
 $$
 f(t\,x_1, t\,x_2,\cdots,t\,x_n) = t^a\,f(x_1,x_2,\cdots,x_n)
 $$
 for all $t>0$, and all values of the $x_i$, is termed a {\em homogenous function of degree $a$}. 
 Prove the following theorem regarding homogeneous functions:
 $$
 \sum_{i=1,n} x_i\,\frac{\partial f}{\partial x_i} = a\,f
 $$
 
  \item Consider an isolated system of $N$ point objects interacting via
attractive central forces. Let the mass and position vector of the $i$th object be
 $m_i$ and  ${\bf r}_i$, respectively. Suppose that magnitude of the force exerted on object $i$ by
object $j$ is $k_i\,k_j\,|{\bf r}_i-{\bf r}_j|^{-n}$. Here, the $k_i$ measure
some constant physical
property of the particles ({\em e.g.}, their electric charges). Write
an expression for the total potential energy $U$ of the system. Is
this a homogenous function? If so, what is its degree?
Write the equation of motion of the $i$th particle. Use the mathematical
theorem from the previous exercise to demonstrate that
$$
\frac{1}{2}\frac{d^2 I}{dt^2} = 2\,K + (n-1)\,U,
$$
where $I=\sum_{i=1,N} m_i\, r_i^{\,2}$, and $K$ is the kinetic energy.
This result is known as the {\em virial theorem}.
Demonstrate that there are no bound steady-states for the system
when $n\geq 3$. 
 
\item A star can be through of as a spherical system that consists of a very large number of particles interacting
via gravity.  Show that, for such a system, the virial theorem, introduced in the previous exercise, implies that
$$
\frac{d^2 I}{dt^2} = -2\,U + c,
$$
where $c$ is a constant, and the $r_i$ are measured from the geometric center. Hence, deduce that the angular frequency of small amplitude radial pulsations
of the star (in which the radial displacement is directly proportional to the radial distance from the center) takes the form
$$
\omega = \left(\frac{|U_0|}{I_0}\right)^{1/2},
$$
where $U_0$ and $I_0$ are the equilibrium values of $U$ and $I$. Finally, show that if the mass
density within the star varies as $r^{-\alpha}$, where $r$ is the radial distance from the geometric center, and where $\alpha<5/2$, then
$$
\omega = \left(\frac{5-\alpha}{5-2\,\alpha}\,\frac{G\,M}{R^{\,3}}\right)^{1/2},
$$
where $M$ and $R$ are the stellar mass and radius, respectively.
 
\item Consider a system of $N$ point particles. Let the $i$th particle have mass $m_i$, electric
 charge $q_i$, and position vector ${\bf r}_i$. Suppose that the charge to
 mass ratio, $q_i/m_i$, is the same for all particles. The system is placed
 in a uniform magnetic field ${\bf B}$. Write the equation
 of motion of the $i$th particle. You may neglect any magnetic fields generated by the motion of the particles. Demonstrate that the total momentum
 ${\bf P}$ of the system precesses  about ${\bf B}$ at the frequency
 ${\mit\Omega} = q_i\,B/m_i$. Demonstrate that $L_\parallel + {\mit\Omega}\,I_\parallel/2$ is a constant of the motion. Here, $L_\parallel$
 is the total angular momentum of the system parallel to the magnetic
 field, and $I_\parallel$ is the moment of inertia of the system about
 an axis parallel to ${\bf B}$ which passes through the origin.
\end{enumerate}
}


