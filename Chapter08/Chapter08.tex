\chapter{Rigid Body Rotation}\label{srigid}
\section{Introduction}
This chapter examines the rotation of rigid bodies in
three dimensions.

\section{Fundamental Equations}
We can think of a rigid body as a collection of a large number of small mass elements
which all maintain a fixed spatial relationship with respect to one another.
Let there be $N$ elements, and let the $i$th element be of mass $m_i$, and instantaneous position
vector ${\bf r}_i$. The equation of motion of the $i$th element
is written
\begin{eqnarray}\label{e9.1}
m_i\,\frac{d^2{\bf r}_i}{dt^2} = \sum_{j=1,N}^{j\neq i}
{\bf f}_{ij} + {\bf F}_i.
\end{eqnarray}
Here, ${\bf f}_{ij}$ is the internal force exerted on the $i$th  element by the
$j$th element, and ${\bf F}_i$  the external force acting on the $i$th
element. The internal forces ${\bf f}_{ij}$ represent the 
stresses which develop within the body in order to ensure that its various
elements maintain a constant spatial relationship with respect to one another.
Of course, ${\bf f}_{ij} = - {\bf f}_{ji}$, by Newton's third law.
The external forces represent forces which originate outside the body.

Repeating the analysis of Section~\ref{new4}, we can
sum Equation~(\ref{e9.1}) over all mass elements to obtain
\begin{equation}\label{e9.2}
M\,\frac{d^2{\bf r}_{cm}}{dt^2} = {\bf F}.
\end{equation}
Here,  $M=\sum_{i=1,N} m_i$ is the total mass, ${\bf r}_{cm}$ 
the position vector of the center of mass [see Equation~(\ref{e3.27})],
and ${\bf F}  = \sum_{i=1,N} {\bf F}_i$  the total external force.
It can be seen that the center of mass of a rigid body moves under the action of the external forces like
a point particle whose mass is identical with that of the body.

Again repeating the analysis of Section~\ref{new4}, we can sum ${\bf r}_i\times$ Equation~(\ref{e9.1}) over all mass elements to obtain
\begin{equation}\label{e9.3}
\frac{d{\bf L}}{dt} = {\bf T}.
\end{equation}
Here, ${\bf L}=\sum_{i=1,N} m_i\,{\bf r}_i\times d{\bf r}_i/dt$ is the
total angular momentum of the body (about the origin), and
${\bf T} = \sum_{i=1, N} {\bf r}_i\times{\bf F}_i$ the
total external torque (about the origin). Note that the above equation is
only valid if the internal forces are {\em central}\/ in nature. However, this
is not a particularly onerous constraint. Equation~(\ref{e9.3}) describes
how the angular momentum of a rigid body evolves in time under the action
of
the external torques.

In the following, we shall only consider the {\em rotational}\/ motion of rigid bodies, since their translational motion is similar to that
of  point particles [see Equation~(\ref{e9.2})], and, therefore, fairly straightforward
in nature.

\section{Moment of Inertia Tensor}\label{mom}
Consider a rigid body rotating with fixed angular velocity \mbox{\boldmath$\omega$} about
an axis which passes through the origin---see Figure~\ref{rig}.
Let ${\bf r}_i$ be the position vector of the $i$th mass element, whose mass
is $m_i$.
We expect this position vector to {\em precess}\/ about the axis of rotation
(which is parallel to \mbox{\boldmath$\omega$})
with angular velocity $\omega$. It, therefore, follows from Equation~(\ref{e239})
that
\begin{equation}\label{e9.4}
\frac{d{\bf r}_i}{dt} =  \mbox{\boldmath$\omega$}\times {\bf r}_i.
\end{equation}
Thus, the above equation  specifies the velocity, ${\bf v}_i = d{\bf r}_i/dt$, 
of each mass element as the body rotates  with fixed angular velocity \mbox{\boldmath$\omega$} about
an axis passing  through the origin.

\begin{figure}
\epsfysize=2.5in
\centerline{\epsffile{Chapter08/fig8.01.eps}}
\caption{\em A rigid rotating body.}\label{rig}
\end{figure}

The total angular momentum of the body (about the origin) is written
\begin{equation}\label{e9.5}
{\bf L} = \sum_{i=1,N} m_i\,{\bf r}_i\times\frac{d{\bf r}_i}{dt} = 
\sum_{i=1,N}m_i\,{\bf r}_i\times(\mbox{\boldmath$\omega$}\times {\bf r}_i) = \sum_{i=1,N}m_i\left[r_i^{\,2}\,\mbox{\boldmath$\omega$}- ({\bf r}_i\cdot\mbox{\boldmath$\omega$})\,{\bf r}_i\right],
\end{equation}
where use has been made of Equation~(\ref{e9.4}), and some standard vector
identities (see Section~\ref{svtp}). The above formula can be written as a matrix equation
of the form
\begin{equation}\label{e9.6}
\left(\begin{array}{c}L_x\\L_y\\L_z\end{array}\right)=
\left(\begin{array}{ccc}
I_{xx}&I_{xy}&I_{xz}\\
I_{yx}&I_{yy}&I_{yz}\\
I_{zx}&I_{zy}&I_{zz}
\end{array}\right)\left(\begin{array}{c}\omega_x\\\omega_y\\\omega_z\end{array}\right),
\end{equation}
where
\begin{eqnarray}\label{e9.7}
I_{xx} &=& \sum_{i=1,N}(y_i^{\,2}+z_i^{\,2}) \,m_i= \int(y^2+ z^2)\,dm,\\[0.5ex]
I_{yy} &=& \sum_{i=1,N}(x_i^{\,2}+z_i^{\,2}) \,m_i= \int(x^2+ z^2)\,dm,\\[0.5ex]
I_{zz} &=& \sum_{i=1,N}(x_i^{\,2}+y_i^{\,2}) \,m_i= \int(x^2+ y^2)\,dm,\\[0.5ex]
I_{xy}=I_{yx} &=& - \sum_{i=1,N}x_i\,y_i \,m_i=- \int x\,y\,dm,\label{e9.11}\\[0.5ex]
I_{yz}=I_{zy} &=& - \sum_{i=1,N}y_i\,z_i \,m_i= -\int y\,z\,dm,\label{e9.12}\\[0.5ex]
I_{xz}=I_{zx} &=& - \sum_{i=1,N}x_i\,z_i \,m_i= -\int x\,z\,dm.\label{e9.13a}
\end{eqnarray}
Here, $I_{xx}$ is called the {\em moment of inertia}\/ about the $x$-axis,
$I_{yy}$  the moment of inertia about the $y$-axis, $I_{xy}$ 
the $xy$ {\em product of inertia}, $I_{yz}$  the $yz$ product of
inertia, {\em etc}. The matrix of the $I_{ij}$ values is
known as the {\em moment of inertia tensor}.\footnote{A tensor is the two-dimensional generalization of a vector. However, for present purposes, we can
simply think of a tensor as another name for a matrix.} Note that each component
of the moment of inertia tensor can be written as either a sum over separate
mass elements, or as an integral over infinitesimal mass elements.
In the integrals, $dm = \rho\,dV$, where $\rho$ is the mass density, and 
$dV$ a volume element.
Equation~(\ref{e9.6}) can be written more succinctly as
\begin{equation}\label{e9.13}
{\bf L} = \tilde{\bf I}\,\mbox{\boldmath$\omega$}.
\end{equation}
Here, it is understood that ${\bf L}$ and \mbox{\boldmath$\omega$} are
both {\em column vectors}, and $\tilde{\bf I}$ is the {\em matrix}\/ of the $I_{ij}$ values.
Note that $\tilde{\bf I}$ is a {\em real symmetric}\/ matrix:
{\em i.e.}, $I_{ij}^{\,\ast} = I_{ij}$ and $I_{ji} = I_{ij}$. 

In general, 
the angular momentum vector, ${\bf L}$, obtained from Equation (\ref{e9.13}),
points in a different direction to the angular velocity vector, \mbox{\boldmath$\omega$}. In other words, ${\bf L}$
is generally {\em not parallel}\/ to \mbox{\boldmath$\omega$}.

Finally, although the above results  were obtained assuming a
fixed angular velocity, they remain valid at each instant in time if the angular velocity varies.

\section{Rotational Kinetic Energy}
The instantaneous rotational kinetic energy of a rotating rigid body is written
\begin{equation}
K = \frac{1}{2}\sum_{i=1,N} m_i\left(\frac{d{\bf r}_i}{dt}\right)^2.
\end{equation}
Making use of Equation~(\ref{e9.4}), and some vector identities (see Section~\ref{sstp}),
the kinetic energy takes the form
\begin{equation}
K = \frac{1}{2}\sum_{i=1,N} m_i\,(\mbox{\boldmath$\omega$}\times
{\bf r}_i)\cdot(\mbox{\boldmath$\omega$}\times
{\bf r}_i) = \frac{1}{2}\,\mbox{\boldmath$\omega$} \cdot\!\! \sum_{i=1,N} 
m_i\,{\bf r}_i\times (\mbox{\boldmath$\omega$}\times {\bf r}_i).
\end{equation}
Hence, it follows from (\ref{e9.5}) that
\begin{equation}\label{e9.16}
K = \frac{1}{2}\,\, \mbox{\boldmath$\omega$} \cdot {\bf L}.
\end{equation}
Making use of Equation~(\ref{e9.13}), we can also
write
\begin{equation}\label{e9.17}
K = \frac{1}{2}\,\mbox{\boldmath$\omega$}^T\,\tilde{\bf I}\,\mbox{\boldmath$\omega$}.
\end{equation}
Here, $\mbox{\boldmath$\omega$}^T$ is the {\em row vector}\/ of the Cartesian components
$\omega_x$, $\omega_y$, $\omega_z$, which is, of course, the transpose
(denoted $~^T$) of the column vector \mbox{\boldmath$\omega$}.
When written in component form, the above equation yields
\begin{equation}\label{e9.18}
K = \frac{1}{2}\left(I_{xx}\,\omega_x^{\,2}+ I_{yy}\,\omega_y^{\,2}+I_{zz}\,\omega_z^{\,2} + 2\,I_{xy}\,\omega_x\,\omega_y + 2\,I_{yz}\,\omega_y\,\omega_z + 2\,I_{xz}\,\omega_x\,\omega_z\right).
\end{equation}

\section{Matrix Eigenvalue Theory}\label{smatrix}
It is time to review a little matrix theory. Suppose that ${\bf A}$ is
a {\em real symmetric}\/ matrix of dimension $n$. If follows that
${\bf A}^\ast = {\bf A}$ and ${\bf A}^T = {\bf A}$, where $~^\ast$
denotes a complex conjugate, and $~^T$ denotes a transpose.
Consider the matrix equation
\begin{equation}\label{e9.21}
{\bf A} \,{\bf x} = \lambda\,{\bf x}.
\end{equation}
Any column vector ${\bf x}$ which satisfies the above equation is called
an {\em eigenvector}\/ of ${\bf A}$. Likewise, the associated number $\lambda$ is called an {\em eigenvalue}\/ of ${\bf A}$. Let us investigate the properties of the eigenvectors and eigenvalues of a real
symmetric matrix.

Equation~(\ref{e9.21}) can be rearranged to give
\begin{equation}
({\bf A} - \lambda\,{\bf 1})\,{\bf x} = {\bf 0},
\end{equation}
where ${\bf 1}$ is the unit matrix. The above matrix equation is essentially a set of $n$ homogeneous
simultaneous algebraic equations for the $n$ components of ${\bf x}$. 
A well-known property of such a set of equations is that it only has a non-trivial
solution when the determinant of the associated matrix is set to zero. 
Hence, a necessary condition for the above set of equations to have a non-trivial
solution is that
\begin{equation}
|{\bf A} - \lambda\,{\bf 1}| = 0.
\end{equation}
The above formula is essentially an $n$th-order {\em polynomial}\/ equation
for $\lambda$. We know that such an equation has $n$ (possibly complex)
roots. Hence, we conclude that there are $n$ eigenvalues, and $n$ associated eigenvectors, of the $n$-dimensional matrix ${\bf A}$. 

Let us now demonstrate that the $n$ eigenvalues and eigenvectors of the real symmetric matrix
${\bf A}$ are all {\em real}. We have
\begin{equation}\label{e9.24}
{\bf A}\,{\bf x}_i = \lambda_i\,{\bf x}_i,
\end{equation}
and, taking the transpose and complex conjugate, 
\begin{equation}\label{e9.25}
{\bf x}_i^{\ast\,T}\,{\bf A} = \lambda_i^{\,\ast}\,{\bf x}_i^{\ast\,T},
\end{equation}
where ${\bf x}_i$ and $\lambda_i$ are the $i$th eigenvector and eigenvalue
of ${\bf A}$, respectively. Left multiplying Equation~(\ref{e9.24}) by
${\bf x}_i^{\ast\,T}$, we obtain
\begin{equation}
{\bf x}_i^{\ast\,T} {\bf A}\,{\bf x}_i = \lambda_i\,{\bf x}_i^{\ast\,T}{\bf x}_i.
\end{equation}
Likewise, right multiplying (\ref{e9.25}) by ${\bf x}_i$, we get
\begin{equation}
{\bf x}_i^{\ast\,T}\,{\bf A}\,{\bf x}_i = \lambda_i^{\,\ast}\,{\bf x}_i^{\ast\,T}{\bf x}_i.
\end{equation}
The difference of the previous two equations yields
\begin{equation}
(\lambda_i - \lambda_i^{\,\ast})\,{\bf x}_i^{\ast\,T} {\bf x}_i = 0.
\end{equation}
It follows that $\lambda_i=\lambda_i^{\,\ast}$, since ${\bf x}_i^{\ast\,T}{\bf x}_i$ (which is ${\bf x}_i^{\,\ast}\cdot{\bf x}_i$ in vector notation) is positive definite. Hence, $\lambda_i$ is real.
It immediately follows that ${\bf x}_i$ is real.

Next, let us show that two eigenvectors corresponding to two {\em different}\/ eigenvalues are {\em mutually orthogonal}. Let
\begin{eqnarray}
{\bf A}\,{\bf x}_i &=& \lambda_i\,{\bf x}_i,\\[0.5ex]
{\bf A}\,{\bf x}_j&=& \lambda_j\,{\bf x}_j,
\end{eqnarray}
where $\lambda_i\neq \lambda_j$. Taking the transpose of the first equation and right multiplying by ${\bf x}_j$, and left multiplying the second
equation by ${\bf x}_i^T$, we obtain
\begin{eqnarray}
{\bf x}_i^T\,{\bf A}\,{\bf x}_j &=& \lambda_i\,{\bf x}_i^T{\bf x}_j,\\[0.5ex]
{\bf x}_i^T\,{\bf A}\,{\bf x}_j&=& \lambda_j\,{\bf x}_i^T{\bf x}_j.
\end{eqnarray}
Taking the difference of the above two equations, we get
\begin{equation}
(\lambda_i-\lambda_j)\,{\bf x}_i^T{\bf x}_j = 0.
\end{equation}
Since, by hypothesis, $\lambda_i\neq \lambda_j$, it follows
that ${\bf x}_i^T{\bf x}_j = 0$. In vector notation, this is the same
as ${\bf x}_i \cdot{\bf x}_j=0$. Hence, the eigenvectors ${\bf x}_i$ and
${\bf x}_j$ are mutually orthogonal. 

Suppose that $\lambda_i=\lambda_j=\lambda$. In this case, we cannot conclude
that ${\bf x}_i^T{\bf x}_j = 0$ by the above argument. However, it is easily seen that any
linear combination of ${\bf x}_i$ and ${\bf x}_j$ is an eigenvector
of ${\bf A}$ with eigenvalue $\lambda$. Hence, it is possible
to define two new eigenvectors of ${\bf A}$, with the eigenvalue
$\lambda$, which are mutually orthogonal. For instance,
\begin{eqnarray}
{\bf x}_i' &=& {\bf x}_i,\\[0.5ex]
{\bf x}_j'&=& {\bf x}_j - \left(\frac{{\bf x}_i^T{\bf x}_j}{{\bf x}_i^T{\bf x}_i}\right) {\bf x}_i.
\end{eqnarray}
It should be clear that this argument can be generalized to deal with any
number of eigenvalues which take the same value.

In conclusion, a real symmetric $n$-dimensional matrix
possesses $n$ {\em real}\/ eigenvalues, with $n$ associated {\em real}\/ eigenvectors,
which are, or can be chosen to be, {\em mutually orthogonal}.

\section{Principal Axes of Rotation}
We have seen that the moment of inertia tensor, $\tilde{\bf I}$, defined in Section~\ref{mom}, takes the form of a  real symmetric three-dimensional
matrix. It therefore follows, from the matrix theory that we have just reviewed,
that the moment of inertia tensor possesses {\em three  mutually orthogonal eigenvectors}\/ which are associated with {\em three  real eigenvalues}. Let the $i$th eigenvector (which can be normalized to
be a unit vector) be denoted $\hat{\mbox{\boldmath$\omega$}}_i$, and the $i$th eigenvalue $\lambda_i$. It then
follows that
\begin{equation}\label{e9.34}
\tilde{\bf I}\, \hat{\mbox{\boldmath$\omega$}}_i = \lambda_i\,\hat{\mbox{\boldmath$\omega$}}_i ,
\end{equation}
for $i=1, 3$. 

The directions of the three mutually orthogonal unit vectors $\hat{\mbox{\boldmath$\omega$}}_i $ define the three so-called {\em principal axes
of rotation}\/ of the rigid body under investigation. These axes are special because when the body  rotates about
one of them ({\em i.e.}, when \mbox{\boldmath$\omega$} is parallel to one of them)  the angular momentum vector ${\bf L}$
becomes {\em parallel}\/ to the angular velocity vector \mbox{\boldmath$\omega$}.
This can be seen from a comparison of Equation~(\ref{e9.13}) and Equation~(\ref{e9.34}). 

Suppose that we reorient our Cartesian coordinate
axes so the they coincide with the mutually orthogonal principal axes of rotation. In this new reference frame, the eigenvectors of $\tilde{\bf I}$ are the unit vectors,
${\bf e}_x$, ${\bf e}_y$, and ${\bf e}_z$, and the eigenvalues
are the moments of inertia about these axes, $I_{xx}$, $I_{yy}$, and $I_{zz}$, respectively. These latter quantities are referred to as the
{\em principal moments of inertia}.
Note that the products of inertia are all {\em zero}\/ in the new
reference frame. Hence, in this frame, the moment
of inertia tensor takes the form of a  {\em diagonal}\/ matrix: {\em i.e.},
\begin{equation}
\tilde{\bf I} = 
\left(\begin{array}{ccc}
I_{xx}&0&0\\
0&I_{yy}&0\\
0&0&I_{zz}
\end{array}\right).
\end{equation}
Incidentally, it is easy to verify that ${\bf e}_x$, ${\bf e}_y$, and ${\bf e}_z$ are indeed
the eigenvectors of the above matrix, with the eigenvalues $I_{xx}$, $I_{yy}$, and $I_{zz}$, respectively, and that ${\bf L} = \tilde{\bf I}\,\mbox{\boldmath$\omega$}$ is indeed parallel to \mbox{\boldmath$\omega$} whenever \mbox{\boldmath$\omega$}
is directed along ${\bf e}_x$, ${\bf e}_y$, or ${\bf e}_z$.

When expressed in our new coordinate system, Equation~(\ref{e9.13})
yields
\begin{equation}
{\bf L} = \left(I_{xx}\,\omega_x,\,I_{yy}\,\omega_y,I_{zz}\,\omega_z\right),
\end{equation}
whereas Equation~(\ref{e9.18}) reduces to
\begin{equation}
K = \frac{1}{2}\left(I_{xx}\,\omega_x^{\,2} + I_{yy}\,\omega_y^{\,2}
+ I_{zz}\,\omega_z^{\,2}\right).
\end{equation}

In conclusion,  there are many great simplifications to be had by choosing a coordinate system whose axes coincide with the principal axes of rotation of the
rigid body under investigation. But how do we determine the directions of the
principal axes in practice?

Well, in general, we have to solve the eigenvalue equation
\begin{equation}
\tilde{\bf I}\,\hat{\mbox{\boldmath$\omega$}} = \lambda\,\hat{\mbox{\boldmath$\omega$}},
\end{equation}
or
\begin{equation}\label{e9.39}
\left(\begin{array}{ccc}
I_{xx}-\lambda&I_{xy}&I_{xz}\\
I_{yx}&I_{yy}-\lambda&I_{yz}\\
I_{zx}&I_{zy}&I_{zz}-\lambda
\end{array}\right)\left(\begin{array}{c}\cos\alpha\\\cos\beta\\\cos\gamma\end{array}\right) = \left(\begin{array}{c}0\\0\\0\end{array}\right),
\end{equation}
where $\hat{\mbox{\boldmath$\omega$}} = (\cos\alpha,\,\cos\,\beta,\,\cos\gamma)$, and $\cos^2\alpha+\cos^2\beta+\cos^2\gamma=1$. Here, $\alpha$ is the angle the unit eigenvector subtends with the $x$-axis, $\beta$ the angle it
subtends with the $y$-axis, and $\gamma$ the angle it subtends with the $z$-axis. Unfortunately, the analytic solution of the above matrix equation
is generally quite difficult.

Fortunately, however, in many instances the rigid body under investigation possesses some
kind of symmetry, so that at least one principal axis can be found by
inspection. In this case, the other two principal axes can be determined
as follows. 

Suppose that the $z$-axis is known to be a principal axes (at the origin)
in some coordinate system. It follows that the two products of inertia $I_{xz}$
and $I_{yz}$ are zero [otherwise, $(0,\,0,\,1)$ would not be an eigenvector in Equation~(\ref{e9.39})]. The other two principal axes must lie in the
$x$-$y$ plane: {\em i.e.}, $\cos\gamma=0$. It then follows that $\cos\beta=\sin\alpha$, since $\cos^2\alpha+\cos^2\beta+ \cos^2\gamma=1$. 
The first two rows in the matrix equation (\ref{e9.39}) thus
reduce to
\begin{eqnarray}\label{e9.40}
(I_{xx}-\lambda)\,\cos\alpha + I_{xy}\,\sin\alpha &=& 0,\\[0.5ex]
I_{xy}\,\cos\alpha + (I_{yy}-\lambda)\,\sin\alpha &=& 0.
\end{eqnarray}
Eliminating $\lambda$ between the above two equations, we obtain 
\begin{equation}\label{e9.42}
I_{xy} \,(1-\tan^2\alpha) = (I_{xx}-I_{yy})\,\tan\alpha.
\end{equation}
But, $\tan( 2\,\alpha) \equiv 2\,\tan\alpha/(1-\tan^2\alpha)$. Hence, Equation~(\ref{e9.42}) yields
\begin{equation}\label{e9.43}
\tan (2\,\alpha) = \frac{2\,I_{xy}}{I_{xx} - I_{yy}}.
\end{equation}
There are two values of $\alpha$, lying between $-\pi/2$ and $\pi/2$, which
satisfy the above equation. These specify the angles, $\alpha$, that
 the two mutually orthogonal principal axes lying in the $x$-$y$ plane
make with the $x$-axis. Hence, we have now determined the directions of
all three principal axes. Incidentally, once we have determined the orientation angle, $\alpha$, of a principal axis, 
we can then substitute back into Equation~(\ref{e9.40}) to obtain the corresponding
principal moment of inertia, $\lambda$. 

\begin{figure}
\epsfysize=2.5in
\centerline{\epsffile{Chapter08/fig8.02.eps}}
\caption{\em A uniform rectangular laminar.}\label{lam}
\end{figure}

As an example, consider a uniform rectangular lamina of mass $m$ and sides $a$ and $b$ which lies
in the $x$-$y$ plane, as shown in Figure~\ref{lam}. Suppose that the axis of
rotation passes through the origin ({\em i.e.}, through a corner of the lamina).
Since $z=0$ throughout the lamina, it follows from Equations~(\ref{e9.12}) and
(\ref{e9.13a}) that $I_{xz}= I_{yz} = 0$.  Hence, the $z$-axis
is a principal axis. After some straightforward integration, Equations~(\ref{e9.7})--(\ref{e9.11}) yield
\begin{eqnarray}
I_{xx} &=& \frac{1}{3}\,m\,b^2,\\[0.5ex]
I_{yy} &=& \frac{1}{3}\,m\,a^2,\\[0.5ex]
I_{xy} &=& - \frac{1}{4}\,m\,a\,b.
\end{eqnarray}
Thus, it follows from Equation~(\ref{e9.43}) that
\begin{equation}
\alpha = \frac{1}{2}\tan^{-1}\left(\frac{3}{2}\frac{a\,b}{a^2-b^2}\right).
\end{equation}
The above equation specifies the orientation of the two principal axes that
lie in the $x$-$y$ plane.
For the special case where $a=b$, we get  $\alpha=\pi/4,\,
3\pi/4$: {\em i.e.}, the two in-plane principal axes of a square lamina (at a corner) are parallel to the two diagonals of the
lamina.

\section{Euler's Equations}
The fundamental equation of motion of a rotating body [see Equation~(\ref{e9.3})],
\begin{equation}\label{e9.48}
{\bf T} = \frac{d{\bf L}}{dt},
\end{equation}
is only valid in an {\em inertial}\/ frame. However, we have seen that  ${\bf L}$
is most simply expressed in a frame of reference whose axes are aligned
along the principal axes of rotation of the body. Such a frame of reference
{\em rotates}\/ with the body, and is, therefore, {\em non-inertial}. Thus, it is helpful to define {\em two}\/ Cartesian coordinate systems, with the same origins. The first,
with coordinates $x$, $y$, $z$, is a fixed inertial frame---let us denote
this the {\em fixed frame}. The second, with coordinates $x'$, $y'$, $z'$,
co-rotates with the body in such a manner that the $x'$-, $y'$-, and $z'$-axes are always pointing along its principal axes of rotation---we shall
refer to this as the {\em body frame}. Since the body frame co-rotates with the body, its instantaneous angular velocity is the same as that of the
body. Hence, it follows from the analysis in Section~\ref{srot} that
\begin{equation}\label{e9.49}
\frac{d{\bf L}}{dt} = \frac{d{\bf L}}{dt'} + \mbox{\boldmath$\omega$}\times{\bf L}.
\end{equation}
Here, $d/dt$ is the time derivative in the fixed frame, and $d/dt'$ the
time derivative in the body frame.
Combining Equations~(\ref{e9.48}) and (\ref{e9.49}), we obtain
\begin{equation}\label{e9.50}
{\bf T} = \frac{d{\bf L}}{dt'} + \mbox{\boldmath$\omega$}\times{\bf L}.
\end{equation}
Now, in the body frame let ${\bf T}= (T_{x'},\,T_{y'},T_{z'})$ and  $\mbox{\boldmath$\omega$}= (\omega_{x'},\,\omega_{y'},\,\omega_{z'})$.
It follows that
 ${\bf L} =  (I_{x'x'}\,\omega_{x'},\,I_{y'y'}\,\omega_{y'},\,I_{z'z'}\,\omega_{z'})$,
 where $I_{x'x'}$, $I_{y'y'}$ and $I_{z'z'}$ are the principal
 moments of inertia. Hence, in the body frame, the components of Equation~(\ref{e9.50}) yield
 \begin{eqnarray}\label{e9.51}
 T_{x'} &=& I_{x'x'}\,\dot{\omega}_{x'} - (I_{y'y'}-I_{z'z'})\,\omega_{y'}\,\omega_{z'},\\[0.5ex]
 T_{y'} &=& I_{y'y'}\,\dot{\omega}_{y'} - (I_{z'z'}-I_{x'x'})\,\omega_{z'}\,\omega_{x'},\\[0.5ex]
 T_{z'} &=& I_{z'z'}\,\dot{\omega}_{z'} - (I_{x'x'}-I_{y'y'})\,\omega_{x'}\,\omega_{y'},\label{e9.53}
 \end{eqnarray}
 where $\dot{~}=d/dt'$. 
Here, we have made use of the fact that the  moments of inertia
of a rigid body are {\em constant}\/ in time in the co-rotating body frame.
The above equations are known as {\em Euler's equations}.
 
 Consider a rigid body which is constrained to rotate about a fixed
 axis with {\em constant}\/ angular velocity. It follows that
 $\dot{\omega}_{x'}=\dot{\omega}_{y'} = \dot{\omega}_{z'}=0$.
 Hence, Euler's equations, (\ref{e9.51})--(\ref{e9.53}), reduce to
 \begin{eqnarray}\label{e9.54}
 T_{x'} &=&  - (I_{y'y'}-I_{z'z'})\,\omega_{y'}\,\omega_{z'},\\[0.5ex]
 T_{y'} &=&  - (I_{z'z'}-I_{x'x'})\,\omega_{z'}\,\omega_{x'},\\[0.5ex]
 T_{z'} &=&  - (I_{x'x'}-I_{y'y'})\,\omega_{x'}\,\omega_{y'}.\label{e9.56}
 \end{eqnarray}
 These equations specify the components of the steady (in the body frame) torque  exerted on the body by 
 the constraining supports. The steady (in the
 body frame) angular momentum is written
 \begin{equation}\label{e9.57x}
 {\bf L} = (I_{x'x'}\,\omega_{x'},\,I_{y'y'}\,\omega_{y'},\,
 I_{z'z'}\,\omega_{z'}).
 \end{equation}
 It is easily demonstrated that ${\bf T}= \mbox{\boldmath$\omega$}\times{\bf L}$. Hence,
 the torque is perpendicular to both the angular velocity and the angular
 momentum vectors.
 Note that if the axis of rotation is a principal
 axis then two of the three components of 
\mbox{\boldmath$\omega$} are zero (in the body frame). It follows from Equations~(\ref{e9.54})--(\ref{e9.56}) that
all three components of the torque are zero. In other words, {\em zero}\/ external torque
is required to make the body  rotate steadily about a {\em principal axis}.

Suppose that the body is {\em freely rotating}: {\em i.e.}, there are no external
torques. Furthermore, let the body be {\em rotationally symmetric}\/
about the $z'$-axis. It follows that $I_{x'x'} = I_{y'y'} = I_\perp$.
Likewise, we can write $I_{z'z'} = I_\parallel$. In general, however, $I_\perp\neq I_\parallel$. Thus, Euler's equations  yield
\begin{eqnarray}\label{e9.57}
I_\perp\,\frac{d\omega_{x'}}{dt'} + (I_{\parallel}-I_{\perp})\,\omega_{z'}\,\omega_{y'} &=& 0,\\[0.5ex]
I_\perp\,\frac{d\omega_{y'}}{dt'} - (I_{\parallel}-I_{\perp})\,
\omega_{z'}\,\omega_{x'} &=& 0,\label{e9.58}\\[0.5ex]
\frac{d\omega_{z'}}{dt'} &=& 0.
\end{eqnarray}
Clearly, $\omega_{z'}$ is a constant of the motion.
Equation~(\ref{e9.57}) and (\ref{e9.58}) can be written
\begin{eqnarray}
\frac{d\omega_{x'}}{dt'} + {\mit\Omega}\,\omega_{y'} &=& 0,\\[0.5ex]
\frac{d\omega_{y'}}{dt'} - {\mit\Omega}\,\omega_{x'} &=& 0,
\end{eqnarray}
where ${\mit\Omega}= (I_{\parallel}/I_\perp-1)\,\omega_{z'}$. As is easily
demonstrated, the solution to the above equations is
\begin{eqnarray}
\omega_{x'} &=& \omega_\perp\,\cos({\mit\Omega}\,t'),\\[0.5ex]
\omega_{y'}&=&\omega_\perp\,\sin({\mit\Omega}\,t'),
\end{eqnarray}
where $\omega_\perp$ is a constant. Thus, the projection of the
angular velocity vector onto the $x'$-$y'$ plane has the fixed length
$\omega_\perp$, and rotates steadily about the $z'$-axis with angular
velocity ${\mit\Omega}$.
It follows that the length of the angular
velocity vector, $\omega=(\omega_{x'}^2+\omega_{y'}^2+\omega_{z'}^2)^{1/2}$,  is a constant of the motion.
Clearly,  the angular
velocity vector makes some constant angle, $\alpha$,  with the $z'$-axis, which implies that
$\omega_{z'} = \omega\,\cos\alpha$ and $\omega_\perp = \omega\,\sin\alpha$.
Hence, the components of the angular velocity vector are
\begin{eqnarray}
\omega_{x'} &=& \omega\,\sin\alpha\,\cos({\mit\Omega}\,t'),\\[0.5ex]
\omega_{y'} &=& \omega\,\sin\alpha\,\sin({\mit\Omega}\,t'),\\[0.5ex]
\omega_{z'} &=& \omega\,\cos\alpha,
\end{eqnarray}
where
\begin{equation}\label{e9.68a}
{\mit\Omega} =\omega\,\cos\alpha \left(\frac{I_\parallel}{I_\perp}-1\right).
\end{equation}
We conclude that, in the body frame, the angular velocity vector {\em precesses}\/ about the
symmetry axis ({\em i.e.}, the $z'$-axis) with the angular
frequency ${\mit\Omega}$. Now, the components of the angular momentum vector are
\begin{eqnarray}
L_{x'} &=& I_\perp\,\omega\,\sin\alpha\,\cos({\mit\Omega}\,t'),\\[0.5ex]
L_{y'} &=& I_\perp\,\omega\,\sin\alpha\,\sin({\mit\Omega}\,t'),\\[0.5ex]
L_{z'} &=& I_\parallel\,\omega\,\cos\alpha.
\end{eqnarray}
Thus, in the body frame, the angular momentum vector is also of constant length, and
precesses about the symmetry axis with the angular frequency
${\mit\Omega}$. Furthermore, the angular momentum vector makes a constant angle $\theta$ with the symmetry
axis, where
\begin{equation}\label{e9.73a}
\tan\theta = \frac{I_\perp}{I_\parallel}\,\tan\alpha.
\end{equation}
Note that the angular momentum vector, the angular velocity vector, and
the symmetry axis all lie in the {\em same plane}:
{\em i.e.}, ${\bf e}_{z'}\cdot{\bf L}\times \mbox{\boldmath$\omega$}=0$,
as can easily be verified. Moreover, the
angular momentum vector lies between the angular velocity vector and
the symmetry axis ({\em i.e.}, $\theta<\alpha$) for a flattened (or oblate) body
({\em i.e.}, $I_\perp< I_\parallel$), whereas the angular velocity vector lies
between the angular momentum vector and the symmetry axis ({\em i.e.}, $\theta>\alpha$) for  an elongated (or prolate) body ({\em i.e.}, $I_\perp>I_\parallel$). 

Let us now consider the most general motion of a freely rotating
{\em asymmetric}\/ rigid body, as seen in the body frame. Since a freely rotating
body experiences no external torques, its angular momentum vector
${\bf L}$ is a constant of the motion in the inertial fixed frame. In general, the direction of this vector varies with time in the non-inertial body frame, but its
length remains fixed. This can be seen from Equation~(\ref{e9.49}):
if $d{\bf L}/dt=0$ then the scalar product of this equation
with ${\bf L}$ implies that $d L^2/dt'=0$. It follows from Equation~(\ref{e9.57x}) that
\begin{equation}\label{e9.73x}
L^2 = I_{x'x'}^{\,2}\,\omega_{x'}^{\,2} + I_{y'y'}^{\,2}\,\omega_{y'}^{\,2} 
+ I_{z'z'}^{\,2}\,\omega_{z'}^{\,2} ={\rm constant}.
\end{equation}
The above constraint can also be derived directly from Euler's equations,
(\ref{e9.51})--(\ref{e9.53}), by setting $T_{x'}=T_{y'}=T_{z'}=0$.
A freely rotating body subject to no external torques clearly has a constant rotational kinetic energy.
Hence, from Equation~(\ref{e9.16}), 
\begin{equation}\label{e9.74x}
\bomega\cdot {\bf L} =  I_{x'x'}\,\omega_{x'}^{\,2} + I_{y'y'}\,\omega_{y'}^{\,2} 
+ I_{z'z'}\,\omega_{z'}^{\,2} ={\rm constant}.
\end{equation}
This constraint can also be derived directly from Euler's equations.
We conclude that, in the body frame, the components of $\bomega$ must
simultaneously satisfy the two constraints (\ref{e9.73x}) and (\ref{e9.74x}).
These constraints are the equations of two ellipsoids whose principal 
axes coincide with the principal axes of the body, and whose principal
radii are in the ratio $I_{x'x'}^{-1}:I_{y'y'}^{-1}:I_{z'z'}^{-1}$
and $I_{x'x'}^{-1/2}:I_{y'y'}^{-1/2}:I_{z'z'}^{-1/2}$, respectively. 
In general, the intersection of these two ellipsoids is a {\em closed curve}. 
Hence, we conclude that the most general motion of a freely rotating
asymmetric body, as seen in the body frame, is a form of irregular
{\em precession}\/ in which the tip of the angular
velocity vector $\bomega$   periodically traces out the aforementioned closed curve. It is easily demonstrated that the tip of the angular momentum vector ${\bf L}$
periodically traces out a different closed curve.

\section{Eulerian Angles}\label{seuler}
We have seen how we can solve Euler's equations to determine the properties
of a rotating body in the co-rotating {\em body frame}. Let us now investigate how we can determine the same properties in the inertial
{\em fixed frame}.

The fixed frame and the body frame share the
same origin. Hence, we can transform from one to the other by
means of an appropriate {\em rotation}\/ of our coordinate axes.
In general, if we restrict ourselves to rotations about one of the Cartesian 
axes,   {\em three}\/ successive rotations are required to
transform the fixed frame into the body frame. There are, in fact, many different
ways to combined three successive rotations in order to achieve this. In the following, we shall describe the
most widely used method, which is due to Euler.

We start in the fixed frame, which has coordinates $x$, $y$, $z$, and
unit vectors ${\bf e}_x$, ${\bf e}_y$, ${\bf e}_z$.  Our first rotation
is counterclockwise (looking down the axis) through an angle $\phi$ about the $z$-axis. The new frame has coordinates $x''$, $y''$, $z''$, and
unit vectors ${\bf e}_{x''}$, ${\bf e}_{y''}$, ${\bf e}_{z''}$. According
to Equations~(\ref{t1})--(\ref{t3}), the transformation of coordinates can be represented as
follows:
\begin{equation}\label{e9.72}
\left(\begin{array}{c}x''\\y''\\z''\end{array}\right)=
\left(\begin{array}{ccc}
\cos\phi&\sin\phi&0\\
-\sin\phi&\cos\phi&0\\
0&0&1
\end{array}\right)\left(\begin{array}{c}x\\y\\z\end{array}\right).
\end{equation}
The angular velocity vector associated with $\phi$ has the magnitude $\dot{\phi}$, 
and is directed along ${\bf e}_z$ ({\em i.e.}, along the axis of rotation).
Hence, we can write
\begin{equation}\label{e9.73}
\mbox{\boldmath$\omega$}_\phi = \dot{\phi}\,{\bf e}_z.
\end{equation}
Clearly, $\dot{\phi}$ is the precession rate about the ${\bf e}_z$ axis,
as seen in the fixed frame.

The second rotation is counterclockwise (looking down the axis) through
an angle $\theta$ about the $x''$-axis. The new frame has coordinates
$x'''$, $y'''$, $z'''$, and unit vectors ${\bf e}_{x'''}$, ${\bf e}_{y'''}$, ${\bf e}_{z'''}$. By analogy with Equation~(\ref{e9.72}), the transformation
of coordinates can be represented as follows:
\begin{equation}
\left(\begin{array}{c}x'''\\y'''\\z'''\end{array}\right)=
\left(\begin{array}{ccc}
1&0&0\\
0&\cos\theta&\sin\theta\\
0&-\sin\theta&\cos\theta
\end{array}\right)\left(\begin{array}{c}x''\\y''\\z''\end{array}\right).
\end{equation}
The angular velocity vector associated with $\theta$ has the magnitude $\dot{\theta}$, 
and is directed along ${\bf e}_{x''}$ ({\em i.e.}, along the axis of rotation).
Hence, we can write
\begin{equation}\label{e9.75}
\mbox{\boldmath$\omega$}_\theta = \dot{\theta}\,{\bf e}_{x''}.
\end{equation}

The third rotation is counterclockwise (looking down the axis) through
an angle $\psi$ about the $z'''$-axis. The new frame is the body frame, which has coordinates
$x'$, $y'$, $z'$, and unit vectors ${\bf e}_{x'}$, ${\bf e}_{y'}$, ${\bf e}_{z'}$. The transformation of coordinates can be represented as
follows:
\begin{equation}
\left(\begin{array}{c}x'\\y'\\z'\end{array}\right)=
\left(\begin{array}{ccc}
\cos\psi&\sin\psi&0\\
-\sin\psi&\cos\psi&0\\
0&0&1
\end{array}\right)\left(\begin{array}{c}x'''\\y'''\\z'''\end{array}\right).
\end{equation}
The angular velocity vector associated with $\psi$ has the magnitude $\dot{\psi}$, 
and is directed along ${\bf e}_{z''}$ ({\em i.e.}, along the axis of rotation).
Note that ${\bf e}_{z'''}={\bf e}_{z'}$, since the third rotation is about
${\bf e}_{z'''}$.
Hence, we can write
\begin{equation}\label{e9.77}
\mbox{\boldmath$\omega$}_\psi= \dot{\psi}\,{\bf e}_{z'}.
\end{equation}
Clearly, $\dot{\psi}$ is {\em minus}\/ the precession rate about the ${\bf e}_{z'}$ axis, as seen in the body frame.

The full transformation between the fixed frame and the body frame
is rather complicated. However, the following results can easily be
verified:
\begin{eqnarray}\label{e9.78}
{\bf e}_z &=& \sin\psi\,\sin\theta\,{\bf e}_{x'} + \cos\psi\,\sin\theta\,{\bf e}_{y'}
+ \cos\theta\,{\bf e}_{z'},\\[0.5ex]\label{e9.79}
{\bf e}_{x''} &=& \cos\psi\,{\bf e}_{x'} -\sin\psi\,{\bf e}_{y'}.
\end{eqnarray}
It follows from Equation~(\ref{e9.78}) that ${\bf e}_z\!\cdot\!{\bf e}_{z'} =\cos\theta$. In other words, $\theta$ is the angle of inclination between the
$z$- and $z'$-axes.
Finally, since the total angular velocity can be written
\begin{equation}\label{e9.81aa}
\mbox{\boldmath$\omega$} = \mbox{\boldmath$\omega$}_\phi
+\mbox{\boldmath$\omega$}_\theta+\mbox{\boldmath$\omega$}_\psi,
\end{equation}
Equations~(\ref{e9.73}), (\ref{e9.75}), and (\ref{e9.77})--(\ref{e9.79})
yield
\begin{eqnarray}\label{e9.81}
\omega_{x'} &=& \sin\psi\,\sin\theta\,\dot{\phi} +\cos\psi\,\dot{\theta},\\[0.5ex]
\omega_{y'} &=& \cos\psi\,\sin\theta\,\dot{\phi}-\sin\psi\,\dot{\theta},\label{e9.82}\\[0.5ex]\label{e9.83}
\omega_{z'} &=& \cos\theta\,\dot{\phi} +\dot{\psi}.
\end{eqnarray}

The angles $\phi$, $\theta$, and $\psi$ are termed {\em Eulerian angles}. Each has a clear physical interpretation: $\phi$ is the angle of precession
about the ${\bf e}_z$ axis in the fixed frame, $\psi$ is minus the angle of precession about the
${\bf e}_{z'}$ axis in the body frame, and $\theta$ is the angle of inclination
between the ${\bf e}_z$ and ${\bf e}_{z'}$ axes. Moreover, we can
express the components of the angular velocity vector \mbox{\boldmath$\omega$} in the body frame entirely in terms of the Eulerian angles, and their time derivatives  [see Equations~(\ref{e9.81})--(\ref{e9.83})].

Consider a rigid body which is constrained to rotate about a fixed axis with
the constant angular velocity $\omega$. Let the fixed angular velocity
vector point along the $z$-axis.  In the previous section, we saw that
the angular momentum and the torque were both steady in the body frame.
Since there is no precession of quantities in the body frame, it follows that
the Eulerian angle $\psi$ is  constant. Furthermore, since the angular velocity vector is
fixed in the body frame, as well as the fixed frame [as can be seen by applying Equation~(\ref{e9.49})
to \mbox{\boldmath$\omega$} instead of ${\bf L}$], it must subtend a constant angle  with the
${\bf e}_{z'}$ axis. Hence, the Eulerian angle $\theta$ is also constant.
It follows from Equations~(\ref{e9.81})--(\ref{e9.83}) that
\begin{eqnarray}
\omega_{x'} &=& \sin\psi\,\sin\theta\,\dot{\phi},\\[0.5ex]
\omega_{y'} &=& \cos\psi\,\sin\theta\,\dot{\phi},\\[0.5ex]
\omega_{z'} &=& \cos\theta\,\dot{\phi},
\end{eqnarray}
which implies that $ \omega\equiv (\omega_{x'}^{\,2}+\omega_{y'}^{\,2}+\omega_{z'}^{\,2})^{1/2}=\dot{\phi}$. In other words, the precession
rate, $\dot{\phi}$,  in the fixed frame is equal to $\omega$. Hence, in the fixed frame,
the constant torque and angular momentum vectors found in the body
frame {\em precess}\/ about the angular velocity vector ({\em i.e.}, about the
$z$-axis) at the rate $\omega$. As discussed in the previous section, for the special case where the angular
velocity vector is {\em parallel}\/ to one of the  principal axes of the body, the angular
momentum vector is parallel to the angular velocity vector, and the torque is {\em zero}. Thus,
in this case, there is no precession in the fixed frame.

Consider a rotating device such as a flywheel or a propeller. If the device
is {\em statically balanced}\/ then its center of mass lies on the axis
of rotation. This is desirable since, otherwise, gravity, which effectively
acts at the center of mass, exerts a varying torque about the
axis of rotation as the device rotates, giving rise to unsteady rotation. If the device is
{\em dynamically balanced}\/ then the axis of rotation is also
a principal axis, so that, as the device rotates its angular momentum vector,
${\bf L}$,
remains parallel to the axis of rotation. This is desirable since, otherwise,
the angular momentum vector is not parallel to the axis of rotation, and, therefore, precesses around it. Since $d{\bf L}/dt$ is equal to the
torque, a precessing torque must also be applied to the device (at right-angles
to both the axis and ${\bf L}$). The result is a reaction
on the bearings which can give rise to violent vibration and
wobbling, even when the device is statically balanced.

Consider a freely rotating body which is rotationally symmetric about one axis (the $z'$-axis). In the absence of an external torque, the
angular momentum vector ${\bf L}$ is a constant of the motion [see Equation~(\ref{e9.3})]. Let ${\bf L}$ point along the $z$-axis. In the
previous section, we saw that the angular momentum vector subtends a
constant angle $\theta$ with the axis of symmetry: {\em i.e.}, with the $z'$-axis. Hence, the time derivative
of the Eulerian angle $\theta$ is zero. We also saw that the angular momentum
vector, the axis of symmetry, and the angular velocity vector are coplanar.
Consider an instant in time at which all of these vectors lie in the $y'$-$z'$
plane. This implies that $\omega_{x'}=0$. According to the
previous section, the angular velocity vector subtends a constant
angle $\alpha$ with the symmetry axis.  It follows that
$\omega_{y'}=\omega\,\sin\alpha$ and $\omega_{z'}=\omega\,\cos\alpha$. Equation (\ref{e9.81}) yields
$\psi=0$. Hence, Equation~(\ref{e9.82}) yields
\begin{equation}\label{e9.88}
\omega\,\sin\alpha = \sin\theta\,\dot{\phi}.
\end{equation}
This can be combined with Equation~(\ref{e9.73a}) to give
\begin{equation}\label{e9.89}
\dot{\phi} = \omega\left[1 + \left(\frac{I_\parallel^{\,2}}{I_\perp^{\,2}}-1\right)\cos^2\alpha\right]^{1/2}.
\end{equation}
Finally, Equations~(\ref{e9.83}), together with (\ref{e9.73a}) and (\ref{e9.88}),
yields
\begin{equation}
\dot{\psi} = \omega\,\cos\alpha-\cos\theta\,\dot{\phi} = \omega\,\cos\alpha\left(1-\frac{\tan\alpha}{\tan\theta}\right)=
\omega\,\cos\alpha\left(1-\frac{I_\parallel}{I_\perp}\right).
\end{equation}
A comparison of the above equation with Equation~(\ref{e9.68a}) gives
\begin{equation}
\dot{\psi}=-{\mit\Omega}.
\end{equation}
Thus, as expected, $\dot{\psi}$ is minus the precession rate (of the angular
momentum and angular velocity vectors) in the body frame. On the other hand, $\dot{\phi}$ is the precession rate (of the angular velocity vector
and the symmetry axis) in the
fixed frame. Note that $\dot{\phi}$ and ${\mit\Omega}$ are
quite dissimilar. For instance, ${\mit\Omega}$ is negative for elongated
bodies ($I_\parallel<I_\perp$) whereas $\dot{\phi}$ is positive definite. It follows that the precession is always in the
same sense as $L_z$ in the fixed frame, whereas the
precession in the body frame is in the opposite sense to $L_{z'}$ for elongated bodies. We
found, in the previous section, that for a {\em flattened}\/ body the angular
momentum  vector lies between the angular velocity vector and the symmetry
axis. This means that, in the fixed frame, the angular velocity vector
and the symmetry axis lie on {\em opposite}\/ sides of the fixed angular
momentum vector.  On the other hand, for an {\em elongated}\/ body
we found that the angular velocity vector lies between the angular momentum
vector and the symmetry axis. This means that, in the fixed frame, the
angular velocity vector and the symmetry axis lie on the {\em same}\/ side of
the fixed angular momentum vector. (Recall that the angular
momentum vector, the angular velocity vector, and the symmetry
axis, are coplanar.)

As an example, consider the free rotation of a thin disk. It is easily
demonstrated (from the perpendicular axis theorem)
that
\begin{equation}
I_\parallel = 2\,I_\perp
\end{equation}
for such a disk. Hence, from Equation~(\ref{e9.68a}), the precession rate in the
body frame is
\begin{equation}
{\mit\Omega} = \omega\,\cos\alpha.
\end{equation}
According to Equation~(\ref{e9.89}), the precession rate in the fixed frame is
\begin{equation}
\dot{\phi} = \omega\left[1+ 3\,\cos^2\alpha\right]^{1/2}.
\end{equation}
In the limit in which $\alpha$ is small ({\em i.e.}, in which the
angular velocity vector is almost parallel to the symmetry axis),
we obtain
\begin{eqnarray}
{\mit\Omega}&\simeq & \omega,\\[0.5ex]
\dot{\phi}&\simeq & 2\,\omega.
\end{eqnarray}
Thus, the symmetry axis precesses in the fixed frame at approximately
twice the angular speed of rotation. This precession is manifest as
a wobbling motion.

It is known that the axis of rotation of the Earth is  very slightly inclined to
its symmetry axis (which passes through the two geographic poles). The angle $\alpha$ is approximately $0.2$ seconds of an arc
(which corresponds to a distance of about $6\,{\rm m}$ on the Earth's surface).
It is also known that the ratio of the terrestrial moments of inertia 
is about $I_\parallel/I_\perp=1.00327$, as determined from the
Earth's oblateness---see Section~\ref{smcl}. Hence, from (\ref{e9.68a}), the precession rate of the
angular velocity vector about the symmetry axis, as viewed in a geostationary reference frame, is
\begin{equation}
{\mit\Omega} = 0.00327\,\omega,
\end{equation}
giving a precession period of
\begin{equation}
T'= \frac{2\pi}{{\mit\Omega}} = 305\,\,{\rm days}.
\end{equation}
(Of course, $2\pi/\omega = 1$ day.)
The observed period of precession is about 440 days. The disagreement
between theory and observation is attributed to the fact that the Earth
is not perfectly rigid. 
The Earth's symmetry axis subtends an angle $\theta\simeq \alpha = 0.2''$ [see (\ref{e9.73a})] with 
its  angular momentum vector, but lies on the opposite side of this vector to the angular velocity vector. 
This implies that, as viewed from space, the Earth's angular velocity vector is almost parallel to its
fixed angular momentum vector, whereas its symmetry axis subtends an angle of $0.2"$ with both vectors, and
precesses about them.
The (theoretical) precession rate of the Earth's symmetry
axis, as seen from space, is given by Equation~(\ref{e9.89}):
\begin{equation}
\dot{\phi} = 1.00327\,\omega.
\end{equation}
The associated precession period is
\begin{equation}
T = \frac{2\pi}{\dot{\phi}} = 0.997\,\,{\rm days}.
\end{equation}
The free precession of the Earth's symmetry axis in space, which is  known as the {\em Chandler wobble}, since it
was discovered by the American astronomer Seth Chandler in 1891, 
 is superimposed on
a much slower  forced precession, with a period of about 26,000 years, caused by
the small gravitational torque exerted on the Earth by the Sun and the
Moon, as a consequence of the Earth's slight oblateness---see Section~\ref{sprec}. 

\section{Gyroscopic Precession}\label{sgyro}
Let us now study the motion of a {\em rigid rotationally symmetric top}\/ which
is free to turn about a fixed point (without friction), but which is
subject to a gravitational torque---see Figure~\ref{top}. Suppose that the $z'$-axis
coincides with the symmetry axis. Let the principal moment of inertia about the symmetry axis
be $I_\parallel$, and let the other principal moments both take the value $I_\perp$.
Suppose that the
$z$-axis points vertically upward, and let the common origin, $O$,  of  the
fixed and body frames coincide with the fixed point about which the top
turns. Suppose that
the center of mass of the top lies a distance $l$ along its symmetry axis from point $O$, and
that the mass of the top is $m$. Let the symmetry axis of the
top subtend an angle $\theta$ (which is an Eulerian angle) with
the upward vertical. 

\begin{figure}
\epsfysize=2.5in
\centerline{\epsffile{Chapter08/fig8.03.eps}}
\caption{\em A symmetric top.}\label{top}
\end{figure}

Consider an instant in time at which the Eulerian angle $\psi$ is zero.
This implies that the $x'$-axis is horizontal [see Equation~(\ref{e9.78})], as shown in the diagram.
The gravitational force, which acts at the
center of mass, thus exerts a torque $m\,g\,l\,\sin\theta$ in the
$x'$-direction. Hence, the components of the torque in the body
frame are
\begin{eqnarray}
T_{x'} &=& m\,g\,l\,\sin\theta,\\[0.5ex]
T_{y'} &=& 0,\\[0.5ex]
T_{z'} &=& 0.
\end{eqnarray}
The components of the angular velocity vector in the body frame
are given by Equations~(\ref{e9.81})--(\ref{e9.83}). 
 Thus, Euler's equations
(\ref{e9.51})--(\ref{e9.53}) take the form:
\begin{eqnarray}
m\,g\,l\,\sin\theta &=& I_\perp\,(\ddot{\theta}- \cos\theta\,\sin\theta\,\dot{\phi}^{\,2}) + L_\psi\,\sin\theta\,\dot{\phi},\label{e9.104}\\[0.5ex]
0&=& I_\perp\,(2\,\cos\theta\,\dot{\theta}\,\dot{\phi} +\sin\theta\,\ddot{\phi}) - L_\psi\,\dot{\theta},\label{e9.105}\\[0.5ex]
0&=& \dot{L}_\psi,\label{e9.106}
\end{eqnarray}
where
\begin{equation}
L_\psi = I_\parallel\,(\cos\theta\,\dot{\phi}+\dot{\psi})= I_\parallel\,{\mit\Omega},
\end{equation}
and ${\mit\Omega} = \omega_{z'}$ is the angular velocity of the top.
Multiplying Equation (\ref{e9.105}) by $\sin\theta$, we obtain
\begin{equation}\label{e9.108}
\dot{L}_\phi = 0,
\end{equation}
where
\begin{equation}\label{e9.109}
L_\phi = I_\perp\,\sin^2\theta\,\dot{\phi} + L_\psi\,\cos\theta.
\end{equation}
According to Equations~(\ref{e9.106}) and (\ref{e9.108}), the
two quantities $L_\psi$ and $L_\phi$ are constants
of the motion. These two quantities are the {\em angular momenta}\/ of the
system about the $z'$- and $z$-axis, respectively. (To be more exact,
they are the generalized momenta conjugate to the coordinates $\psi$ and
$\phi$, respectively---see Section~\ref{s10.8}.) They
are conserved because the gravitational torque has no component
along either the $z'$-  or the $z$-axis. (Alternatively, they are
conserved because the Lagrangian of the top does not depend explicitly
on the coordinates $\psi$ and $\theta$---see Section~\ref{s10.8}.)

If there are no frictional forces acting on the top then the total
energy, $E=K+U$, is also a constant of the motion. Now,
\begin{equation}
E = \frac{1}{2}\,\left(I_\perp\,\omega_{x'}^{\,2} + I_\perp\,\omega_{y'}^{\,2}+
I_\parallel\,\omega_{z'}^{\,2}\right) + m\,g\,l\,\cos\theta.
\end{equation}
When written in terms of the Eulerian angles (with $\psi=0$), this
becomes
\begin{equation}\label{e9.111}
E=\frac{1}{2}\left(I_\perp\,\dot{\theta}^{\,2} + I_\perp\,\sin^2\theta\,\dot{\phi}^{\,2} + L_\psi^{\,2}/I_\parallel\right) + m\,g\,l\,\cos\theta.
\end{equation}
Eliminating $\dot{\phi}$ between Equations~(\ref{e9.109}) and (\ref{e9.111}),
we obtain the following differential equation for $\theta$:
\begin{equation}
E = \frac{1}{2}\,I_\perp\,\dot{\theta}^{\,2} + \frac{(L_\phi-L_\psi\,\cos\theta)^2}{2\,I_\perp\,\sin^2\theta} + \frac{1}{2}\frac{L_\psi^{\,2}}{I_\parallel} + m\,g\,l\,\cos\theta.
\end{equation}
Let
\begin{equation}
E' = E - \frac{1}{2}\frac{L_\psi^{\,2}}{I_\parallel},
\end{equation}
and $u=\cos\theta$. It follows that
\begin{equation}
\dot{u}^{\,2} = 2\,(E' -m\,g\,l\,u)\,(1-u^2)\,I_\perp^{-1} - (L_\phi-L_\psi\,u)^2\,I_\perp^{-2},
\end{equation}
or
\begin{equation}\label{e9.115}
\dot{u}^{\,2} = f(u),
\end{equation}
where $f(u)$ is a {\em cubic}\/ polynomial. In principal, the above equation
can be integrated to give $u$ (and, hence, $\theta$) as a function of $t$:
\begin{equation}\label{e9.116}
t = \int_{u_0}^u\frac{du'}{\sqrt{f(u')}}.
\end{equation}

Fortunately, we do not have to perform the above integration (which is very ugly)
in order to discuss the general properties of the solution to
Equation~(\ref{e9.115}). It is clear, from Equation~(\ref{e9.116}), that $f(u)$
needs to be {\em positive}\/ in order to obtain a physical solution. Hence, the
limits of the motion in $\theta$ are determined by the three roots of the
equation $f(u)=0$. Since $\theta$ must lie between $0$ and $\pi/2$,
it follows that $u$ must lie between 0 and 1. It can easily be demonstrated that 
$f\rightarrow\pm\infty$ as $u\rightarrow\pm\infty$. It can also be shown
that the
largest root $u_3$ lies in the region $u_3>1$, and the two smaller
roots $u_1$ and $u_2$ (if they exist) lie in the region $-1\leq u\leq +1$.
It follows that, in the region $-1\leq u\leq 1$, $f(u)$ is only positive between $u_1$ and $u_2$. 
Figure~\ref{fcurve} shows
a case where $u_1$ and $u_2$ lie in the range
0 to 1. The corresponding values of $\theta$---$\theta_1$ and $\theta_2$, say---are then the limits of the vertical motion.
The axis of the top oscillates backward and forward between these two
values of $\theta$ as the top precesses about the vertical axis. This
oscillation is called {\em nutation}. Incidentally, if $u_1$ becomes
negative then the nutation will cause the top to strike the ground (assuming
that it is spinning on a level surface).

\begin{figure}
\epsfysize=2.5in
\centerline{\epsffile{Chapter08/fig8.04.eps}}
\caption{\em The function $f(u)$.}\label{fcurve}
\end{figure}

If there is a double root of $f(u)=0$ ({\em i.e.}, if $u_1 = u_2$) then
there is no nutation, and the top precesses steadily.  However, the
criterion for steady precession is most easily obtained directly from
Equation~(\ref{e9.104}). In the absence of nutation, $\dot{\theta}=\ddot{\theta}=0$.
Hence, we obtain
\begin{equation}
m\,g\,l = -I_\perp\,\cos\theta\,\dot{\phi}^{\,2}+ L_\psi\,\dot{\phi},
\end{equation}
or 
\begin{equation}\label{e9.118}
{\mit\Omega}= \frac{m\,g\,l}{I_\parallel\,\dot{\phi}} + \frac{I_\perp}{I_\parallel}\,\cos\theta\,\dot{\phi}.
\end{equation}
The above equation is the criterion for steady precession. 
Since the right-hand side of Equation~(\ref{e9.118}) possesses a minimum
value, which is given by  $2\,\sqrt{m\,g\,l\,I_\perp\,\cos\theta}/I_\parallel$, it follows that
\begin{equation}
{\mit\Omega} >  {\mit\Omega}_{\rm crit} = \frac{2\,\sqrt{m\,g\,l\,I_\perp\,\cos\theta}}{I_\parallel}
\end{equation}
is a necessary condition for obtaining steady precession at the
inclination angle $\theta$. For ${\mit\Omega}>{\mit\Omega}_{\rm crit}$, there are two
roots to Equation~(\ref{e9.118}), corresponding to a slow and a fast steady
precession rate for a given inclination angle $\theta$. If ${\mit\Omega}\gg {\mit\Omega}_{\rm crit}$ then these two roots are approximately given by
\begin{eqnarray}
(\dot{\phi})_{\rm slow} &\simeq & \frac{m\,g\,l}{I_\parallel\,{\mit\Omega}},\\[0.5ex]
(\dot{\phi})_{\rm fast} &\simeq & \frac{I_\parallel\,{\mit\Omega}}{I_\perp\,\cos\theta}.
\end{eqnarray}
The slower of these two precession rates is the one which is
generally observed.

\section{Rotational Stability}
Consider a rigid body for which all of the principal moments of inertia
are distinct. Let $I_{z'z'} > I_{y'y'} > I_{x'x'}$. Suppose that the
body is rotating freely  about one of its principal axes. What happens when the
body is slightly disturbed? 

Let the body be  initially rotating about the $x'$-axis, so that
\begin{equation}
\mbox{\boldmath$\omega$} = \omega_{x'}\,{\bf e}_{x'}.
\end{equation}
If we apply a slight perturbation then the angular velocity becomes
\begin{equation}
\mbox{\boldmath$\omega$} = \omega_{x'}\,{\bf e}_{x'} + \lambda\,{\bf e}_{y'} + \mu\,{\bf e}_{z'},
\end{equation}
where $\lambda$ and $\mu$ are both assumed to be small.
Euler's equations (\ref{e9.51})--(\ref{e9.53}) take the
form
\begin{eqnarray}
I_{x'x'}\,\dot{\omega}_{x'} - (I_{y'y'}-I_{z'z'})\,\lambda\,\mu&=& 0,\\[0.5ex]
I_{y'y'}\,\dot{\lambda} - (I_{z'z'}-I_{x'x'})\,\omega_{x'}\,\mu &=& 0,\\[0.5ex]
I_{z'z'} \,\dot{\mu} - (I_{x'x'}-I_{y'y'})\,\omega_{x'}\,\lambda&=& 0.
\end{eqnarray}
Since $\lambda\,\mu$ is second-order in small quantities---and, therefore, negligible---the first of the above equations tells us that $\omega_{x'}$
is an approximate constant of the motion. The other two equations
can be written
\begin{eqnarray}
\dot{\lambda} &=& \left[\frac{(I_{z'z'}-I_{x'x'})\,\omega_{x'}}{I_{y'y'}}\right]\mu,\\[0.5ex]
\dot{\mu} &=& - \left[\frac{(I_{y'y'}-I_{x'x'})\,\omega_{x'}}{I_{z'z'}}\right]\lambda.
\end{eqnarray}
Differentiating the first equation with respect to time, and then eliminating
$\dot{\mu}$, we obtain
\begin{equation}
\ddot{\lambda} + \left[\frac{(I_{y'y'}-I_{x'x'})\,(I_{z'z'}-I_{x'x'})}{I_{y'y'}\,I_{z'z'}}\right]\omega_{x'}^{\,2} \,\lambda = 0.
\end{equation}
It is easily demonstrated that $\mu$ satisfies the same differential equation.
Since the term in square brackets in the above equation is {\em positive},
the equation takes the form of a {\em simple harmonic equation}, and, thus,
has the bounded solution:
\begin{equation}
\lambda = \lambda_0 \,\cos({\mit\Omega}_{x'}\,t' - \alpha).
\end{equation}
Here, $\lambda_0$ and $\alpha$ are constants of integration, and
\begin{equation}
{\mit\Omega}_{x'} = \left[\frac{(I_{y'y'}-I_{x'x'})\,(I_{z'z'}-I_{x'x'})}{I_{y'y'}\,I_{z'z'}}\right]^{1/2}\! \omega_{x'}.
\end{equation}
Thus, the body oscillates sinusoidally about its initial state
with the angular frequency ${\mit\Omega}_{x'}$.
It follows that  the body  is {\em stable}\/ to small perturbations
when rotating about the $x'$-axis, in the sense that the amplitude of
such perturbations does not grow in time.

Suppose that the body is initially rotating about the $z'$-axis, and
is subject to a small perturbation. A similar argument to the above allows
us to conclude that the body oscillates sinusoidally about its initial state
with angular frequency
\begin{equation}
{\mit\Omega}_{z'} = \left[\frac{(I_{z'z'}-I_{x'x'})\,(I_{z'z'}-I_{y'y'})}{I_{x'x'}\,I_{y'y'}}\right]^{1/2}\! \omega_{z'}.
\end{equation}
Hence, the body is also stable to small perturbations when rotating about the
$z'$-axis.

Suppose, finally, that the body is initially rotating about the $y'$-axis,
and is subject to a small perturbation, such that
\begin{equation}
\mbox{\boldmath$\omega$} = \lambda\,{\bf e}_{x'} + \omega_{y'}\,{\bf e}_{y'} + \mu\,{\bf e}_{z'}.
\end{equation}
It is easily demonstrated that $\lambda$ satisfies the following differential
equation:
\begin{equation}
\ddot{\lambda} - \left[\frac{(I_{y'y'}-I_{x'x'})\,(I_{z'z'}-I_{y'y'})}{I_{x'x'}\,I_{z'z'}}\right]\omega_{y'}^{\,2} \,\lambda = 0.
\end{equation}
Note that the term in square brackets is {\em positive}. Hence, the
above equation is {\em not}\/ the simple harmonic equation. Indeed
its solution takes the form
\begin{equation}
\lambda = A\,{\rm e}^{\,k\,t'} + B\,{\rm e}^{-k\,t'}.
\end{equation}
Here, $A$ and $B$ are constants of integration, and
\begin{equation}
k= \left[\frac{(I_{y'y'}-I_{x'x'})\,(I_{z'z'}-I_{y'y'})}{I_{x'x'}\,I_{z'z'}}\right]^{1/2}\omega_{y'}.
\end{equation}
In this case, the amplitude of the perturbation grows exponentially in time.
Hence,  the body is {\em unstable}\/ to small perturbations
when rotating about the $y'$-axis.

In conclusion,  a rigid body with three distinct principal moments of inertia is stable to small perturbations when rotating about the
principal axes with the largest and smallest  moments, but
is unstable when rotating about the axis with the intermediate
 moment.

Finally, if two of the principal moments are the same then it can be shown
that the body is only stable to small perturbations when rotating
about the principal axis whose  moment is distinct from the other two.

\section{Exercises}
{\small
\renewcommand{\theenumi}{8.\arabic{enumi}}
\begin{enumerate}
\item Find the principal axes of rotation and the principal moments of inertia for a
thin uniform rectangular plate of mass $m$ and dimensions $2\,a$ by $a$ for rotation
about  axes passing through (a) the center of mass, and (b) a corner.
\item A rigid body having an axis of symmetry rotates freely about a fixed point under no
torques. If $\alpha$ is the angle between the symmetry axis and the instantaneous axis of
rotation, show that the angle between the axis of rotation and the invariable line
(the ${\bf L}$ vector) is
$$
\tan^{-1}\left[\frac{(I_\parallel-I_\perp)\,\tan\alpha}{I_\parallel-I_\perp\,\tan^2\alpha}\right]
$$
where $I_\parallel$ (the moment of inertia about the symmetry axis) is greater than $I_\perp$
(the moment of inertia about an axis normal to the symmetry axis).
\item Since the greatest value of $I_\parallel/I_\perp$ is 2 (symmetrical lamina) show
from the previous result that the angle between the angular velocity and angular momentum
vectors cannot exceed $\tan^{-1}(1/\sqrt{8})\simeq19.5^\circ$. Find the corresponding
value of $\alpha$. 
\item A thin uniform rod of length $l$ and mass $m$ is constrained to rotate
with constant angular velocity $\omega$ about an axis passing through the
center of the rod, and making an angle $\alpha$ with the rod.
Show that the angular momentum about the center of the rod is perpendicular to the rod, and is of magnitude $(m\,l^2\,\omega/12)\,\sin\alpha$. Show that
the torque is perpendicular to both the rod and the angular momentum vector,
and is of magnitude $(m\,l^2\omega^2/12)\,\sin\alpha\,\cos\alpha$. 

\item A thin uniform disk of radius $a$ and mass $m$ is constrained to rotate
with constant angular velocity $\omega$ about an axis passing through its center, and making an angle $\alpha$ with the normal to the disk.
Find the angular momentum about the center of the disk, as
well as the torque acting on the disk. 
\item Demonstrate that for an isolated rigid body which possesses an
axis of symmetry, and rotates about one of its principal axes, the motion is
only stable to small perturbations if the principal axis is that which corresponds to the
symmetry axis. 

\end{enumerate}

}