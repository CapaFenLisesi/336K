\chapter{Two-Body Dynamics}\label{ctwo}
\section{Introduction}
This chapter examines the motion of dynamical systems consisting
of two freely moving and mutually interacting  point objects. 

\section{Reduced Mass}\label{sredm}
Suppose that our first object is of mass $m_1$, and is located
at position vector ${\bf r}_1$. Likewise, our second object
is of mass $m_2$, and is located at position vector ${\bf r}_2$. 
Let the first object exert a force ${\bf f}_{21}$ on the
second. By Newton's third law, the second object exerts an
equal and opposite force, ${\bf f}_{12} = -{\bf f}_{21}$, on the 
first. Suppose that there are no other forces in the problem. The equations of motion of our two objects are thus
\begin{eqnarray}\label{e7.1}
m_1\,\frac{d^2{\bf r}_1}{dt^2} &=& -{\bf f},\\[0.5ex]
m_2\,\frac{d^2{\bf r}_2}{dt^2} &=&{\bf f},\label{e7.2}
\end{eqnarray}
where ${\bf f} = {\bf f}_{21}$. 

Now, the center of mass of our system is located at
\begin{equation}
{\bf r}_{cm} = \frac{m_1\,{\bf r}_1+ m_2\,{\bf r}_2}{m_1 + m_2}.
\end{equation}
Hence, we can write
\begin{eqnarray}
{\bf r}_1 &=& {\bf r}_{cm} - \frac{m_2}{m_1+m_2}\,{\bf r},\\[0.5ex]
{\bf r}_2 &=& {\bf r}_{cm} + \frac{m_1}{m_1+m_2}\,{\bf r},
\end{eqnarray}
where ${\bf r}= {\bf r}_2-{\bf r}_1$. 
Substituting the above two equations into Equations~(\ref{e7.1}) and (\ref{e7.2}),
and making use of the fact that the center of mass of an isolated system
{\em does not accelerate}\/ (see Section~\ref{new3}), we find that both equations yield
\begin{equation}\label{e7.6}
\mu\,\frac{d^2{\bf r}}{d t^2} = {\bf f},
\end{equation}
where 
\begin{equation}
\mu = \frac{m_1\,m_2}{m_1+ m_2}
\end{equation}
is called the {\em reduced mass}. Hence, we have effectively converted our
original two-body problem into an equivalent one-body problem. In the equivalent problem, the
force ${\bf f}$ is the {\em same}\/ as that acting on both objects  in the original problem (modulo a minus sign). However, the mass, $\mu$, is {\em different}, and
is less than either of $m_1$ or $m_2$ (which is why it is called the ``reduced'' mass). 

\section{Binary Star Systems}\label{sbin}
Approximately half of the stars in our galaxy are members of so-called {\em binary star
systems}. Such systems consist of two stars orbiting about their common
center of mass. The distance separating the  stars is always  much less than
the distance to the nearest neighbour star. Hence, a binary star system
can be treated as a two-body dynamical system to a very good approximation.

In a binary star system, the gravitational force which the first star exerts on
the second is
\begin{equation}
{\bf f} = - \frac{G\,m_1\,m_2}{r^3}\,{\bf r},
\end{equation}
where ${\bf r}= {\bf r}_2-{\bf r}_1$.
As we have seen, a two-body system can be reduced to an equivalent
one-body system whose equation of motion is of the form (\ref{e7.6}),
where $\mu= m_1\,m_2/(m_1+m_2)$. 
Hence, in this particular case, we can write
\begin{equation}
\frac{m_1\,m_2}{m_1+m_2}\, \frac{d^2{\bf r}}{dt^2} =-
\frac{G\,m_1\,m_2}{r^3} \,{\bf r},
\end{equation}
which gives
\begin{equation}\label{e7.10}
\frac{d^2{\bf r}}{dt^2} =-
\frac{G\,M}{r^3} \,{\bf r},
\end{equation}
where
\begin{equation}
M = m_1+ m_2.
\end{equation}

Equation~(\ref{e7.10}) is identical to Equation~(\ref{e6.3}), which we have already
solved. Hence, we can immediately write down the solution:
\begin{equation}
{\bf r} = (r\,\cos\theta,\,r\,\sin\theta,\, 0),
\end{equation}
where
\begin{equation}
 r = \frac{a\,(1-e^2)}{1-e\,\cos\theta},
\end{equation}
and
\begin{equation}
\frac{d\theta}{dt} = \frac{h}{r^2},
\end{equation}
with
\begin{equation}
a = \frac{h^2}{(1-e^2)\,G\,M}.
\end{equation}
Here, $h$ is a constant, and we have aligned our Cartesian axes so that the plane of the orbit
coincides with the $x$-$y$ plane.
According to the above solution, the second star executes a Keplerian
elliptical orbit, with major radius $a$ and eccentricity $e$,
relative to the first star, and {\em vice versa}. From Equation~(\ref{e6.49}), the period of revolution, $T$, is given by
\begin{equation}\label{e7.18}
T = \sqrt{\frac{4\pi^2\,a^3}{G\,M}}.
\end{equation}

In the {\em inertial}\/ frame of reference whose origin always coincides with the center of mass---the so-called {\em center of mass frame}---the position vectors of the two stars are
\begin{eqnarray}\label{e7.16}
{\bf r}_1 &=& -\frac{m_2}{m_1+m_2}\,{\bf r},\\[0.5ex]
{\bf r}_2 &=& \frac{m_1}{m_1+m_2}\,{\bf r},\label{e7.17}
\end{eqnarray}
where ${\bf r}$ is specified above. Figure~\ref{double} shows an example binary star orbit, in the center of mass frame, calculated with $m_1/m_2=0.5$
and $e=0.2$. Here, the triangles and squares denote the positions of the
first and second star, respectively (which are always diagrammatically opposite one another, as indicated by the arrows). It can be seen that both stars execute
elliptical orbits about their common center of mass.

\begin{figure}
\epsfysize=3in
\centerline{\epsffile{Chapter06/fig6.01.eps}}
\caption{\em An example binary star orbit.}\label{double}
\end{figure}

Binary star systems have been very useful to astronomers, since it is
possible to determine the masses of both stars in such a system
by careful observation.
 The {\em sum}\/ of the masses of the two stars, $M=m_1+m_2$, can be found
from Equation~(\ref{e7.18}) after a measurement of the major radius, $a$ (which
is the mean of the greatest and smallest distance apart of the two
stars during their orbit), and the orbital period, $T$. The {\em ratio}\/ of the
masses of the two stars, $m_1/m_2$,  can be determined from Equations~(\ref{e7.16}) and (\ref{e7.17}) by
observing the fixed ratio of the relative distances of the two stars from the common
center of mass about which they both appear to rotate. Obviously, given the sum
of the masses, and the ratio of the masses, the individual masses themselves can
then be calculated.

\section{Scattering in the Center of Mass Frame}\label{cofm}
Let us now consider scattering due to the collision of two particles. We shall
restrict our discussion to particles which interact via {\em conservative central
forces}. It turns out that scattering looks particularly simple when
viewed in the {\em  center of mass frame}. Let us, therefore, start our investigation
by considering two-particle scattering in the center of mass frame.

As before, the first particle is of mass $m_1$, and is located at position
vector ${\bf r}_1$, whereas the second particle is of mass $m_2$,
and is located at ${\bf r}_2$.
By definition,  there is {\em zero net linear momentum}\/ in the center of mass
frame at all times. Hence, if the first particle approaches the collision
point with momentum ${\bf p}$ then the second must approach
with momentum ${\bf -p}$. Likewise, after the collision, if the
first particle recedes from the collision point with momentum ${\bf p}'$ 
then the second  must recede with momentum $-{\bf p}'$---see Figure~\ref{coll}. Furthermore, since the interaction force is
{\em conservative}, the total kinetic energy before and after the collision must
be the {\em same}. It follows that the {\em magnitude}\/ of the final momentum
vector, ${\bf p}'$, is equal to the magnitude of the initial momentum vector, ${\bf p}$. Because of this, the collision event is completely specified
once the angle $\theta$ through which the
first particle is scattered  is given. Of course, in the center of mass frame, the second particle
is scattered through the same angle---see Figure~\ref{coll}.

\begin{figure}
\epsfysize=2.5in
\centerline{\epsffile{Chapter06/fig6.02.eps}}
\caption{\em A collision viewed in the center of mass frame.}\label{coll}
\end{figure}

Suppose that the two particles interact via the potential $U(r)$, where $r$
is the distance separating the particles. As we have seen, the two-body
problem sketched in Figure~\ref{coll} can be converted into the
equivalent one-body problem sketched in Figure~\ref{coll1}. In this
equivalent problem, a particle of mass $\mu=m_1\,m_2/(m_1+m_2)$ is scattered
in the fixed potential $U(r)$, where $r$ is now the distance from the origin.
The vector position ${\bf r}$ of the particle in the equivalent problem corresponds to the relative position vector ${\bf r}_2-{\bf r}_1$ in the original problem.
It follows that the angle $\theta$ through which the particle is scattered in the equivalent
problem is the same as the scattering angle $\theta$ in the original problem.

\begin{figure}
\epsfysize=2.5in
\centerline{\epsffile{Chapter06/fig6.03.eps}}
\caption{\em The one-body equivalent to the previous figure.}\label{coll1}
\end{figure}

The scattering angle, $\theta$, is largely
determined by the so-called {\em impact parameter}, $b$, which is the
distance of closest approach of the two particles in the {\em absence}\/ of an
interaction potential. In the equivalent problem, $b$ is the distance of
closest approach to the origin in the absence of an interaction
potential---see Figure~\ref{coll1}. If $b=0$ then we have a head-on collision. In this case, we expect
the two particles to reverse direction after colliding: {\em i.e.}, we expect 
$\theta=\pi$. Likewise, if $b$ is large then we expect the two particles
to miss one another entirely, in which case $\theta=0$. It follows that the
scattering angle, $\theta$, is a {\em decreasing}\/ function of the impact parameter,
$b$. 

Suppose that the  polar coordinates of the particle in the equivalent problem 
are $(r,\,\vartheta)$. Let the particle approach the origin from the direction
$\vartheta=0$, and attain its closest distance to the origin when
$\vartheta={\mit\Theta}$. From symmetry, the angle $\alpha$ in Figure~\ref{coll1} is equal to the angle $\beta$. However, from simple geometry,
$\alpha={\mit\Theta}$. Hence,
\begin{equation}\label{e7.19}
\theta = \pi - 2\,{\mit\Theta}.
\end{equation}
Now, by analogy with Equation~(\ref{e6.55}), the conserved total energy $E$ in the equivalent
problem, which can easily be shown to be the same as the total energy in
the original problem, is given by
\begin{equation}\label{e7.20}
E = \frac{\mu\,h^2}{2}\left[\left(\frac{du}{d\vartheta}\right)^2 + u^2\right] + U(u),
\end{equation}
where $u=r^{-1}$, and $h$ is the angular momentum per unit mass in the
equivalent problem. It is easily seen that
\begin{equation}
h = b\,v_\infty = b\left(\frac{2\,E}{\mu}\right)^{1/2},
\end{equation}
where $v_\infty$ is the approach velocity in the equivalent problem
at large $r$. 
It follows that
\begin{equation}\label{e7.22}
E = E\,b^2\left[\left(\frac{du}{d\vartheta}\right)^2 + u^2\right] + U(u).
\end{equation}
The above equation can be rearranged to
give
\begin{equation}\label{e7.23}
\frac{d\vartheta}{d u } = \frac{b}{\sqrt{1 - b^2\,u^2-U(u)/E}}.
\end{equation}
Integration yields
\begin{equation}\label{e7.24}
{\mit\Theta} = \int_0^{u_{max}}\frac{b\,du}{\sqrt{1-b^2\,u^2- U(u)/E}}.
\end{equation}
Here, $u_{max}=1/r_{min}$, where $r_{min}$ is the distance of closest approach. Since, by symmetry,  $(du/d\vartheta)_{u_{max}}=0$, it follows from Equation~(\ref{e7.23}) that
\begin{equation}\label{e7.25}
1 - b^2\,u_{max}^2 - U(u_{max})/E = 0.
\end{equation}
Equations (\ref{e7.19}) and (\ref{e7.24}) enable us to calculate the
function $b(\theta)$ for a given interaction potential, $U(r)$, and a
given total energy, $E$, of the two particles  in the center of mass frame. The function $b(\theta)$
tells us which impact parameter corresponds to which scattering angle,
and {\em vice versa}. 

Instead of two particles, suppose that we now have  two counter-propag\-ating {\em beams}\/ of  identical particles (with the same properties as the
two particles described above) which scatter one another via binary collisions. What
is the angular distribution of the scattered particles?
Consider pairs of particles whose impact parameters lie in the range $b$ to $b+db$. These particles are scattered in such a manner that their scattering
angles lie in the range $\theta$ to $\theta+d\theta$, where $\theta$
is determined from inverting the function $b(\theta)$, and
\begin{equation}
d\theta = \frac{db}{|db(\theta)/d\theta|}.
\end{equation}
Incidentally, we must take the modulus of $db(\theta)/d\theta$ because $b(\theta)$ is
a decreasing function of $\theta$. Assuming, as seems reasonable, that the scattering
is azimuthally symmetric, the range of {\em solid angle}\/ into which
the particles are scattered is
\begin{equation}
d{\mit\Omega} = 2\pi\,\sin\theta\,d\theta = \frac{2\pi\,\sin\theta\,db}{|db/d\theta|}
\end{equation}
Finally, the {\em cross-sectional area}\/ of the {\em annulus}\/ through which incoming
particles must pass if they are to have impact parameters in the
range $b$ to $b+db$ is
\begin{equation}
d\sigma = 2\pi\,b\,db.
\end{equation}
The previous two equations allow us to define the {\em differential scattering
cross-section}:
\begin{equation}\label{e7.29}
\frac{d\sigma}{d{\mit\Omega}} = \frac{b}{\sin\theta}\left|\frac{db}{d\theta}\right|
\end{equation}
The differential scattering cross-section has units of area per steradian,
and specifies the effective target area for scattering into a given
range of solid angle. For two {\em uniform}\/ beams scattering
off one another, the differential scattering cross-section thus 
effectively specifies the {\em probability}\/ of scattering into a given range of
solid angle. The {\em total scattering cross-section}\/ is the integral
of the differential cross-section over all solid angles,
\begin{equation}
\sigma = \int \frac{d\sigma}{d{\mit\Omega}}\,d{\mit\Omega},\label{e7.29a}
\end{equation}
and measures the effective target area for scattering in {\em any}\/ direction. 
Thus, if the flux of particles per unit area per unit time, otherwise known as the {\em intensity}, of the two beams is $I$, then the number of particles of a given type scattered
per unit time is simply $I\,\sigma$. 

Let us now calculate the scattering cross-section for the following
very simple interaction potential:
\begin{equation}
U(r) = 
\left\{\begin{array}{lll}
0&\mbox{\hspace{1cm}} & r>a\\[0.5ex]
\infty && r\leq a
\end{array}\right..\label{e7.31}
\end{equation}
This is the interaction potential of {\em impenetrable spheres}\/ which only
exert a force on one another when they are in physical contact 
({\em e.g.}, billiard balls). If the particles in the first beam have
radius $R_1$, and the particles in the second beam have radius $R_2$,
then $a=R_1+R_2$. In other words, the centers of two particles, one from either
beam, can never be less than a distance $a$ apart, where
$a$ is the sum of their radii (since the particles
are impenetrable spheres). 

Equations~(\ref{e7.19}), (\ref{e7.24}), and (\ref{e7.31})
yield
\begin{equation}
\theta = \pi - 2\int_0^{1/a}\frac{b\,du}{\sqrt{1-b^2\,u^2}} = \pi - 2\,\sin^{-1}(b/a).
\end{equation}
The above formula can be rearranged to give
\begin{equation}
b(\theta) = a\,\cos(\theta/2).
\end{equation}
Note that
\begin{equation}\label{e7.34}
b\left|\frac{db}{d\theta}\right| = \frac{1}{2}\left|\frac{d b^2}{d\theta}\right| = \frac{a^2}{2}\,\sin(\theta/2)\,\cos(\theta/2) = \frac{a^2}{4}\,\sin\theta.
\end{equation}
Hence, Equations~(\ref{e7.29}) and (\ref{e7.34}) yield
\begin{equation}\label{e7.36}
\frac{d\sigma}{d{\mit\Omega}} = \frac{a^2}{4}.
\end{equation}
We thus conclude that when two beams of impenetrable spheres
collide, in the center of mass frame, the particles in the two beams have 
an equal probability of being scattered in any direction. The total scattering
cross-section is
\begin{equation}\label{e7.37}
\sigma = \int \frac{d\sigma}{d{\mit\Omega}}\,d{\mit\Omega} = \pi\,a^2.
\end{equation}
Obviously, this result makes a lot of sense---the total scattering cross-section
for two  impenetrable spheres is simply the area of a circle
whose radius is the sum of the radii of the two spheres.

Let us now consider scattering by an inverse-square interaction force
whose potential takes the form
\begin{equation}
U(r) = \frac{k}{r}.
\end{equation}
It follows from Equations~(\ref{e7.24}) and (\ref{e7.25}) that
\begin{equation}
{\mit\Theta} = \int_0^{u_{max}}\frac{b\,du}{\sqrt{1-b^2\,u^2
- k\,u/E}} = \int_0^{x_{max}}\frac{dx}{\sqrt{1-x^2- \alpha\,x}},
\end{equation}
where $\alpha = k/(E\,b)$, and
\begin{equation}
1 - x_{max}^{\,2} - \alpha\,x_{max}= 0.
\end{equation}
Integration yields
\begin{equation}
{\mit\Theta} = \frac{\pi}{2} - \sin^{-1}\left(\frac{\alpha}{\sqrt{4+\alpha^2}}\right).
\end{equation}
Hence, from Equation~(\ref{e7.19}), we obtain
\begin{equation}
\theta = 2\,\sin^{-1}\left(\frac{\alpha}{\sqrt{4+\alpha^2}}\right).
\end{equation}
The above equation can be rearranged to give
\begin{equation}
b^2 = \frac{k^2}{4\,E^{\,2}}\,\cot^2(\theta/2).
\end{equation}
Thus,
\begin{equation}
2\, b\left|\frac{db}{d\theta}\right| = \frac{k^2}{8\,E^{\,2}}\frac{\sin\theta}{\sin^4(\theta/2)}.
\end{equation}
Finally, using Equation~(\ref{e7.29}), we get
\begin{equation}\label{e7.45}
\frac{d\sigma}{d{\mit\Omega}} = \frac{k^2}{16\,E^{\,2}}\frac{1}{\sin^4(\theta/2)}.
\end{equation}
There are a number of things to note about the above formula. First, the
scattering cross-section is proportional to $k^2$. This means that
{\em repulsive}\/ ($k>0$) and {\em attractive}\/ ($k<0$) inverse-square interaction forces
of the same strength give rise to {\em identical}\/ angular distributions of scattered
particles. Second, the scattering cross-section is proportional to $E^{-2}$.
This means that inverse-square interaction forces are much
more effective at scattering low energy, rather than high energy, particles.
Finally, the differential scattering cross-section is proportional to
$\sin^{-4}(\theta/2)$. This means that, with an inverse-square
interaction force, the overwhelming majority of
``collisions''  consist of  {\em small angle}\/ scattering events ({\em i.e.}, $\theta\ll 1$).

Let us now consider a specific case. Suppose that we have particles
of electric charge $q$ scattering off  particles of the same charge. The
interaction potential due to the Coulomb force between the particles
is simply
\begin{equation}
U(r) = \frac{q^2}{4\pi\,\epsilon_0\,r}.
\end{equation}
Thus, it follows from Equation~(\ref{e7.45}) [with $k=q^2/(4\pi\,\epsilon_0)$]
that the differential scattering cross-section takes the form
\begin{equation}\label{ruthf}
\frac{d\sigma}{d{\mit\Omega}} = \frac{q^4}{16\,(4\,\pi\,\epsilon_0)^2\,E^{\,2}}\frac{1}{\sin^4(\theta/2)}.
\end{equation}
This very famous formula is known as the {\em Rutherford scattering cross-section}, since it was first derived by Earnst Rutherford for use in his
celebrated $\alpha$-particle scattering experiment.

Note, finally, that if we try to integrate the Rutherford formula to obtain
the {\em total}\/ scattering cross-section then we find that the integral is {\em divergent},
due to the very strong increase in $d\sigma/d{\mit\Omega}$ as $\theta\rightarrow 0$. This implies that the Coulomb potential (or any
other inverse-square-law potential)
 has an effectively {\em infinite}\/ range. In practice, however, an electric
charge  is generally surrounded by charges of the opposite sign which
{\em shield}\/ the Coulomb potential of the charge beyond a certain distance.
This shielding effect allows the charge to have a {\em finite}\/ total
scattering cross-section (for the scattering of other electric charges). However, the total scattering cross-section of the charge
depends (albeit, logarithmically) on the shielding distance, and, hence, on the
nature and distribution of the charges surrounding it.

\section{Scattering in the Laboratory Frame}
We have seen that two-particle scattering looks fairly simple when viewed in the
center of mass frame. Unfortunately, we are not usually in a position
to do this. In the laboratory, the most
common scattering scenario is one in which the second particle is
initially at rest. Let us now investigate this situation.

Suppose that, in the center of mass frame, the first particle has velocity
${\bf v}_1$ before the collision, and velocity ${\bf v}_1'$ after the
collision. Likewise, the second particle has velocity ${\bf v}_2$ before the
collision, and ${\bf v}_2'$ after the collision. We know that
\begin{equation}
m_1\,{\bf v}_1 + m_2\,{\bf v}_2 = m_1\,{\bf v}_1'+m_2\,{\bf v}_2' = {\bf 0}
\end{equation}
in the center of mass frame. Moreover, since the collision is assumed to be elastic ({\em i.e.}, energy conserving),
\begin{eqnarray}
v_1' &= &v_1,\\[0.5ex]
v_2' &=& v_2.
\end{eqnarray}

Let us transform
to a new inertial frame of reference---which we shall call the {\em laboratory frame}---which is moving with the uniform velocity $-{\bf v}_2$ with respect to
the center of mass frame. In the new reference frame, the first
particle has initial velocity ${\bf V}_1= {\bf v}_1-{\bf v}_2$, and
final velocity ${\bf V}_1' = {\bf v}_1'-{\bf v}_2$. Furthermore, the second particle
is initially at {\em rest}, and has the final velocity ${\bf V}_2' = {\bf v}_2'-{\bf v}_2$. The relationship between scattering in the center of mass frame
and scattering in the laboratory frame is illustrated in Figure~\ref{lab}.

\begin{figure}
\epsfysize=1.8in
\centerline{\epsffile{Chapter06/fig6.04.eps}}
\caption{\em Scattering in the center of mass and laboratory frames.}\label{lab}
\end{figure}

In the center of mass frame, both particles are scattered through the same angle $\theta$. However, in the laboratory frame, the first and second particles
are scattered by the (generally different) angles $\psi$ and $\zeta$,
respectively.

Defining $x$- and $y$-axes, as indicated in Figure~\ref{lab}, it is easily
seen that the Cartesian components of the various velocity vectors in the
two frames of reference are:
\begin{eqnarray}\label{e7.51}
{\bf v}_1 &=& v_1\,(1,\, 0),\\[0.5ex]
{\bf v}_2 &=&(m_1/m_2)\,v_1\,(-1,\,0),\\[0.5ex]
{\bf v}_1' &=& v_1\,(\cos\theta,\,\sin\theta),\\[0.5ex]
{\bf v}_2' &=&(m_1/m_2)\,v_1(-\cos\theta,\,-\sin\theta),\\[0.5ex]
{\bf V}_1 &=& (1+m_1/m_2)\,v_1\,(1,\,0),\\[0.5ex]
{\bf V}_1'&=& v_1\,(\cos\theta+m_1/m_2,\,\sin\theta),\\[0.5ex]
{\bf V}_2' &=& (m_1/m_2)\,v_1\,(1-\cos\theta,\,-\sin\theta).\label{e7.57}
\end{eqnarray}

In the center of mass frame, let $E$ be the total energy,  let $E_1=(1/2)\,m_1\,v_1^{\,2}$ and $E_2=(1/2)\,m_2\,v_2^{\,2}$ be the kinetic energies of the first and second particles, respectively, before the collision, and let
$E_1'=(1/2)\,m_1\,v_1'^{\,2}$ and $E_2'=(1/2)\,m_2\,v_2'^{\,2}$ be the kinetic energies of the first and second particles, respectively, after the collision. Of course, $E=E_1+E_2=E_1'+E_2'$. 
In the laboratory frame, let ${\cal E}$ be the total energy. This is, of course,
equal to the kinetic energy of the first particle before the collision. Likewise,
let ${\cal E}_1'=(1/2)\,m_1\,V_1'^{\,2}$ and ${\cal E}_2'=(1/2)\,m_2\,V_2'^{\,2}$ be the kinetic energies of the first and second particles, respectively, after the collision. Of course, ${\cal E} = {\cal E}_1' + {\cal E}_2'$. 

The following results can easily be obtained from the above definitions and
Equations~(\ref{e7.51})--(\ref{e7.57}). First,
\begin{equation}\label{e7.58}
{\cal E} = \left(\frac{m_1+m_2}{m_2}\right)E.
\end{equation}
Hence, the total energy in the laboratory frame is always {\em greater}\/ than
that in the center of mass frame. In fact, it can be demonstrated that the
total energy in the center of mass frame is less than the total energy in {\em any}\/
other inertial frame. Second,
\begin{eqnarray}
E_1 &=& E_1'= \left(\frac{m_2}{m_1+m_2}\right) E,\label{e7.59}\\[0.5ex]
E_2 &=& E_2'=\left(\frac{m_1}{m_1+m_2}\right) E.\label{e7.60}
\end{eqnarray}
These equations specify how the total energy in the center of mass
frame is distributed
between the two particles. Note that this distribution is {\em unchanged}\/ by the
collision.
Finally,
\begin{eqnarray}
{\cal E}_1' &=& \left[\frac{m_1^{\,2}+2\,m_1\,m_2\,\cos\theta+m_2^{\,2}}{(m_1+m_2)^2}\right]{\cal E},\label{e7.61}\\[0.5ex]
{\cal E}_2'&=& \left[\frac{2\,m_1\,m_2\,(1-\cos\theta)}{(m_1+m_2)^2}\right] {\cal E}.\label{e7.62}
\end{eqnarray}
These equations specify how the total energy in the laboratory frame is distributed between
the two particles after the collision. Note that
the energy distribution in the laboratory frame is {\em different}\/ before and after the collision.

Equations~(\ref{e7.51})--(\ref{e7.57}), and some simple trigonometry, yield
\begin{equation}\label{e7.63}
\tan\psi = \frac{\sin\theta}{\cos\theta+m_1/m_2},
\end{equation}
and
\begin{equation}
\tan\zeta = \frac{\sin\theta}{1-\cos\theta} = \tan\left(\frac{\pi}{2}-\frac{\theta}{2}\right).
\end{equation}
The last equation implies that
\begin{equation}\label{e7.65}
\zeta = \frac{\pi}{2}- \frac{\theta}{2}.
\end{equation}
Differentiating Equation~(\ref{e7.63}) with respect to $\theta$, we obtain
\begin{equation}
\frac{d\tan\psi}{d\theta} = \frac{1+(m_1/m_2)\,\cos\theta}{(\cos\theta+m_1/m_2)^2}.
\end{equation}
Thus, $\tan\psi$ attains an extreme value, which can be shown to correspond to a {\em maximum}\/
possible value of $\psi$, when the numerator of the above expression is zero:
{\em i.e.}, when
\begin{equation}
\cos\theta = -\frac{m_2}{m_1}.
\end{equation}
Note that it is only possible to solve the above equation when $m_1>m_2$.
If this is the case then Equation~(\ref{e7.63}) yields
\begin{equation}
\tan\psi_{max} = \frac{m_2/m_1}{\sqrt{1-(m_2/m_1)^2}},
\end{equation}
which reduces to
\begin{equation}
\psi_{max} = \sin^{-1}\left(\frac{m_2}{m_1}\right).
\end{equation}
Hence, we conclude that when $m_1>m_2$ there is a {\em maximum}\/ possible
value of the scattering angle, $\psi$, in the laboratory frame. This maximum
value is always less than $\pi/2$, which implies that there is
{\em no backward scattering}\/ ({\em i.e.}, $\psi>\pi/2$) at all when $m_1>m_2$. For the special case when
$m_1=m_2$, the maximum scattering angle is $\pi/2$. However, for
$m_1<m_2$ there is no maximum value, and the scattering angle in the laboratory frame can thus
range all the way to $\pi$. 

Equations~(\ref{e7.58})--(\ref{e7.65}) enable us to relate the
particle energies and scattering angles in the laboratory frame to those
in the center of mass frame. In general, these relationships are fairly complicated. 
However, there are two special cases in which the relationships become
much simpler. 

The first special case is when $m_2\gg m_1$. In this limit, it is easily seen
from Equations~(\ref{e7.58})--(\ref{e7.65}) that the second mass is {\em stationary}\/
both before and after the collision, and that the center of mass frame
{\em coincides}\/ with the laboratory frame (since the energies and
scattering angles in the two frames are the same). Hence, the simple analysis
outlined in Section~\ref{cofm} is applicable in  this case.

The second special case is when $m_1=m_2$. In this case, Equation~(\ref{e7.63})
yields
\begin{equation}
\tan\psi = \frac{\sin\theta}{\cos\theta+1} =\tan(\theta/2).
\end{equation}
Hence,
\begin{equation}\label{e7.71}
\psi = \frac{\theta}{2}.
\end{equation}
In other words, the scattering angle of the first particle in the laboratory
frame is {\em half}\/ of the scattering angle in the center of mass frame.
The above equation can be combined with Equation~(\ref{e7.65}) to
give
\begin{equation}
\psi + \zeta =\frac{\pi}{2}.
\end{equation}
Thus, in the laboratory frame, the two particles move off at {\em right-angles}\/ to one another after the collision. Equation~(\ref{e7.58})
yields
\begin{equation}\label{e7.73}
{\cal E} = 2\,E.
\end{equation}
In other words, the total energy in the laboratory frame is {\em twice}\/ that
in the center of mass frame. According to Equations~(\ref{e7.59}) and (\ref{e7.60}),
\begin{equation}
E_1 = E_1' = E_2=E_2' = \frac{E}{2}.
\end{equation}
Hence,  the total energy in the center of mass frame is divided {\em equally}\/ between the two particles. Finally, Equations~(\ref{e7.61}) and (\ref{e7.62})
give
\begin{eqnarray}
{\cal E}_1' &=& \left(\frac{1+\cos\theta}{2}\right){\cal E} = \cos^2(\theta/2)\,\,{\cal E} = \cos^2\psi\,\,{\cal E},\\[0.5ex]
{\cal E}_2'&=& \left(\frac{1-\cos\theta}{2}\right){\cal E} = \sin^2(\theta/2)\,\,{\cal E} = \sin^2\psi\,\,{\cal E}.
\end{eqnarray}
Thus, in the laboratory frame, the unequal energy distribution between the two particles
after the collision  is simply related to
the scattering angle $\psi$.

What is the angular distribution of scattered particles when a beam
of particles of the first type  scatter off stationary particles of the second type?
Well, we can define a differential scattering cross-section, $d\sigma(\psi)/d{\mit\Omega}'$, in the
laboratory frame, where $d{\mit\Omega}'=2\pi\,\sin\psi\,d\psi$ is an element of solid angle in
this frame. Thus, $(d\sigma(\psi)/d{\mit\Omega}')\,d{\mit\Omega}'$
is the effective cross-sectional area in the laboratory  frame
for scattering into the range of scattering angles $\psi$ to $\psi+d\psi$.
Likewise, $(d\sigma(\theta)/d{\mit\Omega})\,d{\mit\Omega}$ is
the effective cross-sectional area in the center of mass frame for
scattering into the range of scattering angles $\theta$ to $\theta+d\theta$.
Note that $d{\mit\Omega} =2\pi\,\sin\theta\,d\theta$.
However, a cross-sectional area is not changed when we transform between
different inertial frames. Hence, we can write
\begin{equation}
\frac{d\sigma(\psi)}{d{\mit\Omega}'} \,d{\mit\Omega}' =
\frac{d\sigma(\theta)}{d{\mit\Omega}}\,d{\mit\Omega},
\end{equation}
provided that $\psi$ and $\theta$ are related via Equation~(\ref{e7.63}).
This equation can be rearranged to give
\begin{equation}
\frac{d\sigma(\psi)}{d{\mit\Omega}'} = \frac{d{\mit\Omega}}{d{\mit\Omega}'}\,\frac{d\sigma(\theta)}{d{\mit\Omega}},
\end{equation}
or
\begin{equation}\label{e7.79}
\frac{d\sigma(\psi)}{d{\mit\Omega}'} = \frac{\sin\theta}{\sin\psi}\,\frac{d\theta}{d\psi}\,\frac{d\sigma(\theta)}{d{\mit\Omega}}.
\end{equation}
The above equation allows us to relate the differential scattering cross-section
in the laboratory frame to that in the center of mass frame. In general, this
relationship is extremely complicated. However, for the special
case where the masses of the two types of particles are {\em equal}, we
have seen that $\psi=\theta/2$ [see Equation~(\ref{e7.71})]. Hence, it follows from Equation~(\ref{e7.79})
that
\begin{equation}\label{e7.80}
\frac{d\sigma(\psi)}{d{\mit\Omega}'} = 4\,\cos\psi\,\,\frac{d\sigma(\theta=2\,\psi)}{d{\mit\Omega}}.
\end{equation}

Let us now consider some specific examples. We saw earlier that, in the
center of mass frame, the
differential scattering cross-section for {\em impenetrable spheres}\/  is [see Equation~(\ref{e7.36})]
\begin{equation}\label{e7.81}
\frac{d\sigma(\theta)}{d{\mit\Omega}} = \frac{a^2}{4},
\end{equation}
where $a$ is the sum of the radii. According to Equation~(\ref{e7.80}),
the differential scattering cross-section (for equal mass spheres) in the laboratory frame is 
\begin{equation}\label{e7.82}
\frac{d\sigma(\psi)}{d{\mit\Omega}'} = a^2\,\cos\psi.
\end{equation}
Note that this cross-section is {\em negative}\/ for $\psi>\pi/2$. This
just tells us that there is {\em no scattering}\/ with scattering angles
greater than $\pi/2$ ({\em i.e.}, there is no backward scattering).
Comparing Equations~(\ref{e7.81}) and (\ref{e7.82}), we can see that
the scattering is {\em isotropic}\/ in the center of mass frame, but appears
concentrated in the forward direction in the laboratory frame.
We can integrate Equation~(\ref{e7.82}) over all solid angles to obtain the
total scattering cross-section in the laboratory frame. Note that we
only integrate over angular regions where the differential scattering
cross-section is {\em positive}. Doing this, we get
\begin{equation}
\sigma = \pi\,a^2,
\end{equation}
which is the same as the total scattering cross-section in the center of mass
frame [see Equation~(\ref{e7.37})]. This is a general result. The total scattering cross-section is
{\em frame independent}, since a cross-sectional area is not modified by
switching between different frames of reference.

As we have seen, the Rutherford scattering cross-section  takes
the form [see Equation~(\ref{ruthf})]
\begin{equation}
\frac{d\sigma}{d{\mit\Omega}} = \frac{q^4}{16\,(4\,\pi\,\epsilon_0)^2\,E^{\,2}}\frac{1}{\sin^4(\theta/2)}
\end{equation}
in the center of mass frame.
It follows, from Equation~(\ref{e7.80}), that the Rutherford scattering
cross-section (for equal mass particles) in the laboratory frame is written
\begin{equation}
\frac{d\sigma}{d{\mit\Omega}'} = \frac{q^4}{(4\,\pi\,\epsilon_0)^2\,{\cal E}^{\,2}}\frac{\cos\psi}{\sin^4\psi}.
\end{equation}
Here, we have made use of the fact that ${\cal E} = 2\,E$ for equal
mass particles [see Equation~(\ref{e7.73})]. Note, again, that this cross-section is negative
for $\psi>\pi/2$, indicating the absence of backward scattering.

\section{Exercises}
{\small
\renewcommand{\theenumi}{6.\arabic{enumi}}
\begin{enumerate}
\item A particle subject to a repulsive force varying as $1/r^3$ is projected from infinity with a velocity
that would carry it to a distance $a$ from the center of force, if it were directed toward the latter. Actually, it
is projected along a line whose closest distance from the center of force would be $b$ if there were no repulsion. Prove that the
particle's least distance from
the center is $\sqrt{a^2+b^2}$, and that the angle between the two asymptotes of its path
is $\pi\,b/\sqrt{a^2+b^2}$. 
\item A particle subject to a repulsive force varying as $1/r^5$ is projected from infinity with a velocity $V$ that
would carry it to a distance $a$ from the center of force, if it were directed toward the latter. Actually, it
is projected along a line whose closest distance from the center of force would be $b$ if there were no repulsion. 
Show that the least velocity of the particle is
$$
V\,\frac{b^2}{a^2}\left[\left(\frac{a^4}{b^4}+\frac{1}{4}\right)^{1/2}-\frac{1}{2}\right]^{1/2}.
$$
\item  Using the notation of Section~\ref{sredm}, show that
the angular momentum of a  two-body system takes the
form
$$
{\bf L} = M\,{\bf r}_{cm}\times \dot{\bf r}_{cm} + \mu\,{\bf r}\times \dot{\bf r},
$$
where $M=m_1+m_2$.
\item Consider the case of Rutherford scattering in the event that $m_1\gg m_2$. Demonstrate that the differential scattering cross-section in the
laboratory frame is approximately
$$
\frac{d\sigma}{d\Omega'}\simeq \frac{q_1^{\,2}\,q_2^{\,2}}{4\,(4\pi\epsilon_0)^2\,{\cal E}^2}\frac{\psi_{\rm max}^{-4}}{[1-\sqrt{1-(\psi/\psi_{\rm max})^2}]^2\,\left[1-(\psi/\psi_{\rm max})^2\right]^{1/2}},
$$
where $\psi_{\rm max}=m_2/m_1$.
 \item Show that the energy distribution of particles recoiling from an elastic
collision is always directly proportional to the differential scattering cross-section
in the center of mass frame.
\item It is found experimentally that in the elastic scattering of neutrons
by protons ($m_n\simeq m_p$) at relatively low energies the energy distribution
of the recoiling protons in the laboratory frame is constant up to
a maximum energy, which is the energy of the incident neutrons. What is the
angular distribution of the scattering in the center of mass frame?
\item The most energetic $\alpha$-particles available to Earnst Rutherford and his colleagues for the
famous Rutherford scattering experiment were $7.7$\,MeV. For the scattering of 7.7\,MeV $\alpha$-particles from Uranium 238 nuclei (initially at rest) at a scattering angle in the laboratory frame of
$90^\circ$, find the following (in the laboratory frame, unless otherwise specified):
\begin{enumerate}
\item The recoil scattering angle of the Uranium nucleus. 
\item The scattering angles of the $\alpha$-particle and Uranium nucleus in the center of mass
frame.
\item The kinetic energies of the scattered $\alpha$-particle and Uranium nucleus (in MeV).
\item The impact parameter, $b$.
\item The distance of closest approach.
\item The differential scattering cross-section at $90^\circ$. 
\end{enumerate}
\item Consider scattering by the repulsive potential $U=k/r^2$ (where $k>0$)
viewed in the center of mass frame. Demonstrate that the differential
scattering cross-section is
$$
\frac{d\sigma}{d\Omega } = \frac{k}{E\,\pi}\, \frac{1}{\sin\theta}\,
\frac{(1-\theta/\pi)}{(\theta/\pi)^2\,(2-\theta/\pi)^2}.
$$
\end{enumerate}
}