\chapter{Planetary Motion}\label{skepler}
\section{Introduction}
Newtonian dynamics was initially developed in order  to account for the motion of the Planets around the Sun. Let us now investigate
this problem.

\section{Kepler's Laws}
As is well-known, Johannes Kepler was the first astronomer to correctly describe planetary motion in the Solar System
(in works published between 1609 and 1619). The motion of the Planets is  summed up
in three simple laws:
\begin{enumerate}
\item The planetary orbits are all ellipses which are confocal with the Sun
({\em i.e.}, the Sun lies at one of the focii of each ellipse).
\item The radius vectors connecting each planet to the Sun sweep out  equal areas in  equal time intervals.
\item The squares of the orbital periods of the planets are proportional to
the cubes of their orbital  major radii.
\end{enumerate} 
Let us now see if we can derive Kepler's laws from Newton's laws of
motion.

\section{Newtonian Gravity}\label{snewt}
The force which maintains the Planets in orbit around the Sun is
called {\em gravity}, and was first correctly described by Isaac
Newton (in 1687). According to Newton, any two point mass objects (or spherically
symmetric objects of finite extent)
exert a force of attraction on one another.  This force points along the
line of centers joining the objects, is directly proportional to the product
of the objects' masses, and inversely proportional to the square of the distance between them. Suppose that the first object is the Sun, which is of mass $M$,
and is located at the origin of our coordinate system. Let the
second object be some planet,  of mass $m$, which is located
at position vector ${\bf r}$. The gravitational force exerted on the 
planet by the Sun is thus written
\begin{equation}\label{e6.1}
{\bf f} = - \frac{G\,M\,m}{r^3}\,{\bf r}.
\end{equation}
The constant of proportionality, $G$, is called the {\em gravitational constant},
and takes the value
\begin{equation}
G = 6.67300\times 10^{-11}\,{\rm m^3\,kg^{-1}\,s^{-2}}.
\end{equation}
An equal and opposite force to (\ref{e6.1}) acts on the Sun. However, we shall assume that
the Sun is so much more massive than the planet in question that this force does not
cause the Sun's position to shift  appreciably. Hence, the Sun will always remain
at the origin of our coordinate system. Likewise, we shall neglect the
 gravitational forces exerted on our planet by the other planets in the Solar System, compared to the much larger gravitational
force exerted by the Sun.

Incidentally, there is something rather curious about Equation~(\ref{e6.1}). According to
this law, the gravitational force acting on an object is directly proportional
to its inertial mass. But why should inertia be related to the force of
gravity? After all, inertia measures the reluctance of a given body to
deviate from its preferred state of uniform motion in a straight-line,
in response to some external force. What has this got to do with gravitational
attraction?
This question perplexed physicists for many years, and was only
answered when Albert Einstein published his general theory of relativity
in 1916. According to Einstein, inertial mass acts as a sort
of gravitational charge since it impossible to distinguish 
an acceleration produced by a gravitational field from an apparent acceleration
generated by observing in a non-inertial reference frame. The assumption that these
two types of acceleration are indistinguishable leads directly to all
of the strange predictions of general relativity: {\em e.g.}, 
clocks in different gravitational potentials run at different rates, 
mass bends space, {\em etc.}

According to Equation~(\ref{e6.1}), and Newton's second law, the equation
of motion of our planet takes the form
\begin{equation}\label{e6.3}
\frac{d^2{\bf r}}{dt^2} = -\frac{G\,M}{r^3}\,{\bf r}.
\end{equation}
Note that the planetary mass, $m$, has canceled out on both sides of
the above equation.

\section{Conservation Laws}
Now gravity is a {\em conservative force}. Hence, the gravitational force (\ref{e6.1}) can be written (see Section~\ref{new2})
\begin{equation}
{\bf f} = - \nabla U,
\end{equation}
where the potential energy, $U({\bf r})$,  of our planet in the Sun's gravitational field takes the form
\begin{equation}\label{e6.5}
U({\bf r}) = - \frac{G\,M\,m}{r}.
\end{equation}
It follows that the {\em total energy}\/ of our planet is a conserved quantity---see Section~\ref{new2}. In other words,
\begin{equation}\label{e6.6}
{\cal E} = \frac{v^2}{2} - \frac{G\,M}{r}
\end{equation}
is constant in time. Here, ${\cal E}$ is actually the planet's total energy {\em per unit
mass}, and ${\bf v} = d{\bf r}/dt$.

Gravity is also a {\em central force}. Hence, the {\em angular momentum}\/
of our planet is a conserved quantity---see Section~\ref{new3}. In other
words, 
\begin{equation}\label{e6.7}
{\bf h} = {\bf r}\times {\bf v},
\end{equation}
which is actually the planet's angular momentum {\em per unit mass}, is constant
in time. Taking the scalar product of the above equation with ${\bf r}$, we
obtain
\begin{equation}
{\bf h}\cdot{\bf r} = 0.
\end{equation}
This is the equation of a {\em plane}\/ which passes through the origin, and
whose normal is parallel to ${\bf h}$. Since ${\bf h}$  is a constant vector,
it always points in the {\em same}\/ direction. We, therefore, conclude that
the motion of our planet is {\em two-dimensional}\/ in nature: {\em i.e.}, it is confined to some fixed plane which passes through the origin. Without loss of generality, we can let this plane coincide with the $x$-$y$ plane.

\begin{figure}
\epsfysize=2.25in
\centerline{\epsffile{Chapter05/fig5.01.eps}}
\caption{\em Polar coordinates.}\label{polar}
\end{figure}

\section{Polar Coordinates}\label{s6.5}
We can determine the instantaneous position of our planet in the
$x$-$y$ plane in terms of standard Cartesian coordinates, ($x$, $y$),
or polar coordinates, ($r$, $\theta$), as illustrated in Figure~\ref{polar}. Here, $r=\sqrt{x^2+y^2}$ and $\theta=\tan^{-1}(y/x)$. 
It is helpful to define  two unit vectors, ${\bf e}_r\equiv {\bf r}/r$ and
${\bf e}_\theta\equiv {\bf e}_z\times {\bf e}_r$, at the
instantaneous position of the planet. The first always points radially away from the origin,
whereas the second is normal to the first, in the direction of increasing $\theta$. As is easily demonstrated, the Cartesian components of
${\bf e}_r$ and ${\bf e}_\theta$ are
\begin{eqnarray}\label{e6.9}
{\bf e}_r &=& (\cos\theta,\, \sin\theta),\\[0.5ex]
{\bf e}_\theta &=& (-\sin\theta,\, \cos\theta),\label{e6.10}
\end{eqnarray}
respectively. 

\begin{figure}
\epsfysize=2.25in
\centerline{\epsffile{Chapter05/fig5.02.eps}}
\caption{\em An ellipse.}\label{ellipse}
\end{figure}

We can  write the position vector of our planet as
\begin{equation}
{\bf r} = r\,{\bf e}_r.
\end{equation}
Thus, the planet's velocity becomes
\begin{equation}\label{e6.12}
{\bf v} = \frac{d{\bf r}}{dt} = \dot{r}\,{\bf e}_r + r\,\dot{\bf e}_r,
\end{equation}
where $\dot{~}$ is shorthand for $d/dt$. Note that ${\bf e}_r$
has a non-zero time-derivative (unlike a Cartesian unit vector) because its
direction {\em changes}\/ as the planet moves around. As is easily demonstrated,
from differentiating Equation~(\ref{e6.9}) with respect to time, 
\begin{equation}
\dot{\bf e}_r = \dot{\theta}\,(-\sin\theta,\,\cos\theta) = \dot{\theta}\,\,{\bf e}_\theta.
\end{equation}
Thus,
\begin{equation}\label{e6.14}
{\bf v} = \dot{r}\,\,{\bf e}_r + r\,\dot{\theta}\,\,{\bf e}_\theta.
\end{equation}
Now, the planet's acceleration is written
\begin{equation}
{\bf a} = \frac{d{\bf v}}{dt} = \frac{d^2{\bf r}}{dt^2}= \ddot{r}\,{\bf e}_r + \dot{r}\,\dot{\bf e}_r+(\dot{r}\,\dot{\theta}
+ r\,\ddot{\theta})\,{\bf e}_\theta + r\,\dot{\theta}\,\,\dot{\bf e}_\theta.
\end{equation}
Again, ${\bf e}_\theta$
has a non-zero time-derivative because its
direction {\em changes}\/ as the planet moves around. 
Differentiation of Equation~(\ref{e6.10}) with respect to time yields
\begin{equation}
\dot{\bf e}_\theta = \dot{\theta}\,(-\cos\theta,\,-\sin\theta) = - \dot{\theta}\,{\bf e}_r.
\end{equation}
Hence,
\begin{equation}
{\bf a} = (\ddot{r}-r\,\dot{\theta}^{\,2})\,{\bf e}_r + (r\,\ddot{\theta} + 2\,\dot{r}\,\dot{\theta})\,{\bf e}_\theta.
\end{equation}
It follows that the equation of motion of our planet, (\ref{e6.3}), can be written
\begin{equation}
{\bf a} =  (\ddot{r}-r\,\dot{\theta}^{\,2})\,{\bf e}_r + (r\,\ddot{\theta} + 2\,\dot{r}\,\dot{\theta})\,{\bf e}_\theta = - \frac{G\,M}{r^2}\,{\bf e}_r.
\end{equation}
Since ${\bf e}_r$ and ${\bf e}_\theta$ are mutually orthogonal, we can separately equate the coefficients of both, in the above equation, to give
a {\em radial equation of motion}, 
\begin{equation}\label{e6.19}
\ddot{r}-r\,\dot{\theta}^{\,2} = - \frac{G\,M}{r^2},
\end{equation}
and a {\em tangential equation of motion},
\begin{equation}\label{e6.20}
r\,\ddot{\theta} + 2\,\dot{r}\,\dot{\theta} = 0.
\end{equation}

\section{Conic Sections}
The ellipse, the parabola, and the hyperbola are collectively known
as {\em conic sections}, since these three  types of curve  can be obtained
by taking various different plane sections of a right cone. It turns out that the possible
solutions of Equations~(\ref{e6.19}) and (\ref{e6.20}) are all conic sections. It is,
therefore, appropriate for us to briefly review these curves.

An {\em ellipse}, centered on the origin, of major radius $a$ and minor radius $b$, which are aligned
along the $x$- and $y$-axes, respectively (see Figure~\ref{ellipse}), satisfies the following
well-known equation:
\begin{equation}\label{e6.21}
\frac{x^2}{a^2} + \frac{y^2}{b^2} = 1.
\end{equation}

Likewise, a parabola which is aligned along the $+x$-axis, and passes through
the origin (see Figure~\ref{parabola}), satisfies:
\begin{equation}\label{e6.22}
y^2 - b\,x = 0,
\end{equation}
where $b>0$. 

\begin{figure}
\epsfysize=2.25in
\centerline{\epsffile{Chapter05/fig5.03.eps}}
\caption{\em A parabola.}\label{parabola}
\end{figure}

Finally, a hyperbola which is aligned along the $+x$-axis, and whose
asymptotes intersect at the origin (see Figure~\ref{hyperbola}), satisfies:
\begin{equation}\label{e6.23}
\frac{x^2}{a^2} - \frac{y^2}{b^2} = 1.
\end{equation}
Here, $a$ is the distance of closest approach to the origin. The
asymptotes subtend an angle $\phi=\tan^{-1}(b/a)$ with the $x$-axis.

\begin{figure}
\epsfysize=2.25in
\centerline{\epsffile{Chapter05/fig5.04.eps}}
\caption{\em A hyperbola.}\label{hyperbola}
\end{figure}

It is not clear, at this stage, what  the ellipse, the parabola, and the hyperbola
have  in common (other than being conic sections). Well, it turns out that what these three curves
have in common is that they can all be represented as the locus of a movable point whose distance from
a fixed point is in a constant ratio to its perpendicular distance to some
fixed straight-line. Let the fixed point (which is termed the {\em focus}
of the ellipse/parabola/hyperbola) lie at the origin, and let
the fixed line correspond to  $x=-d$ (with $d>0$). Thus, the distance of a general point ($x$,\, $y$) (which lies to the right of the line $x=-d$) from the origin is $r_1=\sqrt{x^2+y^2}$, whereas the perpendicular distance of the point from
the line $x=-d$ is $r_2 = x+d$---see Figure~\ref{conic}. 
In polar coordinates, $r_1=r$ and $r_2 =  r\,\cos\theta+d$.
Hence, the locus of a point for which
$r_1$ and $r_2$ are in a fixed ratio satisfies the following equation:
\begin{equation}\label{e6.24}
\frac{r_1}{r_2} = \frac{\sqrt{x^2+y^2}}{x+d}= \frac{r}{r\,\cos\theta+d}=e,
\end{equation}
where $e\geq 0$ is a constant. When expressed in terms of 
polar coordinates, the above equation can be rearranged to give
\begin{equation}\label{e6.25}
r = \frac{r_c}{1-e\,\cos\theta},
\end{equation}
where $r_c=e\,d$. 

\begin{figure}
\epsfysize=2.25in
\centerline{\epsffile{Chapter05/fig5.05.eps}}
\caption{\em Conic sections in polar coordinates.}\label{conic}
\end{figure}

When written in terms of Cartesian coordinates, (\ref{e6.24})
can be rearranged to give
\begin{equation}\label{e6.26}
\frac{(x-x_c)^2}{a^2} + \frac{y^2}{b^2} = 1,
\end{equation}
for $e<1$. Here, 
\begin{eqnarray}
a &=& \frac{r_c}{1-e^2},\label{e6.27}\\[0.5ex]
b &=& \frac{r_c}{\sqrt{1-e^2}}=\sqrt{1-e^2}\,a,\label{e6.28}\\[0.5ex]
x_c &=& \frac{e\,r_c}{1-e^2}= e\,a.
\end{eqnarray}
Equation~(\ref{e6.26}) can be recognized as the equation of an {\em ellipse}
whose center lies at ($x_c$,\, $0$), and whose major and minor radii,
$a$ and $b$, are aligned along the $x$- and $y$-axes, respectively
[{\em cf.}, Equation~(\ref{e6.21})].

When again written in terms of Cartesian coordinates, Equation~(\ref{e6.24})
can be rearranged to give
\begin{equation}
y^2 - 2\,r_c\,(x-x_c) = 0,
\end{equation}
for $e=1$. Here, $x_c = -r_c/2$. This is the equation of a {\em parabola}
which passes through the point ($x_c$,\, $0$), and which is aligned
along the $+x$-direction [{\em cf.}, Equation (\ref{e6.22})].

Finally, when written in terms of Cartesian coordinates, Equation~(\ref{e6.24})
can be rearranged to give
\begin{equation}\label{e6.31}
\frac{(x-x_c)^2}{a^2}  - \frac{y^2}{b^2} = 1,
\end{equation}
for $e>1$. Here,
\begin{eqnarray}
a &=& \frac{r_c}{e^2-1},\\[0.5ex]
b &=& \frac{r_c}{\sqrt{e^2-1}}=\sqrt{e^2-1}\,a,\\[0.5ex]
x_c &=& -\frac{e\,r_c}{e^2-1}=-e\,a.
\end{eqnarray}
Equation~(\ref{e6.31}) can be recognized as the equation of a {\em hyperbola} whose asymptotes intersect at ($x_c$,\, $0$), and which
is aligned along the $+x$-direction. The asymptotes subtend an angle
\begin{equation}
\phi = \tan^{-1}\left(\frac{b}{a}\right) = \tan^{-1}(\sqrt{e^2-1})
\end{equation}
with the $x$-axis [{\em cf.}, Equation~(\ref{e6.23})].

In conclusion, Equation~(\ref{e6.25}) is the   polar equation of a {\em general conic
section}\/ which is {\em confocal with the origin}. For $e<1$, the conic section
is an {\em ellipse}. For $e=1$, the conic section is a {\em parabola}. Finally, for
$e>1$, the conic section is a {\em hyperbola}.

\section{Kepler's Second Law}
Multiplying our planet's tangential equation of motion, (\ref{e6.20}),
by $r$, we obtain
\begin{equation}
r^2\,\ddot{\theta} + 2\,r\,\dot{r}\,\dot{\theta} = 0.
\end{equation}
However, the above equation can be also written
\begin{equation}
\frac{d(r^2\,\dot{\theta})}{dt} = 0,
\end{equation}
which implies that
\begin{equation}\label{e6.38}
h = r^2\,\dot{\theta}
\end{equation}
is constant in time. It is easily demonstrated that $h$ is the magnitude
of the vector ${\bf h}$ defined in Equation~(\ref{e6.7}). Thus, the fact that
$h$ is constant in time is equivalent to the statement that the angular
momentum of our planet is a constant of its motion. As we have already mentioned, this is the case
because gravity is a central force.

\begin{figure}
\epsfysize=1.5in
\centerline{\epsffile{Chapter05/fig5.06.eps}}
\caption{\em Kepler's second law.}\label{area}
\end{figure}

Suppose that the radius vector connecting our planet to the origin ({\em i.e.}, the Sun) sweeps
out an angle $\delta\theta$ between times $t$ and $t+\delta t$---see Figure~\ref{area}. The approximately triangular region swept out by the radius vector has the area
\begin{equation}
\delta A \simeq \frac{1}{2}\,r^2\,\delta\theta,
\end{equation}
since the area of a triangle is half its base ($r\,\delta\theta$) times its
height ($r$). Hence, the rate at which the radius vector sweeps out area
is
\begin{equation}\label{e6.40}
\frac{dA}{dt} = \lim_{\delta t\rightarrow 0}\frac{r^2\,\delta\theta}{2\,\delta{t}}= \frac{r^2}{2}\,\frac{d\theta}{dt} = \frac{h}{2}.
\end{equation}
Thus, the radius vector sweeps out area at a constant rate (since $h$ is
constant in time)---this is Kepler's second law. We conclude that Kepler's
second law of planetary motion is a direct consequence of {\em angular
momentum conservation}.

\section{Kepler's First Law}
Our planet's radial equation of motion, (\ref{e6.19}), can be combined with
Equation~(\ref{e6.38}) to give
\begin{equation}\label{e6.41}
\ddot{r} -\frac{h^2}{r^3}= - \frac{G\,M}{r^2}.
\end{equation}
Suppose that $r = u^{-1}$. It follows that
\begin{equation}\label{e6.42}
\dot{r} = - \frac{\dot{u}}{u^2} = - r^2\,\frac{du}{d\theta}\,\frac{d\theta}{dt} = - h\,\frac{du}{d\theta}.
\end{equation}
Likewise,
\begin{equation}
\ddot{r} = - h \,\frac{d^2 u}{d\theta^2}\,\dot{\theta} = - u^2\,h^2\,\frac{d^2 u}{d\theta^2}.
\end{equation}
Hence, Equation~(\ref{e6.41}) can be written in the {\em linear}\/ form
\begin{equation}\label{e6.44}
\frac{d^2 u}{d\theta^2} + u = \frac{G\,M}{h^2}.
\end{equation}
The general solution to the above equation is
\begin{equation}\label{e6.45}
u(\theta) = \frac{G\,M}{h^2}\left[1 - e\,\cos(\theta-\theta_0)\right],
\end{equation}
where $e$ and $\theta_0$ are arbitrary constants. Without loss of generality, we can
set $\theta_0=0$ by rotating our coordinate system about the $z$-axis. Thus,
we obtain
\begin{equation}\label{e6.46}
r(\theta) = \frac{r_c}{1 - e\,\cos\theta},
\end{equation}
where
\begin{equation}\label{e6.47}
r_c = \frac{h^2}{G\,M}.
\end{equation}
We immediately recognize Equation~(\ref{e6.46}) as the equation of a conic
section which is confocal with the origin ({\em i.e.}, with the Sun).
Specifically, for $e<1$, Equation~(\ref{e6.46}) is the equation of an {\em ellipse}\/
which is {\em confocal with the Sun}. Thus, the orbit of our planet
around the Sun in a confocal ellipse---this is Kepler's first law
of planetary motion. Of course, a planet cannot have a parabolic
or a hyperbolic orbit, since such orbits are only appropriate to objects which are ultimately able to escape from the Sun's gravitational field.

\section{Kepler's Third Law}
We have seen that the radius vector connecting our planet to the
origin sweeps out area at the constant rate $dA/dt=h/2$ [see Equation~(\ref{e6.40})].
We have also seen that the planetary orbit is an ellipse. Suppose that
the major and minor radii of the ellipse are $a$ and $b$, respectively. It follows that the area of the ellipse is $A=\pi\,a\,b$. Now, we expect the
radius vector to sweep out the whole area of the ellipse in a single
orbital period, $T$. Hence,
\begin{equation}
T = \frac{A}{(dA/dt)} = \frac{2\pi\,a\,b}{h}.
\end{equation}
It follows from Equations~(\ref{e6.27}), (\ref{e6.28}), and (\ref{e6.47})
that
\begin{equation}\label{e6.49}
T^2 = \frac{4\pi^2\,a^3}{G\,M}.
\end{equation}
In other words, the {\em square}\/ of the orbital {\em period}\/ of our planet is proportional to the {\em cube}\/
of its orbital {\em major radius}---this is Kepler's third law.

Note that for an elliptical orbit the closest distance to the Sun---the so-called
{\em perihelion}\/ distance---is [see Equations~(\ref{e6.27}) and (\ref{e6.46})]
\begin{equation}\label{e6.50}
r_p = \frac{r_c}{1+e} = a\,(1-e).
\end{equation}
Likewise, the  furthest distance from the Sun---the so-called {\em aphelion}\/ distance---is
\begin{equation}\label{e6.51}
r_a = \frac{r_c}{1-e} = a\,(1+e).
\end{equation}
It follows that the major radius, $a$, is simply the mean of the perihelion
and aphelion distances,
\begin{equation}
a = \frac{r_p+r_a}{2}.
\end{equation}
The parameter 
\begin{equation}\label{e6.53}
e = \frac{r_a-r_p}{r_a+r_p}
\end{equation}
is called the {\em eccentricity}, and measures the deviation of the orbit
from circularity. Thus, $e=0$ corresponds to a circular orbit, whereas
$e\rightarrow 1$ corresponds to an infinitely elongated elliptical orbit.

As is easily demonstrated from the above analysis, Kepler laws of planetary motion can be written in the convenient form
\begin{eqnarray}
r &=& \frac{a\,(1-e^2)}{1-e\,\cos\theta},\\[0.5ex]
r^2\,\dot{\theta} &=& (1-e^2)^{1/2}\,n\,a^2,\\[0.5ex]
G\,M &=& n^2\,a^3,
\end{eqnarray}
where $a$ is the mean orbital radius ({\em i.e.}, the major radius), $e$ the orbital eccentricity, and
$n=2\pi/T$  the mean orbital angular velocity.

\section{Orbital Energies}
According to Equations~(\ref{e6.6}) and (\ref{e6.14}), the total energy per unit mass
of an object in orbit around the Sun is
given by
\begin{equation}\label{e6.54}
{\cal E} = \frac {\dot{r}^{\,2} + r^2\,\dot{\theta}^{\,2}}{2} - \frac{G\,M}{r}.
\end{equation}
It follows from Equations~(\ref{e6.38}), (\ref{e6.42}), and (\ref{e6.47}) that
\begin{equation}\label{e6.55}
{\cal E} = \frac{h^2}{2}\left[\left(\frac{du}{d\theta}\right)^2
+ u^2 - 2\,u\,u_c\right],
\end{equation}
where $u=r^{-1}$, and $u_c = r_c^{-1}$. However, according to Equation~(\ref{e6.46}),
\begin{equation}
u(\theta) = u_c\,(1-e\,\cos\theta).
\end{equation}
The previous two equations can be combined with Equations~(\ref{e6.47})
and (\ref{e6.50}) to give
\begin{equation}\label{e6.57}
{\cal E} = \frac{u_c^{\,2}\,h^2}{2}\, (e^2 -1) = \frac{G\,M}{2\,r_p}\,(e-1).
\end{equation}
We conclude that {\em elliptical}\/ orbits ($e<1$) have {\em negative}\/ total energies,
whereas {\em parabolic}\/ orbits ($e=1$) have {\em zero}\/ total energies,
and {\em hyperbolic}\/ orbits ($e>1$) have {\em positive}\/ total energies. This
makes sense, since in a conservative system in which the potential
energy at infinity is set to zero [see Equation~(\ref{e6.5})] we expect {\em bounded}\/ orbits to have {\em negative}\/ total energies, and {\em unbounded}\/ orbits to have {\em positive}\/
total energies---see Section~\ref{gpotn}. Thus, elliptical orbits, which are clearly bounded, should indeed have
negative total energies, whereas hyperbolic orbits, which are clearly
unbounded, should indeed have positive total energies. Parabolic orbits
are marginally bounded ({\em i.e.}, an object executing a parabolic orbit only just escapes from the Sun's gravitational field), and thus have zero total energy. For the special case of an elliptical orbit, whose major radius $a$ is {\em finite}, we can write
\begin{equation}
{\cal E} = - \frac{G\,M}{2\,a}.
\end{equation}
It follows that the energy of such an orbit is completely determined by the orbital {\em major radius}.

Consider an artificial satellite in an elliptical orbit around the
Sun (the same considerations also apply to satellites in orbit around the Earth). At perihelion, $\dot{r}=0$, and Equations~(\ref{e6.54}) and (\ref{e6.57})
can be combined to give
\begin{equation}\label{e6.57a}
\frac{v_t}{v_c} = \sqrt{1+e}.
\end{equation}
Here, $v_t=r\,\dot{\theta}$ is the satellite's tangential velocity, and
$v_c=\sqrt{G\,M/r_p}$ is the tangential velocity that it would need in order to
maintain a circular orbit at the perihelion distance. 
Likewise, at aphelion,
\begin{equation}\label{e6.57b}
\frac{v_t}{v_c} = \sqrt{1-e},
\end{equation}
where $v_c=\sqrt{G\,M/r_a}$ is now the tangential velocity that the
satellite would need in order to maintain a circular orbit at the aphelion distance.

Suppose that our satellite is initially in a circular orbit of radius $r_1$, and that we wish
to transfer it into a circular orbit of radius $r_2$, where $r_2> r_1$. We can
achieve this by temporarily placing the satellite in an  elliptical orbit
whose perihelion distance is $r_1$, and whose aphelion distance is $r_2$. 
It follows, from Equation~(\ref{e6.53}), that the required eccentricity of the elliptical
orbit is
\begin{equation}
e = \frac{r_2-r_1}{r_2+r_1}.
\end{equation}
According to Equation~(\ref{e6.57a}), we can transfer our satellite from its
initial circular orbit into the
temporary elliptical orbit by increasing its tangential velocity (by briefly
switching on the satellite's rocket motor) by a factor
\begin{equation}
\alpha_1 = \sqrt{1+e}.
\end{equation}
We must next allow the satellite to execute half an orbit, so that it attains its aphelion
distance, and then boost the tangential velocity by a
factor [see Equation~(\ref{e6.57b})]
\begin{equation}
\alpha_2 = \frac{1}{\sqrt{1-e}}.
\end{equation}
The satellite will now be in a circular orbit at the aphelion distance, $r_2$.
This process is illustrated in Figure~\ref{fftrans}. Obviously, we can transfer
our satellite from a larger to a smaller circular orbit by performing
the above process in reverse. Note, finally, from Equation~(\ref{e6.57a}), that if we
increase the tangential velocity of a satellite in a circular orbit about the Sun by a
factor greater than $\sqrt{2}$ then we will transfer it into a
hyperbolic orbit ($e>1$), and it will eventually escape from the Sun's
gravitational field.

\begin{figure}
\epsfysize=3.in
\centerline{\epsffile{Chapter05/fig5.07.eps}}
\caption{\em A transfer orbit between two circular orbits.}\label{fftrans}
\end{figure}

\section{Kepler Problem}
In a nutshell, the so-called {\em Kepler problem}\/ consists of determining
the radial and angular coordinates, $r$ and $\theta$, respectively, of
an object in a Keplerian orbit about the Sun as a function of time.

Consider an object in a general Keplerian orbit about the Sun which
passes through its perihelion point, $r=r_p$ and $\theta=0$, at $t=0$. It
follows from the previous analysis  that
\begin{equation}\label{e6.63x}
r = \frac{r_p\,(1+e)}{1+e\,\cos\theta},
\end{equation}
and
\begin{equation}
{\cal E} = \frac{\dot{r}^{\,2}}{2} + \frac{h^2}{2\,r^2} - \frac{G\,M}{r},
\end{equation}
where $e$, $h = \sqrt{G\,M\,r_p\,(1+e)}$, and ${\cal E} = G\,M\,(e-1)/(2\,r_p)$ are the orbital eccentricity, angular momentum per unit mass, and
energy per unit mass, respectively. The above equation can be rearranged to
give
\begin{equation}
\dot{r}^{\,2} = (e-1)\,\frac{G\,M}{r_p} - (e+1)\,\frac{r_p\,G\,M}{r^2}
+ \frac{2\,G\,M}{r}.
\end{equation}
Taking the square-root, and integrating, we obtain
\begin{equation}\label{e6.66x}
\int_{r_p}^r\frac{r\,dr}{[2\,r + (e-1)\,r^2/r_p - (e+1)\,r_p]^{1/2}} =
\sqrt{G\,M}\,\,t.
\end{equation}

Consider an elliptical orbit characterized by $0<e < 1$. Let us write
\begin{equation}\label{e6.67x}
r = \frac{r_p}{1-e}\,(1-e\,\cos E),
\end{equation}
where $E$ is termed the {\em elliptic anomaly}. In fact, $E$ is an angle which
varies between $-\pi$ and $+\pi$. Moreover, the perihelion point corresponds to
$E=0$, and the aphelion point to $E=\pi$. Now,
\begin{equation}
dr = \frac{r_p}{1-e}\,e\,\sin E\,dE,
\end{equation}
whereas
\begin{eqnarray}
2\,r + (e-1)\,\frac{r^2}{r_p}- (e+1)\,r_p &=& \frac{r_p}{1-e}\,e^2\,(1-\cos^2 E)\nonumber\\[0.5ex]
& =& \frac{r_p}{1-e}\,e^2\,\sin^2E.
\end{eqnarray}
Thus, Equation~(\ref{e6.66x}) reduces to
\begin{equation}
\int_0^E (1-e\,\cos E)\,dE = \left(\frac{G\,M}{a^3}\right)^{1/2} t,
\end{equation}
where $a = r_p/(1-e)$. This equation can immediately be integrated to give
\begin{equation}\label{e6.71x}
E - e\,\sin E = 2\pi\left(\frac{t}{T}\right),
\end{equation}
where
$T= 2\pi\,(a^3/GM)^{1/2}$ is the orbital period. Equation~(\ref{e6.71x}),
which  is known as {\em Kepler's equation},  is a transcendental equation 
which does not possess a simple analytic solution. Fortunately, it is fairly  straightforward to
solve numerically. For instance, using an iterative approach,
if $E_n$ is the $n$th guess then
\begin{equation}
E_{n+1} = 2\pi\left(\frac{t}{T}\right) + e\,\sin E_n.
\end{equation}
The above iteration scheme  converges very rapidly (except in the limit
as $e\rightarrow 1$). 

Equations~(\ref{e6.63x}) and (\ref{e6.67x}) can be combined
to give
\begin{equation}
\cos\theta = \frac{\cos E - e}{1-e\,\cos E}.
\end{equation}
Thus,
\begin{equation}
1+\cos\theta = 2\,\cos^2(\theta/2) = \frac{2\,(1-e)\,\cos^2( E/2)}{1-e\,\cos E},
\end{equation}
and
\begin{equation}
1-\cos\theta = 2\,\sin^2(\theta/2) = \frac{2\,(1+e)\,\sin^2 (E/2)}{1-e\,\cos E}.
\end{equation}
The previous two equations imply that
\begin{equation}
\tan (\theta/2) = \left(\frac{1+e}{1-e}\right)^{1/2}\tan (E/2).
\end{equation}

We conclude that, in the case of an elliptical orbit, the solution of the Kepler problem reduces to the solution of the following three equations:
\begin{eqnarray}
E - e\,\sin E &=& 2\pi\left(\frac{t}{T}\right),\\[0.5ex]
r &=&a\,(1-e\,\cos E),\\[0.5ex]
\tan(\theta/2) &=&\left(\frac{1+e}{1-e}\right)^{1/2} \tan (E/2).
\end{eqnarray}
Here, $T=2\pi\,(a^3/GM)^{1/2}$ and $a=r_p/(1-e)$. Incidentally, it is clear that if $t\rightarrow t+T$ then $E\rightarrow E + 2\pi$, and $\theta\rightarrow
\theta+2\pi$. In other words, the motion is periodic with period
$T$. 

For the case of a parabolic orbit, characterized by $e=1$, similar analysis to
the above yields:
\begin{eqnarray}
P + \frac{P^3}{3} &=& \left(\frac{G\,M}{2\,r_p^{\,3}}\right)^{1/2} t,\label{e6.80x}\\[0.5ex]
r &=& r_p\,(1+P^2),\\[0.5ex]
\tan(\theta/2) &=& P.\label{e6.82x}
\end{eqnarray}
Here, $P$ is termed the {\em parabolic anomaly}, and varies between
$-\infty$ and $+\infty$, with the perihelion point corresponding to $P=0$. Note that Equation~(\ref{e6.80x}) is a cubic equation,
possessing a single real root,
which can, in principle, be solved analytically. However, a numerical
solution is generally more convenient.

Finally, for the case of a hyperbolic orbit, characterized by $e>1$,
we obtain:
\begin{eqnarray}
e\,\sinh H - H &=& \left(\frac{G\,M}{a^3}\right)^{1/2} t,\label{e6.83x}\\[0.5ex]
r &=& a\,(e\,\cosh H - 1),\\[0.5ex]
\tan(\theta/2) &=& \left(\frac{e+1}{e-1}\right)^{1/2} \tanh (H/2).\label{e6.85x}
\end{eqnarray}
Here, $H$ is termed the {\em hyperbolic anomaly}, and varies between
$-\infty$ and $+\infty$, with the perihelion point corresponding to $H=0$.  Moreover, $a=r_p/(e-1)$. As in the elliptical
case, Equation~(\ref{e6.83x}) is a transcendental equation which is most easily solved numerically.

\section{Motion in  a General Central Force-Field}
Consider the motion of an object in a general (attractive) central force-field characterized by the potential energy {\em per unit mass}\/ function $V(r)$. Since the force-field
is central, it still remains true that
\begin{equation}\label{6.63}
h = r^2\,\dot{\theta}
\end{equation}
is a constant of the motion. As is easily demonstrated, Equation~(\ref{e6.44})
generalizes to 
\begin{equation}\label{e6.64}
\frac{d^2 u}{d\theta^2} + u = - \frac{1}{h^2}\frac{dV}{du},
\end{equation}
where $u=r^{-1}$. 

Suppose, for instance, that we wish to find the potential $V(r)$ which causes
an object to execute the spiral orbit
\begin{equation}\label{e6.65}
r = r_0\,\theta^{\,2}.
\end{equation}
Substitution of $u = (r_0\,\theta^2)^{-1}$ into Equation~(\ref{e6.64}) yields
\begin{equation}
\frac{d V}{du} = - h^2\left(6\,r_0\,u^2 + u\right).
\end{equation}
Integrating, we obtain
\begin{equation}
V(u) = -h^2\left(2\,r_0\,u^3 + \frac{u^2}{2}\right),
\end{equation}
or
\begin{equation}
V(r) = - h^2\left(\frac{2\,r_0}{r^3} + \frac{1}{2\,r^2}\right).
\end{equation}
In other words, the spiral pattern (\ref{e6.65}) is obtained from a mixture
of an inverse-square and inverse-cube potential.

\section{Motion in a Nearly Circular Orbit}
In principle, a circular orbit is a possible orbit for any attractive central force.
However, not all forces result in {\em stable}\/ circular orbits.
Let us now consider the stability of circular orbits in a general central force-field. Equation (\ref{e6.41}) generalizes to
\begin{equation}\label{e6.69}
\ddot{r} - \frac{h^2}{r^3} = f(r),
\end{equation}
where $f(r)$ is the radial force {\em per unit mass}. For a circular orbit,
$\ddot{r}=0$, and the above equation reduces to
\begin{equation}\label{e6.70}
-\frac{h^2}{r_c^{\,3}} = f(r_c),
\end{equation}
where $r_c$ is the radius of the orbit. 

Let us now consider {\em small}\/ departures from circularity. Let
\begin{equation}
x = r - r_c.
\end{equation}
Equation~(\ref{e6.69}) can be written
\begin{equation}\label{e6.72}
\ddot{x} - \frac{h^2}{(r_c+x)^3} = f(r_c+x).
\end{equation}
Expanding the two terms involving $r_c+x$ as power series in $x/r_c$,
and keeping all terms up to first order, we obtain
\begin{equation}
\ddot{x} - \frac{h^2}{r_c^{\,3}}\left(1-3\,\frac{x}{r_c}\right)=
f(r_c) + f'(r_c)\,x,
\end{equation}
where $'$ denotes a derivative. Making use of Equation~(\ref{e6.70}), 
the above equation reduces to
\begin{equation}\label{e6.74}
\ddot{x} + \left[-\frac{3\,f(r_c)}{r_c} - f'(r_c)\right] x = 0.
\end{equation}
If the term in square brackets is {\em positive}\/ then we obtain a simple harmonic
equation, which we already know has bounded solutions---{\em i.e.}, the orbit is {\em stable}\/ to small  perturbations. On the other hand,
if the term is square brackets is {\em negative} then we obtain an equation
whose solutions grow exponentially in time---{\em i.e.}, the orbit
is {\em unstable}\/ to small oscillations. Thus, the stability criterion for a circular
orbit of radius $r_c$ in a central force-field characterized by a radial force 
(per unit mass) function $f(r)$ is
\begin{equation}
f(r_c) + \frac{r_c}{3}\,f'(r_c)<0.
\end{equation}

For example, consider an attractive power-law force function of the form
\begin{equation}
f(c) = -c\,r^n,
\end{equation}
where $c>0$. Substituting into the above stability criterion, we obtain
\begin{equation}
-c\,r_c^{\,n} -\frac{c\,n}{3}\,r_c^{\,n} < 0,
\end{equation}
or 
\begin{equation}
n > -3.
\end{equation}
We conclude that circular orbits in attractive central force-fields which decay
faster than $r^{-3}$ are unstable. The case $n =-3$ is special, since the first-order terms in the expansion of Equation~(\ref{e6.72}) cancel out exactly, and it
is necessary  to retain the second-order terms. Doing this, it
is easily demonstrated that circular orbits are also unstable for
inverse-cube ($n=-3$) forces. 

An {\em apsis}\/ is a point in an orbit at which the radial distance, $r$, assumes either a
{\em maximum}\/  or a {\em minimum}\/ value. Thus, the perihelion and aphelion points
are the apsides of planetary orbits. The angle through which the radius vector
rotates in going between two consecutive apsides is called the {\em apsidal
angle}. Thus, the apsidal angle for elliptical orbits in an inverse-square
force-field is $\pi$. 

For the case of stable nearly circular orbits, we have seen that $r$ oscillates sinusoidally 
about its mean value, $r_c$. Indeed, it is clear from Equation~(\ref{e6.74}) that
the period of the oscillation is
\begin{equation}
T = \frac{2\pi}{\left[-3\,f(r_c)/r_c - f'(r_c)\right]^{1/2}}.
\end{equation}
The apsidal angle is the amount by which $\theta$ increases in going
between a maximum and a minimum of $r$. The time taken
to achieve this is clearly $T/2$.  Now, $\dot{\theta} = h/r^2$, where $h$
is a constant of the motion, and $r$ is almost constant. Thus, $\dot{\theta}$
is approximately constant. In fact,
\begin{equation}
\dot{\theta} \simeq \frac{h}{r_c^{\,2}} = \left[-\frac{f(r_c)}{r_c}\right]^{1/2},
\end{equation}
where use has been made of Equation~(\ref{e6.70}). Thus, the apsidal angle,
$\psi$, 
is given by
\begin{equation}\label{e6.81}
\psi = \frac{T}{2}\, \dot{\theta} = \pi \left[3+r_c\,\frac{f'(r_c)}{f(r_c)}\right]^{-1/2}
\end{equation}

For the case of attractive power-law central forces of the form $f(r) = -c\,r^n$, where
$c>0$, the apsidal angle becomes
\begin{equation}
\psi = \frac{\pi}{(3+n)^{1/2}}.
\end{equation}
Now, it should be clear that if an orbit is going to close on itself then the apsidal angle needs to be a {\em rational}\/ fraction of $2\pi$. There are, in fact,
only two small integer values of the power-law index, $n$, for which this
is the case. As we have seen, for an inverse-square force law ({\em i.e.}, $n=-2$), the
apsidal angle is $\pi$. For a linear force law ({\em i.e.}, $n=1$),
the apsidal angle is $\pi/2$---see Section~\ref{sn1}. However, for quadratic ({\em i.e.}, $n=2$) or cubic ({\em i.e.}, $n=3$) force laws, the apsidal angle is an {\em irrational}\/
fraction of $2\pi$, which means that non-circular orbits in such force-fields
never close on themselves.

Let us, finally, calculate the apsidal angle for a nearly circular orbit of radius $r_c$ in a slightly modified (attractive) inverse-square force law of the
form
\begin{equation}
f(r) = - \frac{k}{r^2} - \frac{\epsilon}{r^4},
\end{equation}
where $\epsilon/(k\,r_c^{\,2})$ is small. Substitution into Equation~(\ref{e6.81}) yields
\begin{equation}
\psi = \pi\left[3 + r_c\,\frac{2\,k\,r_c^{-3} + 4\,\epsilon\,r_c^{-5}}{-k\,r_c^{-2} -\epsilon\,r_c^{-4}}\right]^{-1/2} = \pi\left[3- 2\,\frac{1 + 2\,\epsilon/(k\,r_c^{\,2})}{1+\epsilon/(k\,r_c^{\,2})}\right]^{-1/2}.
\end{equation}
Expanding to first-order in $\epsilon/(k\,r_c^{\,2})$, we obtain
\begin{equation}
\psi\simeq \pi\left(3-2\,[1+\epsilon/(k\,r_c^{\,2})]\right)^{-1/2}=
\pi\left[1-2\,\epsilon/(k\,r_c^{\,2})\right]^{-1/2} \simeq
\pi\,[1+\epsilon/(k\,r_c^{\,2})].
\end{equation}
We conclude that if $\epsilon>0$ then the perihelion (or aphelion) of the orbit {\em advances}\/ by an angle $2\,\epsilon/(k\,r_c^{\,2})$ every rotation period.
It turns out that the general relativistic corrections to
Newtonian gravity give rise to a small $1/r^4$ modification (with $\epsilon>0$) to
the Sun's gravitational field. Hence, these corrections generate
a small precession in the perihelion of each planet orbiting the Sun. This
effect is particularly large for Mercury---see Section~\ref{smerc}.

\section{Exercises}
{\small
\renewcommand{\theenumi}{5.\arabic{enumi}}
\begin{enumerate}
\item Halley's comet has an orbital eccentricity of $e=0.967$ and a perihelion
distance of $55,000,000$ miles. Find the orbital period, and the comet's speed at
perihelion and aphelion.
\item A comet is first seen at a distance of $d$ astronomical units (1 astronomical unit is the mean Earth-Sun distance) from the Sun, and
is traveling with a speed of $q$ times the Earth's mean speed. Show that the orbit of the
comet is hyperbolic, parabolic, or elliptical, depending on whether the quantity $q^2\,d$ is
greater than, equal to, or less than 2, respectively.
\item Consider a planet in a Keplerian orbit of major radius $a$ and
eccentricity $e$ about the Sun. Suppose that the eccentricity of the orbit is small
({\em i.e.}, $0<e\ll 1$), as is indeed the case for all of the planets except Mercury
and Pluto. Demonstrate that, to first-order in $e$, the orbit can be approximated as a {\em circle}\/ whose center is shifted a distance $e\,a$ from
the Sun, and that the planet's angular motion appears {\em uniform}\/ when
viewed from a point (called the Equant) which is shifted a distance
$2\,e\,a$ from the Sun, in the same direction as the center of the circle.
This theorem is the
basis of the Ptolomaic model of planetary motion. 
\item How long (in days) does it take the Sun-Earth  radius vector to
rotate through $90^\circ$, starting at the perihelion point? How long does it take starting at the aphelion point? The period and eccentricity of the Earth's orbit are $T=365.24$ days, and $e=0.01673$, respectively.
\item Solve the Kepler problem for a parabolic orbit to obtain  Equations~(\ref{e6.80x})--(\ref{e6.82x}).
\item Solve the Kepler problem for a hyperbolic orbit to obtain Equations~(\ref{e6.83x})--(\ref{e6.85x}).
\item A comet is in a parabolic orbit lying in the plane of the Earth's
orbit. Regarding the Earth's orbit as a circle of radius $a$, show that the points
where the comet intersects the Earth's orbit are given by
$$
\cos\theta = -1 + \frac{2\,p}{a},
$$
where $p$ is the perihelion distance of the comet, defined at $\theta=0$. 
Show that the time interval that the comet remains inside the Earth's orbit is the
faction
$$
\frac{2^{1/2}}{3\,\pi}\left(\frac{2\,p}{a}+1\right)\left(1-\frac{p}{a}\right)^{1/2}
$$
of a year, and that the maximum value of this time interval is $2/3\pi$ year, or
about 11 weeks.
\item Prove that in the case of a central force varying inversely as the cube of the
distance
$$
r^2 = A\,t^2+B\,t+C,
$$
where $A$, $B$, $C$ are constants.
\item The orbit of a particle moving in a central field is a circle passing
through the origin, namely $r=r_0\,\cos\theta$. Show that the force law
is inverse-fifth power.
\item A particle moving in a central field describes a spiral orbit $r=r_0\,\exp(k\,\theta)$.
Show that the force law is inverse-cube, and that $\theta$ varies logarithmically with $t$.
Show that there are two other possible types of orbit in this force-field, and give their
equations.
\item A particle moves in a spiral orbit given by $r=a\,\theta$. Suppose that $\theta$ increases linearly
with $t$. Is the force acting on the particle central in nature? If not, determine how $\theta$ would have to
vary with $t$ in order to make the force  central.
\item A particle moves in a circular orbit of radius $r_0$ in an attractive
central force-field of the form $f(r) = -c\,\exp(-r/a)/r^2$, where $c>0$ and $a>0$.
Demonstrate that the orbit is only stable provided that $r_0<a$.
\item A particle moves in a circular orbit in an attractive
central force-field of the form $f(r) = -c\,r^{-3}$, where $c>0$. Demonstrate
that the orbit is unstable to small perturbations.
\item If the Solar System were embedded in a uniform dust cloud, what would the apsidal angle
of a planet be for motion in a nearly circular orbit? Express your answer in terms of the ratio of the mass of dust contained in a sphere, centered on the Sun, whose radius is that of the orbit, to the mass of the Sun.
This model was once suggested as a possible
explanation for the advance of the perihelion of Mercury. 

\item The potential energy per unit mass of a particle in the gravitational
field of an oblate spheroid, like the Earth, is
$$
V(r) = - \frac{G\,M}{r}\left(1+\frac{\epsilon}{r^2}\right),
$$
where $r$ refers to distances in the equatorial plane, $M$ is the Earth's mass, and
$\epsilon= (2/5)\,R\,{\mit\Delta}R$. Here, $R=4000\,{\rm mi}$ is the Earth's equatorial radius, and ${\mit\Delta}R = 13\,{\rm mi}$ the difference between the equatorial and polar radii.
Find the apsidal angle for a satellite moving in a nearly circular orbit in the equatorial
plane of the Earth.
\end{enumerate}

}