\chapter{Multi-Dimensional Motion}
\section{Introduction}
This chapter employs Newton's laws to 
investigate various aspects  of multi-dimensional motion. 

\section{Motion in a Two-Dimensional Harmonic Potential}\label{sn1}
Consider a particle of mass $m$ moving in 
the two-dimensional harmonic potential
\begin{equation}\label{e5.1}
U(x,y) = \frac{1}{2}\,k\,r^2,
\end{equation} 
where $r=\sqrt{x^2+y^2}$, and $k>0$. It follows that the particle is subject to
a force,
\begin{equation}
{\bf f} = - \nabla U = -k\,(x,\, y) = - k\,{\bf r},
\end{equation}
which always points {\em towards}\/ the origin, and whose magnitude increases {\em linearly}\/ with
increasing distance from the origin. According to Newton's second law, the
equation of motion of the particle is 
\begin{equation}
m\,\frac{d^2{\bf r}}{dt^2} = {\bf f} = - k\,{\bf r}.
\end{equation}
When written in component form, the above equation reduces to
\begin{eqnarray}\label{e5.4}
\frac{d^2 x}{d t^2} &=& - \omega_0^{\,2}\,x,\\[0.5ex]
\frac{d^2 y}{d t^2} &=& - \omega_0^{\,2}\,y,\label{e5.5}
\end{eqnarray}
where $\omega_0 = \sqrt{k/m}$. 

Since Equations~(\ref{e5.4}) and (\ref{e5.5}) are both {\em simple harmonic equations},
we can immediately write their general solutions:
\begin{eqnarray}
x &=& A\,\cos(\omega_0\,t-\phi_1),\\[0.5ex]
y&=& B\,\cos(\omega_0\,t-\phi_2).
\end{eqnarray}
Here, $A$, $B$, $\phi_1$, and $\phi_2$ are arbitrary constants of integration. We can simplify the above equations slightly by shifting the
origin of time (which is, after all, arbitrary): {\em i.e.},
\begin{equation}
t\rightarrow t' + \phi_1/\omega_0.
\end{equation}
Hence, we obtain
\begin{eqnarray}\label{e5.9}
x &=& A\,\cos(\omega_0\,t'),\\[0.5ex]
y&=& B\,\cos(\omega_0\,t'-{\mit\Delta}),\label{e5.10}
\end{eqnarray}
where ${\mit\Delta}=\phi_2-\phi_1$. 
Note that the motion is clearly {\em periodic}\/ in time, with period $T=2\,\pi/\omega_0$.
Thus, the particle must trace out some {\em closed trajectory}\/ in the
$x$-$y$ plane.
The question, now, is what does this
trajectory look like  as a function of
the relative phase-shift, ${\mit\Delta}$, between the oscillations in the
$x$- and $y$-directions?

Using standard trigonometry, we can write Equation~(\ref{e5.10})
in the form
\begin{equation}
y = B\left[\cos(\omega_0\,t')\,\cos{\mit\Delta} + \sin(\omega_0\,t')\,\sin{\mit\Delta}\right].
\end{equation}
Hence, using Equation~(\ref{e5.9}), we obtain
\begin{equation}
\left(\frac{y}{B} - \frac{x}{A}\,\cos{\mit\Delta}\right)^2=\sin^2(\omega_0\,t')\,\sin^2{\mit\Delta} = \left(1-\frac{x^2}{A^2}\right)\sin^2{\mit\Delta},
\end{equation}
which simplifies to give
\begin{equation}\label{e5.13}
\frac{x^2}{A^2}  - 2\,\frac{x\,y}{A\,B} \,\cos{\mit\Delta} + \frac{y^2}{B^2}
= \sin^2{\mit\Delta}.
\end{equation}
Unfortunately, the above equation is not immediately recognizable as being
the equation of any particular geometric curve: {\em e.g.}, a circle, an ellipse, or
a parabola, {\em etc}.

Perhaps our problem is that we are using the wrong coordinates.
Suppose that we rotate our coordinate axes about the $z$-axis by an
angle $\theta$, as illustrated in Figure~\ref{f3}. According to Equations~(\ref{t1}) and (\ref{t2}), our old coordinates  ($x$, $y$) are related to our new coordinates
($x'$, $y'$) via
\begin{eqnarray}
x&=& x'\,\cos\theta - y'\,\sin \theta,\label{e5.14}\\[0.5ex]
y &=& x'\,\sin\theta + y'\,\cos\theta.\label{e5.15}
\end{eqnarray}
Let us see whether Equation~(\ref{e5.13}) takes a simpler form when expressed
in terms of our new coordinates. Equations (\ref{e5.13})--(\ref{e5.15})
yield
\begin{eqnarray}
x'^{\,2}\left[\frac{\cos^2\theta}{A^2} - \frac{2\,\cos\theta\,\sin\theta\,\cos{\mit\Delta}}{A\,B} + \frac{\sin^2\theta}{B^2}\right]&&\nonumber\\[0.5ex]
+ y'^{\,2}\left[\frac{\sin^2\theta}{A^2} +\frac{2\,\cos\theta\,\sin\theta\,\cos{\mit\Delta}}{A\,B} + \frac{\cos^2\theta}{B^2}\right]&&\label{e5.16}\\[0.5ex]
+x' \,y'\left[-\frac{2\,\sin\theta\,\cos\theta}{A^2} + \frac{2\,(\sin^2\theta-\cos^2\theta)\,\cos{\mit\Delta}}{A\,B} + \frac{2\,\cos\theta\,\sin\theta}{B^2}\right] &=& \sin^2{\mit\Delta}.\nonumber
\end{eqnarray}
We can simplify the above equation by setting the term involving $x' \,y'$ to
zero. Hence,
\begin{equation}
-\frac{\sin (2\,\theta)}{A^2}  - \frac{2\,\cos (2\,\theta)\,\cos{\mit\Delta}}{A\,B} + \frac{\sin (2\,\theta)}{B^2}= 0,
\end{equation}
where we have made use of some simple trigonometric identities. Thus, the $x' \,y'$ term disappears when $\theta$ takes the special value
\begin{equation}
\theta = \frac{1}{2}\,\tan^{-1}\left(\frac{2\,A\,B\,\cos{\mit\Delta}}{A^2-B^2}\right).
\end{equation}
In this case, Equation~(\ref{e5.16}) reduces to
\begin{equation}\label{e5.19}
\frac{x'^{\,2}}{a^2} + \frac{y'^{\,2}}{b^2} = 1,
\end{equation}
where
\begin{eqnarray}
\frac{1}{a^2}& =& \frac{1}{\sin^2{\mit\Delta}}\left[\frac{\cos^2\theta}{A^2} - \frac{2\,\cos\theta\,\sin\theta\,\cos{\mit\Delta}}{A\,B} + \frac{\sin^2\theta}{B^2}\right],\\[0.5ex]
\frac{1}{b^2}& =& \frac{1}{\sin^2{\mit\Delta}}\left[\frac{\sin^2\theta}{A^2} +\frac{2\,\cos\theta\,\sin\theta\,\cos{\mit\Delta}}{A\,B} + \frac{\cos^2\theta}{B^2}\right].
\end{eqnarray}
Of course, we immediately recognize Equation~(\ref{e5.19}) as the equation of
an {\em ellipse}, centered on the origin, whose major and minor axes are aligned along the
$x'$- and $y'$-axes, and whose major and minor radii are $a$ and $b$,
respectively (assuming that $a>b$). 

We conclude that, in general, a particle of mass $m$ moving in the two-dimensional harmonic potential (\ref{e5.1}) executes a {\em closed elliptical
orbit}\/ (which is not necessarily aligned along the $x$- and $y$-axes), centered on the origin, with
period $T = 2\,\pi/\omega_0$, where $\omega_0=\sqrt{k/m}$. 

\begin{figure}[h]
\epsfysize=3.25in
\centerline{\epsffile{Chapter04/fig4.01.eps}}
\caption{\em Trajectories in a two-dimensional harmonic oscillator potential.}\label{f24}
\end{figure}

Figure~\ref{f24} shows  some example trajectories calculated for $A=2$, $B=1$, and
the following values of the phase difference, ${\mit\Delta}$: (a) ${\mit\Delta}=0$; (b) ${\mit\Delta}=\pi/4$; (c) ${\mit\Delta}=\pi/2$;
(d) ${\mit\Delta}= 3\pi/4$. Note that when ${\mit\Delta}=0$ the
trajectory degenerates into a straight-line (which can be thought of as an
ellipse whose minor radius is zero).

Perhaps, the main lesson to be learned from the above study of two-dimensional
motion in a harmonic potential is that comparatively simple patterns of
motion can be made to look complicated when expressed 
in terms of ill-chosen coordinates.

\section{Projectile Motion with Air Resistance}
Suppose that a
projectile of mass $m$ is launched, at $t=0$, from ground level (in a flat plain), making an angle $\theta$ to the horizontal. Suppose, further, that, in addition
to the force of gravity, the projectile is subject to an air resistance
force which acts in the opposite direction to  its instantaneous
direction of motion, and whose magnitude is {\em directly proportional}\/ to its  instantaneous speed. This is not a particularly accurate model
of the drag force due to air resistance (the magnitude of the drag force is typically proportion
to the square of the speed---see Section~\ref{svelyd}), but it does lead to
tractable equations of motion. Hence, by using this model we can, at least,
get some idea of how air resistance modifies projectile trajectories.

Let us adopt a Cartesian coordinate system whose origin coincides with the launch point, and whose $z$-axis
points vertically upward. Let the initial velocity of the projectile
lie in the $x$-$z$ plane. Note that, since neither gravity nor the drag force 
cause the projectile to  move out of the $x$-$z$ plane,
we can effectively ignore the $y$ coordinate in this problem.

The equation of motion of our projectile is written
\begin{equation}
m\,\frac{d{\bf v}}{dt} = m\,{\bf g} - c\,{\bf v},
\end{equation}
where ${\bf v} = (v_x, v_z)$ is the projectile velocity,
${\bf g} = (0, -g)$ the acceleration due to gravity, and
$c$ a positive constant. In component form, the
above equation becomes
\begin{eqnarray}
\frac{dv_x}{dt} &=& - g\,\frac{v_x}{v_t},\label{e5.23u}\\[0.5ex]
\frac{d v_z}{dt} &=& - g\left(1+\frac{v_z}{v_t}\right).\label{e5.24u}
\end{eqnarray}
Here, $v_t = m\,g/c$ is the {\em terminal velocity}: {\em i.e.}, the
velocity at which the drag force balances the gravitational force (for a
projectile falling vertically downward).

Integrating Equation~(\ref{e5.23u}), we obtain
\begin{equation}
\int_{v_{x\,0}}^v\frac{dv_x}{v_x} = - \frac{g}{v_t}\,t,
\end{equation}
where $v_{x\,0} = v_0\,\cos\theta$ is the $x$-component of the
launch velocity. 
Hence,
\begin{equation}
\ln\left(\frac{v_x}{v_{x\,0}}\right) = - \frac{g}{v_t}\,t,
\end{equation}
or
\begin{equation}\label{e5.27u}
v_x = v_0\,\cos\theta\,{\rm e}^{-g\,t/v_t}.
\end{equation}
It is clear, from the above equation, that air drag causes the  projectile's horizontal
velocity, which would otherwise be constant, to {\em decay}\/ exponentially on a time-scale of order $v_t/g$.

Integrating Equation~(\ref{e5.24u}), we get
\begin{equation}
\int_{v_{z\,0}}^{v_z} \frac{dv_z}{v_t+v_z} = - \frac{g}{v_t}\,t,
\end{equation}
where $v_{z\,0}=v_0\,\sin\theta$ is the $z$-component of the launch velocity. Hence,
\begin{equation}
\ln\left(\frac{v_t+v_z}{v_t+v_{z\,0}}\right) = - \frac{g}{v_t}\,t,
\end{equation}
or
\begin{equation}\label{e5.30u}
v_z = v_0\,\sin\theta\,{\rm e}^{-g\,t/v_t} - v_t\left(1-e^{-g\,t/v_t}\right).
\end{equation}
It thus follows, from Equations~(\ref{e5.27u}) and (\ref{e5.30u}), that if the
projectile stays in the air much longer than a time of order $v_t/g$ then
it ends up falling vertically downward at the terminal velocity, $v_t$, irrespective
of its initial launch angle.

Integration of (\ref{e5.27u}) yields
\begin{equation}
x = \frac{v_0\,v_t\,\cos\theta}{g}\left(1-{\rm e}^{-g\,t/v_t}\right).
\end{equation}
In the limit $t\ll v_t/g$, the above equation reduces to
\begin{equation}\label{e5.32u}
x = v_0\,\cos\theta\,t,
\end{equation}
which is the standard result in the absence of air drag.
In the opposite limit, $t\gg v_t/g$, we get
\begin{equation}\label{e5.33u}
x = \frac{v_0\,v_t\,\cos\theta}{g}.
\end{equation}
The above expression clearly sets an effective upper limit on how far the
projectile can travel in the horizontal direction.

Integration of (\ref{e5.30u}) gives
\begin{equation}
z = \frac{v_t}{g}\,\left(v_0\,\sin\theta + v_t\right)\left(1-{\rm e}^{-g\,t/v_t}\right) - v_t\,t.
\end{equation}
In the limit $t\ll v_t/g$, this equation reduces to
\begin{equation}
z = v_0\,\sin\theta\,t - \frac{g}{2}\,t^2,
\end{equation}
which is the standard result in the absence of air drag. In the opposite
limit, $t\gg v_t/g$, we get
\begin{equation}
z =  \frac{v_t}{g}\,\left(v_0\,\sin\theta + v_t\right) - v_t\,t.
\end{equation}
Incidentally,   the above analysis implies that air resistance
only starts to have an appreciable effect on the trajectory after
the projectile has been in the air a time of order $v_t/g$.

It is clear, from the previous two equations, that the time
of flight of the projectile ({\em i.e.}, the time at which $z=0$, excluding the
trivial result $t=0$) is
\begin{equation}
t_f = \frac{2\,v_0\,\sin\theta}{g}
\end{equation}
when $t\ll v_t/g$, which implies that $v_0\,\sin\theta \ll v_t$,
and
\begin{equation}
t_f = \frac{v_0\,\sin\theta}{g}
\end{equation}
when  $t\gg v_t/g$, which implies that $v_0\,\sin\theta \gg v_t$ ({\em i.e.}, the vertical component of the
launch velocity is much greater than the terminal velocity).
It thus follows, from Equations~(\ref{e5.32u}) and (\ref{e5.33u}), that
the horizontal range [{\em i.e.}, $x(t_f)$] of the projectile
is 
\begin{equation}\label{e5.39u}
R = \frac{v_0^{\,2}\,\sin(2\,\theta)}{g}
\end{equation}
when $v_0\,\sin\theta \ll v_t$, and
\begin{equation}\label{e5.40u}
R= \frac{v_0\,v_t\,\cos\theta}{g}
\end{equation}
when $v_0\,\sin\theta \gg v_t$.
Equation~(\ref{e5.39u}) is, of course, the standard result without
air resistance. This result implies that, in the absence of air resistance, the maximum horizontal range, $v_0^{\,2}/g$,
is achieved when the launch angle $\theta$ takes the value $45^\circ$.
On the other hand, Equation~(\ref{e5.40u}) implies that, in the presence of
air resistance, the maximum horizontal
range, $v_0\,v_t/g$, is achieved when $\theta$ is made as small
as possible. However, $\theta$ cannot be made too small, since
expression (\ref{e5.40u}) is only valid when $v_0\,\sin\theta \gg v_t$.
In fact, assuming  that $v_0\gg v_t$,  the maximum horizontal
range, $v_0\,v_t/g$, is achieved when $\theta\sim v_t/v_0\ll 1$. We thus
conclude that if air resistance is significant then it causes the horizontal range of the
projectile to scale {\em linearly}, rather than quadratically, with the
launch velocity, $v_0$. Moreover, the maximum horizontal  range is achieved
with a launch angle which is {\em much shallower}\/ than the standard
result, $45^\circ$. 

\begin{figure}[h]
\epsfysize=3in
\centerline{\epsffile{Chapter04/fig4.02.eps}}
\caption{\em Projectile trajectories in the presence of air resistance.}\label{fdrag}
\end{figure}

Figure~\ref{fdrag} shows some example trajectories calculated, from the above model, with the same launch
angle, $45^\circ$, but with different values of the ratio $v_0/v_t$. Here,
$X=x/(v_0^{\,2}/g)$ and $Z=z/(v_0^{\,2}/g)$. The solid, short-dashed,
long-dashed, and dot-dashed curves correspond to $v_0/v_t = 0$, $1$, $2$,
and $4$, respectively. It can be seen that as the air resistance strength
increases  ({\em i.e.}, as $v_0/v_t$ increases), the range of the
projectile decreases.  Furthermore, there is always an initial time interval
during which the trajectory is identical to that calculated in the absence
of air resistance ({\em i.e.}, $v_0/v_t=0$). Finally, in the presence of
air resistance, the projectile tends to fall more steeply than it rises.
Indeed, in the presence of strong air resistance ({\em i.e.}, $v_0/v_t=4$), the projectile falls almost
vertically.

\section{Charged Particle Motion in  Electric and Magnetic Fields}
Consider a particle of mass $m$ and electric charge $q$ moving
in the {\em uniform}\/ electric and magnetic fields, ${\bf E}$ and ${\bf B}$. 
Suppose that the fields are ``crossed'' ({\em i.e.}, perpendicular to
one another), so that ${\bf E}\cdot{\bf B} = 0$. 

The force acting on the particle is given by the familiar Lorentz law:
\begin{equation}
{\bf f} = q\,\left({\bf E} + {\bf v}\times{\bf B}\right),
\end{equation}
where ${\bf v}$ is the particle's instantaneous velocity. Hence, from
Newton's second law, the particle's equation of motion can be written
\begin{equation}\label{e5.23}
m\,\frac{d{\bf v}}{dt} = q\,\left({\bf E} + {\bf v}\times{\bf B}\right).
\end{equation}

It turns out that we can eliminate the electric field from the above equation by
transforming to a different inertial frame. Thus, writing
\begin{equation}
{\bf v} = \frac{{\bf E}\times{\bf B}}{B^2} + {\bf v}',
\end{equation}
Equation~(\ref{e5.23}) reduces to
\begin{equation}\label{e5.24}
m\,\frac{d{\bf v}'}{dt} = q\,{\bf v}'\times{\bf B},
\end{equation}
where we have made use of a standard vector identity (see Section~\ref{svtp}), as well
as the fact that ${\bf E}\cdot{\bf B} = 0$.
Hence, we conclude that the addition of an electric field perpendicular to a
given magnetic field simply causes the particle to drift perpendicular to both the electric and magnetic
field with the fixed velocity
\begin{equation}\label{e5.26}
{\bf v}_{EB} = \frac{{\bf E}\times{\bf B}}{B^2},
\end{equation}
irrespective of its
charge or mass.
It follows that the electric field has no effect on the particle's
motion in  a frame of reference which is co-moving with the so-called {\em E-cross-B velocity}\/ given above.

Let us suppose that the magnetic field is directed along the $z$-axis. 
As we have just seen, in the ${\bf E}\times{\bf B}$ frame, the particle's equation of motion reduces to Equation~(\ref{e5.24}), which can be written:
\begin{eqnarray}\label{e5.27}
\frac{d v_x'}{dt} &=& {\mit\Omega}\,v_y',\\[0.5ex]
\frac{d v_y'}{dt} &=& -{\mit\Omega}\,v_x',\\[0.5ex]
\frac{d v_z'}{dt} &=& 0.\label{e5.29}
\end{eqnarray}
Here,
\begin{equation}
{\mit\Omega} = \frac{q\,B}{m}
\end{equation}
is the so-called {\em cyclotron frequency}.
Equations~(\ref{e5.27})--(\ref{e5.29}) can be integrated to give
\begin{eqnarray}\label{e5.31}
v_x'&=& v_\perp\,\sin({\mit\Omega}\,t),\\[0.5ex]
v_y'&=&v_\perp\,\cos({\mit\Omega}\,t)\\[0.5ex]
v_z' &=& v_\parallel,\label{e5.33}
\end{eqnarray}
where we have judiciously chosen the origin of time so as to eliminate
any phase offset in the arguments of the above trigonometrical functions.
According to Equations~(\ref{e5.31})--(\ref{e5.33}), in the ${\bf E}\times {\bf B}$
frame, our charged particle gyrates at the cyclotron frequency in the plane perpendicular to the
magnetic field with some fixed speed $v_\perp$, and drifts parallel to the magnetic
field with  some  fixed speed $v_\parallel$. 
The fact that the cyclotron frequency is positive for positively charged
particles, and negative for negatively charged particles, just means that
oppositely charged particles gyrate in opposite directions in the
plane perpendicular to the magnetic field.

Equations (\ref{e5.31})--(\ref{e5.33}) can be integrated to give
\begin{eqnarray}\label{e5.34}
x'&=& -{\mit\rho}\,\cos({\mit\Omega}\,t),\\[0.5ex]
y'&=&{\mit\rho}\,\sin({\mit\Omega}\,t)\\[0.5ex]
z' &=& v_\parallel\,t,\label{e5.37}
\end{eqnarray}
where we have  judiciously chosen the origin of our coordinate system so
as to eliminate any constant offsets in the above equations.
Here,
\begin{equation}
{\bf \rho} = \frac{v_\perp}{{\mit\Omega}}
\end{equation}
is called the {\em Larmor radius}. Equations~(\ref{e5.34})--(\ref{e5.37})
are the equations of a {\em spiral}\/ of radius ${\mit\rho}$, aligned along the direction
of the magnetic field ({\em i.e.}, the $z$-direction).

\begin{figure}[h]
\epsfysize=1in
\centerline{\epsffile{Chapter04/fig4.03.eps}}
\caption{\em The spiral trajectory of a negatively charged particle in a magnetic field.}\label{f5.3}
\end{figure}

We conclude that the general motion of a charged particle in crossed
electric and magnetic field is a combination of ${\bf E}\times{\bf B}$
drift [see Equation (\ref{e5.26})] and spiral motion aligned along the
direction of the magnetic field---see Figure~\ref{f5.3}. Particles drift parallel to the magnetic field
with constant speeds, and gyrate at the cyclotron frequency in the plane perpendicular to the magnetic field with constant speeds.
Oppositely charged particles gyrate in opposite directions.

\section{Exercises}
{\small
\renewcommand{\theenumi}{4.\arabic{enumi}}
\begin{enumerate}
\item An electron of mass $m$ and charge $-e$ moves in a uniform $y$-directed electric field of magnitude $E$,
and a uniform $z$-directed magnetic field of magnitude $B$. The electron is situated at the
origin at $t=0$ with an initial $x$-directed velocity of magnitude $v_0$. Show that the
electron traces out a cycloid of the general form
\begin{eqnarray}
x &=& a\,\sin(\omega\,t) + b\,t,\nonumber\\[0.5ex]
y &=& c\,[1-\cos(\omega\,t)],\nonumber\\[0.5ex]
z &=& 0.\nonumber
\end{eqnarray}
Find the values of $a$, $b$, $c$, and $\omega$, and sketch the electron's trajectory
in the $x$-$y$ plane when $v_0<E/B$, $E/B < v_0 < 2\,E/B$, and
$v_0> 2\,E/B$.

\item A particle of mass $m$ and charge $q$ moves in the $x$-$y$ plane
under the influence of a constant amplitude rotating electric field
which is such that $E_x= E_0\,\cos(\omega\,t)$ and $E_y=E_0\,\sin(\omega\,t)$. The particle starts at rest from the origin. Determine its subsequent motion. What shape is the particle's trajectory?

\item A particle of mass $m$ slides on a frictionless surface whose height is a
function of $x$ only: {\em i.e.}, $z=z(x)$. The function $z(x)$
is specified by the parametric equations
\begin{eqnarray}
x&= &A\,[2\,\phi + \sin (2\,\phi)],\nonumber\\[0.5ex]
z& = &A\,[1-\cos (2\,\phi)],\nonumber
\end{eqnarray}
where $\phi$ is the parameter. Show that the total energy of the
particle can be written
$$
E = \frac{m}{2}\left(\frac{ds}{dt}\right)^{2} + \frac{1}{2}\,\frac{m\,g}{4\,A}\,s^2,
$$
where $s=4\,A\,\sin\phi$. Deduce that the particle undergoes periodic
motion whose frequency is amplitude independent (even when the amplitude
is large). Demonstrate that the frequency of the motion is given by $4\pi\,(A/g)^{1/2}$. 
\end{enumerate}
}