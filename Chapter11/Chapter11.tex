\chapter{Coupled Oscillations}
\section{Introduction} 
This chapter examines the motion of a many degree of
freedom dynamical system which is perturbed from some equilibrium state.

\section{Equilibrium State}
Consider an ${\cal F}$ degree of freedom dynamical system described by the
generalized coordinates $q_i$, for $i=1,{\cal F}$. Suppose that the
kinetic energy $K$ and the potential energy $U$ are not explicit
functions of time. This implies that the system in question
is {\em isolated}: {\em i.e.}, it is not subject to any external forces or
time-varying constraints. In virtually all dynamical systems of interest, the kinetic
energy can be expressed as a {\em quadratic form}: {\em i.e.}, 
\begin{equation}\label{e12.1}
K = \frac{1}{2}\sum_{i,j=1,{\cal F}} m_{ij}(q_1,q_2,\cdots, q_{\cal F})\,\dot{q}_i\,\dot{q}_j.
\end{equation}
Without loss of generality, we can specify that the weights $m_{ij}$ in the above form are
invariant under interchange of the indices $i$ and $j$: {\em i.e.},
\begin{equation}
m_{ij} = m_{ji}.
\end{equation}
Finally, the potential energy is written $U=U(q_1,q_2,\cdots, q_{\cal F})$. 

Suppose that $q_i=q_{i\,0}$, for $i=1,{\cal F}$, corresponds to an
{\em equilibrium state} of the system. It follows that
$q_i=q_{i\,0}$ and $\dot{q}_i=\ddot{q}_i=0$, for $i=1,{\cal F}$, should be a possible solution of the equations of motion. 
Now, Lagrange's equations of motion for the system take the form [see
Equation~(\ref{e10.23})]
\begin{equation}\label{e12.3}
\frac{d}{dt}\!\left(\frac{\partial K}{\partial \dot{q}_i}\right) - \frac{\partial K}{\partial q_i}+ \frac{\partial U}{\partial q_i} = 0,
\end{equation}
for $i=1,{\cal F}$. Here, we have made use of the definition $L=K-U$, and
the fact that $U$ is independent of the $\dot{q}_i$.  Now, it is clear, from
an examination of Equation~(\ref{e12.1}), that every component making up
the first two terms in the above equation contains either a $\dot{q}_j$ or
a $\ddot{q}_j$, for some $j$.  But, we can set all of the generalized velocities
and accelerations  to zero in an equilibrium state of the system.
Hence, the first two terms in the above equation are zero, and the
condition for equilibrium reduces to
\begin{equation}\label{e12.2}
Q_{i\,0} = - \frac{\partial U(q_{1\,0}, q_{2\,0},\cdots,q_{{\cal F}\,0})}{\partial q_i} = 0,
\end{equation}
for $i=1,{\cal F}$. In other words,  $q_i=q_{i\,0}$, for $i=1,{\cal F}$, is an equilibrium
state provided that all of the generalized forces, $Q_i$ [see Equation~(\ref{e10.8})],  evaluated at $q_i=q_{i\,0}$,
are zero. Let us suppose that this is the case. 

\section{Stability Equations}
It is evident that if our system is initialized in some equilibrium state, with
all of the $\dot{q}_i$ set to zero, then it will remain in this state for ever. 
But what happens if the system is slightly perturbed from the equilibrium
state?

Let
\begin{equation}
q_i = q_{i\,0} + \delta q_i,
\end{equation}
for $i=1,{\cal F}$, where the $\delta q_i$ are {\em small}. To lowest order in $\delta q_i
$, the
kinetic energy (\ref{e12.1}) can be written
\begin{equation}\label{e12.6}
K \simeq \frac{1}{2}\sum_{i,j=1,{\cal F}} M_{ij}\,\delta\dot{q}_i\,\delta\dot{q}_j,
\end{equation}
where 
\begin{equation}\label{e12.7}
M_{ij} = m_{ij}(q_{1\,0},q_{2\,0},\cdots,q_{{\cal F}\,0}),
\end{equation}
and
\begin{equation}\label{e12.8}
M_{ij} = M_{ji}.
\end{equation}
Note that the weights $M_{ij}$ in the quadratic form (\ref{e12.6}) are now {\em constants}. 

Taylor expanding the potential energy function about the equilibrium state, up to second-order in the $\delta q_i$, we obtain
\begin{equation}\label{e12.9}
U \simeq U_0 - \sum_{i=1,{\cal F}} Q_{i\,0}\,\delta q_i - \frac{1}{2}\sum_{i,j=1,{\cal F}} G_{ij}\,\delta q_i \,\delta q_j,
\end{equation}
where $U_0=U(q_{1\,0},q_{2\,0},\cdots, q_{{\cal F}\,0})$, the $Q_{i\,0}$
are specified in Equation~(\ref{e12.2}), and
\begin{equation}\label{e12.10}
G_{ij} = -\frac{\partial^2 U(q_{1\,0},q_{2\,0},\cdots, q_{{\cal F}\,0})}{\partial q_i\,\partial q_j}.
\end{equation}
Now, we can set $U_0$ to zero without loss of generality. Moreover, according to Equation~(\ref{e12.2}), the
$Q_{i\,0}$ are all zero. Hence, the  expression (\ref{e12.9}) reduces to
\begin{equation}\label{e12.11}
U \simeq  - \frac{1}{2}\sum_{i,j=1,{\cal F}} G_{ij}\,\delta q_i \,\delta q_j.
\end{equation}
Note that, since $\partial^2 U/\partial q_i\,\partial q_j\equiv
\partial^2 U/\partial q_j\,\partial q_i$, the constants weights $G_{ij}$  in the
above quadratic form are invariant under interchange of the indices $i$ and $j$: {\em i.e.},
\begin{equation}\label{e12.12}
G_{ij} = G_{ji}.
\end{equation}

With $K$ and $U$ specified by the quadratic forms (\ref{e12.6}) and (\ref{e12.11}), respectively, Lagrange's equations of motion (\ref{e12.3}) reduce to
\begin{equation}\label{e12.13}
\sum_{j=1,{\cal F}}\left(M_{ij}\,\delta\ddot{q}_j - G_{ij}\,\delta q_j\right) = 0,
\end{equation}
for $i=1,{\cal F}$. 
Note that the above coupled differential equations are {\em linear} in the $\delta q_i$. It follows
that the solutions are {\em superposable}.
Let us search for  solutions of the above equations in which all of the perturbed coordinates $\delta q_i$
have a common time variation of the form
\begin{equation}\label{e12.14}
\delta q_i(t) = \delta q_i\,{\rm e}^{\,\gamma\,t},
\end{equation}
for $i=1,{\cal F}$. 
Now, Equations~(\ref{e12.13}) are a set of ${\cal F}$ second-order differential equations.
Hence, the most general solution contains $2{\cal F}$ arbitrary constants of integration. Thus, if we can find sufficient independent solutions of the form (\ref{e12.14}) to Equations~(\ref{e12.13})  that the superposition
of these solutions contains $2{\cal F}$ arbitrary constants then we can be sure that we
have found the most general solution. Equations
(\ref{e12.13}) and (\ref{e12.14}) yield
\begin{equation}\label{e12.15}
\sum_{j=1,{\cal F}}(G_{ij}- \gamma^2\,M_{ij})\,\delta q_j = 0,
\end{equation}
which can be written more succinctly as a matrix equation:
\begin{equation}\label{e12.16}
({\bf G} - \gamma^2\,{\bf M})\,\delta\bf{q} = {\bf 0}.
\end{equation}
Here, ${\bf G}$ is the real [see Equation~(\ref{e12.10})] symmetric [see Equation~(\ref{e12.12})] ${\cal F}\times {\cal F}$ matrix of the $G_{ij}$ values. 
Furthermore, ${\bf M}$ is the real [see Equation~(\ref{e12.1})] symmetric [see Equation~(\ref{e12.8})] ${\cal F}\times {\cal F}$ matrix of the $M_{ij}$ values. Finally,
$\delta {\bf q}$ is the $1\times {\cal F}$ vector of the $\delta q_i$ values, and
${\bf 0}$ is a null vector.

\section{More Matrix Eigenvalue Theory}\label{smat2}
Equation (\ref{e12.16}) takes the form of a matrix eigenvalue equation:
\begin{equation}\label{e12.17}
({\bf G} - \lambda\,{\bf M})\,\bf{x} = {\bf 0}.
\end{equation}
Here, ${\bf G}$ and ${\bf M}$ are both real symmetric matrices, whereas
$\lambda$ is  termed the {\em eigenvalue}, and ${\bf x}$ the associated {\em eigenvector}. 
The above matrix equation is essentially a set of ${\cal F}$ homogeneous simultaneous algebraic equations for the components of ${\bf x}$. As is well-known, a necessary condition for such a set of equations to possess a
non-trivial solution is that the {\em determinant}\/ of the matrix must
be zero: {\em i.e.}, 
\begin{equation}
|{\bf G} -\lambda\,{\bf M}| = 0.
\end{equation}
The above formula reduces to an ${\cal F}$th-order {\em polynomial}\/ equation
for $\lambda$. Hence, we conclude that Equation~(\ref{e12.17}) is satisfied
by ${\cal F}$ eigenvalues, and ${\cal F}$ associated eigenvectors.

We can easily demonstrate that the eigenvalues are all {\em real}. Suppose
that $\lambda_k$ and ${\bf x}_k$ are the $k$th eigenvalue and
eigenvector, respectively. Then we have
\begin{equation}\label{e12.19}
{\bf G}\,{\bf x}_k = \lambda_k\,{\bf M}\,{\bf x}_k.
\end{equation}
Taking the transpose and complex conjugate of the above equation, and
right multiplying by ${\bf x}_k$, we obtain
\begin{equation}
{\bf x}_k^{\ast\,T}\,{\bf G}^{\ast\,T}\,{\bf x}_k = \lambda_k^\ast\,
{\bf x}_k^{\ast\,T}\,{\bf M}^{\ast\,T}\,{\bf x}_k.
\end{equation}
Here, $~^T$ denotes a transpose, and $~^\ast$ a complex conjugate.
However, since ${\bf G}$ and ${\bf M}$ are both real symmetric
matrices, it follows that ${\bf G}^{\ast\,T} = {\bf G}$ and ${\bf M}^{\ast\,T}={\bf M}$. Hence,
\begin{equation}
{\bf x}_k^{\ast\,T}\,{\bf G}\,{\bf x}_k = \lambda_k^\ast\,
{\bf x}_k^{\ast\,T}\,{\bf M}\,{\bf x}_k.
\end{equation}
Next, left multiplying Equation~(\ref{e12.19}) by ${\bf x}_k^{\ast\,T}$,
we obtain
\begin{equation}
{\bf x}_k^{\ast\,T}\,{\bf G}\,{\bf x}_k = \lambda_k\,
{\bf x}_k^{\ast\,T}\,{\bf M}\,{\bf x}_k.
\end{equation}
Taking the difference between the above two expressions, we get
\begin{equation}
(\lambda_k^\ast-\lambda_k)\,{\bf x}_k^{\ast\,T}\,{\bf M}\,{\bf x}_k = 0.
\end{equation}
Since ${\bf x}_k^{\ast\,T}\,{\bf M}\,{\bf x}_k$ is not generally zero, except in the trivial case where
${\bf x}_k$ is a null vector, we conclude that $\lambda_k^\ast = \lambda_k$
for all $k$. In other words, the eigenvalues are all real. It immediately
follows that the eigenvectors can also be chosen to be all real.

Consider two distinct eigenvalues, $\lambda_k$ and $\lambda_l$, with
the associated eigenvectors ${\bf x}_k$ and ${\bf x}_l$, respectively. 
We have
\begin{eqnarray}\label{e12.24}
{\bf G}\,{\bf x}_k &=& \lambda_k\,{\bf M}\,{\bf x}_k,\\[0.5ex]
{\bf G}\,{\bf x}_l &=& \lambda_l\,{\bf M}\,{\bf x}_l.\label{e12.25}
\end{eqnarray}
Right multiplying the transpose of Equation~(\ref{e12.24}) by ${\bf x}_l$,
and left multiplying Equation (\ref{e12.25}) by ${\bf x}_k^T$, we obtain
\begin{eqnarray}
{\bf x}_k^T\,{\bf G}\,{\bf x}_l&=& \lambda_k\,{\bf x}_k^T\,{\bf M}\,{\bf x}_l,\\[0.5ex]
{\bf x}_k^T\,{\bf G}\,{\bf x}_l &=& \lambda_l\,{\bf x}_k^T\,{\bf M}\,{\bf x}_l.
\end{eqnarray}
Taking the difference between the above two expressions, we get
\begin{equation}
(\lambda_k-\lambda_l)\,{\bf x}_k^T\,{\bf M}\,{\bf x}_l = 0.
\end{equation}
Hence, we conclude that
\begin{equation}
{\bf x}_k^T\,{\bf M}\,{\bf x}_l = 0,
\end{equation}
provided $\lambda_k\neq \lambda_l$. 
In other words, two eigenvectors corresponding to two different eigenvalues
are ``orthogonal'' to one another (in the sense specified in the above equation).
Moreover, it is easily demonstrated that different eigenvectors corresponding to the same
eigenvalue can be chosen in such a manner that they are also orthogonal to one another---see Section~\ref{smatrix}. Thus, we conclude
that all of the eigenvectors are {\em mutually orthogonal}. Since Equation~(\ref{e12.17}) only specifies the directions, and not the lengths, of the
eigenvectors, we are free to normalize our eigenvectors such that
\begin{equation}\label{e12.30}
{\bf x}_k^T\,{\bf M}\,{\bf x}_l = \delta_{kl},
\end{equation}
where $\delta_{kl}=1$ when $k=l$, and $\delta_{kl}=0$ otherwise. 
Note, finally, that since the ${\bf x}_k$, for $k=1,{\cal F}$, are mutually orthogonal,
they are {\em independent}\/ ({\em i.e.}, one eigenvector cannot be expressed as a linear combination of the others), and completely span ${\cal F}$-dimensional vector space.

\section{Normal Modes} 
It follows from Equation~(\ref{e12.14}) and (\ref{e12.15}), plus the
mathematical results contained in the previous section, that the
most general solution to Equation~(\ref{e12.13}) can be written
\begin{equation}\label{e12.31}
\delta {\bf q}(t) = \sum_{k=1,{\cal F}}\delta {\bf q}_k(t),
\end{equation}
where
\begin{equation}\label{e12.32}
\delta {\bf q}_k(t) = \left(\alpha_k \,{\rm e}^{+\sqrt{\lambda_k}\,t}+\beta_k\,{\rm e}^{-\sqrt{\lambda_k}\,t}\right)
{\bf x}_k. 
\end{equation}
Here, the $\lambda_k$ and the ${\bf x}_k$ are the eigenvalues and eigenvectors obtained by solving Equation~(\ref{e12.17}). Moreover, the $\alpha_k$ and $\beta_k$ are arbitrary constants. Finally, we have made
use of the fact that the two roots of $\gamma^2=\lambda_k$ are
$\gamma=\pm \sqrt{\lambda}_k$. 

According to Equation~(\ref{e12.31}), the most general perturbed motion of the
system consists of a {\em linear combination}\/ of ${\cal F}$ different modes. These
modes are generally termed {\em normal modes}, since they are mutually
orthogonal (because the ${\bf x}_k$ are mutually orthogonal).  Furthermore, it follows
from the independence of the ${\bf x}_k$ that the normal
modes are also {\em independent}\/ ({\em i.e.}, one mode cannot be expressed as a
linear combination of the others). The $k$th normal mode has a specific
pattern of motion which is specified by the $k$th  eigenvector, ${\bf x}_k$. 
Moreover, the $k$th  mode has a specific time variation which is determined by the associated eigenvalue, $\lambda_k$. Recall that
$\lambda_k$ is {\em real}. Hence, there are only two possibilities. Either $\lambda_k$ is {\em positive}, in which case we can write
\begin{equation}
\delta {\bf q}_k(t) = \left(\alpha_k \,{\rm e}^{+\gamma_k\,t}+\beta_k\,{\rm e}^{-\gamma_k\,t}\right)
{\bf x}_k,
\end{equation}
where $\lambda_k = \gamma_k^{\,2}$, or 
$\lambda_k$ is {\em negative}, in which case we can write
\begin{equation}
\delta {\bf q}_k(t) = \left(\alpha_k \,{\rm e}^{+{\rm i}\,\omega_k\,t}+\beta_k\,{\rm e}^{-{\rm i}\,\omega_k\,t}\right)
{\bf x}_k,
\end{equation}
where $\lambda_k = -\omega_k^{\,2}$. In other words, if $\lambda_k$ is
positive then the $k$th normal mode {\em grows}\/ or decays secularly in time, whereas
if $\lambda_k$ is negative then the $k$th normal mode {\em oscillates}\/  in time.
Obviously, if the system possesses one or more normal modes which grow
secularly in time then the equilibrium about which we originally expanded the
equations of motion must be an {\em unstable}\/ equilibrium. On the
other hand, if all of the normal modes oscillate in time then the equilibrium is
{\em stable}. Thus, we conclude that whilst Equation~(\ref{e12.2})
is the condition for the existence of an equilibrium state in a many degree of freedom system, the condition for the
equilibrium to be {\em stable}\/ is that all of the eigenvalues of the stability
equation (\ref{e12.17}) must be {\em negative}.

The arbitrary constants $\alpha_k$ and $\beta_k$ appearing in expression (\ref{e12.32})
are determined from the {\em initial conditions}. Thus, if $\delta {\bf q}^{(0)}
= \delta {\bf q}(t=0)$ and $\delta \dot{\bf q}^{(0)} = \delta \dot{\bf q}(t=0)$
then  it is easily demonstrated from Equations~(\ref{e12.30})--(\ref{e12.32})
that
\begin{equation}
{\bf x}_k^{T}\,{\bf M}\,\delta {\bf q}^{(0)} = \alpha_k + \beta_k,
\end{equation}
and
\begin{equation}
{\bf x}_k^{T}\,{\bf M}\,\delta \dot{\bf q}^{(0)} =\sqrt{\lambda}_k\,(\alpha_k - \beta_k).
\end{equation}
Hence,
\begin{eqnarray}
\alpha_k &=& \frac{{\bf x}_k^{T}\,{\bf M}\,\delta{\bf q}^{(0)} +{\bf x}_k^{T}\,{\bf M}\,\delta \dot{\bf q}^{(0)}/\sqrt{\lambda}_k}{2},\\[0.5ex]
\beta_k &=& \frac{{\bf x}_k^{T}\,{\bf M}\,\delta {\bf q}^{(0)} -{\bf x}_k^{T}\,{\bf M}\,\delta \dot{\bf q}^{(0)}/\sqrt{\lambda}_k}{2}.
\end{eqnarray}
Note, finally, that since there are $2{\cal F}$ arbitrary constants (two for each of the ${\cal F}$ normal modes), we can be sure that Equation~(\ref{e12.31}) represents the
most general solution to Equation~(\ref{e12.13}).

\section{Normal Coordinates}
Since the eigenvectors ${\bf x}_k$, for $k=1,{\cal F}$, span ${\cal F}$-dimensional
vector space, we can always write the displacement vector $\delta{\bf q}$
as some linear combination of the ${\bf x}_k$: {\em i.e.}, 
\begin{equation}\label{e12.39}
\delta {\bf q}(t) = \sum_{k=1,{\cal F}}\eta_k(t)\,{\bf x}_k.
\end{equation}
We can regard the $\eta_k(t)$ as a new set of generalized coordinates,
since specifying the $\eta_k$ is equivalent to specifying the $\delta q_k$
(and, hence, the $q_k$). The $\eta_k$ are usually termed {\em normal coordinates}. According to Equations~(\ref{e12.30}) and (\ref{e12.39}), the
normal coordinates can be written in terms of the $\delta q_k$ as
\begin{equation}\label{e12.40}
\eta_k = {\bf x}_k^T\,{\bf M}\,\delta{\bf q}.
\end{equation}
Let us now try to express $K$, $U$, and the equations of motion
in terms of the $\eta_k$. 

The kinetic
energy can be written 
\begin{equation}
K = \frac{1}{2}\,\delta \dot{\bf q}^T\,{\bf M}\,\delta\dot{\bf q},
\end{equation}
where use has been made of Equation~(\ref{e12.6}).
It follows from (\ref{e12.39}) that
\begin{equation}
K = \frac{1}{2}\sum_{k,l=1,{\cal F}} \dot{\eta}_k\,\dot{\eta}_l\,{\bf x}_k^T\,{\bf M}\,{\bf x}_l.
\end{equation}
Finally, making use of the orthonormality condition (\ref{e12.30}),
we obtain
\begin{equation}\label{e12.44}
K = \frac{1}{2}\sum_{k=1,{\cal F}}\dot{\eta}_k^{\,2}.
\end{equation}
Hence,  the kinetic energy $K$ takes the form of a {\em diagonal}\/ quadratic form when expressed in terms of the normal coordinates.

The potential energy can be written
\begin{equation}
U=- \frac{1}{2}\,\delta {\bf q}^T\,{\bf G}\,\delta{\bf q},
\end{equation}
where use has been made of Equations~(\ref{e12.11}). It follows from (\ref{e12.39}) that
\begin{equation}
U =- \frac{1}{2}\sum_{k,l=1,{\cal F}} \eta_k\,\eta_l\,{\bf x}_k^T\,{\bf G}\,{\bf x}_l.
\end{equation}
Finally, making use of Equation~(\ref{e12.19}) and the orthonormality condition (\ref{e12.30}),
we obtain
\begin{equation}\label{e12.47}
U = -\frac{1}{2}\sum_{k=1,{\cal F}}\lambda_k\, \eta_k^{\,2}.
\end{equation}
Hence, the potential energy $U$ also takes the form of a {\em diagonal}\/ quadratic form when expressed in terms of the normal coordinates.

Writing Lagrange's equations of motion in terms of the normal
coordinates, we obtain [{\em cf.},  Equation~(\ref{e12.3})]
\begin{equation}
\frac{d}{dt}\!\left(\frac{\partial K}{\partial \dot{\eta}_k}\right) - \frac{\partial K}{\partial \eta_k}+ \frac{\partial U}{\partial \eta_k} = 0,
\end{equation}
for $k=1,{\cal F}$. Thus, it follows from Equations~(\ref{e12.44}) and (\ref{e12.47})
that
\begin{equation}
\ddot{\eta}_k = \lambda_k\,\eta_k,
\end{equation}
for $k=1,{\cal F}$. In other words, Lagrange's equations reduce to a set of
${\cal F}$ {\em uncoupled}\/ simple harmonic equations when expressed in terms
of the normal coordinates. The solutions to the above equations are obvious:
{\em i.e.}, 
\begin{equation}\label{e12.50}
\eta_k(t) = \alpha_k \,{\rm e}^{+\sqrt{\lambda_k}\,t}+\beta_k\,{\rm e}^{-\sqrt{\lambda_k}\,t},
\end{equation}
where $\alpha_k$ and $\beta_k$ are arbitrary constants. Hence, it is clear from Equations~(\ref{e12.39}) and (\ref{e12.50}) that the most general solution to
the perturbed equations of motion is indeed given by Equations~(\ref{e12.31})
and (\ref{e12.32}).

In conclusion, the  equations of motion of a many degree of
freedom dynamical system which is slightly perturbed from an equilibrium state take a particularly simple form when expressed
in terms of the normal coordinates. Each normal
coordinate specifies the instantaneous displacement of an independent 
mode of oscillation (or secular growth) of the system. Moreover, each
normal coordinate oscillates at a characteristic frequency (or grows at a characteristic rate), and is completely unaffected by the other coordinates.

\section{Spring-Coupled Masses}
Consider the two degree of freedom dynamical system pictured in Figure \ref{spring}. In this system, two point objects of mass $m$ are free
to move in one dimension.  Furthermore, the masses are 
connected together by a spring of spring constant $k$, and are also each attached to
 fixed supports via  springs of spring constant $k'$. 
\begin{figure}
\epsfysize=1.in
\centerline{\epsffile{Chapter11/fig11.01.eps}}
\caption{\em Two spring-coupled masses.}\label{spring}
\end{figure}

Let $q_1$ and $q_2$ be the displacements of the first and second masses,
respectively, from the equilibrium state. It follows that the
extensions of the left-hand, middle, and right-hand springs are 
$q_1$, $q_2-q_1$, and $-q_2$, respectively. The kinetic energy of the system takes the
form
\begin{equation}\label{e12.51}
K = \frac{m}{2} \,(\dot{q}_1^{\,2} + \dot{q}_2^{\,2}),
\end{equation}
whereas the potential energy is written
\begin{equation}
U= \frac{1}{2}\left[k'\,q_1^{\,2} + k\,(q_2-q_1)^2+ k'\,q_2^{\,2}\right].
\end{equation}
The above expression can be rearranged to give
\begin{equation}\label{e12.53}
U= \frac{1}{2}\left[(k+k')\,q_1^{\,2} -2\,k\,q_1\,q_2 + (k+k')\,q_2^{\,2}\right].
\end{equation}

A comparison of Equations~(\ref{e12.51}) and (\ref{e12.53}) with the standard
forms (\ref{e12.6}) and (\ref{e12.11}) yields the following
expressions for the mass matrix, ${\bf M}$, and the force matrix, ${\bf G}$:
\begin{eqnarray}\label{e12.54}
{\bf M} &=& \left(
\begin{array}{cc}
m& 0\\
0& m\end{array}
\right),\\[1ex]
{\bf G} &=& \left(\begin{array}{cc}
-k-k'& k\\
k& -k-k'\end{array}\right).
\end{eqnarray}
Now, the equation of motion of the system takes the form [see Equation (\ref{e12.17})]
\begin{equation}\label{e12.56}
({\bf G} -\lambda\,{\bf M})\,{\bf x} = {\bf 0},
\end{equation}
where ${\bf x}$ is the column vector of the $q_1$ and $q_2$ values.
The solubility condition for the above equation is
\begin{equation}
\left|{\bf G} - \lambda\,{\bf M}\right| = 0,
\end{equation}
or
\begin{equation}
\left|\begin{array}{cc}
-k-k'-\lambda\,m& k\\[0.5ex]
k&-k-k'-\lambda\,m
\end{array}\right| = 0,
\end{equation}
which yields the following quadratic equation for the eigenvalue $\lambda$:
\begin{equation}
m^2\,\lambda^2 + 2\,m\,(k+k')\,\lambda + k'\,(k'+2\,k) = 0.
\end{equation}

The two roots of the above equation are
\begin{eqnarray}
\lambda_1 &=& - \frac{k'}{m},\\[0.5ex]
\lambda_2 &=& - \frac{(2\,k+k')}{m}.
\end{eqnarray}
The fact that the roots are negative implies that both normal modes are
{\em oscillatory}\/ in nature: {\em i.e.}, the original equilibrium is {\em stable}. 
The characteristic oscillation frequencies of the modes are
\begin{eqnarray}\label{e12.62}
\omega_1 &=& \sqrt{-\lambda_1} = \sqrt{\frac{k'}{m}},\\[0.5ex]
\omega_2 &=& \sqrt{-\lambda_2} = \sqrt{\frac{2\,k+k'}{m}}.\label{e12.63}
\end{eqnarray}

Now, the first row of Equation~(\ref{e12.56}) gives
\begin{equation}
\frac{q_1}{q_2}= \frac{k}{k+k'+\lambda\,m}.
\end{equation}
Moreover, Equations~(\ref{e12.30}) and (\ref{e12.54}) yield the following
normalization condition for the eigenvectors:
\begin{equation}
{\bf x}_k^T\,{\bf x}_k = m^{-1},
\end{equation}
for $k=1,2$.  It follows that the two eigenvectors are
\begin{eqnarray}\label{e12.66}
{\bf x}_1 &=& (2\,m)^{-1/2}\,(1,\,\,\,\, 1),\\[0.5ex]
{\bf x}_2 &=& (2\,m)^{-1/2}\,(1,-1).\label{e12.67}
\end{eqnarray}

According to Equations~(\ref{e12.62})--(\ref{e12.63}) and (\ref{e12.66})--(\ref{e12.67}), our two degree of freedom system possesses two normal modes. The first mode oscillates at the frequency $\omega_1$, and is a purely {\em symmetric}\/
mode: {\em i.e.}, $q_1 = q_2$. Note that such a mode does not stretch
the middle spring. Hence, $\omega_1$ is independent of $k$. In fact, $\omega_1$ is simply the characteristic oscillation frequency of a mass $m$ on the end of a spring of spring constant $k'$. The second mode
oscillates at the frequency $\omega_2$, and is a purely
{\em anti-symmetric}\/ mode: {\em i.e.}, $q_1=-q_2$. Since such a mode
stretches the middle spring, the second mode experiences a greater restoring force than the first, and hence has a higher oscillation frequency: {\em i.e.},
$\omega_2> \omega_1$. 

Note, finally, from Equations~(\ref{e12.40}) and (\ref{e12.54}), that the normal
coordinates of the system are:
\begin{eqnarray}
\eta_1 &=& \sqrt{\frac{m}{2}}\,(q_1 + q_2),\\[0.5ex]
\eta_2 &=& \sqrt{\frac{m}{2}}\,(q_1-q_2).
\end{eqnarray}
When expressed in terms of these normal coordinates, the kinetic
and potential energies of the system reduce to
\begin{eqnarray}
K &=& \frac{1}{2}\,(\dot{\eta}_1^{\,2} + \dot{\eta}_2^{\,2}),\\[0.5ex]
U &=&\frac{1}{2}\,(\omega_1^{\,2}\,\eta_1^{\,2} + \omega_2^{\,2}\,\eta_2^{\,2}),
\end{eqnarray}
respectively.

\section{Triatomic Molecule}
Consider the simple model of a linear triatomic molecule ({\em e.g.}, carbon
dioxide) illustrated in Figure~\ref{triatomic}. The molecule consists
of a central atom of mass $M$ flanked by two identical atoms of
mass $m$. The atomic bonds are represented as springs of spring constant $k$.
The linear displacements of the flanking atoms are $q_1$ and $q_2$,
whilst that of the central atom is $q_3$. Let us investigate the linear  modes of oscillation our model molecule.
\begin{figure}
\epsfysize=1.25in
\centerline{\epsffile{Chapter11/fig11.02.eps}}
\caption{\em A model triatomic molecule.}\label{triatomic}
\end{figure}

The kinetic energy of the molecule is written
\begin{equation}\label{e12.72}
K = \frac{m}{2}\,(\dot{q}_1^{\,2} + \dot{q}_2^{\,2})+ \frac{M}{2}\,\dot{q}_3^{\,2},
\end{equation}
whereas the potential energy takes the form
\begin{equation}\label{e12.73}
U = \frac{k}{2}\,(q_3-q_1)^2 + \frac{k}{2}\,(q_2-q_3)^2.
\end{equation}
Clearly, we have a three degree of freedom dynamical system. However, we
can reduce this to a two degree of freedom system by only considering
{\em oscillatory}\/ modes of motion, and, hence, neglecting {\em translational}\/ modes. We can achieve this by demanding that the center
of mass of the system remains stationary. In other words, we require that
\begin{equation}
m\,(q_1+q_2) + M\,q_3 = 0.
\end{equation}
This constraint can be rearranged to give
\begin{equation}\label{e12.75}
q_3 = - \frac{m}{M}\,(q_1+q_2).
\end{equation}
Eliminating $q_3$ from Equations~(\ref{e12.72}) and (\ref{e12.73}), we obtain
\begin{equation}
K = \frac{m}{2}\left[(1+\alpha)\,\dot{q}_1^{\,2} + 2\,\alpha\,\dot{q}_1\,\dot{q}_2 + (1+\alpha)\,\dot{q}_2^{\,2}\right],
\end{equation}
and
\begin{equation}
U = \frac{k}{2}\left[(1+2\,\alpha+2\,\alpha^2)\,q_1^{\,2} + 4\,\alpha\,(1+\alpha)\,q_1\,q_2 +(1+2\,\alpha+2\,\alpha^2)\,q_2^{\,2}\right],
\end{equation}
respectively, where $\alpha =m/M$. 

A comparison of the above expressions with the standard
forms (\ref{e12.6}) and (\ref{e12.11}) yields the following
expressions for the mass matrix, ${\bf M}$, and the force matrix, ${\bf G}$:
\begin{eqnarray}
{\bf M} &=& m\left(
\begin{array}{cc}
1+\alpha& \alpha\\
\alpha& 1+\alpha\end{array}
\right),\\[1ex]
{\bf G} &=& -k\left(\begin{array}{cc}
1+2\,\alpha+2\,\alpha^2& 2\,\alpha\,(1+\alpha)\\
2\,\alpha\,(1+\alpha)&1+2\,\alpha+2\,\alpha^2 \end{array}\right).
\end{eqnarray}
Now, the equation of motion of the system takes the form [see Equation (\ref{e12.17})]
\begin{equation}\label{e12.80}
({\bf G} -\lambda\,{\bf M})\,{\bf x} = {\bf 0},
\end{equation}
where ${\bf x}$ is the column vector of the $q_1$ and $q_2$ values.
The solubility condition for the above equation is
\begin{equation}
\left|{\bf G} - \lambda\,{\bf M}\right| = 0,
\end{equation}
which yields the following quadratic equation for the eigenvalue $\lambda$:
\begin{equation}
(1+2\,\alpha)\,\left[m^2\,\lambda^2 + 2\,m\,k\,(1+\alpha)\,\lambda + k^2\,(1+2\alpha)\right] = 0.
\end{equation}

The two roots of the above equation are
\begin{eqnarray}
\lambda_1 &=& - \frac{k}{m},\\[0.5ex]
\lambda_2 &=& - \frac{k\,(1+2\alpha)}{m}.
\end{eqnarray}
The fact that the roots are negative implies that both normal modes are indeed
{\em oscillatory} in nature. The characteristic oscillation frequencies are
\begin{eqnarray}\label{e12.85}
\omega_1 &=& \sqrt{-\lambda_1} = \sqrt{\frac{k}{m}},\\[0.5ex]
\omega_2 &=& \sqrt{-\lambda_2} = \sqrt{\frac{k\,(1+2\,\alpha)}{m}}.
\end{eqnarray}
Equation~(\ref{e12.80}) can now be solved, subject to the normalization
condition (\ref{e12.30}), to give the two eigenvectors:
\begin{eqnarray}
{\bf x}_1 &=& (2\,m)^{-1/2}\,(1,-1),\\[0.5ex]
{\bf x}_2 &=& (2\,m)^{-1/2}\,(1+2\,\alpha)^{-1/2}\,(1,\,\,\,\,1).\label{e12.88}
\end{eqnarray}

Thus, we conclude from Equations~(\ref{e12.75}) and (\ref{e12.85})--(\ref{e12.88}) that our model molecule possesses two normal modes of oscillation. The first mode oscillates at the frequency $\omega_1$, and
is an {\em anti-symmetric}\/ mode in which $q_1=-q_2$ and $q_3=0$. 
In other words, in this mode of oscillation, the two end atoms move in opposite
directions whilst the central atom remains stationary. The second mode oscillates at the frequency $\omega_2$, and is a mixed symmetry mode in which
 $q_1=q_2$ but $q_3=-2\,\alpha\,q_1$. In other words, in this mode of oscillation,  the
 two end atoms move in the same direction whilst the central atom moves
 in the opposite direction.
 
 Finally, it is easily demonstrated that the normal coordinates of the system
 are 
 \begin{eqnarray}
 \eta_1 &=& \sqrt{\frac{m}{2}}\,(q_1-q_2),\\[0.5ex]
 \eta_2 &=& \sqrt{\frac{m\,(1+2\,\alpha)}{k}}\,(q_1+q_2).
 \end{eqnarray}
 When expressed in terms of these coordinates, $K$ and $U$ reduce to
 \begin{eqnarray}
K &=& \frac{1}{2}\,(\dot{\eta}_1^{\,2} + \dot{\eta}_2^{\,2}),\\[0.5ex]
U &=&\frac{1}{2}\,(\omega_1^{\,2}\,\eta_1^{\,2} + \omega_2^{\,2}\,\eta_2^{\,2}),
\end{eqnarray}
respectively.

\section{Exercises}
{\small
\renewcommand{\theenumi}{11.\arabic{enumi}}
\begin{enumerate}
\item A particle of mass $m$ is attached to a rigid support by means of a spring of
spring constant $k$. At equilibrium, the spring hangs vertically
downward. An identical oscillator is added to this system, the
spring of the former being attached to the mass of the latter.
Calculate the characteristic frequencies for one-dimensional vertical
oscillations, and describe the associated normal modes.

\item A simple pendulum consists of a bob of mass $m$ suspended
by an inextensible light string of length $l$. From the bob of
this pendulum, a second identical
pendulum is suspended. Consider the case of small angle oscillations
in the same vertical plane. Calculate the characteristic frequencies, and describe the associated normal modes.

\item A thin hoop of radius $a$ and mass $m$ oscillates in its
own plane (which is constrained to be vertical) hanging from a single fixed point. A small mass $m$ slides
without friction along the hoop. Consider the case of small oscillations.
Calculate the characteristic frequencies, and describe the associated normal modes.

\item A simple pendulum of mass $m$ and length $l$ is suspended from
 a block of mass $M$ which is constrained to slide along a frictionless
horizontal track. Consider the case of small oscillations. Calculate the characteristic frequencies, and describe the associated normal modes.

\item A thin uniform rod of mass $m$ and length $a$ is suspended from
one end by a light string of length $l$. The other end of the string
is attached to a fixed support. Consider the case of small oscillations
in a vertical plane. 
Calculate the characteristic frequencies, and describe the associated normal modes.

\item A triatomic molecule consists of three atoms of equal mass.
Each atom is attached to the other two atoms via identical chemical bonds.
The equilibrium state of the molecule is such that the masses are at the vertices of 
an equilateral triangle of side $a$. Modeling the chemical bonds as
springs of spring constant $k$, and only considering motion in the plane of the molecule, find the vibrational frequencies and normal modes of the molecule. Exclude translational and rotational modes. 
\end{enumerate}
}
